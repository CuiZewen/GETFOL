\gfcommand{falsei}{implication elimination for negation or false introduction}
\index{falsei}\index{fi}

\gfsyntax{
   falsei \ALT fi \ARG{fact1} \OPT{,} \ARG{fact2}; 
}

\gfdescription{
   \renewcommand{\arraystretch}{0.5}
   \[
   \begin{array}{l}
      \imp E_{\neg}  
   \end{array}
   \ \ 
   \begin{array}{c}
      \fraz{A \ \ \neg A}
           {\bot}
   \end{array}
   \]
   \renewcommand{\arraystretch}{1}

   One fact must be the negation of the other. 
   The wff derived is {\tt FALSE} and its dependencies are the union of the
   dependencies of both facts.
}

\gfrecap{
   One fact must be the negation of the other. 
   The wff derived is `FALSE' and its dependencies are the union of the
   dependencies of both facts.
}

\gfexample+
   ***** declare sentconst A;
   [...]

   ***** assume A not A;
   1  A  (1)
   2  not A  (2)

   ***** falsei 1 2;
   3  FALSE  (1 2)

   ***** falsei 2 1;
   4  FALSE  (1 2)
+

\gfnotes{
   This rule can be seen as a special case of introduction elimination
   in which the main symbol of one of the premises must be {\tt not} ($\neg$)
   rather than {\tt imp} ($\imp$). 
}
