\section{Language}
\label{sec-lang}
\label{sec-decl}


A {\GF} language is a first-order language.
Its terms, atomic formulae, and
well-formed formulae are built from primitive symbols representing
variables, constants, functions, predicates,
connectives and quantifiers. The various syntactic categories of expressions,
corresponding to the different logical categories, are
recursively defined in terms of each other,
these definitions determining the written syntax of the
language\footnote{By written syntax we mean the syntax of the
expressions as they are typed rather than as they are represented
internally.}.

Each primitive symbol has a {\bf syntype}
which corresponds to its
logical status. Symbols of certain {\bf syntype}s
must also have other associated
information (such as their arity) which together with the {\bf syntype}
determines
how the symbol is to be used in the construction of compound expressions.

As every {\GF} language is a first-order language there are certain
logical symbols (such as connectives and quantifiers)
which are predefined. The other primitive
symbols of a language must be user defined by means of declarations 
(see the command {\bf declare}) which
specify a symbol together with its {\bf syntype}
(and any other information
required).

\subsection{{\GF} symbols}

{\GF} commands accept symbols that we will identify with
{\em sym}. The {\GF} scanner recognizes different {\em types} of
characters: {\em identifiers}, {\em numbers} (positive integers),
{\em escape characters},
{\em delimiters} and special delimiters that 
may be regarded as tokens: {\em identifier-delimiters}.
{\GF} symbols {\em sym}s are sequences of {\em identifiers}
and {\em numbers}. Identifiers are for instance:
\begin{verbatim}
                       a b c d
                       A B C D
                       ! " # $ % &
\end{verbatim}    %%% $ (fake emacs' hilite mode)
{\em delimiters} are used to separate symbols.
Some delimiters are:
\begin{verbatim}
                       ( ) , . : ; [ ] 
\end{verbatim}
{\em identifier-delimiters} may be used to separate
symbols or may identify a particular token, as for instance
\begin{verbatim}
                       < > = + - * / ? @ 
\end{verbatim}
Escape characters turn a character into an identifier.
The character $\backslash$ is {\GF}'s escape character.

Types are assigned to characters in the file {\tt asciitab.fol}
in the {\tt fol} directory.

An example:
\begin{verbatim}
***** declare sentconst a!b;
a!b has been declared to be a Sentconst
***** declare sentconst a_b;
a_b has been declared to be a Sentconst
***** declare sentconst a-b;
a has been declared to be a Sentconst
- has been declared to be a Sentconst
b has been declared to be a Sentconst
***** declare sentconst c/d;
c has been declared to be a Sentconst
/ has been declared to be a Sentconst
d has been declared to be a Sentconst

***** declare sentconst e\f;
ef has been declared to be a Sentconst
\end{verbatim}

\subsection{{\GF} special symbols}

The following strings are treated as special symbols:

\begin{itemize}
\item The logical connectives:
{\tt not, and, or, imp, iff, wffif, trmif, forall, exists};
\item The sentential constants for truth and falsity:
{\tt FALSE, TRUE}; 
\item The equality predicate symbol: {\tt =};
\item Sorts: {\tt UNIVERSAL}, {\tt NATNUMSORT};
\item Numerals: {\tt 1}, {\tt 2}, {\tt 3}, ...;
\item Representations: {\tt UNIVERSALREP}, {\tt NATNUMREP};
\item Natural numbers: {\tt 1}, {\tt 2} {\tt 3}, ...;
\item Various: {\tt TRUTHSORT}, {\tt TRUTHREP}, {\tt NOTNAMED\&}, {\tt UNDEF\&};
\end{itemize}



Using these symbols not in ``reading mode",
that is modifying their meaning,
is very dangerous. The behavior
of the system becomes unpredictable.






\subsection{Syntypes of primitive symbols}

The different {\bf syntype}s of primitive symbols are:

\gap
\begin{center}
\fbox{
\parbox{15cm}{
\begin{itemize}

\item {\indvar} --- individual variables.

\item {\indpar} --- individual parameters.

\item {\indconst} --- individual constants.

\item {\funpar} --- function parameters.

\item {\funconst} --- function constants.

\item {\predpar} --- predicate parameters.

\item {\predconst} --- predicate constants.

\item {\sentpar} --- sentential parameters.

\item {\sentconst} --- sentential constants.

\item {\sentconn} --- sentential connectives.

\item {\quant} --- quantifiers.

\end{itemize}
}}
\end{center}

\gap
Some of the above {\bf syntype}s are grouped together into hybrid syntactic
categories according to the role they play
in defining logical expressions.

\gap
\begin{center}
\fbox{
\parbox{15cm}{
{\indsym} ::= {\indvar} \I {\indpar} \I {\indconst}

{\funsym} ::= {\funpar} \I {\funconst}

{\predsym} ::= {\predpar} \I {\predconst}

{\sentsym} ::= {\sentpar} \I {\sentconst}
}}
\end{center}

\subsection{Logical expressions}

The various categories of compound logical expressions are defined
recursively in terms of each other and in terms of the categories of
primitive symbols. The most important categories, from a logical point of
view, are: {\wff} (the category of well-formed formulae), {\awff} (the
category of atomic formulae) and {\term} (the category of terms).

The following definition gives the 
grammar of expressions accepted by {\GF}. In all cases $n$
is a natural number $\geq 1$.
Superscripts of
---$^n$, ---$^P$, ---$^I$ are used with certain categories,
the categories of operator symbols, to specify other properties which the
permitted symbols must possess; the three superscripts require
respectively: that the
symbol has arity $n$, that the symbol is a unary prefix operator, that the
symbol is a binary infix operator. Operator symbols and their associated
properties are treated in full in section \ref{opsym}.

\gap
\begin{center}
\fbox{
\parbox{15cm}{
{\wff} ::= {\bf (} {\wff} {\bf )} \I {\connappl} \I {\quantwff}
\I {\wffif} \I {\awff}

{\connappl} ::= {\sentconn}$^{P}$ {\wff} \I {\wff}$_1$
{\sentconn}$^{I}$ {\wff}$_2$

{\quantwff} ::= {\quantprefix} {\wff}

{\quantprefix} ::= {\quant} {\indvar}$_1$ ... {\indvar}$_n$ {\bf .}

{\wffif} ::= {\bf wffif} {\wff}$_1$ {\bf then} {\wff}$_2$ {\bf else}
{\wff}$_3$

{\awff} ::= {\sentsym} \I {\predappl}

{\predappl} ::= {\predsym}$^{P}$ {\term} \I
                {\term}$_1$ {\predsym}$^{I}$ {\term}$_2$ \I
                {\predsym}$^n${\bf (} {\term}$_1$ [{\bf ,}] ...
                [{\bf ,}] {\term}$_n$ {\bf )}

{\term} ::= {\bf (} {\term} {\bf )} \I {\funappl} \I {\termif} \I {\indsym}

{\termif} ::= {\bf trmif} {\wff} {\bf then} {\term}$_1$ {\bf else}
{\term}$_2$

{\funappl} ::= {\funsym}$^{P}$ {\term} \I
               {\term}$_1$ {\funsym}$^{I}$ {\term}$_2$ \I
               {\funsym}$^n${\bf (} {\term}$_1$ [{\bf ,}] ...
               [{\bf ,}] {\term}$_n$ {\bf )}
}}
\end{center}

\subsection{Operator symbols}
\label{opsym}

The primitive symbols of the categories  {\funsym},
\predsym, {\sentconn} are operator symbols; each operator
symbol has an associated arity which is a natural number $\geq 1$.
\funsym$^n$, \predsym$^n$, and \sentconn$^n$ are used to denote the
subclasses of {\funsym}, {\predsym} and {\sentconn} of operators with
arity $n$.

An operator symbol with arity $1$ must be defined to be prefix. \funsym$^P$,
\predsym$^P$, and \sentconn$^P$ are used to denote the categories of prefix
operators, these are subclasses of \funsym$^1$, \predsym$^1$, and
\sentconn$^1$ respectively.

An operator symbol with arity $2$ may be defined to be infix. \funsym$^I$,
\predsym$^I$, and \sentconn$^I$  are used to denote the categories of infix
operators, these will be
subclasses of \funsym$^2$, \predsym$^2$, and \sentconn$^2$ respectively.

The use of prefix and infix operators allows the following ambiguous
expressions:


\begin{enumerate}

\item {\term}$_1$ {\funsym}$^I_1$ {\term}$_2$ {\funsym}$^I_2$ {\term}$_3$

\item {\wff}$_1$ {\sentconn}$^I_1$ {\wff}$_2$ {\sentconn}$^I_2$ {\wff}$_3$

\item {\funsym}$^P$ {\term}$_1$ {\funsym}$^I$ {\term}$_2$

\item {\sentconn}$^P$ {\wff}$_1$ {\sentconn}$^I$ {\wff}$_2$

\item {\quantprefix} {\wff}$_1$ {\sentconn}$^I$ {\wff}$_2$~\footnote{This is
a special case as a {\quantprefix} behaves like a prefix operator.}

\end{enumerate}

These ambiguities are avoided through the assignment of additional
information to certain operator symbols declared to be prefix or infix.
Each symbol in \funsym$^P$ and \sentconn$^P$ must
have an associated prefix binding priority (a natural number
$\geq 1$)\footnote{In fact each prefix and infix {\predsym} also has
associated binding priorities but this information is redundant as
ambiguities can not arise.}, this
will be denoted by prbp(---). Each symbol in \funsym$^I$ and \sentconn$^I$
must have two associated binding priorities: the left binding
priority and the right binding priority (both natural numbers $\geq 1$);
these will be denoted by lbp(---) and rbp(---) respectively.

The five ambiguous cases above are disambiguated in the following way:

\begin{enumerate}

\item If rbp({\funsym}$^I_1$) $>$ lbp({\funsym}$^I_2$)
then the expression is parsed as:

({\term}$_1$ {\funsym}$^I_1$ {\term}$_2$) {\funsym}$^I_2$ {\term}$_3$

Otherwise, if
rbp({\funsym}$^I_1$) $\leq$ lbp({\funsym}$^I_2$)
then the expression is parsed as:

{\term}$_1$ {\funsym}$^I_1$ ({\term}$_2$ {\funsym}$^I_2$ {\term}$_3$)

\item If rbp({\sentconn}$^I_1$) $>$ lbp({\sentconn}$^I_2$)
then the expression is parsed as:

({\wff}$_1$ {\sentconn}$^I_1$ {\wff}$_2$) {\sentconn}$^I_2$ {\wff}$_3$

Otherwise, if
rbp({\sentconn}$^I_1$) $\leq$ lbp({\sentconn}$^I_2$)
then the expression is parsed as:

{\wff}$_1$ {\sentconn}$^I_1$ ({\wff}$_2$ {\sentconn}$^I_2$ {\wff}$_3$)

\item If prbp({\funsym}$^P$) $>$ lbp({\funsym}$^I$)
then the expression is parsed as:

({\funsym}$^P$ {\term}$_1$) {\funsym}$^I$ {\term}$_2$

Otherwise, if
rbp({\sentconn}$^I_1$) $\leq$ lbp({\sentconn}$^I_2$)
then the expression is parsed as:

{\wff}$_1$ {\sentconn}$^I_1$ ({\wff}$_2$ {\sentconn}$^I_2$ {\wff}$_3$)

\item If prbp({\sentconn}$^P$) $>$ lbp({\sentconn}$^I$)
then the expression is parsed as:

({\sentconn}$^P$ {\wff}$_1$) {\sentconn}$^I$ {\wff}$_2$

Otherwise, if
prbp({\sentconn}$^P$) $\leq$ lbp({\sentconn}$^I$)
then the expression is parsed as:

{\sentconn}$^P$ ({\wff}$_1$ {\sentconn}$^I$ {\wff}$_2$)

\item If $1000 >$ lbp({\sentconn}$^I$)
then the expression is parsed
as \footnote{1000 is the default prefix binding
priority.}:

({\quantprefix} {\wff}$_1$) {\sentconn}$^I$ {\wff}$_2$

Otherwise, if
$1000 \leq$ lbp({\sentconn}$^I$)
then the expression is parsed as:

{\quantprefix}  ({\wff}$_1$ {\sentconn}$^I$ {\wff}$_2$)

\end{enumerate}


\subsection{The logical symbols}
\label{sec-log-symb}

The logical symbols are hardwired into the system.
The two categories {\quant} and {\sentconn}
consist entirely of logical constants and are thus fixed in advance for each
language.

\gap
\begin{center}
\fbox{
\parbox{15cm}{
{\quant} ::= {\bf forall} \I {\bf exists}


{\sentconn} ::= {\bf not} \I
                {\bf and} \I {\bf or} \I {\bf imp} \I {\bf iff}
}}
\end{center}

The logical connectives are all operator symbols
of arity 2 and declared as infix with the exception of {\bf not}
which has arity 1 and is prefix. The binding priorities of these symbols are
given in the table below.

\begin{center}
\begin{tabular}{|l|c|c|ccc|} \hline
\multicolumn{1}{|c}{Symbol} &
\multicolumn{1}{c}{Arity} &
\multicolumn{1}{c}{Type} &
\multicolumn{1}{c}{prbp} & lbp & rbp \\ \hline
{\bf not} & 1 & prefix & 1000 & n/a & n/a \\ \hline
{\bf and} & 2 & infix & n/a  & 750 & 755 \\ \hline
{\bf or} & 2 & infix & n/a  & 740 & 745 \\ \hline
{\bf imp} & 2 & infix & n/a  & 730 & 735 \\ \hline
{\bf iff} & 2 & infix & n/a  & 720 & 725 \\ \hline
\end{tabular}
\end{center}

The two truth values {\tt TRUE} and {\tt FALSE} and the equality
predicate, {\tt =}, are also logical symbols, {\tt TRUE} and
{\tt FALSE} belong to the category {\sentconst} and {\tt =} belongs to the
category {\predconst}; however, the categories {\sentconst} and
{\predconst} are not fixed
(because other primitive symbols in these categories
can be user declared).

The equality predicate, {\tt =}, has arity 2 and is declared as infix.



