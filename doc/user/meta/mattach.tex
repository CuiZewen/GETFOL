\gfcommand{mattach}{semantic meta attachment}
\index{mattach}

\gfsyntax{
   mattach \ARG{indconst} to \ALT dar \OPT{[rep]}
   \ARG{cname}:\ARG{pname}:\ARG{sort}:\ARG{object};
}

\gfdescription{
   Defines an attachment for a constant of the context {\em meta}.

   \ARG{rep}, if present, can be a representation function or \verb+*+.
   If it is \verb+*+ or no representation is specified, then the default representation function 
   {\tt UNIVERSALREP} is taken.
   \ARG{indconst} is a symbol declared to be an INDCONST in {\em meta}; \ARG{cname} is the name
   of the context to which \ARG{object} belongs; \ARG{pname} is the name of the proof in which
   \ARG{object} is present; \ARG{sort} is a sort of the meta-context associated to one of the
   syntactic categories reported with the {\tt reflect} command; \ARG{object} is an object of
   type \ARG{sort}.

   This command implements the mechanism of ``naming'' symbols or objects belonging to the 
   context \ARG{cname}, {\em ie.} the creation of names denoting objects of \ARG{cname}.
   \ARG{indconst} can be attached ``one way'' (using {\tt to}) or ``two ways'' (using {\tt dar}).
   The ``one way'' attachment tells the semantic interpretation mechanism that \ARG{indconst} is
   the name in {\meta} of the {\GF} object in the context \ARG{cname} corresponding to
   \ARG{object}.
   The two ways attachment tells the semantic interpretation mechanism that whenever the 
   (data structure representing) the \ARG{object} is computed as the representation of a 
   term {\em t}, then \ARG{indconst} should be returned as the simplified version of {\em t}.
}

\gfrecap{
This command implements the mechanism of ``naming'' symbols or objects belonging to the 
context `cname', ie. the creation of names denoting objects of `cname'.
`indconst' can be attached ``one way'' (using `to') or ``two ways'' (using `dar').
The ``one way'' attachment tells the semantic interpretation mechanism that `indconst' is
the name in meta of the GETFOL object in the context `cname' corresponding to
`object'.
The two ways attachment tells the semantic interpretation mechanism that whenever the 
(data structure representing) the `object' is computed as the representation of a 
term `t', then `indconst' should be returned as the simplified version of `t'.
}

\gfexample+
   ***** namecontext META;
   ***** nameproof P1;

   ***** declare indconst sc [SENTCONST];
   ***** declare indconst ic [INDCONST];
   ***** declare indconst vl [FACT];
   ***** declare indconst f1 [FACT];

   ***** DECREP  SENTCONST INDCONST FACT;

   ***** represent { SENTCONST } as SENTCONST;
   ***** represent { INDCONST } as INDCONST;
   ***** represent { FACT } as FACT;

   ***** makecontext C;
   ***** switchcontext C;
   ***** declare indconst c;
   ***** declare sentconst A;
   ***** nameproof P1;
   You have named the current proof: P1

   ***** assume c=c;
   1   c = c     (1)

   ***** makeproof P2;
   You have created the empty proof: P2

   ***** switchproof P2;
   You are now using the proof: P2

   ***** assume A imp A;
   1   A imp A     (1)

   ***** label fact ax = 1;

   ***** switchcontext META;

   ***** MATTACH sc TO  C::SENTCONST:A;
   ctext-get: I changed context to: C
   ctext-get: I changed context to: META
   sc attached to 'A

   ***** MATTACH ic DAR C:P2:INDCONST:c;
   ctext-get: I changed context to: C
   ctext-get: I changed context to: META
   ic attached to 'c
   ic is the preferred name of c

   ***** MATTACH vl DAR [SENTCONST] C:P1:FACT:1;
   ctext-get: I changed context to: C
   proof-get: I changed proof to: P1
   proof-get: I changed proof to: P2
   ctext-get: I changed context to: META
   vl attached to '(1 (= c c) (1) ASSUME (%WFF% = c c))
   vl is the preferred name of (1 (= c c) (1) ASSUME (%WFF% = c c))

   ***** MATTACH f1 TO  C:P2:FACT:1;
   ctext-get: I changed context to: C
   ctext-get: I changed context to: META
   f1 attached to '(1 (imp A A) (1) ASSUME (%WFF% imp A A))

   ***** MATTACH f1 DAR C:P2:FACT:ax;
   ctext-get: I changed context to: C
   ctext-get: I changed context to: META
   f1 attached to '(1 (imp A A) (1) ASSUME (%WFF% imp A A))
   f1 is the preferred name of (1 (imp A A) (1) ASSUME (%WFF% imp A A))

+

\gfnotes{}

