\gfcommand{cut}{cut rule}
\index{cut}

\gfsyntax{
	cut \ARG{fact1} \ARG{fact2};\\
	cut \ARG{fact1} \ARG{fact2} \OPT{keep \ARG{assumption1} \SEQ
	\ARG{assumptionN}};
}

\gfdescription{
	This command generates a new proof line obtained from \ARG{fact2} 
	by eliminating dependencies of all facts whose wff is equal to \ARG{fact1}'s
	wff, and then adding the dependencies of \ARG{fact1}.
	In the second form, every \ARG{assumptionI} in the list of dependencies
	is kept. 
	Every \ARG{assumptionI} must occur in the list of dependencies of
	\ARG{fact2} . 
}

\gfrecap{
This command generates a new proof line obtained from `fact2'
by eliminating dependencies of all facts whose wff is equal to `fact1''s
wff, and then adding the dependencies of `fact1'.
In the second form, every `assumptionI' in the list of dependencies is kept. 
Every `assumptionI' must occur in the list of dependencies of `fact2'. 
}


\gfexample+
   ***** declare sentconst A B C;
   ***** axiom AAA : A;
   AAA : A
   ***** assume A A A A B C;
   1   A     (1)
   2   A     (2)
   3   A     (3)
   4   A     (4)
   5   B     (5)
   6   C     (6)
   ***** wk 5 by 1 2 3 4;
   7   B     (1 2 3 4 5)
   ***** wk AAA by 6;
   8   A     (6)
   ***** cut 8 7;
   9   B     (5 6)
   ***** wk 7 by 6;
   10   B     (1 2 3 4 5 6)
   ***** cut AAA 10;
   11   B     (5 6)
   ***** cut 8 7 keep 3 2;
   12   B     (2 3 5 6)
+
