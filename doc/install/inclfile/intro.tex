\section{Introduction}

The distribution package allows the installation of the following three
applications:
%
\begin{itemize}
	\item
		the {\tt HGKM} interpreter \cite{giunchiglia35}, that is a LISP-like
		language acting as the implementation language of both {\tt FOL}
		and {\tt GETFOL}.
	\item
		the {\tt GETFOL} system \cite{giunchiglia12,giunchiglia29}.
	\item
		the {\tt AGETFOL} system ({\tt GETFOL} with {\em abstraction}
		\cite{giunchiglia7}.
\end{itemize}

From now on, we will write ``{\tt the system}" to mean {\tt HGKM},
{\tt GETFOL} and {\tt AGETFOL}.
The distribution package is provided with a set of general purpose
configuration and installation facilities which allow you to install your
favourite application on your machine.

In order to install {\tt the system} read section~\ref{requirements}
(``{\it Minimal System Requirements}'') and verify that your machine
has all the required features.
Then read section~\ref{instproc} (``{\it The installation procedure}'').
This section gives the instructions to follow in order to install
{\tt the system} on your machine.

If you are an expert you might be interested also in the features
of the installation and configuration facilities.
Section~\ref{sysmod} (``{\it Systems and Modules}'') introduces
to the data structures ({\it systems} and {\it modules}) devised
to store and/or modify the configuration of the applications.
