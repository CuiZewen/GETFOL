\section{Semantic simplification}
\label{sec-comp}

The subsections \ref{sec-ss-intro}, \ref{sec-ss-model} and \ref{sec-ss-repr} 
of this section have been taken from \cite{rww3}.


\subsection{Introduction}
\label{sec-ss-intro}

{\GF} is intended to express a variety of methods of human reasoning.
Though the word "reasoning" usually connotes a logical deductive process of 
using facts and assertions to obtain conclusions, much of human intelligence 
relies more upon observation than upon deduction.
We look at a book. The book is seen to be "green", as an immediate observation,
not as a deduction involving, say, analysis of wavelengths of light and 
sensory receptors  in the eye. Similarly, humans cross streets without 
conscious analysis of the traffic flow, add numbers without resorting to basic
set theory, and play chess without considering each move in terms of the 
geometry of the board. 

Any system which hopes to express a variety of reasoning processes, therefore 
needs a method of doing purely computational tasks.
In {\GF}, the {\bf semantic interpretation mechanism}, which provides this 
ability, consists of two parts:
\begin{itemize}
\item {\GF}'s {\bf semantic attachment mechanism} permits the user to define a
      ``correspondence'' between the various constants (function symbols,
      predicate constants, individual constants) of the language and
      corresponding objects of the programming language {\HG}.
\item facts about the {\HG} structure can be used directly in the proof
      via the {\bf semantic simplification mechanism}, eliminating the 
      necessity of a possibly complicated deduction.
\end{itemize}
For example, obvious attachments to the function symbol $+$ and to the 
individual constants $17$, $34$, $51$ would allow to conclude $17+34=51$ in 
one step, instead of computing $34$ successors of $17$.
In order to explain this more clearly we first give an informal account of the 
technical details.

\subsection{``Intended'' and ``computational'' models}
\label{sec-ss-model}

The declarations made by a {\GF} user specify a first order language 
$L=\langle P,F,C\rangle$, where $P$ is the list of {\predconst}s, $F$ the list 
of {\funconst}s, and $C$ the list of {\indconst}s (see section \ref{sec-decl}).

A model for such a language is a structure $M=\langle D,P',F',C'\rangle$ where
$D$ is a set and $P'$,$F'$ and $C'$ are lists of predicates over $D$, functions 
over $D$, and individuals of $D$ such that the arities of the symbols in $P$ and 
$F$ match the arities of the predicates and functions at the correspondent 
positions in $P'$ and $F'$.
The idea here is that the language $L$ is used for making statements about 
structures such as $M$ (what we call {\bf ``intended'' or ``standard'' model}). 
In particular, when the user writes down a theory in {\GF}, he generally has 
in mind some particular model for his language, and the axioms of his theory 
are intended to express the properties of this particular model.

The fact that {\GF} is really a {\HG} program running in a LISP 
environment, inspires the following idea: some parts of a model for a {\GF} 
language can often be expressed computationally in the sense that the elements 
of $D$ can be represented by s-expressions, and the predicates and functions 
on $D$ can be represented by {\HG} functions and predicates.
It should then be possible to use the computational representation to aid 
{\GF} deductions concerning the model.
For example, suppose the theory we are interested in, is first order number 
theory, and the model that we have in mind is the set of natural numbers 
together with the operations of successor, addition and multiplication.
The numerals have natural representations as {\HG} numbers, and the 
functions in question have {\tt PLUS1}, {\tt PLUS}, {\tt TIMES} as their {\HG}
counterparts.
As mentioned above it should then be possible to use the computational 
representation to provide swift deductions of such statements as $25+37=52$.

The semantic attachment mechanism in {\GF} allows the user to set up these 
computational representations of his subject matter, and the semantic
interpretation mechanism allows to use these representations to aid deduction 
in {\GF}.

With the above overview in mind, let us proceed to the details.

Given a language $L=\langle P,F,C\rangle$ and a model 
$M=\langle D,P',F',C'\rangle$, we define an interpretation function $I$.
For each {\term} $t$ of $L$ in which no free variable occurs, $I(t)$ is the 
individual in $D$ which $t$ denotes.
In particular we define the interpretation of an {\indconst} $c$ to be the 
individual $c'$ in $D$, and where $f$ is a {\funconst}, and the interpretation 
of {\term} $t_1,\ldots,t_n$ are defined, we inductively define the 
interpretation of the {\term} 
$f(t_1,\ldots,t_n)$ to be $f'(I(t_1),\ldots,I(t_n))$.
We may extend the interpretation function to formulas (again without free 
variables) over $L$ by defining $I(w)$ to be the object {\tt TRUE} exactly 
when the formula $w$ is true of the model (for a technical definition 
see \cite{kleene2}).

When $f'$ is the function in a model corresponding to the {\funconst} $f$ in 
$L$, we will also say that $f'$ is the interpretation of $f$, and similarly 
for predconsts.

Now we define a {\bf computational model} to be an object
$K=\langle D',P'',F'',C''\rangle$, where it is understood that $D'$ is a set
of s-expressions, and $P''$, $F''$ and $C''$ are lists of {\HG} predicates,
functions and s-expressions respectively, with the appropriate restrictions on 
arities.

From the extensional point of view, a computational model is for a language
just like a set-theoretic model for a language, except that we do not require
that the functions and predicates concerned be total; that is functions and
predicates may be undefined (non-terminating) for some elements
of $D'$.

We define an {\bf attachment map} $att$ from terms and formulas of $L$ into
$K$ in a manner exactly analogous to the definition of $I$ given above.

We have one last map to worry about, the map {\bf $rep$} which gives, for each
object in the domain $D'$ of the computational model $K$, the object it
represents in the domain $D$ of the model $M$.

Now we may define precisely the meaning of attachments made in the {\GF}
system: the attachment of an {\indconst} $c$ to an s-expression $c''$
signifies that $c$ and $c''$ represent the same object in the model, that is
to say, $I(c)=rep(c'')$.
Similarly, the attachment of a {\funconst} $f$ to a {\HG} function $f''$
signifies that the result of applying $f''$ to an s-expression $c''$ which
represents an individual $c'$ in the model, is a s-expression which represents
the individual $f'(c')$ in the model.
The analogous statements hold for attachments to {\predconst}s.

The above conditions are equivalent to the statement that the 
diagram in figure \ref{fig-ss} commutes.

\begin{figure}[htb]
\begin{center}
\setlength{\unitlength}{0.0125in}%
\begin{picture}(288,260)(92,540)
\thicklines
\put(340,580){\oval(80,80)}
\put(160,580){\oval(80,80)}
\put(160,760){\oval(80,80)}
\put(200,580){\vector( 1, 0){100}}
\put(194,726){\vector( 1,-1){112}}
\put(160,720){\vector( 0,-1){100}}
\put(340,559){\makebox(0,0)[b]{\raisebox{0pt}[0pt][0pt]{\twlrm model}}}
\put(340,577){\makebox(0,0)[b]{\raisebox{0pt}[0pt][0pt]{\twlrm of intended}}}
\put(340,595){\makebox(0,0)[b]{\raisebox{0pt}[0pt][0pt]{\twlrm Domain }}}
\put(160,562){\makebox(0,0)[b]{\raisebox{0pt}[0pt][0pt]{\twlrm sexpr}}}
\put(160,580){\makebox(0,0)[b]{\raisebox{0pt}[0pt][0pt]{\twlrm {\HG}}}}
\put(160,752){\makebox(0,0)[b]{\raisebox{0pt}[0pt][0pt]{\twlrm Terms}}}
\put(160,770){\makebox(0,0)[b]{\raisebox{0pt}[0pt][0pt]{\twlrm {\GF}}}}
\put(250,560){\makebox(0,0)[b]{\raisebox{0pt}[0pt][0pt]{\twlrm Representation}}}
\put(280,660){\makebox(0,0)[b]{\raisebox{0pt}[0pt][0pt]{\twlrm I}}}
\put(120,660){\makebox(0,0)[b]{\raisebox{0pt}[0pt][0pt]{\twlrm attachment}}}
\end{picture}
\end{center}
\label{fig-ss}
\caption{intended model - computational model mappings}
\end{figure}


\subsection{Multiple representation functions}
\label{sec-ss-repr} 

The semantic attachment mechanism allows several representation of the model 
by {\HG} s-expressions to be in force at the same time.
We will seek to motivate this aspect of the semantic attachment mechanism 
by means of an example: consider a theory of chess with includes a general 
theory of lists as a subtheory (this subtheory would be applied in arguments 
about lists of pieces, lists of game positions and so on).
The intended model of such a theory includes at least two kinds of objects: 
chess positions and lists.
Lists and positions form disjoint domains in the model, though it may be 
possible to build lists of chess position.
If we are going to build a computational representation of this model, we will
need to represent positions and lists by s-expressions in such a way that no 
s-expression represents both a list and a position.
The natural representation of a chess position as an s-expression is as a 
list of eight lists, each of which is a list of eight piece names (one of 
which is "empty" or some such), and the natural representation of lists as 
s-expressions is the direct representation as {\HG} lists.
This representation scheme cannot be used, since it will not be possible to 
decide whether a given list of eight lists of eight piece names represents a 
chess board or a list of list of pieces. 
That is to say, the map $rep$ will not be well defined. 
It is of course not hard to solve this problem by the use of some slightly 
fancier coding, but a general solution to the problem of disambiguating 
computational representations is available.
Suppose that the intended model of a {\GF} theory $T$ includes the disjoint 
domains $D_1,\ldots,D_n$, and suppose further that we have a different coding 
function for each of these domains.
That is we have $n$ different {\bf representation functions} $rep_i$ which map 
the domain of s-expressions into domain of the model, with the property that 
the range of $rep_i$ is a subset of $D_i$.
Then it is possible that a single s-expression codes two different objects 
$d_i$, $d_j$ in the model, but as long as we know what coding function $rep_i$
to apply, there is no ambiguity. 


Then the definition of the $att$ map may be extended to take account of the 
possibility of multiple representations in the following way: the domain of 
the $att$ map will still consist of the set of {\GF} terms and formulas, but
its range will now lie in the set of pairs of the form $\langle$ representation
function, s-expression $\rangle$.

The soundness condition for the $att$ map is now that, when 
$att(t)=\langle rep, c'' \rangle$, we have $rep(c'')=I(t)$.
In order to specify this new more complicated $att$ map, the user of the {\GF}
system must give representation information concerning his attachments.

Specifically, each representation function must be given a name and when the 
attachment to an {\indconst} is given, the name of the associate 
representation function must be given as well.
Similarly, when the attachment $f''$ to a {\funconst} $f$ is specified, the 
(names of the) representations of its arguments and of the value it returns 
must be given, and when the attachment to a {\predconst} is specified, the 
representations of its arguments must also be specified.

The significance of specifying that the representations of the arguments and 
value of the attachment $f''$ to a {\funconst} $f$ are 
$R_1,\ldots,R_n$ and $R_{n+1}$ respectively, is that 
$R_{n+1}(f''(c''_1,\ldots,c''_n))=f'(R_1(c''_1),\ldots,R_n(c''_n))$ 
where $f'$ is the interpretation of $f$, whenever $c''_1$,..,$c''_n$ are 
s-expressions in the domains of $R_1$,..,$R_n$.
The same holds for attachments to {\predconst}, mutatis mutandis.
Given the attachments with representation information for individual symbols,
the map $att$ on the domain of terms and formulas is defined inductively in 
the obvious way: if $f$ is attached to $f''$ and the declared representation 
of the arguments of $f''$ are $R_1,\ldots,R_n$ and terms $t_1$,..,$t_n$ have 
attachments with representations $R_1$,..,$R_n$ then 
$att(f(t_1,\ldots,t_n))=f''(att(t_1),\ldots,att(t_n))$.
Under this definition the diagram above commutes for each individual 
representation function.

Note that if the representation of the attachment of any term $t_i$ does not 
match that of its place in the argument list, then 
$f''(att(t_1),\ldots,att(t_n))$ cannot be expected to represent the 
interpretation of $f(t_1,\ldots,t_n)$.
The reason for this is that the correctness of a computation which purports to
represent a mathematical function depends on the representation of the 
arguments of the function as data objects.
For example, no one would expect a floating point multiplication algorithm to 
behave correctly if its arguments were encoded as integers rather than 
floating point numbers.

Finally, note that the attachment map, as well as the s-expressions which 
represent functions, may be partial.
The user is never required to provide an attachment for any {\GF} symbol, nor is
any attachment to a {\funconst} or {\predconst} required to be complete.

The semantic simplification mechanism will use whatever information is 
available and if there will be insufficient information, it will return this 
fact to the user. 


