\gfcommand{extension}{sort extension declaration}
\index{extension}

\gfsyntax{
  extension \ARG{sort} by \ARG{extexpr};
}

\gfdescription{
  The extension of a sort $S$ is the set of all and only the {\tt indconst}s
  of sort $S$. To say that $S$ has extension $\{c_1, \ldots ,c_n\}$,
  where $c_1, \ldots ,c_n$ are {\tt indconst}s, is the same as saying: $\forall
  x.S(x) \liff (x=c_1  \dis\ldots \dis x=c_n$), where $x$ is an {\tt indvar}
  of sort {\tt UNIVERSAL}. 
  The command declares the extension of \ARG{sort} to be \ARG{extexpr}.
  An extension expression \ARG{extexpr} is defined by the syntax below, where
  {\bf uni} stands for set union, {\bf dif} stands for set difference and {\bf
  int} stands for set intersection.
  The binding power is in the following increasing order: {\bf uni}, {\bf
  dif}, {\bf int}.
  %
  \begin{bnf}
    $extexpr$  \sep  $sort$ | {\bf \{}$indsym_1,\ldots,indsym_n${\bf \}}|\\ 
               &     $extexpr$ {\bf uni} $extexpr$      |\\
               &     $extexpr$ {\bf dif} $extexpr$      |\\  
               &     $extexpr$ {\bf int} $extexpr$ 
  \end{bnf}
  %
  A sort in \ARG{extexpr} without extension is treated as if it had an empty
  extension. 
}

\gfrecap{
  Declares the extension of a sort S (that is the set of all and only the
  indconst of sort S).
}

\gfexample+
   ***** declare sort S T R;
   S has been declared to be a sort
   T has been declared to be a sort
   R has been declared to be a sort

   ***** declare indconst a b c;
   UNIVERSAL is a sort
   a has been declared to be an Indconst
   b has been declared to be an Indconst
   c has been declared to be an Indconst

   ***** extension S by {a b c};
   Now the extension of S is fixed to be : (a b c)

   ***** extension T by S int {a c};
   Now the extension of T is fixed to be : (a c)

   ***** extension R by T uni {b};
   Now the extension of R is fixed to be : (a c b)

   ***** extension R by {a b c} dif S;
   You can't set an empty extension for R
+

\gfnotes{
  The command overwrites the previous extension of a sort without 
  warning.
  The syntactic and semantic simplifiers use the extension
  information explicitly if there is any (see the command {\tt simplify} 
  in section \ref{sec-simplify}
  and the command {\tt eval} in section \ref{sec-eval}).
}
