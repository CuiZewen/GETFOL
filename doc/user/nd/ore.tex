\gfcommand{ore}{or elimination rule}.
\index{ore}\index{oe}

\gfsyntax{
   ore \ALT oe \ARG{fact1} \OPT{,} \ARG{fact2} \OPT{,} \ARG{fact3};
}

\gfdescription{
   \renewcommand{\arraystretch}{0.5}
   \[
   \begin{array}{l}
      \\ \\ 
      \dis E  
   \end{array}
   \ \ 
   \begin{array}{ccc}
      &{[A]}&{[B]}\\
      &\vdots&\vdots\\
      A \dis B & C & C\\
      \hline\\
      & C &
   \end{array}
   \]
   \renewcommand{\arraystretch}{1}

   Let \ARG{wff1}, \ARG{wff2} and \ARG{wff3} be the formulas of \ARG{fact1}, 
   \ARG{fact2}, \ARG{fact3} respectively; let {\em wff1} be a disjunction;
   let \ARG{wff2} and \ARG{wff3} be the same formula.
   Then the conclusion is \ARG{wff2} (\ARG{wff3}) with the dependencies of
   \ARG{fact1} along with those of \ARG{fact2} whose wff is not equal to the
   left disjunct of \ARG{wff1} and those of \ARG{fact3} whose wff is not equal
   to the right disjunct of \ARG{wff1}. 
}

\gfrecap{
Let `wff1', `wff2' and `wff3' be the formulas of `fact1', 
`fact2', `fact3' respectively; let {\em wff1} be a disjunction;
let `wff2' and `wff3' be the same formula.
Then the conclusion is `wff2' (`wff3') with the dependencies of
`fact1' along with those of `fact2' whose wff is not equal to the
left disjunct of `wff1' and those of `fact3' whose wff is not equal
to the right disjunct of `wff1'. 
}

\gfexample+
   ***** declare sentconst A B C;
   [...]

   ***** assume B imp A;
   1  B imp A (1)

   ***** assume C imp A;
   2  C imp A (2);

   ***** assume B;
   3  B  (3);

   ***** assume C;
   4  C  (4);

   ***** impe 3 1;
   5  A  (1 3)

   ***** impe 4 2;
   6  A  (2 4)

   ***** assume B or C;
   7  B or C  (7)

   ***** ore 7 5 6;
   8  A   (1 2 7)

   ***** impi 7 8;
   9  (B or C) imp A   (1 2)

   ***** ore 7 6 5;
   10  A   (1 2 3 4 7)
+
