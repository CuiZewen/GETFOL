\gfcommand{impi}{implication introduction rule}
\index{impi}\index{ded}

\gfsyntax{
   impi \ALT ded  \ARG{fact1} \OPT{, \ALT imp} \ARG{fact};\\
   impi \ALT ded  \ARG{wff} \OPT{, \ALT imp} \ARG{fact};
}


\gfdescription{
   \renewcommand{\arraystretch}{0.5}
   \[
   \begin{array}{l}
      \\ \\ \\
      \imp I   
   \end{array}
   \ \  
   %
   \begin{array}{c}
      {[A]}\\
      \vdots\\
      B\\
      \hline\\
      A \imp B
   \end{array}
   \]
   \renewcommand{\arraystretch}{1}

   The wff derived is the implication of the wffs of \ARG{fact1} and
   \ARG{fact2}. It depends on all the dependencies of \ARG{fact2} less 
   {\em all the lines} whose wff is the same as \ARG{fact1}'s wff (in the 
   first form) or \ARG{wff} (in the second form).
}

\gfrecap{
The wff derived is the implication of the wffs of `fact1' and `fact2'.
It depends on all the dependencies of `fact2' less all the lines whose wff is
the same as `fact1''s wff (in the first form) or `wff' (in the second form).
}
   
\gfexample+
   ***** declare sentconst A;
   ***** assume A;
   1  A  (1)

   ***** impi 1 1;
   2  A imp A  

   ***** impi A 1;
   3  A imp A 
+
