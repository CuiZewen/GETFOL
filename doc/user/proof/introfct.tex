\section{Facts: reasoning building blocks}
\label{sec-reas}

Reasoning in {\GF} consists of a sequence of reasoning steps, each step being
performed on {\bf facts}.
A fact is a data structure containing a {\em hook} identifying it and some extra
information characterizing the the fact itself.
A fact can be a {\bf proof line}, an {\bf axiom}, a {\bf theory} or a {\bf
definition}.
These four categories are disjoint.
The hooks of proof lines, axioms, theories and definitions are called labels
({\em label}s), axiom labels ({\em axlabel}s), theory labels ({\em thlabel}s)
and definition labels ({\em deflabel}s) respectively.
We will simply call them {\em label}s whenever the distinction is not relevant.


\subsection{Proof lines: reasoning steps}
\label{sec-proofline}

Proof lines are the basic reasoning steps in {\GF}.
A proof line is a 4-tuple of the form: 

\begin{center}
$<$ {\em label} \ , \ {\em wff} \ , \ {\em reason} \ , \ {\em deplist} $>$
\end{center}

where
{\em label} is a ``hook" identifying the proof line itself; {\em wff} is the
well formed formula of the proof line; {\em reason} records how the proof line
has been asserted; {\em dep} records the assumptions the proof lines depends on.
For instance, the proof line 
%
\begin{center}
	{\tt < 3 , B , impe 1 2 , (1 2) >}
\end{center}
%
tells us that: the proof line {\tt 3} whose formula is ``{\tt B}'' has been
deduced by modus ponens ({\tt impe}) applied to the previously deduced proof
lines whose labels are {\tt 1} and {\tt 2} and that depends on the proof lines
{\tt 1} and {\tt 2}.
Analogously, the proof line 
%
\begin{center}
	{\tt < 4 , s(0) > 0 , alle smonotonic x 0 ,  >}
\end{center}
%
tells us that the proof line {\tt 4} whose formula is ``{\tt s(0) > 0}'' has
been deduced by instantiating the variable ``{\tt x}'' in the {\em fact} whose
{\em hook} is {\tt smonotonic} (probably an axiom asserting that
$\forall x \ (\ s(x) > x\ )$).

A {\GF} {\bf proof} is a sequence of proof lines.
The proof line labels are generated by {\GF} as increasing natural numbers
starting from $1$.
The user can ``name'' proof line labels with desired {\GF} symbols (see the
command {\tt label} in section \ref{sec-adm}) or to refer to the previous {\em
n-th} proof line in the current proof by typing \verb+^n+.


\subsection{Axioms and theories}

A {\GF} axiom is the mechanization of the notion of axiom in logic, a {\GF}
theory is the mechanization of the notion of theory, defined as a set of axioms,
in logic.
 
In {\GF} you can define the axioms of your own theory.
No check is performed on their consistency.
You are thus free to define an unsound theory.
However, the {\GF} logic is complete without need of any axioms.  
An axiom can be used to generate proof lines. 
A {\GF} axiom is a 2-tuple of the form:

\begin{center}
	$<$ {\em axlabel} \ , \ {\em wff} $>$\\
\end{center}

where {\em axlabel} is the ``hook'' and {\em wff} is the well formed formula of
the axiom. 

A theory is a set of axioms.
A theory allows you to use all the axioms of the theory as a whole. 
