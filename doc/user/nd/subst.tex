\gfcommand{subst}{equality substitution}
\index{subst}

\gfsyntax{
  subst \ARG{fact1} \OPT{with} \ARG{fact2};\\
  subst \ARG{fact1} \OPT{with} \ARG{fact2} \OPT{right \ALT left};\\
  subst \ARG{fact1} \OPT{with} \ARG{fact2} 
                    \OPT{occ \ARG{n1} \ARG{n2} \SEQ} \OPT{right \ALT left};
}

\gfdescription{
  \[
  \begin{array}{cc}
    sub_l \ \ 
    \fraz{A(t_1) \ \ \ \ t_1 = t_2}
    {A(t_2)}
    \ \ \ 
    &
    \ \ \ 
    sub_r \ \ 
    \fraz{A(t_1) \ \ \ \ t_2 = t_1}
    {A(t_2)}
  \end{array}
  \]

  The rule substitutes a term in \ARG{fact1}'s wff with another one proved to 
  be equal to the former.
  \ARG{fact2} must be the equality {\tt t1 = t2} and \ARG{fact1} may contain 
  one or more occurrences of {\tt t1}.
  The conclusion is the result of the substitution of {\tt t1} with {\tt t2} in
  \ARG{fact1}'s wff. 
  Dependencies of derived fact are the union of those of \ARG{fact1} and
  \ARG{fact2}.
  If \ARG{fact1} does not contain {\tt t1}, then the conclusion has the same 
  wff and dependencies as \ARG{fact1}.

  The default is the substitution of {\tt t2} with {\tt t1}, corresponding to 
  the option {\bf left}. If {\bf right} is indicated, then {\tt t1} is 
  substituted with {\tt t2}.

  Individual occurrences can be substituted by specifying the optional
  \OPT{{\tt occ} \ARG{n1} \ARG{n2} \SEQ}, where \ARG{n1}, \ARG{n2}, \SEQ are
  the occurrences to be substituted.
  Without this option, all occurrences are substituted. 
}

\gfrecap{
The rule substitutes a term in the formula of `fact1' with another one
proved to be equal to the former.
`fact2' must be the equality `t1 = t2' and `fact1' may contain one or more
occurrences of `t1'.
The conclusion is the result of the substitution of `t1' with `t2' in the
formula of `fact1'. 
Dependencies of derived fact are the union of those of `fact1' and `fact2'.
If `fact1' does not contain `t1', then the conclusion has the same wff and
dependencies as `fact1'.
The default is the substitution of `t2' with `t1', corresponding to the option
`left'.
If `right' is indicated, then `t1' is substituted with `t2'.
Individual occurrences can be substituted by specifying the optional
`[occ n1 n2 ...]', where `n1', `n2', ... are the occurrences to be substituted.
Without this option, all occurrences are substituted. 
}

\gfexample+
   ***** declare predconst P Q 2;
   ***** declare funconst f 1; 
   ***** declare indvar x; declare indvar y;
   [...]

   ***** assume P(x y) imp Q(y x);
   1   P(x,y) imp Q(y,x)     (1)

   ***** assume x = f(x);
   2   x = f(x)     (2)

   ***** subst 1 2;
   3   P(f(x),y) imp Q(y,f(x))     (1 2)

   ***** subst 3 2 right;
   4   P(x,y) imp Q(y,x)     (1 2)

   ***** subst 1 2 occ 1;
   5   P(f(x),y) imp Q(y,x)     (1 2)
+


