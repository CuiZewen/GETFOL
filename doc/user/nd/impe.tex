\gfcommand{impe}{implication elimination rule}
\index{impe}\index{mp}

\gfsyntax{
   impe \ALT mp \ARG{fact1} \OPT{,} \ARG{fact2};
}

\gfdescription{
   \renewcommand{\arraystretch}{0.5}
   \[
   \begin{array}{l}
      \imp E  
   \end{array}
   \ \  
   \begin{array}{c}
      \fraz{A \ \ A \imp B}
           {B}
   \end{array}
   \]
   \renewcommand{\arraystretch}{1}

   One of the two facts must be an implication and the other one
   must be its hypothesis.
   The order of the arguments is not relevant.
   The command creates a proof line whose wff is the conclusion of the
   implication and whose dependencies list is the union of the dependencies of
   \ARG{fact1} and \ARG{fact2}.
}

\gfrecap{
One of the two facts must be an implication and the other one must be its
hypothesis.
The order of the arguments is not relevant.
The command creates a proof line whose wff is the conclusion of the implication
and whose dependencies list is the union of the dependencies of `fact1' and
`fact2'.
}


\gfexample+
   ***** declare sentconst A B;
   ***** assume A A imp B;
   1  A  (1)
   2  A imp B  (2)

   ***** impe 1 2;
   3  B  (1 2)

   ***** impe 2 1;
   4  B  (1 2)
+
