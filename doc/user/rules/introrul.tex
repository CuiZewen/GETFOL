\section{Other rules}

\subsection{Conditional rules}
\label{sec-cond}

Conditional rules allow us to introduce and eliminate the conditional wffs
{\wffif} and conditional terms {\termif}.

\begin{bnf}
	{\wffif}  \sep {\bf wffif} {\wff}$_1$ {\bf then} {\wff}$_2$ {\bf else}
				   {\wff}$_3$ \\
	{\termif} \sep {\bf trmif} {\wff} {\bf then} {\term}$_1$ {\bf else}
				   {\term}$_2$
\end{bnf}

Note that the {\termif} construct is not first order.
However, it can easily be shown that {\termif} can be defined as a conservative
extension of first order logic using an induction on the length of deductions.

\renewcommand{\arraystretch}{0.5}
\[
\begin{array}{|c|c||c|c|} \hline 
\multicolumn{2}{|c||}{\mbox{\bf Introduction rules}} &
\multicolumn{2}{c|}{\mbox{\bf Elimination rules}} \\ \hline
\begin{array}{l}
\\ \\ \\ \\ \\
\mbox{{\em wif}} \  I  
\end{array}
&
\begin{array}{cc}
\\ \\
\begin{array}{c}
{[A]}\\
\vdots\\
B
\end{array}
\ \ \ 
\begin{array}{c}
{[\neg A]}\\
\vdots\\
C
\end{array}
\\
\hline\\
\begin{array}{c}
\mbox{{\em wffif}} \ A \ \mbox{{\em then}} \ B \ \mbox{{\em else}} \ C 
\end{array}
\end{array}
&
\begin{array}{l}
\\ \\ \\ \\ \\
\mbox{{\em wif}} \  E
\end{array}
&
\begin{array}{cc}
\\ \\
\begin{array}{c}
\\ \\ \\ \\
A
\end{array}
\ \ \ 
\begin{array}{c}
\\ \\ \\ \\
\mbox{{\em wffif}} \ A \ \mbox{{\em then}} \ B \ \mbox{{\em else}} \ C 
\end{array}
\\
\hline\\
\begin{array}{c}
B
\end{array}
\end{array}
%\fraz{\Gamma \vdash A \ \ \Delta \vdash \mbox{{\em wffif}} \ A \ \mbox{{\em then}} \ B \ 
%\mbox{{\em else}} \ C}
%     {\Gamma,\Delta \vdash B}
\\ %\hline 
& &
\begin{array}{l}
\\ \\ \\ \\ \\
\mbox{{\em wif}} \  E_{\neg}
\end{array}
&
\begin{array}{cc}
\\ \\
\begin{array}{c}
\\ \\ \\ \\
\neg A
\end{array}
\ \ \ 
\begin{array}{c}
\\ \\ \\ \\
\mbox{{\em wffif}} \ A \ \mbox{{\em then}} \ B \ \mbox{{\em else}} \ C 
\end{array}
\\
\hline\\
\begin{array}{c}
C
\end{array}
\end{array}
\\ %\hline
\begin{array}{l}
\\ \\ \\ \\ \\
\mbox{{\em tif}} \  I    
\end{array}
&
\begin{array}{cc}
\\ \\
\begin{array}{c}
{[A]}\\
\vdots\\
B(t_1)
\end{array}
\ \ \ 
\begin{array}{c}
{[\neg A]}\\
\vdots\\
B(t_2)
\end{array}
\\
\hline\\
\begin{array}{c}
B(\mbox{{\em termif}} \ A \ \mbox{{\em then}} \ t_1 \ \mbox{{\em else}} \ t_2)
\end{array}
\end{array}
&
\begin{array}{l}
\\ \\ \\ \\ \\
\mbox{{\em tif}} \  E
\end{array}
&
\begin{array}{cc}
\\ \\
\begin{array}{c}
\\ \\ \\ \\
A
\end{array}
\ \ \ 
\begin{array}{c}
\\ \\ \\ \\
B(\mbox{{\em termif}} \ A \ \mbox{{\em then}} \ t_1 \ \mbox{{\em else}} \ t_2)
\end{array}
\\
\hline\\
\begin{array}{c}
B(t_1)
\end{array}
\end{array}
\\ %\hline 
& &
\begin{array}{l}
\\ \\ \\ \\ \\
\mbox{{\em tif}} \  E_{\neg}
\end{array}
&
\begin{array}{cc}
\\ \\
\begin{array}{c}
\\ \\ \\ \\
\neg A
\end{array}
\begin{array}{c}
\\ \\ \\ \\
B(\mbox{{\em termif}} \ A \ \mbox{{\em then}} \ t_1 \ \mbox{{\em else}} \ t_2)
\end{array}
\\
\hline\\
\begin{array}{c}
B(t_2)
\end{array}
\end{array}
\\ 
& & & \\ \hline 
\end{array}
\]
\renewcommand{\arraystretch}{1}


\subsection{Structural rules}

Structural rules are useful when performing theorem proving.