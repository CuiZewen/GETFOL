\gfcommand{existe}{existential quantification elimination rule}
\index{existe}\index{es}

\gfsyntax{
   existe \ALT es \ARG{fact}  \OPT{,} \ARG{indvar1} \ALT \ARG{indpar1}
                  \OPT{,} \ARG{indvar2} \ALT \ARG{indpar2} \SEQ;
}


\gfdescription{
   \renewcommand{\arraystretch}{0.5}
   \[
   \begin{array}{l}
      \\ \\ \\ \\
      \exists E 
   \end{array}
   \ \ 
   \begin{array}{cc}
      \\ \\
      &{[A^{x}_{a}]}\\
      &\vdots\\
      \exists x A & B\\
      \hline\\
      B
   \end{array}
   \]
   \renewcommand{\arraystretch}{1}

   The implementation of this rule is the most radically different
   from the formal statement given in Prawitz's ND. This rule corresponds, in
   informal reasoning, to the following kind of argument:
   suppose we have shown that something exists with some particular
   property, e.g. $\exists y P(a,y)$. Then we say ``call this thing $b$''.
   This is like saying assume $P(a,b)$. Then we can reason about $b$.
   As soon as we have a sentence, however, that no longer mentions $b$,
   it does not depend on what we called ``$y$'',
   but only on the dependencies of the existential statement we started
   with. Thus we can discharge $P(a,b)$ from the dependencies
   and replace them with those of $\exists y P(a,y)$. {\GF} thus makes the 
   correct assumption for you, remembers it and automatically removes it at
   the first legitimate opportunity.

   The only difference with sorts is the following:
   the existentially quantified variable must be substituted by a variable
   or a parameter of weakly more general sort.

   In the example below, an existential elimination is done creating step 
   {\tt 6}.
   This fact actually has as {\tt reason}  (see
   subsection~\ref{sec-proofline}) that it was assumed.
   Fact {\tt 8} thus depends on {\tt 6}. When the existential generalization
   is done on the next fact, {\tt b} no longer appears and so fact {\tt 6}
   is removed from the dependencies of fact {\tt 9}. The user should
   convince himself that the {\GF} logic is equivalent to the ND definition
   given at the beginning of the section.
}

\gfrecap{
The implementation of this rule is the most radically different
from the formal statement given in Prawitz's ND.
This rule corresponds, in informal reasoning, to the following kind of 
argument: suppose we have shown that something exists with some particular
property, eg. `exists y P(a,y)'.
Then we say ``call this thing `b''': this is like saying assume `P(a,b)'.
Then we can reason about `b'.
As soon as we have a sentence, however, that no longer mentions `b',
it does not depend on what we called `y', but only on the dependencies of the
existential statement we started with.
Thus we can discharge `P(a,b)' from the dependencies and replace them with 
those of `exists y P(a,y)'.
GETFOL thus makes the correct assumption for you, remembers it and 
automatically removes it at the first legitimate opportunity.
The only difference with sorts is the following: the existentially quantified
variable must be substituted by a variable or a parameter of weakly more
general sort.
}

\gfexample+
   ***** comment ! Sorted example !
     
   ***** declare indvar x y;
   ***** declare indpar a b;
   ***** declare predconst P 2;
   [...]

   ***** assume forall x.exists y.P(x y) and forall x y.(P(x y) imp P(y x));
   1   forall x.exists y.P(x,y) and forall x y.(P(x,y) imp P(y,x))     (1)

   ***** ande 1 1;
   2   forall x.exists y.P(x,y)     (1)

   ***** ande 1 2;
   3   forall x y.(P(x,y) imp P(y,x))     (1)

   ***** alle 2 a;
   4   exists y.P(a,y)     (1)

   ***** alle 3 a b;
   5   P(a,b) imp P(b,a))     (1)

   ***** existe 4 b;
   6   P(a,b)     (6)

   ***** impe 6 5;
   7   P(b,a)     (1 6)

   ***** andi 6 7;
   8   P(a,b) and  P(b,a)     (1 6)

   ***** existi 8 b:y;
   9   exists y.(P(a,y) and P(y,a)     (1)

   ***** alli 9 a:x;
   10   forall x.exists y.(P(x,y) and P(x,x)     (1)

   ***** impi 1 10;
   11   forall x.exists y.P(x,y) and 
        forall x y.(P(x,y) imp P(y,x)) imp 
        forall x.exists y.(P(x,y) and P(x,y)
    
       
   ***** comment ! Sorted example !
    
   ***** reset;
   [...]

   ***** declare indvar x [Sx];
   ***** declare indpar p [S];
   ***** declare predconst A 1;
   ***** declare sentconst B;
   [...]

   ***** axiom ONE: exists x.A(x);
   ***** axiom TWO: A(p) imp B;
   ONE : exists x. A(x)
   TWO : A(p) imp B

   ***** existe ONE p;
   A <var> must be replaced by one with more general sort;

   ***** moregeneral S < Sx >;

   ***** existe ONE p;
   1   A(p)     (1)

   ***** impe TWO 1;
   2   B
+

\gfnotes{
   The rule is often called ``existential instantiation".
}