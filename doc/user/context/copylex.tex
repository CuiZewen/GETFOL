\gfcommand{copylex}{Language declaration through contexts}
\index{copylex}

\gfsyntax{
	copylex	\ARG{ctx-name};
}

\gfdescription{
	This command copies in the current context all the symbols and sorts
	declared in the context {\em ctx-name}.
	The command has no effects in the case there is at least a symbol in the
	current context that has the same name as a symbol in the context
	{\em ctx-name}.
}

\gfrecap{
This command copies in the current context all the symbols and sorts
declared in the context `ctx-name'.
The command has no effects in the case there is at least a symbol in the
current context that has the same name as a symbol in the context `ctx-name'.
}


\gfexample+
   ***** declare indconst a b;
   ***** declare sentconst A B;
   ***** declare sort S1 S2;
   ***** namecontext C1;
   You have named the current context: C1
   ***** makecontext C2;
   You have created the empty context: C2
   ***** switchcontext C2;
   You are now using context: C2
   You are switching to a proof with no name.
   ***** probe declare;
   ***** copylex C1;
   S1 has been declared to be a sort
   S2 has been declared to be a sort
   A has been declared to be a Sentconst
   B has been declared to be a Sentconst
   a has been declared to be an Indconst
   b has been declared to be an Indconst
   ***** copylex C1;
   COPYLEX cannot be done: A has already been declared
   ***** copylex C2;
   You cannot copy the lex of the current context
+
