\gfcommand{declare sort}{sort declaration}
\index{declare}
\index{declare!sort}

\gfsyntax{
  declare sort \ARG{sym1} \SEQ \ARG{symN};
}

\gfdescription{
  The declare sort command declares the \ARG{symI} to be sorts.
  The \ARG{symI} have to be either new symbols or previously declared unary
  predicates. 
}

\gfrecap{
  Declares the `symI's to be sorts.
}

\gfexample+
   ***** declare predconst S1 1;
   S1 has been declared to be a Predconst

   ***** declare sort S2;
   S2 has been declared to be a sort

   ***** declare sort S1 S2 S3;
   The unary predconst S1 has been declared to be a sort
   S2 is a sort
   S3 has been declared to be a sort

   ***** declare indvar x [S1];
   S1 is a sort
   x has been declared to be an Indvar

   ***** simplify exists x. S1(x);
   1   exists x. S1(x)     

   ***** simplify forall x. S1(x);
   2   forall x. S1(x)     
+

\gfnotes{
  Symbols with no declared sort have the default sort {\tt UNIVERSAL}.
  If \ARG{symI} is already a sort (as {\tt S2} is in the last
  example) no error is signaled. If $S$ is a sort, then $\exists x S(x)$
  and $\forall x S(x)$ are theorems where $x$ is an indvar of sort $S$.
}
