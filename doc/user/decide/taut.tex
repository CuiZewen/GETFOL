\gfcommand{taut}{tautological first order decider}
\index{taut}

\gfsyntax{
  taut \ARG{wff} \OPT{by \ARG{fact1} \ARG{fact2} \SEQ};
}

\gfdescription{
  The class of formulae decided by {\tt taut} is the set of first order
  formulas provable using only the introduction and elimination rules for 
  the sentential connectives plus the following rule ({\em congruence-rule}):\\
  %
  \[
      \fraz{\forall x A(x)}{\forall y A(y)}
  \]
}

\gfrecap{
The class of formulae decided by `taut' is the set of first order formulas
provable using only the introduction and elimination rules for the 
sentencial connectives plus the following rule (congruence rule)::
              +------------------------------------+
              | forall x. A(x) imp forall y. A(y)  |
              +------------------------------------+
}

\gfexample+
   ***** declare sentconst A;
   ***** declare predconst P 1;
   ***** declare indconst c;

   ***** taut (A imp (P(c) imp A));
   1   A imp (P(c) imp A)

   ***** declare indvar x y [S1];
   ***** declare indvar z [S2];

   ***** taut forall x.P(x) iff forall y.P(y);
   2   forall x.P(x) iff forall y.P(y);
   ***** taut forall x.P(x) iff forall z.P(z);
   TAUT couldn't prove that forall x. P(x) iff forall z. P(z)
   is a tautology.
+
