\gfcommand{andi}{and introduction rule}
\index{andi}\index{ai}

\gfsyntax{
   andi \ALT ai \ARG{fact1} \OPT{,} \ARG{fact2}; \\
   andi \ALT ai \ARG{fact11}
                \OPT{conj \ALT cj \ARG{fact12} conj \ALT cj \ARG{fact13} \SEQ}
                \OPT{,}
                \ARG{fact21}
                \OPT{conj \ALT cj \ARG{fact22} conj \ALT cj \ARG{fact23} \SEQ};
}

\gfdescription{
   \[
   \con I \ \
   \fraz{A \ \ B}
        {A \wedge B}
   \]

   A proof line is derived whose wff is the conjunction of the wffs
   of \ARG{fact1} and \ARG{fact2} and whose dependencies are the union
   of the dependencies of \ARG{fact1} and \ARG{fact2}.

   In the second form we have ``conjunctions of facts'' rather than facts.
   A ``conjunction of facts'' is any parenthesized conjunctive expression in
   which all conjuncts are facts (\ARG{fact11} {\tt conj} \ARG{fact12} \SEQ).
   A proof line is derived whose wff is the conjunction of each $fact_{ij}$'s
   wff and whose dependencies are the union of the dependencies each
   $fact_{ij}$ depends on.
}

\gfrecap{
Applies `and' introduction to the given arguments.
In its first from a proof line is derived whose formula is the conjuction of
the formulae of `fact1' and `fact2' and whose dependencies are the union of
the dependencies of `fact1' and `fact2'.
In the second form we have ``conjunctions of facts" rather than facts.
A "conjunction of facts" is any parenthesized conjunctive expression in
which all conjuncts are facts (`fact11 conj fact12 ...').
A proof line is derived whose wff is the conjunction of each `factIJ''s
wff and whose dependencies are the union of the dependencies each
`factIJ' depends on.
}


\gfexample+
   ***** declare sentconst A B C D E;
   ***** assume A B;
   1   A     (1)
   2   B     (2)

   ***** andi 1 2;
   3   A and B     (1 2)

   ***** assume C D E;
   4   C     (4)
   5   D     (5)
   6   E     (6)

   ***** andi 1 conj 2   3;
   7   (A and B) and (A and B)     (1 2)

   ***** andi 1 conj 2   3 conj 4;
   8   (A and B) and ((A and B) and C)     (1 2 4)

   ***** andi 1 conj 2 conj 3   4;
   9   (A and (B and (A and B))) and C     (1 2 4)
+

\gfnotes{
   A proof line derived by executing the second form can always be derived by
   a sequence of executions of the command in the first form.
}
