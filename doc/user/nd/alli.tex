\gfcommand{alli}{universal quantification introduction rule}
\index{alli}\index{ug}

\gfsyntax{
   alli \ALT ug \ARG{fact} \OPT{\OPT{,} \ARG{indvar1} \ALT \ARG{indpar1} :} 
                           \ARG{indvar11}
                           \OPT{\OPT{,} \ARG{indvar2} \ALT \ARG{indpar2} :}
                           \ARG{indvar22} \SEQ;
}

\gfdescription{
   \renewcommand{\arraystretch}{0.5}
   \[
   \begin{array}{l}
      \forall  I  
   \end{array}
   \ \ 
   \begin{array}{c}
      \fraz{A}
      {\forall x A^{a}_{x}}
   \end{array}
   \]
   \renewcommand{\arraystretch}{1}

   There is the usual ND restriction on the application of this rule, namely
   the newly quantified variable must not appear free in any of the 
   dependencies of \ARG{fact}.

   Several simultaneous universal generalizations on \ARG{fact}'s wff
   can be carried out with this command.
   Each element of the substitution list may be either an individual variable
   (e.g. {\tt x}) or a pair.
   The pair elements are two individual variables ({\tt y:x}) or an individual
   parameter and an individual variable ({\tt a:x}).
   For each element in the substitution list a new universal quantifier is put
   at the front of \ARG{fact}'s wff. 
   In the case of pairs of the kind {\tt a:x},  the variable {\tt x} is
   substituted for all occurrences of the individual parameter {\tt a}
   which are not within the scope of the universal quantification of {\tt x}. 
   The individual parameter {\tt a} must not occur within the scope of a
   quantifier binding {\tt x}, otherwise an error message is returned. 

   The dependencies of the new created proof line are the same as those of
   \ARG{fact}.

   The rule with sorts must also satisfy the following restriction.
   Let us first consider a substitution list whose only substitution is
   {\tt a:x} or {\tt y:x}, where {\tt x} is a variable of sort {\tt Sx}.
   Then the sorted rule is applicable only if {\tt a} and {\tt y} are terms
   of sort weakly more general than {\tt Sx}.
   In the general case, the specified condition must hold for all the
   substitutions of the substitution list.
}

\gfrecap{
   Applies introduction of universal quantification.
   The usual natural deduction's restrictions apply.
}

\gfexample+
   ***** comment ! An example with generalization from individual parameters
                   to quantified variables !
        
   ***** declare predconst P 1;
   ***** declare indvar x;
   ***** declare indpar a;
   [...]

   ***** assume P(a);
   1   P(a)     (1)
   ***** alli 1 a:x;
   alli 1 a:x;
   Some variables appear free in the assumptions.
   ***** impi 1 1;
   2   P(a) imp P(a)
   ***** alli 2 a:x;
   3   forall x.(P(x) imp P(x))
   ***** alli 2 x;
   4   forall x.(P(a) imp P(a))
    
     
   ***** comment ! The same example when substituting more than one variable !
     
   ***** reset;
   ***** declare predconst Q 2;
   ***** declare indvar x y;
   ***** declare indpar a b;
   [...]

   ***** assume Q(a b);
   1   Q(a,b)     (1)
   ***** impi 1 1;
   2   Q(a,b) imp Q(a,b)
   ***** alli 2 a:x b:y;
   3   forall x y. (Q(x,y) imp Q(x,y))


   ***** comment ! An example with generalization from free variables
                   to quantified variables !

   ***** reset;
   ***** declare predconst P 1;
   ***** declare indvar x;
   [...]

   ***** assume P(x);
   1   P(x)     (1)
   ***** impi 1 1;
   2   P(x) imp P(x)
   ***** alli 2 x:x;
   3   forall x.(P(x) imp P(x))
   ***** alli 2 x;
   4   forall x.(P(x) imp P(x))
     

   ***** comment ! An example of non applicability of generalization from
                   individual parameters to quantified variables with sorts !
      
   ***** declare predconst P 1;
   ***** declare indvar x [Sx];
   ***** declare indpar a [Sa];
   [...]

   ***** assume P(a);
   1   P(a)     (1)

   ***** alli 1 a:x;
   alli 1 a:x;
   Some variables appear free in the assumptions.

   ***** impi 1 1;
   2   P(a) imp P(a)

   ***** alli 2 a:x;
   alli 2 a:x
   A <var> cannot be replaced by one with more general sort

   ***** moregeneral Sa < Sx >;

   ***** alli 2 a:x;
   3   forall x. (P(x) imp P(x))
+


\gfnotes{
   The rule is often called ``universal generalization".
}
