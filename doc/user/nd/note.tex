\gfcommand{note}{falsity rule in classical logic}.
\index{note}\index{ne}

\gfsyntax{
   note \ALT ne \ARG{fact1} \OPT{,} \ARG{wff}; \\
   note \ALT ne \ARG{fact1} \OPT{,} \ARG{fact2};
}

\gfdescription{
   \renewcommand{\arraystretch}{0.5}
   \[
   \begin{array}{l}
      \\ \\ 
      \bot_c 
   \end{array}
   \ \ 
   \begin{array}{c}
      {[\neg A]}\\
      \vdots\\
      \bot\\
      \hline\\
      A 
   \end{array}
   \]
   \renewcommand{\arraystretch}{1}

   The wff of \ARG{fact1} must be {\tt FALSE} and \ARG{wff} must be a negation
   of the form $\neg A$.
   The wff derived is $A$.
   It depends on all the dependencies of \ARG{fact1} less the dependencies of
   all those facts whose wff is equal to \ARG{wff} ($\neg A$) (in the first) 
   form or \ARG{fact1}'s wff (in the second form).
}

\gfrecap{
   The wff of `fact1' must be `FALSE' and `wff' must be a negation
   of the form `not A'.
   The wff derived is `A'.
   It depends on all the dependencies of `fact1' less the dependencies of
   all those facts whose wff is equal to `wff' (`not A') (in the first) 
   form or the formula of `fact1' (in the second form).
}

\gfexample+
   ***** declare sentconst A;
   [...]

   ***** assume not A not not A;
   1   not A     (1)
   2   not (not A)     (2)

   *****  falsei 1 2;
   3   FALSE     (1 2)

   *****  note 3 not A;
   4   A     (2)

   *****  impi 2 4;
   5   (not (not A)) imp A 
+

\gfnotes{
   This rule implements ``reductio ad absurdum''. 
   Any proof using only the ND commands but not this rule is valid
   intuitionistically. 
}
