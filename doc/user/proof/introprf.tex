\section{Multiple proofs}
\label{sec-proof}

\subsection{Introduction}

In {\GF} the user can build multiple distinct proofs (sequences of proof lines).
All the proofs within a context (see section \ref{sec-cxt}) share the language,
the  axioms, and the computational model of the context.
However, proofs (in the same context) differ by their proof lines. A proof  line
belongs to one and only  one proof.
The proof you are working in is the {\em current  proof}.
When you enter the system the current proof does not contain proof lines and has
no name (the {\em un-named} proof).
If you want to leave the proof to build another one, you have to give it a name
(by the command {\tt nameproof}).
This allows us to refer to it later on.
You can create a new proof by using {\tt makeproof}, and  switch to it by using
{\tt switchproof}. The proof you switch to becomes then the current proof.
