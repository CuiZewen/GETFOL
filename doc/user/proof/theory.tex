\gfcommand{theory}{theory definition}
\index{theory}

\gfsyntax{
  theory \ARG{thlabel} : \ARG{wff1} \OPT{\ARG{wff2} \SEQ};\\
  theory \ARG{thlabel} : \ARG{axlabel1} : \ARG{wff}
                         \OPT{\ARG{axlabel2} : \ARG{wff} \SEQ};
}

\gfdescription{
  Creates a theory, {\it ie.} a set of axioms that can be used as a whole
  in a deduction.
  The theory contains as many axioms as
  the number of wffs specified in \ARG{wff1} \OPT{\ARG{wff2} \SEQ}
  or the axioms specified in \ARG{axlabel1} : \ARG{wff}
  \OPT{\ARG{axlabel2} : \ARG{wff} \SEQ}.\\
  %
  In the first form, each axiom is given an axiom label (a name) by {\GF}; 
  the label is generated by taking the theory label {\em thlabel} and 
  concatenating an increasing integer starting from 1.\\
  %
  In the second form, the axiom's label is given by the user
  ({\em axlabel$_i$}).\\
  %
  A theory can be used to perform inference 
  ({\it e.g.} by applying inference rules on the theory itself). In this case,
  it is considered as
  a fact whose formula is the conjunction of the theory axioms' formulas.
  Any axiom in the theory can also be used individually by specifying its
  axiom label.\\
}

\gfrecap{
  Declares a theory (a set of axioms) in the current context; axioms in a
  theory are indexed by `axlabelN'.
}

\gfexample+
   ***** declare sentconst A B C;
   ...

   ***** theory hilbert : 
          A imp (B imp C) 
          (A imp (B imp C)) imp ((A imp B) imp (A imp C));
   hilbert
   hilbert1 : A imp (B imp C)
   hilbert2 : (A imp (B imp C)) imp ((A imp B) imp (A imp C))

   ***** mp hilbert1 hilbert2;
   1   (A imp B) imp (A imp C)

   ***** andi hilbert hilbert;
   4   (((A imp B) imp C) and 
       (((A imp B) imp C) imp ((A imp B) imp (A imp C)))) and 
       (((A imp B) imp C) and 
       (((A imp B) imp C) imp ((A imp B) imp (A imp C))))

   ***** theory tautologies: 
          IMP: A imp A 
          OR: A or not A;
   tautologies
   IMP : A imp A
   OR : A or (not A)
+
