\gfcommand{theorem}{axiom creation from facts with no dependencies}
\index{theorem}

\gfsyntax{
  theorem \ARG{sym} \ARG{hook};
}

\gfdescription{
  The command creates an axiom from the fact whose hook
  is \ARG{hook}. The {\em axlabel} is \ARG{sym} and the {\em wff} is the formula
  of the fact; \ARG{hook} can identify an axiom, a theory 
  or a proof line without dependencies.
}

\gfrecap{
  Adds a fact without dependencies to the list of axioms.
}

\gfexample+
   ***** declare sentconst A;
   ...

   ***** assume A;
   1   A     (1)
   ***** theorem th1 1;
   theorem th1 1;
   A <fact> with dependencies cannot be made into a theorem
   ***** impi 1 1;
   2   A imp A     
   ***** theorem th1 2;
   th1 : A imp A
   ***** show axiom;
   th1 : A imp A
   ***** axiom ax1 : A or not A;
   ax1 : A or (not A)
   ***** theorem ExcludedMiddle ax1;
   ExcludedMiddle : A or (not A)
   ***** show axiom;
   th1 : A imp A
   ax1 : A or (not A)
   ExcludedMiddle : A or (not A)

   ***** declare predpar P 1;
   ***** declare predconst p 1;
   ***** declare indvar x;
   ***** axiom PI: P(x);
   ...
   ***** andi PI P:lambda x.p(x) PI P:lambda x.p(x);
   3   p(x) and p(x)     
   ***** theorem thPI PI;
   thPI : P(x)
   ***** andi thPI P:lambda x.p(x) thPI P:lambda x.p(x);
   4   p(x) and p(x)     
+
   
\gfnotes{
  The axiom may be an axiom schema in which case it becomes then 
  possible to instantiate it.
  In {\GF} proof lines containing {\tt predpar}s
  and/or {\tt funpar}s can {\it not} be instantiated. 
  If the user wants to prove $A \imp A$ and $B \imp B$,
  he can carry out the proof with non-instantiated axioms and get the
  schema $\alpha \imp \alpha$. $\alpha \imp \alpha$ can then be instantiated to
  $A \imp A$ and $B \imp B$. This capability is provided 
  by the command {\tt theorem}.
}




