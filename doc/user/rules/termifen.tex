\gfcommand{termifen}{term conditional elimination (with negation)}
\index{termifen}

\gfsyntax{
	termifen \ARG{fact1} \ARG{fact2} \ARG{termif}; \\
	termifen \ARG{fact1} \ARG{fact2} \ARG{termif} \OPT{occ \ARG{n1} \ARG{n2}
	\SEQ}; 
}

\gfdescription{
	\renewcommand{\arraystretch}{0.5}
	\[
	\mbox{{\em tif}} \  E_{\neg} \ \ 
	\fraz{\neg A \ \ \ \ \ B(\mbox{{\em termif}} \ A \ \mbox{{\em then}} \ t_1 \
	\mbox{{\em else}} \ t_2)} 
	{B(t_2)}
	\]
	\renewcommand{\arraystretch}{1}
	
	If {\em termif} is {\tt iftrm A then t1 else t2}, {\em fact}$_1$'s wff is
	{\tt W(iftrm A then t1 else t2)} and \ARG{fact2}'s wff is {\tt not A},
	then the rule deduces {\tt W(t2)}.
	If \ARG{termif} is not a subexpression of \ARG{fact1}'s wff, then no
	substitution is performed. 
	Individual occurrences can be substituted by specifying the optional 
	\ARG{n1}, \ARG{n2}, \SEQ, where \ARG{n1}, \ARG{n2}, \SEQ are the occurrences
	to be substituted.
	Without this option, all occurrences are substituted. 
	The dependencies of the derived wff are the union of those of \ARG{fact1}
	and \ARG{fact2}.
}

\gfrecap{
Term conditional elimination (with negation).
}

\gfexample+
   ***** declare sentconst A;
   ***** declare predconst P 1;
   ***** declare indpar a b;
   ***** assume P(trmif A then a else b);
   1   P(trmif A then a else b)     (1)
   ***** assume not A;
   2   not A     (2)
   ***** termifen 1 2 trmif A then a else b;
   3   P(b)     (1 2)
   \end{verbatim}
+

\gfnotes{
	\ARG{fact2}'s wff must be negation of \ARG{termif}'s condition and not
	viceversa.
	If \ARG{termif} = {\tt trmif not A then t1 else t2} and \ARG{fact2}'s wff =
	{\tt A}, then the rule is not applicable. 
}
