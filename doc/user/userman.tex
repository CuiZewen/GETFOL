%............................... USER MANUAL .................................
%.............................................................................

\documentstyle[12pt,rawfonts]{../styfiles/GFmanual}

\title{GETFOL Manual}
\author{\bf Fausto Giunchiglia}
\date{7 March 1994}
\version{2.0}
\abstract{
	  {\GF} is an interactive reasoning system.
	  We use it as an environment for studying epistemological issues.
	  We try to look at questions like: which notions are
	  important for the development of mechanized reasoning systems?
	  What kind of conversations do we want to have with them?
	  What parts of logic should we use to represent such notions?
	  How should logic be embedded in a conversational reasoning system?
	}
\addresses{
     \begin{tabular}[c]{l}
       {\bf Fausto Giunchiglia}           \\
       Mechanized Reasoning Group		  \\
		 IRST, Povo, 38050 Trento, Italy  \\
		 e-mail: {\tt fausto@irst.it}     \\
		 phone: +39 461 314436
	\end{tabular}	
}
\published{
  \begin{tabular}{l}
	  DIST Technical Report No. 92-0010 (1994). \\
	  DIST -- University of Genoa,\\
	  Via Opera Pia 11A, 16145 Genova, Italy.\\ \\
  \end{tabular}
}

%% \newcommand{\gfbibliography}{%
%% \bibliography{/home/tarski/staff/mrg/biblio/bib/a-l,%
%% /home/tarski/staff/mrg/biblio/bib/m-z,userman}}
\newcommand{\gfbibliography}{\bibliography{}}

%% \makeindex
\begin{document}
	%  ............................. COVER ..................................
	\thispagestyle{empty}
	\maketitle

	%  ........................ TABLE OF CONTENTS ...........................
	\newpage
	\pagenumbering{roman}
	\tableofcontents

	\newpage
	\pagenumbering{arabic}
	\pagestyle{headings}

	%  ........................... INTRODUCTION ............................
	\section{Introduction}

The distribution package allows the installation of the following three
applications:
%
\begin{itemize}
	\item
		the {\tt HGKM} interpreter \cite{giunchiglia35}, that is a LISP-like
		language acting as the implementation language of both {\tt FOL}
		and {\tt GETFOL}.
	\item
		the {\tt GETFOL} system \cite{giunchiglia12,giunchiglia29}.
	\item
		the {\tt AGETFOL} system ({\tt GETFOL} with {\em abstraction}
		\cite{giunchiglia7}.
\end{itemize}

From now on, we will write ``{\tt the system}" to mean {\tt HGKM},
{\tt GETFOL} and {\tt AGETFOL}.
The distribution package is provided with a set of general purpose
configuration and installation facilities which allow you to install your
favourite application on your machine.

In order to install {\tt the system} read section~\ref{requirements}
(``{\it Minimal System Requirements}'') and verify that your machine
has all the required features.
Then read section~\ref{instproc} (``{\it The installation procedure}'').
This section gives the instructions to follow in order to install
{\tt the system} on your machine.

If you are an expert you might be interested also in the features
of the installation and configuration facilities.
Section~\ref{sysmod} (``{\it Systems and Modules}'') introduces
to the data structures ({\it systems} and {\it modules}) devised
to store and/or modify the configuration of the applications.


	%  .............................. MODULES ..............................
	% loading introduction to the section
\section{Parser}

This section is intended to explain:
%
\begin{itemize}
	\item
		the main functionalities of the {\GF} scanning primitives,
	\item
		what modifications must be performed in the scanner data
		structures in order to be able to change its behavior ({\em e.g.} to
		define a new escape character).
\end{itemize}

The {\GF} scanner is a {\em backupable} scanner.
This feature is necessary as the parser works in a top-down fashion and,
sometimes, needs to backtrack. 

The {\GF} scanner is able to recognize three types of tokens:
%
\begin{enumerate}
	\item {\tt IDTOKEN},
	\item {\tt NUMTOKEN} and
	\item {\tt DELTOKEN}.
\end{enumerate}

The primitives to scan such tokens are \verb+FOLSYM@+, \verb+NATNUM@+ (actually
no dedicated primitive to scan a \verb+DELTOKEN+ exists).

\verb+TK@+ scans a generic token.

The primitives \verb+SCANSTATUS-GET+ and \verb+SCANSTATUS-RESTORE+ allow
respectively to save and restore the scanner status.
They are necessary to restore the status of the scanner in case of
unsuccessful parsing.

The high level functions \verb+FOLSYM@+, \verb+NATNUM@+, \verb+TK@+ (and other
not mentioned here for brevity) use the general \verb+TOKEN-GET-NEXT+ routine
which is also able to {\em bufferize} the tokens already read (reading from
an input stream is a destructive operation so we need at a some level a
buffer mechanism if we want to perform backtracking).

The buffer is implemented by the array \verb+TOKENARRAY+ whose dimension is
given by the macro \verb+TOKENARRAY-DIMENSION+. The max number of tokens
that can be present in a command line is given by the number returned
by this macro.

\verb+TOKEN-GET-NEXT+ routine is based on the lower level primitive
\verb+TOKEN-SCAN+, which reads the next token from the input stream and returns
its type.
\verb+TOKEN-SCAN+ is able to identify the type of a token reading (via 
\verb+CH-GET-NEXT+) its first character and identifying its type (via
\verb+CHTYPE-GET+).

The characters are divided in the following types:

\begin{enumerate}
	\item
		identifiers ({\tt IDCHAR}): see file
		\verb+ascitab.fol+;
	\item
		numbers (positive integers --- {\tt NUMCHAR}):
		\verb+0 1 2 3 4 5 6 7 8 9+;
	\item
		delimiter {\tt DELCHAR}:
		\verb+( ) , . : ; [ ] { }+;
	\item
		ignored char {\tt IGNCHAR}: see file
		\verb+ascitab.fol+;
	\item
		iddelim {\tt IDDELCHAR}:
        \verb$' * + - / < > = ? @ ^ ` |$;
	\item
		escape characters {\tt ESCCHAR}:
		\verb+\+;
	\item
		special handling {\tt SPECCHAR}
\end{enumerate}

The \verb+IDDELCHAR+s are identifier characters having also the functionality
of a delimiter, that is no \verb+IDTOKEN+ can contain a character of such
type but they by themselves may be considered \verb+IDTOKEN+.
For instance the string \verb+A@B+ will be regarded by the scanner a string
formed by the three distinct \verb+IDTOKEN+ \verb+A @ B+.
The escape characters allows to coerce the type of the following
character to be \verb+IDCHAR+.
For istance the string \verb+A\@B+ will be regarded by the scanner a string
formed by the \verb+IDTOKEN A@B+.

The only think important to know at the user level it is how to modify the type
of a character so that you can, for istance, extend the set of \verb+IDDELCHAR+
with the character "\verb+_+". To do this you have only to change type
declaration for the "\verb+_+" character from \verb+IDCHAR+ to \verb+IDDELCHAR+
in the file {\tt asciitab.fol}.

Finally a low level feature: each time a command line is issued to {\GF} the low
level scanning primitives store each read character in the {\tt TUPLE
SCANBUFARRAY}, so that if a syntactic error in the parsing is detected the rest
of the line will be read and the whole line printed out to show, using an
"\verb+^+", where the syntatic error has been detected. This useful feature
imposes (as it is implemented) a limit in the length of a {\GF} command line as
the dimension of \verb+SCANBUFARRAY+ is fixed by the value returned by the macro
\verb+SCANBUFARRAY-DIMENSION+.


% loading command files
\gfcommand{backup}{backup of a {\GF} session}
\index{backup}

\gfsyntax{
  backup \ARG{file} open;\\
  backup \ARG{file} close;  
}

\gfdescription{
   {\GF} stores in \ARG{file} all the successful commands between the command
   ``{\tt backup \ARG{file} open}" and the command ``{\tt backup \ARG{file}
   close}". 
}

\gfrecap{
Stores in a file all successfull commands type in GETFOL.
}

\gfexample+
   ***** backup file open;
   I am starting to backup onto file
   
   ***** declare sentconst A;
   ***** assume A;
   1   A     (1)
   
   ***** backup file close;
   
   ***** andi 1 1;
   2   A and A     (1)
   
   ***** ^D
   
   >Bye.
   
   <host-prompt> more file
   
   declare sentconst A;
   
   assume A;
   
   <host-prompt>
+

\gfnotes{
   Only Unix file names are supported.
   The file path name can be absolute or relative to the current directory.
   Multiple open backup files can exist simultaneously.
   Files must be explicitly closed to guarantee that all the commands
   are properly stored in the backup file.
}

\gfcommand{done}{exiting a {\GF} session}
\index{done}

\gfsyntax{done;}

\gfdescription{
   Returns the control back to the {\HG} environment~\cite{giunchiglia35}.
   You can get back to {\GF} by typing {\tt (SYSBACK)} at the {\HG} prompt.
}

\gfrecap{
Returns the control back to the HGKM environment.
You can get back to GETFOL by typing (SYSBACK) at the HGKM prompt.
}

\gfexample+
   **** done;
   Returning to host
   NIL
+

\gfnotes{}

\gfcommand{fetch}{fetches {\GF} file}
\index{fetch}

\gfsyntax{
   fetch \ARG{file} \OPT{from \ARG{mark1}} \OPT{to \ARG{mark2}};
}

\gfdescription{
   Redirects the standard input to the file \ARG{file}; all {\GF} commands
   between \ARG{mark1} and \ARG{mark2} are executed.
}

\gfrecap{
Fetches commands from the file `file'.
All commands between `mark1' and `mark2' are executed.
}

\gfexample+
  ***** fetch exmarks.tst;
  ...
  ***** fetch exmarks.tst to m1;
  ...
  ***** fetch exmarks.tst from m1 to m2;
  ...
  ***** fetch exmarks.tst from m2;
+

\gfnotes{
   Only Unix file names are supported.
   The file path name can be absolute or relative to the current directory.
   Nested fetches (and marking) are allowed.
   If no marks are specified, then the whole file is fetched.
   See the command {\tt mark} in this section to set marks in a file.
}


\gfcommand{mark}{sets a mark}
\index{mark}

\gfsyntax{
   mark \ARG{sym};
}

\gfdescription{
   Sets a mark in between a sequence of {\GF} commands.
   It can be used to fetch a file from/to a certain mark (see command 
   {\tt fetch} in this section).
}

\gfrecap{
Sets a mark in a file (see fetch).
}

\gfexample+
   <host-prompt> more example
   NAMECONTEXT META;
   DECLARE SORT FACT WFF;
   DECLARE INDVAR fc [FACT];
   DECLARE FUNCONST wffof (FACT) = WFF;
   DECLARE PREDCONST THEOREM 1;
   comment | here we put the first mark m1 |
   mark m1;
   DECREP FACT;
   DECREP WFF;
   REPRESENT \{WFF\} AS WFF;
   REPRESENT \{FACT\} AS FACT;
   comment | here we put the second mark m2 |
   mark m2;
   ATTACH wffof TO [FACT = WFF] fact\-get\-wff;
   AXIOM M2: forall fc. THEOREM(wffof(fc));
   MAKECONTEXT OBJ;
   SWITCHCONTEXT OBJ;
   DECLARE SENTCONST A;
+



\gfcommand{probe}{verbose mode}
\index{probe}

\gfsyntax{
   probe;\\
   probe \ARG{activity};\\
   probe all;\\
}

\gfdescription{
	Some {\GF} {\em activities} can be executed either in verbose
	or in silent mode.
	In the former, {\GF} displays messages describing the execution of the
	command which are not displayed in the silent mode.
	Examples of activities are given below. 

	{\tt probe} lists the probed commands

	{\tt probe} \ARG{activity} sets commands in the activity to be executed
	in a verbose mode. 

	{\tt probe all} sets all activities to verbose mode.
}

\gfrecap{
Set verbose mode for certain activities.
}

\gfexample+
   ***** probe;
   Probing function set : COMMAND no
   Probing function set : IO yes
   Probing function set : DECLARE yes
   Probing function set : PROOF yes
   Probing function set : ATTACH yes
   Probing function set : SIMPLIFY yes
   Probing function set : SIMPSET yes
   Probing function set : REWRITE yes
   Probing function set : EVAL yes
   Probing function set : CONTEXT yes
   Probing function set : REFLECT yes
   
   ***** declare sentconst A B C;
   A has been declared to be a Sentconst
   B has been declared to be a Sentconst
   C has been declared to be a Sentconst
   
   ***** probe command;
   
   ***** probe;
   probe;
   Probing function set : COMMAND yes
   Probing function set : IO yes
   Probing function set : DECLARE yes
   Probing function set : PROOF yes
   Probing function set : ATTACH yes
   Probing function set : SIMPLIFY yes
   Probing function set : SIMPSET yes
   Probing function set : REWRITE yes
   Probing function set : EVAL yes
   Probing function set : CONTEXT yes
   Probing function set : REFLECT yes
   
   ***** declare sentconst D E F;
   declare sentconst D E F;
   D has been declared to be a Sentconst
   E has been declared to be a Sentconst
   F has been declared to be a Sentconst
   
   ***** unprobe all;
   unprobe all;
   
   ***** declare sentconst H G K;
+

\gfnotes{}

\gfcommand{unprobe}{silent mode}
\index{unprobe}

\gfsyntax{
   unprobe \ARG{activity};\\
   unprobe all;
}

\gfdescription{
   Some {\GF} {\em activities} can be executed either in verbose
   or in silent mode.

   {\tt unprobe} \ARG{activity} sets the silent mode for commands in
   \ARG{activity}.

   {\tt unprobe all} sets all activities to silent mode .
}

\gfrecap{
Set/unset verbose mode for certain activities.  
}

\gfexample+
   ***** declare sentconst D E F;
   declare sentconst B E F;
   D has been declared to be a Sentconst
   E has been declared to be a Sentconst
   F has been declared to be a Sentconst
   
   ***** unprobe all;
   unprobe all;

   ***** probe;
   Probing function set : COMMAND no
   Probing function set : IO no
   Probing function set : DECLARE no
   Probing function set : PROOF no
   Probing function set : ATTACH no
   Probing function set : SIMPLIFY no
   Probing function set : SIMPSET no
   Probing function set : REWRITE no
   Probing function set : EVAL no
   Probing function set : CONTEXT no
   Probing function set : REFLECT no
   
   ***** declare sentconst H G K;
+

\gfnotes{}


	% loading introduction to the section
\newpage
\section{Administration}
\label{sec-adm}

The commands described in this section manipulate the proof checker but do
not modify  the ``logical'' state of the deduction or of the computation.
Among the other things, they can be used to give alternative names to proof lines,
to load {\HG} or {\GF} files, to insert comments in {\GF} files, to show the
logical/computational status of the system.


% loading explanation of commands
\gfcommand{comment}{comments in {\GF}}
\index{comment}

\gfsyntax{
   comment \ARG{separator} \OPT{\ARG{text}} \ARG{separator}
}

\gfdescription{
   Defines a comment \ARG{text} between the two \ARG{separator}s.
   Any token can be a separator.
}

\gfrecap{
Defines a comment between two separators.
Any token can be a separator.
}

\gfexample+
   ***** comment ! this is a comment 
   enclosed between two exclamation marks !
   
   ***** comment ? this is a comment 
   enclosed between two question marks ?
   
   ***** comment com this is a comment 
   enclosed between the two words "com" com
+


\gfcommand{deflam}{defining {\HG} functions}
\index{deflam}

\gfsyntax{
   deflam \ARG{funname} \ARG{var-list} \ARG{form};
}

\gfdescription{
   Defines a {\HG} function at the {\GF} prompt.
   The function \ARG{funname} is defined as the {\HG} \ARG{form}
   with parameters the parameters in the list \ARG{var-list}.
}

\gfrecap{
Defines a HGKM function at the GETFOL prompt.
}

\gfexample+
   ***** deflam wffof (fact) (fact\-get\-wff fact);
+


\gfcommand{echo}{echoes a message to the standard output stream}
\index{echo}

\gfsyntax{
   echo \ARG{separator} \OPT{\ARG{text}} \ARG{separator}
}

\gfdescription{
   Echoes \ARG{text} between the two \ARG{separator}s to the current
   output stream.
   Any token can be a \ARG{separator}.
}

\gfrecap{
Echoes text between the two separators.
}

\gfexample+
   ***** echo ! this is an echo
   enclosed between two exclamation marks !
   this is an echo enclosed between two exclamation marks

   ***** comment ? this is an echo
   enclosed between two question marks ?
   this is an echo enclosed between two question marks

   ***** comment com this is a echo
   enclosed between the two words "com" com
   this is an echo enclosed between the two words "com"
+

\gfcommand{hgk}{{\HG} evaluation}
\index{hgk}

\gfsyntax{
   hgk \ARG{s-expr};
}

\gfdescription{
   Runs the {\HG} evaluator on \ARG{s-expr}.
}

\gfrecap{
Runs the HGKM evaluator on `s-expr'.   
}

\gfexample+
   ***** hgk (LOAD (QUOTE "file"))
+

\gfnotes{
	The command hgk tries to evaluate every element of the {\em
	s-expression} passed as argument; therefore \verb+(LOAD "file")+
	causes an error, if \verb+"file"+ has no value.
}
\gfcommand{know natnums}{allows the use of natural numbers}
\index{know natnums}

\gfsyntax{
	know natnums \OPT{\ARG{natnum1}, \SEQ, \ARG{natnumN}};
}

\gfdescription{
	It allows the use of \ARG{natnum1}, \SEQ, \ARG{natnumN}
	(all natural numbers if no natural numbers are explicitly listed).
	It declares the sort {\tt NATNUMSORT}, the representation 
	{\tt NATNUMREP} and optionally defines the extension of {\tt NATNUMSORT} 
	to be \ARG{natnum1}, \SEQ, \ARG{natnumN} (if explicitly listed).
}

\gfexample+
   ***** know natnums;
   ***** simplify (5=7);
   1   not (5 = 7)   
   ***** know natnums 1 2 3 4;
   Warning! You already know natnums.
   Now the extension of NATNUMSORT is fixed to be : (1 2 3 4)
   ***** simplify (5=7);
   simplify (5=7);
            ^
   SIMPLIFY requires a wff,fact or term here
   ***** simplify (1=3);
   2   not (1 = 3)
+

\gfcommand{load}{loading a {\HG} file}
\index{load}

\gfsyntax{
   load \ARG{file};
}

\gfdescription{
   Each form in \ARG{file} is read by the {\HG} reader and evaluated by the 
   {\HG} evaluator \cite{giunchiglia35}.
}

\gfrecap{
Each form in `file' is read and evaluated by HGKM.
}

\gfexample+
   ***** load example.hgk;
+


\gfcommand{resetprompt}{Resets the user defined prompt}
\index{resetprompt}

\gfsyntax{
   resetprompt;
}

\gfdescription{
   Comes back to the default prompt.
}
\gfrecap{
Comes back to the default prompt.
}

\gfexample+
   CARTOONIA:: resetprompt;

   ***** switchcontext Disneyland;
   You are now using context: Disneyland
   You are switching to a proof with no name.

   *****
+

\gfcommand{setprompt}{Redefines the prompt}
\index{setprompt}

\gfsyntax{
   setprompt to \ARG{s-expr};
}

\gfdescription{
   Sets {\GF}'s prompt  to the value of the s-expression
   \ARG{s-expr}  followed by  ":: ". It is particularly  useful when  you are
   working on multiple contexts as you can set the prompt to the value of
   the current context.
}

\gfrecap{
Sets GETFOL's prompt to `s-expr'.
}

\gfexample+
   ***** setprompt to (QUOTE myprompt);

   myprompt:: setprompt to (QUOTE Tweedledee\&Tweedledum);

   Tweedledee&Tweedledum:: setprompt to (CAPITALIZE (curcname\-get));

   NOTNAMED&:: namecontext Disneyland;
   You have named the current context: Disneyland

   DISNEYLAND:: makecontext Cartoonia;
   You have created the empty context: Cartoonia

   DISNEYLAND:: switchcontext Cartoonia;
   You are now using context: Cartoonia
   You are switching to a proof with no name.

   CARTOONIA:: 
+

\gfnotes{
	The command tries to evaluate the {\em s-expression} passed as argument.
	Failure of the evaluation causes a crash to the {\HG} evalautor.
}

\gfcommand{show}{shows {\GF} information}

\gfsyntax{
   show \ARG{option};
}

\gfdescription{
   Shows {\GF} information. In the example we show some of the
   options implemented in a {\GF} version.
   Notice that ``\verb+show show;+" lists the options supported by the show
   command.
}

\gfrecap{
Shows GETFOL information.
}

\gfexample+
   ***** comment | ******* SHOW SHOW ******* |
   ***** show show;
   The list of show options is the following:

   CONTEXT : WHEREAMI 

   REWRITER : SIMPSET 

   SIMPLIFIER : INT REP 

   DEFINITION : DEFINITION 

   PROOF : PREMISES FACT AXIOM 

   LANGUAGE : LGS MGS MEM SORT TYP SYM 

   SYSTEM : COM SHOW 
   
   ***** comment | ******* SHOW WHEREAMI ******* |
   
   ***** show whereami;
   You are now using an unnamed context.
   You are now using an unnamed proof.
   
   ***** comment | ******* SHOW REP and INT ******* |
   
   ***** decrep PIPPOREP;
   ***** attach A dar [PIPPOREP] a;
   ***** attach C dar [PIPPOREP] c;
   ***** attach B to  [PIPPOREP] b;
   
   ***** show rep PIPPOREP;
   The designators for the representation: PIPPOREP are:
   (c . C) (a . A)
   
   ***** show int A;
   The Indconst A is attached to 'a
   with representation PIPPOREP
   
   ***** comment | ******* SHOW SIMPSET ******* |
   
   ***** setbasicsimp simp1 at wffs
   {forall x1.F1(x1)=x1,
   forall x1 x2.F2(x1 x2)=x1,
   forall x1 x2 x3.F3(x1 x2 x3)=x1};
   
   ***** show simpset simp1;
   Wffs :
   forall x1. (F1(x1) = x1)
   forall x1 x2. (F2(x1,x2) = x1)
   forall x1 x2 x3. (F3(x1,x2,x3) = x1)
   
   ***** setbasicsimp simp2 at facts {1 2};
   
   ***** show simpset simp2;
   Proof lines :  1 2
   
   ***** comment | AX1 and AX2 are two axioms |
   ***** setbasicsimp simp3 at facts {AX1 AX2};
   ***** show simpset simp3;
   Axioms :  AX1 AX2
   
   ***** setcompsimp simpA at simp1 uni simp2 uni simp3;
   ***** show simpset simpA;
   simpA is compound by this list of basic simpsets :
   (simp1 simp2 simp3)
   
   ***** SETCOMPSIMP simpB at simpA dif simp1;
   ***** show simpset simpB;
   simpB is compound by this list of basic simpsets :(simp2 simp3)
   
   
   ***** comment | ******* SHOW PROOF, FACT and AXIOM ******* |
   
   ***** show proof;
   1   forall x1. (F1(x1) = x1)     (1)
   2   forall x1 x2. (F2(x1,x2) = x1)     (2)
   
   ***** label fact identity = 1;
   ***** show fact;
   identity   1
   
   ***** show axiom;
   AX1 : forall x1. (F1(x1) = x1)
   AX2 : forall x1 x2. (F2(x1,x2) = x1)
   
   ***** comment | ******* SHOW LSG, MGS , MEM and SORT ******* |
   
   ***** declare sort a b c d e f g h;
   moregeneral a < b c d e f g h >;
   moregeneral b < c d e f g h >;
   
   ***** show lgs a;
   No sort is strictly lessgeneral than a.
   ***** show lgs b;
   No sort is strictly lessgeneral than b.
   
   ***** show mgs a;
   The only sort strictly moregeneral than a is UNIVERSAL
   ***** show mgs f;
   The only sort strictly moregeneral than f is UNIVERSAL
   
   ***** show mem a;
   No <indsym> is declared to be of sort a.
   
   ***** declare indvar x [a];
   a is a sort
   x has been declared to be an Indvar
   ***** declare indvar y [a];
   a is a sort
   y has been declared to be an Indvar
   ***** show mem a;
   The <indsym>'s declared to be of sort a are
       y  x  
   
   ***** show sort;
   The symbols declared to be sorts are
       b  a  UNIVERSAL  
   
   ***** comment | ******* SHOW TYP, SYM ******* |
   
   ***** show typ INDVAR;
   The symbols declared to be Indvars are
       x  y  
   
   ***** show sym a;
   a is declared to be a sort.
   ***** show sym x;
   x is declared to be an Indvar of sort a.
   
   
   ***** comment | **************** SHOW PREMISES *************** |
   
   ***** declare sentconst A B C;
   ***** assume A B;
   1   A     (1)
   2   B     (2)
   ***** andi 1 2;
   3   A and B     (1 2)
   ***** ori 3 C;
   4   (A and B) or C     (1 2)
   ***** andi 3 4;
   5   (A and B) and ((A and B) or C)     (1 2)
   ***** show premises 5;
   5  (A and B) and ((A and B) or C)  (1 2)
      3  A and B  (1 2)
      4  (A and B) or C  (1 2)
   ***** show premises 5 2;
   5  (A and B) and ((A and B) or C)  (1 2)
      3  A and B  (1 2)
         1  A  (1)
         2  B  (2)
      4  (A and B) or C  (1 2)
         3  A and B  (1 2)
   *****  show premises 5 all;
   5  (A and B) and ((A and B) or C)  (1 2)
      3  A and B  (1 2)
         1  A  (1)
         2  B  (2)
      4  (A and B) or C  (1 2)
         3  A and B  (1 2)
            1  A  (1)
            2  B  (2)
   
   
   ***** comment | ******* SHOW COM ******* |
   
   ***** show com;
   The list of commands is the following:

   META : REFLECT MATTACH 

   CONTEXT : COPYLEX SWITCHCONTEXT COPYCONTEXT NAMECONTEXT MAKECONTEXT 

   DECIDER : DECIDE MONADEQ MONAD TAUTEQ TAUT PTAUT 

   EVAL : EVAL 

   SIMPLIFIER : SIMPLIFY LET HARDWARE REPRESENT ATTACH DECREP 

   REWRITER : REWRITE ASSERTSIMP SETCOMPSIMP SETBASICSIMP UNFOLD FOLD CUT
   CTC CONTRACT WK WEAKEN WFFIFI WFFIFEN WFFIFE TERMIFI TERMIFEN TERMIFE  

   Natural-Deduction : ES EXISTE EXISTI US ALLE UG ALLI IE IFFE II IFFI
   NE NOTE NI NOTI FE FALSEE FI FALSEI OE ORE OI ORI AE ANDE AI ANDI MP
   IMPE DED IMPI SUBST THEOREM ASSUME   

   DEFINITION : DEFINE 

   PROOF : LABEL CANCEL AXIOM THEORY 

   LANGUAGE : SWITCHPROOF COPYPROOF NAMEPROOF MAKEPROOF EXTENSION WFF
   AWFF TERM MOREGENERAL MOSTGENERAL SETFMAP DECLARE  

   SYSTEM : COPYLEX RESET PAGER RESETPROMPT SETPROMPT KNOW HGK SHOW ECHO
   COMMENT UNPROBE PROBE DEFLAM LOAD MARK FETCH DONE BACKUP  
+

 	;;;;;;;;;;;;;;;;;;;;;;;;;;;;;;;;;;;;;;;;;;;;;;;;;;;;;;;;;;;;;;;;;;;;;;;;;;;;;;
;;
;; FOL version 2.001
;; This file is an FOL source file: language.cfg
;; Date: Wed Oct 20 10:46:03 MET 1993
;;
;;;;;;;;;;;;;;;;;;;;;;;;;;;;;;;;;;;;;;;;;;;;;;;;;;;;;;;;;;;;;;;;;;;;;;;;;;;;;;
;;                                                                          ;;
;;   Copyright (c) 1986-1987 by Richard Weyhrauch.  All rights reserved.    ;;
;;   Copyright (c) 1987-1988 by Fausto Giunchiglia.  All rights reserved.   ;;
;;                                                                          ;;
;;   This software is being provided to you, the LICENSEE, by Richard       ;;
;;   Weyhrauch and Fausto Giunchiglia, the AUTHORS, under certain rights    ;;
;;   and obligations.  By obtaining, using and/or copying this software,    ;;
;;   you indicate that you have read, understood, and will comply with      ;;
;;   the following terms and conditions:                                    ;;
;;                                                                          ;;
;;   THE AUTHORS MAKE NO REPRESENTATIONS OF WARRANTIES, EXPRESS OR          ;;
;;   IMPLIED.  By way of example, but not limitation, THE AUTHORS MAKE      ;;
;;   NO REPRESENTATIONS OR WARRANTIES OF MERCHANTABILITY OF FITNESS FOR     ;;
;;   ANY PARTICULAR PURPOSE OR THAT THE USE OF THE LICENSED SOFTWARE        ;;
;;   COMPONENTS OR DOCUMENTATION WILL NOT INFRINGE ANY PATENTS,             ;;
;;   COPYRIGHTS, TRADEMARKS OR OTHER RIGHTS.                                ;;
;;                                                                          ;;
;;   The AUTHORS shall not be held liable for any direct, indirect or       ;;
;;   consequential damages with respect to any claim by LICENSEE or any     ;;
;;   third party on account of or arising from this Agreement or use of     ;;
;;   this software.  Permission to use, copy, modify and distribute this    ;;
;;   software and its documentation for any purpose and without fee or      ;;
;;   royalty is hereby granted, provided that the above copyright notice    ;;
;;   and disclaimer appears in and on ALL copies of the software and        ;;
;;   documentation, whether original to the AUTHORS or a modified           ;;
;;   version by LICENSEE.                                                   ;;
;;                                                                          ;;
;;   The name of the AUTHORS may not be used in advertising or publicity    ;;
;;   pertaining to distribution of the software without specific, written   ;;
;;   prior permission.  Notice must be given in supporting documentation    ;;
;;   that such distribution is by permission of the AUTHORS.  The AUTHORS   ;;
;;   make no representations about the suitability of this software for     ;;
;;   any purpose.  It is provided "AS IS" without express or implied        ;;
;;   warranty.  Title to copyright to this software and to any associated   ;;
;;   documentation shall at all times remain with the AUTHORS and LICENSEE  ;;
;;   agrees to preserve same.  LICENSEE agrees to place the appropriate     ;;
;;   copyright notice on any such copies.                                   ;;
;;                                                                          ;;
;;;;;;;;;;;;;;;;;;;;;;;;;;;;;;;;;;;;;;;;;;;;;;;;;;;;;;;;;;;;;;;;;;;;;;;;;;;;;;

;*************************************************************************;
;                                                                         ;
;                    "LANGUAGE" MODULE CONFIGURATION FILE                 ;
;                                                                         ;
;*************************************************************************;

(MODULE-INIT        'LANGUAGE)
(MODULE-SET-NAME    'LANGUAGE "LANGUAGE")
(MODULE-SET-MODE    'LANGUAGE 'COMPILED)

(MODULE-SET-SRCDIR 'LANGUAGE (PATH-CONCAT (SYS-GET-SRCDIR 'GETFOL) "language"))
(MODULE-SET-OBJDIR 'LANGUAGE (SYS-GET-OBJDIR 'GETFOL))
(MODULE-SET-DOCDIR 'LANGUAGE (PATH-CONCAT (SYS-GET-DOCDIR 'GETFOL) "language"))

(MODULE-SET-DOCFILE 'LANGUAGE
   (PATH-CONCAT  (MODULE-GET-DOCDIR 'LANGUAGE) "language.tex"))


;;;     GETHGKM special variables declaration
(MODULE-ADD-FILE 'LANGUAGE   "vlang.cl"     ""         'INTERPRETED)

;;;     Label Spaces
(MODULE-ADD-FILE 'LANGUAGE   "labelspa.hgk" "labspah"  'COMPILED)
(MODULE-ADD-FILE 'LANGUAGE   "labelspa.fol" "labspaf"  'COMPILED)
(MODULE-ADD-FILE 'LANGUAGE   "labelspa.rp"  "labspar"  'COMPILED)

;;;     Signature, symbols and sorts
(MODULE-ADD-FILE 'LANGUAGE   "symls.hgk"    "symlsh"   'COMPILED)
(MODULE-ADD-FILE 'LANGUAGE   "sym.hgk"      "symh"     'COMPILED)
(MODULE-ADD-FILE 'LANGUAGE   "symls.fol"    "symlsf"   'COMPILED)
(MODULE-ADD-FILE 'LANGUAGE   "sym.fol"      "symf"     'COMPILED)
(MODULE-ADD-FILE 'LANGUAGE   "sym.rp"       "symr"     'COMPILED)

;;;     Expressions
(MODULE-ADD-FILE 'LANGUAGE   "exp.hgk"      "exph"     'INTERPRETED)
(MODULE-ADD-FILE 'LANGUAGE   "exp.rp"       "expr"     'COMPILED)
(MODULE-ADD-FILE 'LANGUAGE   "exp.fol"      "expf"     'COMPILED)

;;;     Sorts
(MODULE-ADD-FILE 'LANGUAGE   "sort.hgk"     "sorth"    'COMPILED)
(MODULE-ADD-FILE 'LANGUAGE   "sort.fol"     "sortf"    'COMPILED)

;;;     Probe file
(MODULE-ADD-FILE 'LANGUAGE   "problang.fol" "problaf"  'COMPILED)

;;;     Command files
(MODULE-ADD-FILE 'LANGUAGE   "decsymls.hgk" "decsymlh" 'COMPILED)
(MODULE-ADD-FILE 'LANGUAGE   "decsymls.fol" "decsymlf" 'COMPILED)
(MODULE-ADD-FILE 'LANGUAGE   "language.fol" "langf"    'COMPILED)
(MODULE-ADD-FILE 'LANGUAGE   "cmdlang.fol"  "cmdlangf" 'COMPILED)

;;;     Defaults for the language
(MODULE-ADD-FILE 'LANGUAGE   "langdflt.fol" "langdflf" 'COMPILED)

;;;     Show files
(MODULE-ADD-FILE 'LANGUAGE   "showlang.fol" "showlanf" 'COMPILED)
(MODULE-ADD-FILE 'LANGUAGE   "showlang.rp"  "showlanr" 'COMPILED)


(MODULE-ADD-FILE 'LANGUAGE   "skolem.hgk"   "skolh"    'COMPILED)


;;;     Initialization files
(MODULE-ADD-FILE 'LANGUAGE   "ilang.fol"    ""         'INTERPRETED)

 	;;;;;;;;;;;;;;;;;;;;;;;;;;;;;;;;;;;;;;;;;;;;;;;;;;;;;;;;;;;;;;;;;;;;;;;;;;;;;;
;;
;; FOL version 2.001
;; This file is an FOL source file: proof.cfg
;; Date: Wed Oct 20 10:47:27 MET 1993
;;
;;;;;;;;;;;;;;;;;;;;;;;;;;;;;;;;;;;;;;;;;;;;;;;;;;;;;;;;;;;;;;;;;;;;;;;;;;;;;;
;;                                                                          ;;
;;   Copyright (c) 1986-1987 by Richard Weyhrauch.  All rights reserved.    ;;
;;   Copyright (c) 1987-1988 by Fausto Giunchiglia.  All rights reserved.   ;;
;;                                                                          ;;
;;   This software is being provided to you, the LICENSEE, by Richard       ;;
;;   Weyhrauch and Fausto Giunchiglia, the AUTHORS, under certain rights    ;;
;;   and obligations.  By obtaining, using and/or copying this software,    ;;
;;   you indicate that you have read, understood, and will comply with      ;;
;;   the following terms and conditions:                                    ;;
;;                                                                          ;;
;;   THE AUTHORS MAKE NO REPRESENTATIONS OF WARRANTIES, EXPRESS OR          ;;
;;   IMPLIED.  By way of example, but not limitation, THE AUTHORS MAKE      ;;
;;   NO REPRESENTATIONS OR WARRANTIES OF MERCHANTABILITY OF FITNESS FOR     ;;
;;   ANY PARTICULAR PURPOSE OR THAT THE USE OF THE LICENSED SOFTWARE        ;;
;;   COMPONENTS OR DOCUMENTATION WILL NOT INFRINGE ANY PATENTS,             ;;
;;   COPYRIGHTS, TRADEMARKS OR OTHER RIGHTS.                                ;;
;;                                                                          ;;
;;   The AUTHORS shall not be held liable for any direct, indirect or       ;;
;;   consequential damages with respect to any claim by LICENSEE or any     ;;
;;   third party on account of or arising from this Agreement or use of     ;;
;;   this software.  Permission to use, copy, modify and distribute this    ;;
;;   software and its documentation for any purpose and without fee or      ;;
;;   royalty is hereby granted, provided that the above copyright notice    ;;
;;   and disclaimer appears in and on ALL copies of the software and        ;;
;;   documentation, whether original to the AUTHORS or a modified           ;;
;;   version by LICENSEE.                                                   ;;
;;                                                                          ;;
;;   The name of the AUTHORS may not be used in advertising or publicity    ;;
;;   pertaining to distribution of the software without specific, written   ;;
;;   prior permission.  Notice must be given in supporting documentation    ;;
;;   that such distribution is by permission of the AUTHORS.  The AUTHORS   ;;
;;   make no representations about the suitability of this software for     ;;
;;   any purpose.  It is provided "AS IS" without express or implied        ;;
;;   warranty.  Title to copyright to this software and to any associated   ;;
;;   documentation shall at all times remain with the AUTHORS and LICENSEE  ;;
;;   agrees to preserve same.  LICENSEE agrees to place the appropriate     ;;
;;   copyright notice on any such copies.                                   ;;
;;                                                                          ;;
;;;;;;;;;;;;;;;;;;;;;;;;;;;;;;;;;;;;;;;;;;;;;;;;;;;;;;;;;;;;;;;;;;;;;;;;;;;;;;

;*************************************************************************;
;                                                                         ;
;                    "PROOF" MODULE CONFIGURATION FILE                    ;
;                                                                         ;
;*************************************************************************;

(MODULE-INIT        'PROOF)
(MODULE-SET-NAME    'PROOF   "PROOF")
(MODULE-SET-MODE    'PROOF   'COMPILED)

(MODULE-SET-SRCDIR  'PROOF  (PATH-CONCAT (SYS-GET-SRCDIR 'GETFOL) "proof"))
(MODULE-SET-OBJDIR  'PROOF  (SYS-GET-OBJDIR 'GETFOL))
(MODULE-SET-DOCDIR  'PROOF  (PATH-CONCAT (SYS-GET-DOCDIR 'GETFOL) "proof"))

(MODULE-SET-DOCFILE 'PROOF
    (PATH-CONCAT (MODULE-GET-DOCDIR 'PROOF) "inclfile"    "proof.tex"))



;;;   GETHGKM special variables declaration
(MODULE-ADD-FILE   'PROOF   "vproof.cl"       ""             'INTERPRETED)

;;;   Reason
(MODULE-ADD-FILE    'PROOF  "reason.hgk"      "reasonh"      'COMPILED)

;;;   Proof-steps
(MODULE-ADD-FILE    'PROOF  "pline.hgk"       "plineh"       'COMPILED)
(MODULE-ADD-FILE    'PROOF  "pline.fol"       "plinef"       'COMPILED)
(MODULE-ADD-FILE    'PROOF  "pline.rp"        "pliner"       'COMPILED)

;;;   Facts
(MODULE-ADD-FILE    'PROOF  "fact.hgk"        "facth"        'COMPILED)
(MODULE-ADD-FILE    'PROOF  "fact.fol"        "factf"        'COMPILED)
(MODULE-ADD-FILE    'PROOF  "fact.rp"         "factr"        'COMPILED)

;;;   Axioms
(MODULE-ADD-FILE    'PROOF  "axiom.hgk"       "axiomh"       'COMPILED)
(MODULE-ADD-FILE    'PROOF  "axiom.fol"       "axiomf"       'COMPILED)
(MODULE-ADD-FILE    'PROOF  "axiom.rp"        "axiomr"       'COMPILED)

;;;   Show & probe proofs.
(MODULE-ADD-FILE    'PROOF  "showprf.rp"      "showprfr"     'COMPILED)
(MODULE-ADD-FILE    'PROOF  "probprf.fol"     "probprf"      'COMPILED)

;;;   Proofs
(MODULE-ADD-FILE    'PROOF  "proof.hgk"       "proofh"       'COMPILED)
(MODULE-ADD-FILE    'PROOF  "proof.fol"       "prooff"       'COMPILED)
(MODULE-ADD-FILE    'PROOF  "proof.rp"        "proofr"       'COMPILED)

;;;   Command files
(MODULE-ADD-FILE    'PROOF  "cmdproof.fol"    "cmdprff"      'COMPILED)
(MODULE-ADD-FILE    'PROOF  "cmdlabel.fol"    "cmdlabef"     'COMPILED)

;;;   Initialization files
(MODULE-ADD-FILE    'PROOF  "iproof.fol"      ""             'INTERPRETED)

 	;;;;;;;;;;;;;;;;;;;;;;;;;;;;;;;;;;;;;;;;;;;;;;;;;;;;;;;;;;;;;;;;;;;;;;;;;;;;;;
;;
;; FOL version 2.001
;; This file is an FOL source file: nd.cfg
;; Date: Wed Oct 20 10:46:59 MET 1993
;;
;;;;;;;;;;;;;;;;;;;;;;;;;;;;;;;;;;;;;;;;;;;;;;;;;;;;;;;;;;;;;;;;;;;;;;;;;;;;;;
;;                                                                          ;;
;;   Copyright (c) 1986-1987 by Richard Weyhrauch.  All rights reserved.    ;;
;;   Copyright (c) 1987-1988 by Fausto Giunchiglia.  All rights reserved.   ;;
;;                                                                          ;;
;;   This software is being provided to you, the LICENSEE, by Richard       ;;
;;   Weyhrauch and Fausto Giunchiglia, the AUTHORS, under certain rights    ;;
;;   and obligations.  By obtaining, using and/or copying this software,    ;;
;;   you indicate that you have read, understood, and will comply with      ;;
;;   the following terms and conditions:                                    ;;
;;                                                                          ;;
;;   THE AUTHORS MAKE NO REPRESENTATIONS OF WARRANTIES, EXPRESS OR          ;;
;;   IMPLIED.  By way of example, but not limitation, THE AUTHORS MAKE      ;;
;;   NO REPRESENTATIONS OR WARRANTIES OF MERCHANTABILITY OF FITNESS FOR     ;;
;;   ANY PARTICULAR PURPOSE OR THAT THE USE OF THE LICENSED SOFTWARE        ;;
;;   COMPONENTS OR DOCUMENTATION WILL NOT INFRINGE ANY PATENTS,             ;;
;;   COPYRIGHTS, TRADEMARKS OR OTHER RIGHTS.                                ;;
;;                                                                          ;;
;;   The AUTHORS shall not be held liable for any direct, indirect or       ;;
;;   consequential damages with respect to any claim by LICENSEE or any     ;;
;;   third party on account of or arising from this Agreement or use of     ;;
;;   this software.  Permission to use, copy, modify and distribute this    ;;
;;   software and its documentation for any purpose and without fee or      ;;
;;   royalty is hereby granted, provided that the above copyright notice    ;;
;;   and disclaimer appears in and on ALL copies of the software and        ;;
;;   documentation, whether original to the AUTHORS or a modified           ;;
;;   version by LICENSEE.                                                   ;;
;;                                                                          ;;
;;   The name of the AUTHORS may not be used in advertising or publicity    ;;
;;   pertaining to distribution of the software without specific, written   ;;
;;   prior permission.  Notice must be given in supporting documentation    ;;
;;   that such distribution is by permission of the AUTHORS.  The AUTHORS   ;;
;;   make no representations about the suitability of this software for     ;;
;;   any purpose.  It is provided "AS IS" without express or implied        ;;
;;   warranty.  Title to copyright to this software and to any associated   ;;
;;   documentation shall at all times remain with the AUTHORS and LICENSEE  ;;
;;   agrees to preserve same.  LICENSEE agrees to place the appropriate     ;;
;;   copyright notice on any such copies.                                   ;;
;;                                                                          ;;
;;;;;;;;;;;;;;;;;;;;;;;;;;;;;;;;;;;;;;;;;;;;;;;;;;;;;;;;;;;;;;;;;;;;;;;;;;;;;;

;*************************************************************************;
;                                                                         ;
;                    "ND" MODULE CONFIGURATION FILE                       ;
;                                                                         ;
;*************************************************************************;

(MODULE-INIT        'ND)
(MODULE-SET-NAME    'ND  "ND")
(MODULE-SET-MODE    'ND  'COMPILED)

(MODULE-SET-SRCDIR  'ND  (PATH-CONCAT (SYS-GET-SRCDIR 'GETFOL) "nd"))
(MODULE-SET-OBJDIR  'ND  (SYS-GET-OBJDIR 'GETFOL))
(MODULE-SET-DOCDIR  'ND
     (PATH-CONCAT (SYS-GET-DOCDIR 'GETFOL) "inclfile"  "nd"))
(MODULE-SET-DOCFILE 'ND
     (PATH-CONCAT (MODULE-GET-DOCDIR 'ND)  "inclfile"  "nd.tex"))


;;;   GETHGKM special variables declaration
(MODULE-ADD-FILE 'ND "vnd.cl"       ""         'INTERPRETED)              

;;;   Inference rules for nd
(MODULE-ADD-FILE 'ND "fapfnd.hgk"   "fapfndh"  'COMPILED)
(MODULE-ADD-FILE 'ND "fapfnd.fol"   "fapfndf"  'COMPILED)

;;;   Command files
(MODULE-ADD-FILE 'ND "cmdnd.fol"    "cmdndf"   'COMPILED)

;;;   Initialization files
(MODULE-ADD-FILE 'ND "ind.fol"      ""         'INTERPRETED)

	;;;;;;;;;;;;;;;;;;;;;;;;;;;;;;;;;;;;;;;;;;;;;;;;;;;;;;;;;;;;;;;;;;;;;;;;;;;;;;
;;
;; GETFOL version 1.001
;; This file is a GETFOL source file: rules.cfg
;; Date: Thu Nov 11 14:42:52 MET 1993
;;
;;;;;;;;;;;;;;;;;;;;;;;;;;;;;;;;;;;;;;;;;;;;;;;;;;;;;;;;;;;;;;;;;;;;;;;;;;;;;;
;;                                                                          ;;
;;   Copyright (c) 1987-1988 by Fausto Giunchiglia.  All rights reserved.   ;;
;;                                                                          ;;
;;   This software is being provided to you, the LICENSEE, by Fausto        ;;
;;   Giunchiglia, the AUTHOR, under certain rights and obligations.         ;;
;;   By obtaining, using and/or copying this software, you indicate that    ;;
;;   you have read, understood, and will comply with the following terms    ;;
;;   and conditions:                                                        ;;
;;                                                                          ;;
;;   THE AUTHOR MAKES NO REPRESENTATIONS OF WARRANTIES, EXPRESS OR          ;;
;;   IMPLIED.  By way of example, but not limitation, THE AUTHOR MAKES      ;;
;;   NO REPRESENTATIONS OR WARRANTIES OF MERCHANTABILITY OF FITNESS FOR     ;;
;;   ANY PARTICULAR PURPOSE OR THAT THE USE OF THE LICENSED SOFTWARE        ;;
;;   COMPONENTS OR DOCUMENTATION WILL NOT INFRINGE ANY PATENTS,             ;;
;;   COPYRIGHTS, TRADEMARKS OR OTHER RIGHTS.                                ;;
;;                                                                          ;;
;;   The AUTHOR shall not be held liable for any direct, indirect or        ;;
;;   consequential damages with respect to any claim by LICENSEE or any     ;;
;;   third party on account of or arising from this Agreement or use of     ;;
;;   this software.  Permission to use, copy, modify and distribute this    ;;
;;   software and its documentation for any purpose and without fee or      ;;
;;   royalty is hereby granted, provided that the above copyright notice    ;;
;;   and disclaimer appears in and on ALL copies of the software and        ;;
;;   documentation, whether original to the AUTHOR or a modified            ;;
;;   version by LICENSEE.                                                   ;;
;;                                                                          ;;
;;   The name of the AUTHOR may not be used in advertising or publicity     ;;
;;   pertaining to distribution of the software without specific, written   ;;
;;   prior permission.  Notice must be given in supporting documentation    ;;
;;   that such distribution is by permission of the AUTHOR.  The AUTHOR     ;;
;;   makes no representations about the suitability of this software for    ;;
;;   any purpose.  It is provided "AS IS" without express or implied        ;;
;;   warranty.  Title to copyright to this software and to any associated   ;;
;;   documentation shall at all times remain with the AUTHOR and LICENSEE   ;;
;;   agrees to preserve same.  LICENSEE agrees to place the appropriate     ;;
;;   copyright notice on any such copies.                                   ;;
;;                                                                          ;;
;;;;;;;;;;;;;;;;;;;;;;;;;;;;;;;;;;;;;;;;;;;;;;;;;;;;;;;;;;;;;;;;;;;;;;;;;;;;;;

;*************************************************************************;
;                                                                         ;
;                    "RULES" MODULE CONFIGURATION FILE                    ;
;                                                                         ;
;*************************************************************************;

(MODULE-INIT       'RULES)
(MODULE-SET-NAME   'RULES "RULES")
(MODULE-SET-MODE   'RULES 'COMPILED)

(MODULE-SET-SRCDIR 'RULES (PATH-CONCAT (SYS-GET-SRCDIR 'GETFOL) "rules"))
(MODULE-SET-OBJDIR 'RULES (SYS-GET-OBJDIR 'GETFOL))
(MODULE-SET-DOCDIR 'RULES (PATH-CONCAT (SYS-GET-DOCDIR 'GETFOL) "rules"))

(MODULE-SET-DOCFILE 'RULES
  (PATH-CONCAT (MODULE-GET-DOCDIR 'RULES) "rules.tex"))

;;;
;;;                !!!FILES WITHOUT DEFSUB!!!
;;;

;;;  If rules
(MODULE-ADD-FILE 'RULES "fapfif.fol"   "fapfiff"  'COMPILED)
(MODULE-ADD-FILE 'RULES "cmdif.fol"    "cmdiff"   'COMPILED)

;;;  Structural rules
(MODULE-ADD-FILE 'RULES "fapfstr.fol"  "fapfstrf" 'COMPILED)
(MODULE-ADD-FILE 'RULES "cmdstr.fol"   "cmdstrf"  'COMPILED)

;;;  Initialization files
(MODULE-ADD-FILE 'RULES "iif.fol"      ""         'INTERPRETED)
(MODULE-ADD-FILE 'RULES "istr.fol"     ""         'INTERPRETED)

 	% introduction to the deciders
\newpage
\section{Introduction}

The distribution package allows the installation of the following three
applications:
%
\begin{itemize}
	\item
		the {\tt HGKM} interpreter \cite{giunchiglia35}, that is a LISP-like
		language acting as the implementation language of both {\tt FOL}
		and {\tt GETFOL}.
	\item
		the {\tt GETFOL} system \cite{giunchiglia12,giunchiglia29}.
	\item
		the {\tt AGETFOL} system ({\tt GETFOL} with {\em abstraction}
		\cite{giunchiglia7}.
\end{itemize}

From now on, we will write ``{\tt the system}" to mean {\tt HGKM},
{\tt GETFOL} and {\tt AGETFOL}.
The distribution package is provided with a set of general purpose
configuration and installation facilities which allow you to install your
favourite application on your machine.

In order to install {\tt the system} read section~\ref{requirements}
(``{\it Minimal System Requirements}'') and verify that your machine
has all the required features.
Then read section~\ref{instproc} (``{\it The installation procedure}'').
This section gives the instructions to follow in order to install
{\tt the system} on your machine.

If you are an expert you might be interested also in the features
of the installation and configuration facilities.
Section~\ref{sysmod} (``{\it Systems and Modules}'') introduces
to the data structures ({\it systems} and {\it modules}) devised
to store and/or modify the configuration of the applications.


% user commands for deciders
;;;;;;;;;;;;;;;;;;;;;;;;;;;;;;;;;;;;;;;;;;;;;;;;;;;;;;;;;;;;;;;;;;;;;;;;;;;;;;
;;
;; GETFOL version 2.001
;; Date: Wed Nov 10 15:08:45 MET 1993
;;
;; This    FOL file was modified in GETFOL version 2.001
;;
;;;;;;;;;;;;;;;;;;;;;;;;;;;;;;;;;;;;;;;;;;;;;;;;;;;;;;;;;;;;;;;;;;;;;;;;;;;;;;
;;
;; FOL version 2.001
;; This file is an FOL source file: decide.cfg
;; Date: Wed Oct 20 10:43:51 MET 1993
;;
;;;;;;;;;;;;;;;;;;;;;;;;;;;;;;;;;;;;;;;;;;;;;;;;;;;;;;;;;;;;;;;;;;;;;;;;;;;;;;
;;                                                                          ;;
;;   Copyright (c) 1986-1987 by Richard Weyhrauch.  All rights reserved.    ;;
;;   Copyright (c) 1987-1988 by Fausto Giunchiglia.  All rights reserved.   ;;
;;                                                                          ;;
;;   This software is being provided to you, the LICENSEE, by Richard       ;;
;;   Weyhrauch and Fausto Giunchiglia, the AUTHORS, under certain rights    ;;
;;   and obligations.  By obtaining, using and/or copying this software,    ;;
;;   you indicate that you have read, understood, and will comply with      ;;
;;   the following terms and conditions:                                    ;;
;;                                                                          ;;
;;   THE AUTHORS MAKE NO REPRESENTATIONS OF WARRANTIES, EXPRESS OR          ;;
;;   IMPLIED.  By way of example, but not limitation, THE AUTHORS MAKE      ;;
;;   NO REPRESENTATIONS OR WARRANTIES OF MERCHANTABILITY OF FITNESS FOR     ;;
;;   ANY PARTICULAR PURPOSE OR THAT THE USE OF THE LICENSED SOFTWARE        ;;
;;   COMPONENTS OR DOCUMENTATION WILL NOT INFRINGE ANY PATENTS,             ;;
;;   COPYRIGHTS, TRADEMARKS OR OTHER RIGHTS.                                ;;
;;                                                                          ;;
;;   The AUTHORS shall not be held liable for any direct, indirect or       ;;
;;   consequential damages with respect to any claim by LICENSEE or any     ;;
;;   third party on account of or arising from this Agreement or use of     ;;
;;   this software.  Permission to use, copy, modify and distribute this    ;;
;;   software and its documentation for any purpose and without fee or      ;;
;;   royalty is hereby granted, provided that the above copyright notice    ;;
;;   and disclaimer appears in and on ALL copies of the software and        ;;
;;   documentation, whether original to the AUTHORS or a modified           ;;
;;   version by LICENSEE.                                                   ;;
;;                                                                          ;;
;;   The name of the AUTHORS may not be used in advertising or publicity    ;;
;;   pertaining to distribution of the software without specific, written   ;;
;;   prior permission.  Notice must be given in supporting documentation    ;;
;;   that such distribution is by permission of the AUTHORS.  The AUTHORS   ;;
;;   make no representations about the suitability of this software for     ;;
;;   any purpose.  It is provided "AS IS" without express or implied        ;;
;;   warranty.  Title to copyright to this software and to any associated   ;;
;;   documentation shall at all times remain with the AUTHORS and LICENSEE  ;;
;;   agrees to preserve same.  LICENSEE agrees to place the appropriate     ;;
;;   copyright notice on any such copies.                                   ;;
;;                                                                          ;;
;;;;;;;;;;;;;;;;;;;;;;;;;;;;;;;;;;;;;;;;;;;;;;;;;;;;;;;;;;;;;;;;;;;;;;;;;;;;;;

;*************************************************************************;
;                                                                         ;
;                    "DECIDE" MODULE CONFIGURATION FILE                   ;
;                                                                         ;
;*************************************************************************;

(MODULE-INIT        'DECIDE)
(MODULE-SET-NAME    'DECIDE  "DECIDE")
(MODULE-SET-MODE    'DECIDE  'COMPILED)

(MODULE-SET-SRCDIR  'DECIDE (PATH-CONCAT (SYS-GET-SRCDIR 'GETFOL) "decide"))
(MODULE-SET-OBJDIR  'DECIDE (SYS-GET-OBJDIR 'GETFOL))
(MODULE-SET-DOCDIR  'DECIDE (PATH-CONCAT (SYS-GET-DOCDIR 'GETFOL) "decide"))
(MODULE-SET-DOCFILE 'DECIDE
        (PATH-CONCAT (MODULE-GET-DOCDIR 'GETFOL) "inclfile" "deciders.tex"))


;;;   GETHGKM special variables declaration
(MODULE-ADD-FILE 'DECIDE "vdecide.cl"   ""         'INTERPRETED)

;;;   Low level functions for deciders
(MODULE-ADD-FILE 'DECIDE "orelse.hgk"   "orelseh"  'COMPILED)
(MODULE-ADD-FILE 'DECIDE "decide.hgk"   "decideh"  'COMPILED)
(MODULE-ADD-FILE 'DECIDE "ptaut.hgk"    "ptauth"   'COMPILED)
(MODULE-ADD-FILE 'DECIDE "ptauteq.hgk"  "ptauteqh" 'COMPILED)
(MODULE-ADD-FILE 'DECIDE "tautren.hgk"  "tautrenh" 'COMPILED)
(MODULE-ADD-FILE 'DECIDE "phexp.hgk"    "phexph"   'COMPILED)
(MODULE-ADD-FILE 'DECIDE "reduce.hgk"   "reduceh"  'COMPILED)

;;;   High level functions for deciders
(MODULE-ADD-FILE 'DECIDE "ptaut.fol"    "ptautf"   'COMPILED)
(MODULE-ADD-FILE 'DECIDE "ptauteq.fol"  "ptauteqf" 'COMPILED)
(MODULE-ADD-FILE 'DECIDE "phexp.fol"    "phexpf"   'COMPILED)
(MODULE-ADD-FILE 'DECIDE "reduce.fol"   "reducef"  'COMPILED)
(MODULE-ADD-FILE 'DECIDE "assert.fol"   "assertf"  'COMPILED)
(MODULE-ADD-FILE 'DECIDE "peval.fol"    "pevalf"   'COMPILED)
(MODULE-ADD-FILE 'DECIDE "termif.fol"   "termiff"  'COMPILED)
(MODULE-ADD-FILE 'DECIDE "tautren.fol"  "tautrenf" 'COMPILED)
(MODULE-ADD-FILE 'DECIDE "skolem.fol"   "skolemf"  'COMPILED)
(MODULE-ADD-FILE 'DECIDE "nnf.fol"      "nnff"     'COMPILED)

;;;   Command files
(MODULE-ADD-FILE 'DECIDE "cmdecide.fol" "cmdecidf" 'COMPILED)

;;;   Initialization files
(MODULE-ADD-FILE 'DECIDE "idecide.fol"  ""         'INTERPRETED)

\gfcommand{monad}{first order decider for monadic formulas}
\index{monad}

\gfsyntax{
  monad \ARG{wff} \OPT{by \ARG{fact1} \ARG{fact2} \SEQ};
}

\gfdescription{
It tries to establish whether the input formula is deducible from the
specified facts by using {\tt nnf}, {\tt reduce}, {\tt phexp} and finally
{\tt ptaut}.
}

\gfrecap{
It tries to establish whether the input formula is deducible from the
specified facts by using nnf, reduce, phexp and finally ptaut.
}

\gfexample+
   ***** declare predconst P 1;
   ***** declare funconst f 3;
   ***** declare indvar x y z;
   ***** declare indpar a b;
   
   *****comment | *** MONAD EXAMPLES *** |
   ***** monad forall x. exists y. (P(x) imp P(y));
   1   forall x. exists y. (P(x) imp P(y))
   ***** monad exists y. forall x. (P(x) imp P(y));
   2   exists y. forall x. (P(x) imp P(y))
   ***** monad exists y. forall x. ((P(x) imp P(y)) or P(x));
   3   exists y. forall x. ((P(x) imp P(y)) or P(x))
   
   ***** monad forall x. exists y.
    wffif P(trmif P(y) then x else y)
     then P(trmif P(y) 
             then trmif P(y) 
                   then x 
                   else trmif P(y) 
                         then x 
                         else y 
             else y) 
          or TRUE
     else P(y) or TRUE;
   4   forall x. exists y. 
       (wffif P(trmif P(y) then x else y) 
         then (P(trmif P(y) 
                 then 
                 (trmif P(y) 
                   then x 
                   else (trmif P(y) then x else y)) 
               else y) or TRUE) else (P(y) or TRUE))
   ***** monad forall x. exists y. wffif P(x) then P(y) else not P(y);
   5   forall x. exists y. (wffif P(x) then P(y) else (not P(y)))  
   ***** monad forall y. exists x. (P(f(a,b,x)) or not P(f(a,b,y)));
   6   forall y. exists x. (P(f(a,b,x)) or (not P(f(a,b,y))))
   ***** monad exists z. forall y. exists x. (P(f(z,b,x)) or not P(f(x,b,z)));
   7   exists z. forall y. exists x. (P(f(z,b,x)) or (not P(f(x,b,z)))) 
   ***** monad exists z. forall y. exists x. (P(f(z,b,x)) or not P(f(x,b,y)));
   7   exists z. forall y. exists x. (P(f(z,b,x)) or (not P(f(x,b,y)))) 
+
   
\gfnotes{
   The name ``{\tt monad}" is due to the fact that the monadic predicate
calculus is subset of the class of formulas decided by such a decider.
}   

\gfcommand{monadeq}{first order decider for monadic formulas with equality}
\index{monadeq}

\gfsyntax{
  monadeq \ARG{wff} \OPT{by \ARG{fact1} \ARG{fact2} \SEQ};
}


\gfdescription{
It tries to establish whether the input formula is deducible from the
specified facts by using {\tt nnf}, {\tt reduce}, {\tt phexp} and finally
{\tt ptauteq}.
}

\gfrecap{
It tries to establish whether the input formula is deducible from the
specified facts by using nnf, reduce, phexp and finally ptauteq.
}

\gfexample+
   ***** declare predconst P 1;
   ***** declare funconst f 3;
   ***** declare indvar x y z;
   ***** declare indpar a b;

   *****comment | *** MONADEQ EXAMPLES *** |
   ***** monadeq forall x. exists y. (x=y);
   1   forall x. exists y. (x=y)
   ***** monadeq forall x.  forall y. (x=y imp (P(x) imp P(y)));
   2   forall x.  forall y. (x=y imp (P(x) imp P(y)))
   ***** monad (x=y imp (P(x) imp P(y))) by 2;
   3   (x=y imp (P(x) imp P(y)))  (2)
+

\gfcommand{ptaut}{tautological decider}
\index{ptaut}
\label{sec-decproc}

\gfsyntax{
  ptaut \ARG{wff} \OPT{by \ARG{fact1} \ARG{fact2} \SEQ};
}

\gfdescription{
It decides the quantifier-free formulas provable only by means of the
propositional deductive machinery.
}

\gfrecap{
It decides the quantifier-free formulas provable only by means of the
propositional deductive machinery.
}

\gfexample+
   ***** declare sentconst A B;
   ***** ptaut (A imp (B imp A));
   1   A imp (B imp A)
   ***** assume A B;
   2   A     (2)
   3   B     (3)
   ***** ptaut (A imp (B imp A)) by 2 3;
   4   A imp (B imp A)  (2 3)
   ***** ptaut (A and B) by 2 3;
   5   A and B  (2 3)
   ***** ptaut A by 3;
   PTAUT couldn't prove that A
   is a logical consequence of facts.
   
   ***** declare predconst P 1;
   ***** declare indconst c;
   ***** declare indvar x;

   ***** ptaut (A imp (P(c) imp A));
   6   A imp (P(c) imp A)     
   ***** ptaut (A imp (forall x.P(x) imp A));
   The formula passed to PTAUT is not propositional !
+   

\gfcommand{taut}{tautological first order decider}
\index{taut}

\gfsyntax{
  taut \ARG{wff} \OPT{by \ARG{fact1} \ARG{fact2} \SEQ};
}

\gfdescription{
  The class of formulae decided by {\tt taut} is the set of first order
  formulas provable using only the introduction and elimination rules for 
  the sentential connectives plus the following rule ({\em congruence-rule}):\\
  %
  \[
      \fraz{\forall x A(x)}{\forall y A(y)}
  \]
}

\gfrecap{
The class of formulae decided by `taut' is the set of first order formulas
provable using only the introduction and elimination rules for the 
sentencial connectives plus the following rule (congruence rule)::
              +------------------------------------+
              | forall x. A(x) imp forall y. A(y)  |
              +------------------------------------+
}

\gfexample+
   ***** declare sentconst A;
   ***** declare predconst P 1;
   ***** declare indconst c;

   ***** taut (A imp (P(c) imp A));
   1   A imp (P(c) imp A)

   ***** declare indvar x y [S1];
   ***** declare indvar z [S2];

   ***** taut forall x.P(x) iff forall y.P(y);
   2   forall x.P(x) iff forall y.P(y);
   ***** taut forall x.P(x) iff forall z.P(z);
   TAUT couldn't prove that forall x. P(x) iff forall z. P(z)
   is a tautology.
+

\gfcommand{tauteq}{tautological decider with equality}
\index{tauteq}

\gfsyntax{
  tauteq \ARG{wff} \OPT{by \ARG{fact1} \ARG{fact2} \SEQ};
}

\gfdescription{
  The class of formulae decided by {\tt tauteq} is the set of formulas provable
  using:
  \begin{itemize}
  \item the introduction and elimination rules for the sentential connectives;
  \item the {\em congruence-rule};
  \item the following axioms schemata for equality:
  $$
  \begin{array}{l}
    x=x\\
    (x=y\ \imp\ y=x)\\
    ((x=y\ \con\ y=z)\ \imp\ x=z)\\
    ((x_1=y_1\ \con \ldots \con\ x_n=y_n)\ \imp\ 
    (P(x_1, \ldots \ ,x_n)\ \liff\ P(y_1, \ldots \ ,y_n)))
  \end{array}
  $$
  %
  corresponding to {\em reflexivity, symmetry, transitivity} and {\em substitution 
  into predicates}.
  \end{itemize}
}

\gfrecap{
The class of formulae decided by {\tt tauteq} is the set of formulas provable
using:
* the introduction and elimination rules for the sentential connectives;
* the ``congruence-rule'';
* the following axioms schemata for equality:
     +-------------------------------------------------------------------------+
     |   x = x                                                                 |
     |   x = y imp y = x                                                       |
     |   (x = y and y = z) imp x = z                                           |
     |   (x1 =y1 and ... and xN = yN) imp (P(x1, ..., xN) iff P(y1, ..., yN))  |
     +-------------------------------------------------------------------------+
  corresponding to ``reflexivity'', ``symmetry'', ``transitivity'' and
  ``substitution into predicates''.
}

\gfexample+
   ***** declare predconst P 1;
   ***** declare funconst f 1;
   ***** declare indvar x y;
   ***** declare indvar z;
   ***** tauteq x=x;
   1   x=x
   ***** tauteq x=y imp y=x; 
   2   x=y imp y=x
   ***** tauteq ((x=y and y=z) imp x=z);
   3   (x=y and y=z) imp x=z
   ***** tauteq (x=y imp (P(x) or not P(y)));
   4   x=y imp (P(x) or not P(y))
   ***** tauteq (f(x)=f(y) imp (P(f(x)) iff P(f(y))));
   5   f(x)=f(y) imp (P(f(x)) iff P(f(y)))
   ***** tauteq x=y imp f(x)=f(y);
   TAUTEQ couldn't prove that (x = y) imp (f(x) = f(y))
   is a tautology.
+

 	% introduction to the semantic simplification's section
\newpage
\section{Semantic simplification}
\label{sec-comp}

The subsections \ref{sec-ss-intro}, \ref{sec-ss-model} and \ref{sec-ss-repr} 
of this section have been taken from \cite{rww3}.


\subsection{Introduction}
\label{sec-ss-intro}

{\GF} is intended to express a variety of methods of human reasoning.
Though the word "reasoning" usually connotes a logical deductive process of 
using facts and assertions to obtain conclusions, much of human intelligence 
relies more upon observation than upon deduction.
We look at a book. The book is seen to be "green", as an immediate observation,
not as a deduction involving, say, analysis of wavelengths of light and 
sensory receptors  in the eye. Similarly, humans cross streets without 
conscious analysis of the traffic flow, add numbers without resorting to basic
set theory, and play chess without considering each move in terms of the 
geometry of the board. 

Any system which hopes to express a variety of reasoning processes, therefore 
needs a method of doing purely computational tasks.
In {\GF}, the {\bf semantic interpretation mechanism}, which provides this 
ability, consists of two parts:
\begin{itemize}
\item {\GF}'s {\bf semantic attachment mechanism} permits the user to define a
      ``correspondence'' between the various constants (function symbols,
      predicate constants, individual constants) of the language and
      corresponding objects of the programming language {\HG}.
\item facts about the {\HG} structure can be used directly in the proof
      via the {\bf semantic simplification mechanism}, eliminating the 
      necessity of a possibly complicated deduction.
\end{itemize}
For example, obvious attachments to the function symbol $+$ and to the 
individual constants $17$, $34$, $51$ would allow to conclude $17+34=51$ in 
one step, instead of computing $34$ successors of $17$.
In order to explain this more clearly we first give an informal account of the 
technical details.

\subsection{``Intended'' and ``computational'' models}
\label{sec-ss-model}

The declarations made by a {\GF} user specify a first order language 
$L=\langle P,F,C\rangle$, where $P$ is the list of {\predconst}s, $F$ the list 
of {\funconst}s, and $C$ the list of {\indconst}s (see section \ref{sec-decl}).

A model for such a language is a structure $M=\langle D,P',F',C'\rangle$ where
$D$ is a set and $P'$,$F'$ and $C'$ are lists of predicates over $D$, functions 
over $D$, and individuals of $D$ such that the arities of the symbols in $P$ and 
$F$ match the arities of the predicates and functions at the correspondent 
positions in $P'$ and $F'$.
The idea here is that the language $L$ is used for making statements about 
structures such as $M$ (what we call {\bf ``intended'' or ``standard'' model}). 
In particular, when the user writes down a theory in {\GF}, he generally has 
in mind some particular model for his language, and the axioms of his theory 
are intended to express the properties of this particular model.

The fact that {\GF} is really a {\HG} program running in a LISP 
environment, inspires the following idea: some parts of a model for a {\GF} 
language can often be expressed computationally in the sense that the elements 
of $D$ can be represented by s-expressions, and the predicates and functions 
on $D$ can be represented by {\HG} functions and predicates.
It should then be possible to use the computational representation to aid 
{\GF} deductions concerning the model.
For example, suppose the theory we are interested in, is first order number 
theory, and the model that we have in mind is the set of natural numbers 
together with the operations of successor, addition and multiplication.
The numerals have natural representations as {\HG} numbers, and the 
functions in question have {\tt PLUS1}, {\tt PLUS}, {\tt TIMES} as their {\HG}
counterparts.
As mentioned above it should then be possible to use the computational 
representation to provide swift deductions of such statements as $25+37=52$.

The semantic attachment mechanism in {\GF} allows the user to set up these 
computational representations of his subject matter, and the semantic
interpretation mechanism allows to use these representations to aid deduction 
in {\GF}.

With the above overview in mind, let us proceed to the details.

Given a language $L=\langle P,F,C\rangle$ and a model 
$M=\langle D,P',F',C'\rangle$, we define an interpretation function $I$.
For each {\term} $t$ of $L$ in which no free variable occurs, $I(t)$ is the 
individual in $D$ which $t$ denotes.
In particular we define the interpretation of an {\indconst} $c$ to be the 
individual $c'$ in $D$, and where $f$ is a {\funconst}, and the interpretation 
of {\term} $t_1,\ldots,t_n$ are defined, we inductively define the 
interpretation of the {\term} 
$f(t_1,\ldots,t_n)$ to be $f'(I(t_1),\ldots,I(t_n))$.
We may extend the interpretation function to formulas (again without free 
variables) over $L$ by defining $I(w)$ to be the object {\tt TRUE} exactly 
when the formula $w$ is true of the model (for a technical definition 
see \cite{kleene2}).

When $f'$ is the function in a model corresponding to the {\funconst} $f$ in 
$L$, we will also say that $f'$ is the interpretation of $f$, and similarly 
for predconsts.

Now we define a {\bf computational model} to be an object
$K=\langle D',P'',F'',C''\rangle$, where it is understood that $D'$ is a set
of s-expressions, and $P''$, $F''$ and $C''$ are lists of {\HG} predicates,
functions and s-expressions respectively, with the appropriate restrictions on 
arities.

From the extensional point of view, a computational model is for a language
just like a set-theoretic model for a language, except that we do not require
that the functions and predicates concerned be total; that is functions and
predicates may be undefined (non-terminating) for some elements
of $D'$.

We define an {\bf attachment map} $att$ from terms and formulas of $L$ into
$K$ in a manner exactly analogous to the definition of $I$ given above.

We have one last map to worry about, the map {\bf $rep$} which gives, for each
object in the domain $D'$ of the computational model $K$, the object it
represents in the domain $D$ of the model $M$.

Now we may define precisely the meaning of attachments made in the {\GF}
system: the attachment of an {\indconst} $c$ to an s-expression $c''$
signifies that $c$ and $c''$ represent the same object in the model, that is
to say, $I(c)=rep(c'')$.
Similarly, the attachment of a {\funconst} $f$ to a {\HG} function $f''$
signifies that the result of applying $f''$ to an s-expression $c''$ which
represents an individual $c'$ in the model, is a s-expression which represents
the individual $f'(c')$ in the model.
The analogous statements hold for attachments to {\predconst}s.

The above conditions are equivalent to the statement that the 
diagram in figure \ref{fig-ss} commutes.

\begin{figure}[htb]
\begin{center}
\setlength{\unitlength}{0.0125in}%
\begin{picture}(288,260)(92,540)
\thicklines
\put(340,580){\oval(80,80)}
\put(160,580){\oval(80,80)}
\put(160,760){\oval(80,80)}
\put(200,580){\vector( 1, 0){100}}
\put(194,726){\vector( 1,-1){112}}
\put(160,720){\vector( 0,-1){100}}
\put(340,559){\makebox(0,0)[b]{\raisebox{0pt}[0pt][0pt]{\twlrm model}}}
\put(340,577){\makebox(0,0)[b]{\raisebox{0pt}[0pt][0pt]{\twlrm of intended}}}
\put(340,595){\makebox(0,0)[b]{\raisebox{0pt}[0pt][0pt]{\twlrm Domain }}}
\put(160,562){\makebox(0,0)[b]{\raisebox{0pt}[0pt][0pt]{\twlrm sexpr}}}
\put(160,580){\makebox(0,0)[b]{\raisebox{0pt}[0pt][0pt]{\twlrm {\HG}}}}
\put(160,752){\makebox(0,0)[b]{\raisebox{0pt}[0pt][0pt]{\twlrm Terms}}}
\put(160,770){\makebox(0,0)[b]{\raisebox{0pt}[0pt][0pt]{\twlrm {\GF}}}}
\put(250,560){\makebox(0,0)[b]{\raisebox{0pt}[0pt][0pt]{\twlrm Representation}}}
\put(280,660){\makebox(0,0)[b]{\raisebox{0pt}[0pt][0pt]{\twlrm I}}}
\put(120,660){\makebox(0,0)[b]{\raisebox{0pt}[0pt][0pt]{\twlrm attachment}}}
\end{picture}
\end{center}
\label{fig-ss}
\caption{intended model - computational model mappings}
\end{figure}


\subsection{Multiple representation functions}
\label{sec-ss-repr} 

The semantic attachment mechanism allows several representation of the model 
by {\HG} s-expressions to be in force at the same time.
We will seek to motivate this aspect of the semantic attachment mechanism 
by means of an example: consider a theory of chess with includes a general 
theory of lists as a subtheory (this subtheory would be applied in arguments 
about lists of pieces, lists of game positions and so on).
The intended model of such a theory includes at least two kinds of objects: 
chess positions and lists.
Lists and positions form disjoint domains in the model, though it may be 
possible to build lists of chess position.
If we are going to build a computational representation of this model, we will
need to represent positions and lists by s-expressions in such a way that no 
s-expression represents both a list and a position.
The natural representation of a chess position as an s-expression is as a 
list of eight lists, each of which is a list of eight piece names (one of 
which is "empty" or some such), and the natural representation of lists as 
s-expressions is the direct representation as {\HG} lists.
This representation scheme cannot be used, since it will not be possible to 
decide whether a given list of eight lists of eight piece names represents a 
chess board or a list of list of pieces. 
That is to say, the map $rep$ will not be well defined. 
It is of course not hard to solve this problem by the use of some slightly 
fancier coding, but a general solution to the problem of disambiguating 
computational representations is available.
Suppose that the intended model of a {\GF} theory $T$ includes the disjoint 
domains $D_1,\ldots,D_n$, and suppose further that we have a different coding 
function for each of these domains.
That is we have $n$ different {\bf representation functions} $rep_i$ which map 
the domain of s-expressions into domain of the model, with the property that 
the range of $rep_i$ is a subset of $D_i$.
Then it is possible that a single s-expression codes two different objects 
$d_i$, $d_j$ in the model, but as long as we know what coding function $rep_i$
to apply, there is no ambiguity. 


Then the definition of the $att$ map may be extended to take account of the 
possibility of multiple representations in the following way: the domain of 
the $att$ map will still consist of the set of {\GF} terms and formulas, but
its range will now lie in the set of pairs of the form $\langle$ representation
function, s-expression $\rangle$.

The soundness condition for the $att$ map is now that, when 
$att(t)=\langle rep, c'' \rangle$, we have $rep(c'')=I(t)$.
In order to specify this new more complicated $att$ map, the user of the {\GF}
system must give representation information concerning his attachments.

Specifically, each representation function must be given a name and when the 
attachment to an {\indconst} is given, the name of the associate 
representation function must be given as well.
Similarly, when the attachment $f''$ to a {\funconst} $f$ is specified, the 
(names of the) representations of its arguments and of the value it returns 
must be given, and when the attachment to a {\predconst} is specified, the 
representations of its arguments must also be specified.

The significance of specifying that the representations of the arguments and 
value of the attachment $f''$ to a {\funconst} $f$ are 
$R_1,\ldots,R_n$ and $R_{n+1}$ respectively, is that 
$R_{n+1}(f''(c''_1,\ldots,c''_n))=f'(R_1(c''_1),\ldots,R_n(c''_n))$ 
where $f'$ is the interpretation of $f$, whenever $c''_1$,..,$c''_n$ are 
s-expressions in the domains of $R_1$,..,$R_n$.
The same holds for attachments to {\predconst}, mutatis mutandis.
Given the attachments with representation information for individual symbols,
the map $att$ on the domain of terms and formulas is defined inductively in 
the obvious way: if $f$ is attached to $f''$ and the declared representation 
of the arguments of $f''$ are $R_1,\ldots,R_n$ and terms $t_1$,..,$t_n$ have 
attachments with representations $R_1$,..,$R_n$ then 
$att(f(t_1,\ldots,t_n))=f''(att(t_1),\ldots,att(t_n))$.
Under this definition the diagram above commutes for each individual 
representation function.

Note that if the representation of the attachment of any term $t_i$ does not 
match that of its place in the argument list, then 
$f''(att(t_1),\ldots,att(t_n))$ cannot be expected to represent the 
interpretation of $f(t_1,\ldots,t_n)$.
The reason for this is that the correctness of a computation which purports to
represent a mathematical function depends on the representation of the 
arguments of the function as data objects.
For example, no one would expect a floating point multiplication algorithm to 
behave correctly if its arguments were encoded as integers rather than 
floating point numbers.

Finally, note that the attachment map, as well as the s-expressions which 
represent functions, may be partial.
The user is never required to provide an attachment for any {\GF} symbol, nor is
any attachment to a {\funconst} or {\predconst} required to be complete.

The semantic simplification mechanism will use whatever information is 
available and if there will be insufficient information, it will return this 
fact to the user. 




% user commands for semantic simplification
\gfcommand{attach}{semantic attachment}
\index{attach}

\gfsyntax{
  attach \ARG{indconst}  to \ALT dar [ rep ] \ARG{sexpr};\\
  attach \ARG{sentconst} to T \ALT NIL \ALT UNDEF;\\
  attach \ARG{funconst} \ALT \ARG{predconst} to \ARG{atom};\\
  attach \ARG{funconst}  to [ \ARG{rep1}, \SEQ, \ARG{repN} = \ARG{repM} ]
  \ARG{atom};\\
  attach \ARG{predconst} to [ \ARG{rep1}, \SEQ, \ARG{repN} ] \ARG{atom};
}

\gfdescription{
  Defines the attachment for the {\GF} constants.
  \ARG{repI} can be a representation function or an asterisk; if it is an
  asterisk or no representation is specified, then the default representation
  function {\tt UNIVERSALREP} is taken.
  \ARG{indconst}s can be attached either ``one way'' (using {\tt to}) or ``two ways''
  (using {\tt dar}).
  The ``two ways'' attachment tells the semantic interpretation mechanism 
  that whenever \ARG{sexpr} is computed as the {\HG} representation of 
  a term $t$, then the attached {\GF} \ARG{indconst} should be returned as the 
  simplified version of $t$.
  That is, not only \ARG{sexpr} is the {\HG} representation of \ARG{indconst}, but 
  \ARG{indconst} is the preferred {\GF} name of (the intended model value denoted 
  by) the {\HG} object \ARG{sexpr}.
  \ARG{sentconst}s can be attached to the three possible truth values
  corresponding to true, false and undefined\cite{kleene1}.
  This is done by attaching a sentconst to {\tt T}, {\tt NIL}, {\tt UNDEF}
  respectively.
  \ARG{funconst}s and \ARG{predconst}s can be attached to a {\HG} \ARG{atom}
  \cite{giunchiglia35}.
  \ARG{atom} will be used as the identifier of a {\HG} function, whose number of 
  arguments is supposed to match the arity of the {\GF} symbol.
}

\gfrecap{
Defines the attachment for the GETFOL constants.
`repI' can be a representation function or an asterisk; if it is an
asterisk or no representation is specified, then the default representation
function `UNIVERSALREP' is taken.
`indconst's can be attached either ``one way'' (using `to') or ``two ways''
(using `dar').
The ``two ways'' attachment tells the semantic interpretation mechanism 
that whenever `sexpr' is computed as the HGKM representation of 
a term `t', then the attached GETFOL `indconst' should be returned as the 
simplified version of `t'.
That is, not only `sexpr' is the HGKM representation of `indconst', but 
`indconst' is the preferred GETFOL name of (the intended model value denoted 
by) the HGKM object `sexpr'.
`sentconst's can be attached to the three possible truth values
corresponding to true, false and undefined.
This is done by attaching a sentconst to `T', `NIL', `UNDEF'
respectively.
`funconst's and `predconst's can be attached to a HGKM `atom'
`atom' will be used as the identifier of a HGKM function, whose number of 
arguments is supposed to match the arity of the GETFOL symbol.
}

\gfexample+
   ***** declare indconst a b;
   ***** declare sentconst s;
   ***** declare funconst f 1;
   ***** declare predconst p 1;
   ***** decrep rep;

   ***** attach a to a;
   a attached to 'a
   ***** attach a dar a;
   a attached to 'a
   a is the preferred name of a
   ***** attach a dar [rep]a;
   a attached to 'a
   ***** attach a dar [rep]b;
   a is already an preferred name in this representation
   ***** attach b dar [rep]a;
   a has already a preferred name in this representation

   ***** COMMENT | deflam defines an HGKM function |;
   ***** deflam f(x) x;
   ***** attach f to f;
   f attached to f
   ***** deflam p(x) (IF (EQUAL x (QUOTE a))TRUE FALSE);
   ***** attach p to p;
   p attached to p
   attach f to [repin=repout]f;
   f attached to f
   ***** deflam p1(x) (IF (EQUAL x (QUOTE b))TRUE FALSE);
   ***** attach p to [repin]p1;
   p attached to p1
   ***** attach p to [repin]p1;
   p has already an attachment with these representation informations
   ***** attach s to T;
   s attached to 'T
   ***** attach s to NIL;
   s has already an attachment
+

\gfcommand{decrep}{representation declaration}
\index{decrep}

\gfsyntax{
  decrep \ARG{replabel1} \OPT{\SEQ \ARG{replabelN}};
}

\gfdescription{
  It declares \ARG{replabelI} to be representation functions.\\
  The only builtin representation functions are {\tt NATNUMREP}, {\tt TRUTHREP}
  and {\tt UNIVERSALREP}, the representation functions for natural numbers, for 
  truth values and for default representation respectively.
  Numerals have a builtin attachment to {\HG} numbers in the representation 
  function {\tt NUMERALREP}.
}

\gfrecap{
It declares `replabelI' to be representation functions.
The only builtin representation functions are `NATNUMREP', `TRUTHREP'
and `UNIVERSALREP', the representation functions for natural numbers, for 
truth values and for default representation respectively.
Numerals have a builtin attachment to HGKM numbers in the representation 
function `NUMERALREP'.
}

\gfexample+
   ***** decrep rep1 rep2;
+

\gfnotes{
  Since the intended model itself appears nowhere in the {{\GF}} system, there
  is no need for the user to give any detailed information about the nature of 
  the representation maps which he has in mind.
  {\tt NATNUMREP} is known by {\GF} only after typing {\tt know natnums}.
}

\gfcommand{hardware}{semantic attachment to values of a s-expression}
\index{hardware}

\gfsyntax{
  hardware \ARG{indconst} to \ALT dar \ARG{sexpr};\\
  hardware \ARG{indconst} to \ALT dar [ \ARG{rep} ] \ARG{sexpr};
}

\gfdescription{
  This command is similar to the attach command for \ARG{indconst}.
  The difference is that, if \ARG{sexpr} changes value over time, then so does
  the value of the attachment.
  It is a "dynamic attachment" in the sense that it is attached to the values the
  \ARG{sexpr} assumes over time.
}

\gfrecap{
This command is similar to the attach command for `indconst'.
The difference is that, if `sexpr' changes value over time, then so does
the value of the attachment.
It is a "dynamic attachment" in the sense that it is attached to the values the
`sexpr' assumes over time.
}

\gfexample+
   ***** declare indconst clock t0 t1;
   ***** attach t0 to 0;
   t0 attached to '0
   ***** attach t1 to 1;
   t1 attached to '1
   ***** hardware clock to time;
   clock attached to time
   ***** done;
   >(SETQ time 0)
   0
   >(GETFOL)
   Hi!  Glad your back.  What would you like to talk about now?
   ***** simplify clock = t0;
   1   clock = t0     
   ***** done;
   >(SETQ time 1)
   1
   >(GETFOL)
   Hi!  Glad your back.  What would you like to talk about now?
   ***** simplify clock = t1;
   2   clock = t1     
+

\gfnotes{
  This command gives the possibility of changing the intended model. 
}

\gfcommand{represent}{default representation for sorts}
\index{represent}
\label{sec-rep-sort}

\gfsyntax{
  represent \{ \ARG{sort1}, \SEQ, \ARG{sortN} \} as \ARG{rep} \ALT \ARG{*};
}

\gfdescription{
  Sets the default representation for \ARG{sort1}, \SEQ, \ARG{sortN} to be
  \ARG{rep} ({\tt UNIVERSALREP} if \ARG{*} is specified).
  The default representation is used by the reflect command (see section
  \ref{sec-refl}).
}

\gfrecap{
Sets the default representation for `sort1', ..., `sortN' to be `rep'
(`UNIVERSALREP' if `*' is specified).
The default representation is used by the `reflect' command.
}

\gfexample+
   ***** declare sort s t;
   ***** decrep rep;
   ***** represent {s t} as rep;
   ***** represent {s t} as rep;
   s has already a default representation
+

\gfcommand{simplify}{semantic simplification}
\index{simplify}
\label{sec-simplify}

\gfsyntax{
  simplify \ARG{wff} \ALT \ARG{fact} \ALT \ARG{term};
}

\gfdescription{
  If \ARG{term} is provided as argument, three steps are performed:\\
  %
  \begin{itemize}
  \item 
    the interpretation of \ARG{term} in the computational model is computed;
  \item a preferred name for the interpretation is found;
  \item the equality of \ARG{term} with the preferred name is asserted as the next
    line of the proof.
    No action is taken if \ARG{term} has no interpretation in the computational
    model (has an undefined interpretation) or a preferred name for its
    interpretation does not exist.
  \end{itemize}
  %
  If {\em wff} is provided as an argument, two steps are performed:
  %
  \begin{itemize}
  \item the interpretation of {\em wff} in the computational model is computed 
    (in this case the interpretation will be {\tt TRUE}, {\tt FALSE} or an 
    undefined truth value)
  \item if the interpretation of {\em wff} is {\tt TRUE}, {\em wff} 
    is asserted as the next line of the proof, if it is {\tt FALSE} 
    the negation of {\em wff} is asserted.
  \end{itemize}
  %
  When \ARG{fact} is provided as argument, the simplify command works on the wff
  of \ARG{fact}.
}

\gfrecap{
  If `term' is provided as argument, three steps are performed:
      * the interpretation of `term' in the computational model is computed;
      * a preferred name for the interpretation is found;
      * the equality of $term$ with the preferred name is asserted as the next
        line of the proof. No action is taken if `term' has no interpretation
        in the computational model (has an undefined interpretation) or a
        preferred name for its interpretation does not exist.
  If `wff' is provided as an argument, two steps are performed:
      * the interpretation of `wff' in the computational model is computed 
        (in this case the interpretation will be `TRUE', `FALSE' or an 
        undefined truth value)
      * if the interpretation of `wff' is `TRUE', `wff' is asserted as the next
        line of the proof, if it is `FALSE' the negation of `wff' is asserted.
  When `fact' is provided as argument, the simplify command works on the wff
  of `fact'.
}

\gfexample+
   ***** declare indconst a b c;
   ***** decrep REP;
   ***** attach a dar [REP]a;
   a attached to 'a
   a is the preferred name of a
   ***** attach b dar [REP]b;
   b attached to 'b
   b is the preferred name of b
   ***** attach c to [REP]c;
   c attached to 'c
   ***** declare funconst F 1;
   F has been declared to be a Funconst
   ***** DEFLAM F(x) (IF (EQ x (QUOTE a)) (QUOTE b) 
                         (IF (EQ x (QUOTE b)) (QUOTE c)
                          (QUOTE UNDEF&)));
   ***** attach F to [REP=REP]F;
   F attached to F
   ***** simplify F(a);
   1   F(a) = b    
   ***** simplify F(b);
   F(b) : No simplification is possible.
   ***** simplify F(c);
   F(c) : No simplification is possible.
   ***** declare predconst P 1;
   P has been declared to be a Predconst
   ***** DEFLAM P(x) (IF (EQ x (QUOTE a)) TRUE 
                         (IF (EQ x (QUOTE b)) FALSE 
                          (QUOTE UNDEF&)));
   ***** attach P to [REP]P;
   P attached to P
   ***** simplify P(a);
   2   P(a)  
   ***** simplify P(b);
   3   not P(b)    
   ***** simplify P(c);
   P(c) : No simplification is possible.
   ***** extension UNIVERSAL by {a b c};
   Now the extension of UNIVERSAL is fixed to be : (a b c)
   ***** declare indvar x;
   UNIVERSAL is a sort
   x has been declared to be an Indvar
   ***** simplify exists x.F(x)=b;
   4   exists x. (F(x) = b)
   ***** simplify forall x.P(x);
   5   not forall x. P(x) 
+

\gfnotes{
  In the case of sorts with extensions (see the command {\tt extension} in section 
  \ref{sec-sort}) quantification is considered as {\bf bounded quantification}.
  In other words, let $P$ be a predicate and $x$ an indvar of sort $S$, where
  $S$ has extension $\{s_1,\ldots,s_n\}$.
  Then the following equivalences hold:
  $$ 
  \forall x P(x)\liff (P(s_1) \con \ldots \con P(s_n))
  $$
  $$
  \exists x P(x)\liff (P(s_1) \dis \ldots \dis P(s_n))
  $$
  %
  The command explicitly unfolds universal/existential statements into 
  their propositional equivalents. 
}



% introduction to the syntactic simplification's section
\newpage
\section{Syntactic simplification}
\label{sec-rew}

The subsections \ref{sec-rew-intro} and \ref{sec-rew-simpset} 
of this section have been taken from \cite{rww3}.


\subsection{Introduction}
\label{sec-rew-intro}

The basic idea  of syntactic simplification is repeated substitution of 
selected equalities and equivalences into a given expression.
More precisely, let $E$ be a set of universally quantified equations and 
equivalences ("rewrite rules"), so members of $E$ look like:
\begin{itemize}
\item $\forall\ {\vec x}.(t_1=t_2)$
\item $\forall\ {\vec y}.(F_1\ \liff \ F_2)$
\end{itemize}
where ${\vec x}$ and ${\vec y}$ are the {\indvar} sequences $x_1$...$x_n$
and $y_1$..,$y_m$, $t_1$ and $t_2$ are {\term}s, and $F_1$, $F_2$ are {\wff}s.

A match, or an immediate simplification, of a {{\GF}} expression $exp$ consists
 of replacing an occurrence of $t_1[x \leftarrow u]$($F_1[y \leftarrow v]$) 
in $exp$ by $t_2[x \leftarrow u]$($F_2[y \leftarrow v]$), where $u$($v$) is a 
sequence of terms and where $\leftarrow$ indicates substitution.

There are two problems to solve:
\begin{enumerate}
\item There may be more than one equation (or equivalence) whose left half 
      matches a given expression, so one has to establish a precedence 
      hierarchy for matching.

\item The order used by the algorithm to consider the subexpressions of a 
      given expression.
\end{enumerate}

{{\GF}}'s solution to the first problem is the following ordering expression:
each simplification expression (i.e., left half of a rewrite rule) is 
regarded as a linear string of atoms.
Each atom is either:

\begin{itemize}
\item a {\bf constant} (which is not bound by the universal quantifier in the
                       prefix);

\item an {\bf old variable} (which is bound by the universal quantifier in 
                            the prefix and which has occurred before in the 
                            linear string);

\item a {\bf new variable} (which is bound by the universal quantifiers in the
                           prefix and which has not occurred before in the 
                           linear string);
\end{itemize}

If we think of concatenating different atoms to a given initial string, then 
the atoms have this precedence ordering:
\begin{center}
constants $<$ old variables $<$ new variables
\end{center}
and expressions are ordered lexicographically in accordance with this 
ordering on atoms.

Let's consider, for example, the precedence relations among the simplification
 expressions :
$f(a,b,b)$, $f(a,b,c)$, $f(a,a,x)$, $f(a,x,x)$, $f(a,x,y)$, $f(x,x,x)$,
$f(x,x,y)$, 
where $f$, $a$, $b$, $c$ are constants and $x$, $y$ are variables.

The last four expressions are linearly ordered:
$$
f(a,x,x)<f(a,x,y)<f(x,x,x)f<(x,x,y)
$$
and each of the first three expressions is less than $f(a,x,x)$ and 
incomparable to the other two of the first three expressions:
$$
f(a,b,b)<f(a,x,x)
$$
$$
f(a,b,c)<f(a,x,x)
$$
$$
f(a,a,x)<f(a,x,x)
$$
Together with transitivity, these inequalities completely define the 
precedence relation.

As far as regard the second problem, {{\GF}}'s syntactic simplification code 
basically considers subexpressions of $exp$ in the usual left-to-right order.
The exceptions occur after a subexpression $exp'$ has been matched (and 
substituted for).
The algorithm then begins again at the subexpression one level above $exp'$.

The syntactic simplification algorithm has the usual problems of rewrite rules.
A typical difficulty is the infinitely recurring substitutions: 
for example if one uses $\forall\ x.x+y=y+x$ as simplification equation, 
the algorithm will attempt to make this substitution without end.

\subsection{Simplification sets}
\label{sec-rew-simpset}

Syntactic simplification in {\GF} is performed by using 
{\bf syntactic simplification sets} (called {\bf simpsets} from now on).
Simpsets contain a label ({\em simplabel}) to identify the
rewrite rules used to rewrite expressions.
{\GF} has built-in simpsets, but the user can define his own ones: he can specify a 
set of formulae or facts as rewrite rules in a {\bf basic simspset}
or he can compose already defined simpsets in {\bf compound simpsets}.

The {{\GF}} builtin simpsets (see figure \ref{fig-simpset}) are
{\bf \tt LPROPTREE}, {\bf \tt LQUANTREE}, {\bf \tt LARGIFTREE} and 
{\tt LOGICTREE}.
{\tt LPROPTREE} contains a set of rewrite rules 
corresponding to basic logical equivalences (e.g. $P\con\neg P\liff \bot$). 
{\tt LQUANTREE} contains a set of rewrite rules 
corresponding to logic equivalences for quantified formulas.
{\tt LARGIFTREE} contains a set of rewrite rules corresponding to logic
equalities and equivalences for conditional terms and formulas.
{\tt LOGICTREE} is the union of all the previous builtin simpsets.

\newpage

\begin{figure}[htbp]
\begin{center}
\fbox{
\parbox{16cm}{
$$
\begin{array}{ll}
\neg \neg P \liff P    & \ \ \                         \\
\neg {\tt TRUE} \liff \bot   & \ \ \  \neg \bot \liff {\tt TRUE}    \\
P \con \bot \liff \bot & \ \ \  \bot \con P\liff \bot   \\
P \con {\tt TRUE}  \liff  P  & \ \ \  {\tt TRUE}  \con P\liff P     \\
\neg P\con P\liff\bot  & \ \ \  P \con \neg P\liff \bot \\
P\con P\liff P         & \ \ \                         \\
P     \dis \bot \liff P  & \ \ \  \bot \dis P \liff P       \\
P \dis {\tt TRUE} \liff {\tt TRUE}   & \ \ \  {\tt TRUE}  \dis P \liff {\tt TRUE}   \\
\neg P \dis P\liff {\tt TRUE}  & \ \ \  P \dis \neg P \liff {\tt TRUE}  \\
P     \dis P     \liff P & \ \ \                         \\
P \imp \bot\liff\neg P & \ \ \  \bot \imp P\liff {\tt TRUE}   \\
P\imp {\tt TRUE}\liff {\tt TRUE}   & \ \ \  {\tt TRUE}  \imp P     \liff P\\
\neg P \imp P \liff P  & \ \ \  P \imp \neg P\liff\neg p\\
P \imp P\liff {\tt TRUE}     & \ \ \                         \\
P\liff \bot\liff\neg P  & \ \ \  \bot \liff P\liff\neg P  \\
P\liff {\tt TRUE}\liff P      & \ \ \  {\tt TRUE}  \liff P\liff P     \\
\neg P \liff P\liff\bot & \ \ \  P\liff\neg P\liff\neg\bot\\
P\liff P\liff {\tt TRUE}      & \ \ \                         \\
\end{array}
$$
}}
\fbox{
\parbox{16cm}{
$$
\begin{array}{ll}
\forall x.{\tt TRUE}  \liff {\tt TRUE} & \ \ \  \forall x.\bot \liff \bot    \\
\exists x.{\tt TRUE}  \liff {\tt TRUE} & \ \ \  \exists x.\bot \liff \bot    \\
\end{array}
$$
}}
\vspace{0.3cm} 
\fbox{
\parbox{16cm}{
$$
\begin{array}{ll}
\forall x y.{\em trmif}\ \bot{\em then}\ x\ {\em else}\ y = y    & \ \ \ 
\forall x y.{\em trmif}\ {\tt TRUE}\ {\em then}\ x\ {\em else}\ y\ = x \\
\forall x.  {\em trmif}\ P\ {\em then}\ x\ {\em else}\ x = x & \ \ \  \\
{\em wffif}\ \bot\ {\em then}\ P1\ {\em else}\ P2\ \liff P2 & \ \ \ 
{\em wffif}\ {\tt TRUE}\ {\em then}\ P1\ {\em else}\ P2\ \liff\ P1 \\
{\em wffif}\ P\ {\em then}\ P1\ {\em else}\ P1\ \liff\ P1 & \ \ \  \\
\end{array}
\\
$$
}}
\caption{The rewrite rules of {\tt LPROPTREE}, {\tt LQUANTREE}
\label{fig-simpset}
and {\tt LARGIFTREE}.}
\end{center}
\end{figure}


% user commands for syntactic simplification
\gfcommand{assertsimp}{simpset command}
\index{assertsimp}

\gfsyntax{
  assertsimp \ARG{simplabel}; 
}

\gfdescription{
  It generates a proof step for each formula contained in the rewrite rules
  of \ARG{simplabel}, which cannot be a builtin simpset.
}

\gfrecap{
It generates a proof step for each formula contained in the rewrite rules
of `simplabel', which cannot be a builtin simpset.
}

\gfexample+
   ***** declare indconst a b;
   ***** declare predconst q r 1;
   ***** declare indvar x;
   ***** setbasicsimp s1 at wffs {q(a), forall x.(q(x) iff r(x))};
   ***** assume a=b;
   1   a = b     (1)
   ***** assume q(b);
   2   q(b)     (2)
   ***** setbasicsimp s2 at facts {1,2};
   ***** setbasicsimp s2 at facts {1};
   s2 is already the label of a simpset
   ***** setcompsimp s4 at s1 uni s2;
   ***** assertsimp s1;
   3   q(a)     
   4   forall x. (q(x) iff r(x))     
   ***** assertsimp s2;
   s2 does not contain any wff to assert.
   ***** assertsimp s4;
   5   q(a)
   6   forall x. (q(x) iff r(x))
   ***** assertsimp LOGICTREE;
   LOGICTREE is the label of a builtin simpset, you can't assert it.
+
\gfcommand{rewrite}{syntactic simplifier command}
\index{rewrite}

\gfsyntax{
  rewrite \ARG{wff} \ALT \ARG{fact} \ALT \ARG{term} \OPT{by \ARG{simpexpr}};
}

\gfdescription{
  It rewrites the given expression by using the union of the rewrite rules 
  indicated by the \ARG{simpexpr}.
  If \ARG{simpexpr} is not specified, {\tt LOGICTREE} is used.
  If \ARG{term} is provided as an argument, two steps are performed:
  %
  \begin{itemize}
  \item \ARG{term} is rewritten by using the set of rewrite rules 
    indicated by \ARG{simpexpr}.
  \item the equality of \ARG{term} with its rewritten form is asserted as the next 
    line of the proof. The dependencies depend on the simpsets actually used
    during the syntactic simplification.
  \end{itemize}
  %
  If \ARG{wff} is provided as argument, also two steps are performed:
  %
  \begin{itemize}
  \item \ARG{wff} is rewritten by the set of rewrite rules of the
    \ARG{simpexpr} result.
  \item if \ARG{wff} is rewritten to {\tt TRUE}, \ARG{wff} is asserted as the
    next line in the proof, if \ARG{wff} is rewritten to {\tt FALSE},
    $\neg$ \ARG{wff} is asserted, otherwise the equivalence of \ARG{wff} with 
    its rewritten form is asserted.
    The dependencies depend on the simpsets used during the syntactic
    simplification.
  \end{itemize}
  %
  When \ARG{fact} is provided as an argument, the rewrite command works on the wff
  of the \ARG{fact}.
}

\gfrecap{
It rewrites the given expression by using the union of the rewrite rules 
indicated by the `simpexpr'.
If `simpexpr' is not specified, LOGICTREE is used.
}

\gfexample+
   ***** declare indconst A,B;
   ***** declare indvar X,Y;
   ***** declare funconst F 2;
   ***** declare funconst G 1;
   ***** declare sentconst P;
   ***** assume forall X . F(X,A) = A;
   1   forall X. (F(X,A) = A)     (1)
   ***** assume forall X . F(X,X) = G(X);
   2   forall X. (F(X,X) = G(X))     (2)
   ***** assume forall X Y . F(X,Y) =Y;
   3   forall X Y. (F(X,Y) = Y)     (3)
   ***** axiom F1:forall X . F(X,A) = A;
   F1 : forall X. (F(X,A) = A)
   ***** axiom F2:forall X . F(X,X) = G(X);
   F2 : forall X. (F(X,X) = G(X))
   ***** axiom F3:forall X Y. F(X,Y) = Y;
   F3 : forall X Y. (F(X,Y) = Y)
   ***** setbasicsimp S1 at facts {1};
   ***** setbasicsimp S2 at facts {2};
   ***** setbasicsimp S3 at facts {3};
   ***** setbasicsimp S4 at facts {F1};
   ***** setbasicsimp S5 at facts {F2};
   ***** setbasicsimp S6 at facts {F3};
   ***** setbasicsimp SIMPEQ at wffs {forall X.(X=X iff TRUE)};
   ***** setcompsimp S7 at S1 uni S2 uni S3;
   ***** rewrite F(A,A) by S6;
   4   F(A,A) = A     
   ***** rewrite F(A,A) by S5;
   5   F(A,A) = G(A)     
   ***** rewrite F(A,A) by S4;
   6   F(A,A) = A     
   ***** rewrite F(A,A) by S1;
   7   F(A,A) = A     (1)
   ***** rewrite F(A,A) by S2;
   8   F(A,A) = G(A)     (2)
   ***** rewrite F(A,A) by S3;
   9   F(A,A) = A     (3)
   ***** rewrite F(A,A) by S7;
   10   F(A,A) = A     (1)
   ***** rewrite F(B,B) by S1 uni S3;
   11   F(B,B) = B     (3)
   ***** rewrite F(B,B) by S1;
   F(B,B): No simplification is possible
   ***** rewrite not TRUE by S1;
   not TRUE: No simplification is possible
   ***** rewrite not TRUE;
   12   not (not TRUE)     
   ***** rewrite TRUE imp (P imp X=X);
   13   (TRUE imp (P imp (X = X))) iff (P imp (X = X)) 
   ***** rewrite TRUE imp (P imp X=X) by SIMPEQ uni LOGICTREE;
   14   TRUE imp (P imp (X = X))
   ***** rewrite F(A,A) by S7;
   15   F(A,A) = A     (1)
   ***** rewrite F(A,A)=A by S7;
   16   F(A,A) = A  iff (A = A)   (1)
   ***** rewrite F(A,A)=A by S7 uni SIMPEQ;
   17   F(A,A) = A     (1)
   ***** rewrite F(A,A)=G(A) by S7;
   18   (F(A,A) = G(A)) iff (A = G(A))     (1)
   ***** rewrite F(B,B) by S7;
   19   F(B,B) = G(B)     (2)
   ***** rewrite F(B,B)=G(B) by S7 uni SIMPEQ;
   20   F(B,B) = G(B)     (2)
   ***** rewrite F(B,B)=G(B) and F(A,A)=A by S7 uni SIMPEQ uni LOGICTREE;
   21   (F(B,B) = G(B)) and (F(A,A) = A)     (1 2)
   ***** rewrite F(A,A) by S7 dif S1 ;
   22   F(A,A) = G(A)     (2)
   ***** rewrite F(A,A) by S7 dif (S1 uni S2);
   23   F(A,A) = A     (3)
   ***** rewrite F(A,A)=A by S3 dif (S1 uni S2) uni SIMPEQ;
   24   F(A,A) = A     (3)
+
   


% introduction to the syntactic/semantic simplification's section
\newpage
\section{Syntactic and semantic simplification}

Some of the commands perform both syntactic and semantic simplifications.


% user commands for the syntactic/semantic simplification's section
\gfcommand{eval}{mixed simplifier command}
\index{eval}
\label{sec-eval}

\gfsyntax{
  eval \ARG{wff} \ALT \ARG{fact} \ALT \ARG{term} \OPT{by \ARG{simpexpr}};
}

\gfdescription{
  This command evaluates the expression (\ARG{wff}, the wff of \ARG{fact}
  or \ARG{term} respectively) by combining the semantic evaluation of the
  expression in the simulation structure and the syntactical rewriting
  performed by using the union of the rewrite rules indicated by 
  \ARG{simpexpr}.
}

\gfrecap{
  This command evaluates the expression (`wff', the wff of `fact'
  or `term' respectively) by combining the semantic evaluation of the
  expression in the simulation structure and the syntactical rewriting
  performed by using the union of the rewrite rules indicated by 
  `simpexpr'.
}

\gfexample+
   ***** declare indconst a b c;
   ***** decrep REP;
   ***** attach a dar [REP]a;
   a attached to 'a
   a is the preferred name of a
   ***** attach b dar [REP]b;
   b attached to 'b
   b is the preferred name of b
   ***** attach c dar [REP]c;
   c attached to 'c
   c is the preferred name of c
   ***** declare funconst G 2;
   ***** declare indvar x y;
   ***** setbasicsimp S at wffs {forall x y.G(x y)=x};
   ***** declare predconst P 1;
   ***** DEFLAM P(x) (IF (EQ x (QUOTE a)) TRUE 
                         (IF (EQ x (QUOTE b)) FALSE 
                          (QUOTE UNDEF&)));
   ***** attach P to [REP]P;
   P attached to P
   ***** eval P(G(a,G(b,c))) by S;
   1   P(G(a,G(b,c)))     
   ***** eval P(G(b,c)) and P(c) by S;
   2   not (P(G(b,c)) and P(c))
   ***** eval P(G(c,a)) by S;
   3   P(G(c,a)) iff P(c)   
   ***** eval forall x.P(G(x x)) by S;
   4   forall x. P(G(x,x)) iff forall x. P(x)     
   ***** extension UNIVERSAL by {a b c};
   Now the extension of UNIVERSAL is fixed to be : (a b c)
   ***** eval forall x.P(G(x x)) by S;
   5   not forall x. P(G(x,x)) 
+

\gfnotes{
  In the case of sorts with extensions (see the command {\tt extension} in section 
  \ref{sec-sort}) quantification is considered as  {\bf bounded quantification}.
  In other words, let $P$ be a predicate and $x$ an indvar of sort $S$, where
  $S$ has extension $\{s_1,\ldots,s_n\}$.
  Then the following equivalences hold:
  $$ 
  \forall x(P(x)\liff P(s_1) \con \ldots P(s_n))
  $$
  $$
  \exists x(P(x)\liff P(s_1) \dis \ldots P(s_n))
  $$
  %
  The command explicitly unfolds universal/existential statements into 
  their propositional equivalents. The expansion is performed 
  syntactically, that is the formula $\forall x P(x)$ [$\exists x P(x)$]
  is rewritten as $P(s_1) \con \ldots \con P(s_n)$
  [$P(s_1) \dis \ldots \dis P(s_n)$].
  The expansion mechanism embedded in {\tt simplify} is not used by {\tt eval}.
}


\gfcommand{let}{evaluation plus attachment}
\index{let}

\gfsyntax{
  let \ARG{\indconst} to \ALT dar [ \ARG{rep} ] \ARG{term};
}

\gfdescription{
  This command evaluates the \ARG{term} using the mixed evaluation mechanism
  of the {\tt eval} command.
  If the evaluation returns a {\HG} representation for \ARG{term}, then it is
  attached to  \ARG{indconst} with representation function \ARG{rep}. Then the 
  equality of \ARG{indconst} with \ARG{term} is asserted as the next line in the
  proof, otherwise an error message is given. If \ARG{rep} is not specified the
  representation is the default representation. If this does not happen {\GF}
  outputs an error message.
}

\gfrecap{
This command evaluates the `term' using the mixed evaluation mechanism
of the {\tt eval} command.
If the evaluation returns a HGKM representation for `term', then it is
attached to  `indconst' with representation function `rep'. Then the 
equality of `indconst' with `term' is asserted as the next line in the
proof, otherwise an error message is given. If `rep' is not specified the
representation is the default representation. If this does not happen GETFOL
outputs an error message.
}

\gfexample+
   ***** declare indconst a b c;
   ***** attach b to b;
   b attached to 'b
   ***** attach c to c;
   c attached to 'c
   ***** declare funconst h 2;
   ***** DEFLAM h(x y) (QUOTE d);
   ***** attach h to h;
   h attached to h
   ***** let a dar h(b c);
   a attached to 'd
   a is the preferred name of d
   1   a = h(b,c)     
+



 	% loading introduction to the section
\newpage
\section{Multiple contexts}
\label{sec-cxt}

\subsection{Introduction}

\begin{quote}\em
  ... When reasoning, people seem to be able to switch focus of their
  attention and make always some sort of local reasoning ...
  \cite{giunchiglia2}
\end{quote}

Structuring the knowledge into {\em distinguished partial descriptions of the
world}, has been hinted as a cognitively plausible hypotheses. 
Distinct partial descriptions can be represented by {\GF} {\bf contexts}.
A {\GF} context contains its own language defined by a set of declarations, 
its own axioms and definitions, and its own computational model.
Reasoning can be performed within a context. 
You can type any command defined so far within any context in {\GF}.
Multiple proofs can be performed within a context.
When you work in {\GF} you are always in one context.
The context in which you are working in is called the {\em current context}.
When you enter the system the current context is empty and without name.
If you want to leave the context to work in another one, you have to give the
context a name to refer to it later (by the command {\tt namecontext}).
You can create a new context by using {\tt makecontext}, and  switch to it
by using {\tt switchcontext}.
The context you switch to then becomes the current context.


% loading explanation of commands
\gfcommand{copycontext}{Multiple contexts manipulation}
\index{copycontext}

\gfsyntax{
  copycontext \ARG{ctx-name};
}

\gfdescription{
  A new context with name \ARG{ctx-name} is created, and the current context
  is copied in it.
}

\gfrecap{
  A new context with name `ctx-name' is created, and the current context
  is copied in it.
}

\gfexample+
   ***** show whereami;
   You are now using an unnamed context.
   You are now using an unnamed proof.
   ***** namecontext C1;
   You have named the current context: C1
   ***** show whereami;
   You are now using context: C1
   You are now using an unnamed proof.
   ***** declare sentconst A;
   ***** makecontext C2;
   You have created the empty context: C2
   ***** switchcontext C2;
   You are now using context: C2
   ***** declare indvar A;
+
\gfcommand{copylex}{Language declaration through contexts}
\index{copylex}

\gfsyntax{
	copylex	\ARG{ctx-name};
}

\gfdescription{
	This command copies in the current context all the symbols and sorts
	declared in the context {\em ctx-name}.
	The command has no effects in the case there is at least a symbol in the
	current context that has the same name as a symbol in the context
	{\em ctx-name}.
}

\gfrecap{
This command copies in the current context all the symbols and sorts
declared in the context `ctx-name'.
The command has no effects in the case there is at least a symbol in the
current context that has the same name as a symbol in the context `ctx-name'.
}


\gfexample+
   ***** declare indconst a b;
   ***** declare sentconst A B;
   ***** declare sort S1 S2;
   ***** namecontext C1;
   You have named the current context: C1
   ***** makecontext C2;
   You have created the empty context: C2
   ***** switchcontext C2;
   You are now using context: C2
   You are switching to a proof with no name.
   ***** probe declare;
   ***** copylex C1;
   S1 has been declared to be a sort
   S2 has been declared to be a sort
   A has been declared to be a Sentconst
   B has been declared to be a Sentconst
   a has been declared to be an Indconst
   b has been declared to be an Indconst
   ***** copylex C1;
   COPYLEX cannot be done: A has already been declared
   ***** copylex C2;
   You cannot copy the lex of the current context
+

\gfcommand{makecontext}{Multiple contexts' manipulation}
\index{makecontext}

\gfsyntax{
  makecontext \ARG{ctx-name};
}

\gfdescription{
  A new empty context  with name  \ARG{ctx-name} is created.
}

\gfrecap{
  A new empty context  with name  `ctx-name' is created.
}

\gfexample+
   ***** show whereami;
   You are now using an unnamed context.
   You are now using an unnamed proof.
   ***** namecontext C1;
   You have named the current context: C1
   ***** show whereami;
   You are now using context: C1
   You are now using an unnamed proof.
   ***** declare sentconst A;
   ***** makecontext C2;
   You have created the empty context: C2
   ***** switchcontext C2;
   You are now using context: C2
   ***** declare indvar A;
+

\gfcommand{namecontext}{Multiple contexts' manipulation}
\index{namecontext}

\gfsyntax{
  namecontext \ARG{ctx-name};
}

\gfdescription{
  If the current context has no  name,  it is named with \ARG{ctx-name}.
}

\gfrecap{
  If the current context has no  name,  it is named with `ctx-name'.
}

\gfexample+
   ***** show whereami;
   You are now using an unnamed context.
   You are now using an unnamed proof.
   ***** namecontext C1;
   You have named the current context: C1
   ***** show whereami;
   You are now using context: C1
   You are now using an unnamed proof.
   ***** declare sentconst A;
   ***** makecontext C2;
   You have created the empty context: C2
   ***** switchcontext C2;
   You are now using context: C2
   ***** declare indvar A;
+
\gfcommand{reset}{{\GF} reset}
\index{reset}

\gfsyntax{
  reset;
}

\gfdescription{
  Resets the whole {\GF} system.
}

\gfrecap{
  Resets the whole GETFOL system.
}

\gfexample+
   ***** namecontext c; nameproof p;
   You have named the current context: c
   You have named the current proof: p
   
   ***** show whereami;
   You are now using context: c
   You are now using the proof: p
   
   ***** declare sentconst A;
   A has been declared to be a Sentconst
   ***** assume A;
   1   A     (1)
   ***** show proof;
   1   A     (1)
   ***** makecontext c1;
   You have created the empty context: c1
   ***** switchcontext c1;
   You are now using context: c1
   You are switching to a proof with no name.
   
   ***** reset;
   Resetting the whole GETFOL-system
   
   ***** show whereami;
   You are now using an unnamed context.
   You are now using an unnamed proof.
   
   ***** show proof;
+

\gfcommand{switchcontext}{Multiple contexts' manipulation}
\index{switchcontext}

\gfsyntax{
  switchcontext \ARG{ctx-name};
}

\gfdescription{ 
  Switches from the current context to the context \ARG{ctx-name} which
  becomes the current context.
}

\gfrecap{ 
  Switches from the current context to the context `ctx-name' which
  becomes the current context.
}

\gfexample+
   ***** show whereami;
   You are now using an unnamed context.
   You are now using an unnamed proof.
   ***** namecontext C1;
   You have named the current context: C1
   ***** show whereami;
   You are now using context: C1
   You are now using an unnamed proof.
   ***** declare sentconst A;
   ***** makecontext C2;
   You have created the empty context: C2
   ***** switchcontext C2;
   You are now using context: C2
   ***** declare indvar A;
+

 	% loading introduction to the section
\newpage
\section{Metareasoning}
\label{sec-meta}

\subsection{Introduction}

A special context is {\meta}.
{\GF} recognizes {\meta} as a metatheory of all the other contexts.
The context {\meta} can be used to perform {\em metareasoning}, that is to
describe other contexts and to reason about them. 
Metareasoning in {\meta} is performed by employing the following novel
features:
%
\begin{itemize}
  \item
    the metatheory is, in general, distinct from the object theories it
    describes; 
  \item
    the link between the metalanguage and the object language is not performed
    by encoding, but rather by naming \cite{giunchiglia3}.
    Naming is implemented by using the commands which implement reasoning in
    the computational model of a context (see section \ref{sec-comp}).
    These features are available to the user by the commands \C{attach},
    \C{simplify}, \C{eval} etc.  
  \item
    Metareasoning and object reasoning can be mixed via the reflection rule 
    \cite{giunchiglia3}:

    \begin{equation}
      R_{down}
      \fraz{\der{M} Theorem(``w'')}{\der{O} w}
      \label{refl}
    \end{equation}

    where $M$ and $O$ stand for {\meta} and object theory respectively. 
    This rule is implemented in the {\GF} command \C{reflect}.
    The command knows that some form of metareasoning must be performed in
    {\meta} to deduce the metastatement $Theorem(``w'')$.
    The command can use the reflection rule (\ref{refl}) to assert a new proof
    line in the object level context (the context in which object level 
    reasoning is performed and where the command \C{reflect} is typed in).
  \item
    Any context can be the object level context, {\meta} itself.
    The amalgamation of the object and meta level is a particular case of {\GF}
    metareasoning. 
\end{itemize}

In {\meta}, the user is free to declare any language, any set of axioms and to 
define any computational model. This amounts to say that {\meta} is the 
``metatheory'' of a theory represented in a context as far as the user defines
the appropriate attachments and axioms.
A special unary predicate symbol which can be declared in {\meta} is
{\tt THEOREM}: this is the predicate recognized as meaning theoremhood by the
the reflect rule (\ref{refl}) in the command {\tt reflect}.



% loading explanation of commands
\gfcommand{mattach}{semantic meta attachment}
\index{mattach}

\gfsyntax{
   mattach \ARG{indconst} to \ALT dar \OPT{[rep]}
   \ARG{cname}:\ARG{pname}:\ARG{sort}:\ARG{object};
}

\gfdescription{
   Defines an attachment for a constant of the context {\em meta}.

   \ARG{rep}, if present, can be a representation function or \verb+*+.
   If it is \verb+*+ or no representation is specified, then the default representation function 
   {\tt UNIVERSALREP} is taken.
   \ARG{indconst} is a symbol declared to be an INDCONST in {\em meta}; \ARG{cname} is the name
   of the context to which \ARG{object} belongs; \ARG{pname} is the name of the proof in which
   \ARG{object} is present; \ARG{sort} is a sort of the meta-context associated to one of the
   syntactic categories reported with the {\tt reflect} command; \ARG{object} is an object of
   type \ARG{sort}.

   This command implements the mechanism of ``naming'' symbols or objects belonging to the 
   context \ARG{cname}, {\em ie.} the creation of names denoting objects of \ARG{cname}.
   \ARG{indconst} can be attached ``one way'' (using {\tt to}) or ``two ways'' (using {\tt dar}).
   The ``one way'' attachment tells the semantic interpretation mechanism that \ARG{indconst} is
   the name in {\meta} of the {\GF} object in the context \ARG{cname} corresponding to
   \ARG{object}.
   The two ways attachment tells the semantic interpretation mechanism that whenever the 
   (data structure representing) the \ARG{object} is computed as the representation of a 
   term {\em t}, then \ARG{indconst} should be returned as the simplified version of {\em t}.
}

\gfrecap{
This command implements the mechanism of ``naming'' symbols or objects belonging to the 
context `cname', ie. the creation of names denoting objects of `cname'.
`indconst' can be attached ``one way'' (using `to') or ``two ways'' (using `dar').
The ``one way'' attachment tells the semantic interpretation mechanism that `indconst' is
the name in meta of the GETFOL object in the context `cname' corresponding to
`object'.
The two ways attachment tells the semantic interpretation mechanism that whenever the 
(data structure representing) the `object' is computed as the representation of a 
term `t', then `indconst' should be returned as the simplified version of `t'.
}

\gfexample+
   ***** namecontext META;
   ***** nameproof P1;

   ***** declare indconst sc [SENTCONST];
   ***** declare indconst ic [INDCONST];
   ***** declare indconst vl [FACT];
   ***** declare indconst f1 [FACT];

   ***** DECREP  SENTCONST INDCONST FACT;

   ***** represent { SENTCONST } as SENTCONST;
   ***** represent { INDCONST } as INDCONST;
   ***** represent { FACT } as FACT;

   ***** makecontext C;
   ***** switchcontext C;
   ***** declare indconst c;
   ***** declare sentconst A;
   ***** nameproof P1;
   You have named the current proof: P1

   ***** assume c=c;
   1   c = c     (1)

   ***** makeproof P2;
   You have created the empty proof: P2

   ***** switchproof P2;
   You are now using the proof: P2

   ***** assume A imp A;
   1   A imp A     (1)

   ***** label fact ax = 1;

   ***** switchcontext META;

   ***** MATTACH sc TO  C::SENTCONST:A;
   ctext-get: I changed context to: C
   ctext-get: I changed context to: META
   sc attached to 'A

   ***** MATTACH ic DAR C:P2:INDCONST:c;
   ctext-get: I changed context to: C
   ctext-get: I changed context to: META
   ic attached to 'c
   ic is the preferred name of c

   ***** MATTACH vl DAR [SENTCONST] C:P1:FACT:1;
   ctext-get: I changed context to: C
   proof-get: I changed proof to: P1
   proof-get: I changed proof to: P2
   ctext-get: I changed context to: META
   vl attached to '(1 (= c c) (1) ASSUME (%WFF% = c c))
   vl is the preferred name of (1 (= c c) (1) ASSUME (%WFF% = c c))

   ***** MATTACH f1 TO  C:P2:FACT:1;
   ctext-get: I changed context to: C
   ctext-get: I changed context to: META
   f1 attached to '(1 (imp A A) (1) ASSUME (%WFF% imp A A))

   ***** MATTACH f1 DAR C:P2:FACT:ax;
   ctext-get: I changed context to: C
   ctext-get: I changed context to: META
   f1 attached to '(1 (imp A A) (1) ASSUME (%WFF% imp A A))
   f1 is the preferred name of (1 (imp A A) (1) ASSUME (%WFF% imp A A))

+

\gfnotes{}


\gfcommand{reflect}{reflection}
\index{reflect}
\label{sec-refl}

\gfsyntax{
  reflect \ARG{M-fact} \ARG{arg1} \ARG{arg2} \SEQ \ARG{argN};
}

\gfdescription{
  In the   following description we  call ``object context''  the 
  context where the {\tt reflect} command is executed.

  \ARG{M-fact} is any fact of  the context {\meta} whose wff is of the
  form $\forall x_1 x_2\ldots x_n A(x_1, x_2,\ldots,x_n)$, $(n \geq 0)$,
  where the sorts of the variables $x_1, x_2,\ldots,x_n$ 
  correspond to some {\GF} syntactic category ({\em term, wff, fact} ... ).
  For any syntactic category corresponding to a sort in {\meta},
  {\GF} provides the necessary parsing routine.
  This parsing routine is necessary to run the reflect command.
  The relation between sorts in {\meta} and the associated parsed syntactic
  category is the following:

  \begin{figure}
    \begin{tabular}{|l|l|}
   \hline
   {\bf sort}&  {\bf syntactic category} \hspace{7cm} \\ \hline \hline
   SENTCONST &  a {\em sentconst}; \\
   QUANT     &  a {\em quant} (quantifier: {\tt forall} or {\tt exists}); \\
   SORT      &  a symbol declared as a sort; \\
   DECSYM    &  any declared symbol: {\em sym};\\
   FACT      &  a {\em fact};\\
   WFF       &  a {\em wff}; \\
   WFFIF     &  a {\em wffif};\\
   QUANTWFF  &  a {\em quantwff} (of the form  {\tt forall ... }  or 
                {\tt exists  ... });\\
   AWFF      &  an atomic wff (a wff of the form {\tt P( ... ))};\\
   TERM      &  a {\em term}; \\
   TERMIF    &  a {\em termif};\\
   INDSYM    &  a symbol declared as an {\em indconst} or {\em indvar} or
                {\em indpar}; \\
   INDVAR    &  a symbol declared as an {\em indvar};\\
   INDPAR    &  a symbol declared as an {\em indpar};\\
   INDCONST  &  a symbol declared as an {\em indconst};\\
   SENTSYM   &  a symbol declared as a {\em sentconst} or a {\em sentpar}; \\
   SENTPAR   &  a symbol declared as a {\em sentpar}; \\
   SENTCONST &  a symbol declared as a {\em sentconst}; \\
   APPLSYM   &  a symbol declared as a {\em funconst, funpar, predconst, 
                predpar} \\
             &  or a boolean connective; \\ 
   PREDSYM   &  a symbol declared as a {\em predconst} or {\em predpar}; \\
   PREDPAR   &  a symbol declared as a {\em predpar}; \\
   PREDCONST &  a symbol declared as a {\em predconst}; \\
   FUNSYM    &  a symbol declared as a {\em funconst} or {\em funpar}; \\
   FUNPAR    &  a symbol declared as a {\em funpar}; \\
   FUNCONST  &  a symbol declared as a {\em funconst}; \\ \hline 
   \end{tabular}
   \caption{
	Sorts for {\GF} syntactic categories.
   }
   \end{figure}

   If, for example,  the sort of the variable $x_1$  is SENTCONST,  we
   expect $x_1$ to denote any object that is declared  as  a {\em sentconst}
   in the object  context .  Therefore \ARG{arg1}, \ARG{arg2}, \SEQ, \ARG{argN}
   are objects in the object context, such that all \ARG{arg$_i$} are of
   the syntactic category corresponding to the sort of the variables $x_i$
   in \ARG{M-fact}.

   In the following we give an explanation of the major steps performed during
   the execution of the command {\tt reflect} \cite{giunchiglia3}~\footnote{
   This description is the generalization of the example reported below
   and copied from \cite{giunchiglia3}.}:

   \begin{enumerate}
   \item 
     In the object context, when \C{REFLECT} is executed, {\GF}, after parsing 
     \C{REFLECT}, knows that the next argument is the label of a fact in 
     {\meta}. 
     Thus, {\GF} switches context automatically and goes to {\meta}.
   \item
     In {\meta}, the first argument of the command (\ARG{M-fact}) is parsed 
     and the formula of the fact whose label is \ARG{M-fact} is returned:
     $\forall x_1 x_2\ldots x_n A(x_1, x_2,\ldots,x_n)$.
     The variables $x_1, x_2,\ldots,x_n$ must be instantiated to constants
     in {\meta} which will be the names of the objects in the object context
     \ARG{arg1}, \ARG{arg2}, \SEQ, \ARG{argN}.
     The syntactic type of these objects must correspond to the sort of the 
     variables in {\meta}.
     For instance, if the sort of $x_1$ is WFF, then it must be instantiated
     to a constant denoting a {\it wff} in the object context. At this point
     {\GF} knows that \ARG{arg1} must be a {\it wff} in the object context,
     and so {\em arg}$_1$ can be parsed.
     This step provides {\GF} with the information needed to parse \ARG{arg1},
     \SEQ, \ARG{argN} in the object context.
   \item 
     {\GF} switches to the object context and parses  \ARG{arg1}, \ARG{arg2}, 
     \SEQ, \ARG{argN}, using the parsing functions of the syntactic categories
     corresponding to the variables $x_1, x_2,\ldots,x_n$ respectively.
   \item 
     {\GF} switches to {\meta} and automatically declares $n$ constants in 
     {\meta}. Let them be $c_1, c_2, ... c_n$, with the sorts of $x_1, x_2, 
     ... x_n$ respectively.
     Any constant in $c_1, ... c_n$ is automatically ``attached'' to the objects
     returned by parsing \ARG{arg1}, \ARG{arg2}, \SEQ, \ARG{argN} respectively.
     The representation associated to these constants is the representation
     declared for their sorts (see the command {\bf represent}, section
     \ref{sec-rep-sort}).
   \item 
     Still in {\meta}, an universal elimination is performed on 
     $\forall x_1 x_2\ldots x_n A(x_1, x_2,\ldots,x_n)$ obtaining
     $A(c_1, \ldots ,c_n)$.
   \item 
     Still in {\meta}, the formula $A(c_1,...,c_n)$ is automatically evaluated
     by the command {\tt eval} (see section \ref{sec-eval}).
     In this step {\GF}, by using the command {\bf eval}, performs metalevel
     reasoning by computation in the model.
     This metalevel reasoning is used to compute a formula of the form 
     $Theorem(``w'')$.
     If the metalevel reasoning does not lead to $Theorem(``w'')$, then an error
     message is returned. Otherwise the evaluation of the ground term ``$w$''
     gives $w$, the formula of a fact to be asserted in the object context by
     $R_{down}$
   \item
     At this point the reflection rule $R_{down}$ can be applied.
     Thus {\GF} forgets the constants $c_1 ... c_n$ declared in {\meta} and the
     attachments, switches back to the object context and asserts a new fact
     whose formula is $w$ and whose dependencies are the union of the 
     dependencies of the facts in {\em arg$_1$ ... arg$_n$} if there are any.
   \end{enumerate}

	Let us define two context {\tt OBJ} and {\tt META} as follows:

	\gfsourcefile{example.tst}{
	  NAMECONTEXT META; \\
	  DECLARE SORT TERM WFF;\\
	  DECLARE PREDCONST THEOREM 1;\\
	  DECLARE FUNCONST mkequal (INDVAR, INDVAR) = WFF;\\
	  DECLARE INDVAR x [TERM];\\
	  AXIOM M1: forall x. THEOREM(mkequal(x,x));\\
	  DECREP TERM;\\
	  DECREP WFF;\\
	  REPRESENT \{TERM\} AS TERM;\\
	  REPRESENT \{WFF\} AS WFF;\\
	  ATTACH mkequal TO [TERM,TERM=WFF] mkequ;\\
	  MAKECONTEXT OBJ;\\
	  SWITCHCONTEXT OBJ;\\
	  DECLARE INDCONST c;\\
	  DECLARE INDVAR   x;\\
	  DECLARE FUNCONST f 2;
	}

	Let's now type the following lines in {\GF}:

	\begin{quote}\tt
	   ***** FETCH example.tst;
	   ...

	   ***** REFLECT M1 c;
	   1  c = c;
	   ***** REFLECT M1 f(x,f(c,c));
	   2  f(x,f(c,c)) = f(x,f(c,c))
	\end{quote}

	Let us describe  step by step what happened during  the execution of
	the two  previous command \C{REFLECT}.

	\begin{enumerate}
	\item
    In {\tt OBJ} the word \C{REFLECT}  is parsed.
    The next argument must be  the name of a fact  in {\meta}.  Thus {\GF}
    automatically switches context and goes to {\meta}.
	\item
    In  {\meta}, {\tt M1} is parsed  and the axiom
    {\tt forall  x.THEOREM(mkequ(x,x))} is returned.  The variable {\tt x}
    in {\tt M1} must be instantiated  to a constant  in {\meta} which will
    be the name of a symbol of the language  of {\tt OBJ}.   Since the
    sort of  $x$ is {\em TERM}, then  the symbol of  the language  of {\tt
    OBJ}  must have  syntactic  type {\em term}   (it must be  a term).
	\item
    {\GF} switches to the context {\tt OBJ}.
    In  {\tt OBJ} the   term  {\tt  c} [{\tt f(x,f(c,c))} in  the second case]
    is  parsed.  
	\item
    Since no more arguments are needed, {\GF} switches back to {\meta}.
    In {\meta} a new constant, say {\tt C1} of sort {\tt TERM} is created and
    added to the language of {\meta}.
    {\tt   C1} is {\em attached} to the term {\tt c}  [{\tt f(x,f(c,c))}] 
    of the language of {\tt OBJ}.  
    The representation associated to {\tt C1} is {\tt TERM} which is associated
    to the sort {\tt TERM} by the command {\tt REPRESENT} in example.tst.
    In this way,  {\tt C1} is defined  as the name in  {\meta} 
    of {\tt c} [{\tt f(x,f(c,c))}].   
	\item
    Still in {\meta}, an   universal elimination is performed on {\tt  M1}
    obtaining\\
    {\tt THEOREM(mkequ(C1,C1))}.
	\item
    Still in {\meta}, {\tt  THEOREM(mkequ(C1,C1))} is evaluated in {\meta}'s 
    model.
    {\tt THEOREM} has no interpretation, {\tt mkequ(C1,C1)} evaluates to 
    {\tt c = c} [{\tt f(x,f(c,c)) = f(x,f(c,c))}], namely {\tt mkequ(C1,C1)}
    turns out to be the name of {\tt c = c} [{\tt f(x,f(c,c)) = f(x,f(c,c))}].
    So the result of this step is something that could be written as 
    {\tt THEOREM(``c =  c'')}.
    [{\tt THEOREM(``f(x,f(c,c)) = f(x,f(c,c))'')}],  where {\tt ``c = c''}
    [{\tt ``f(x,f(c,c)) = f(x,f(c,c))''}]  should be read as  the  name  of
    {\tt c = c} [{\tt (f(f(c,c),c),f(x,c)) = f(x,f(c,c))}]
	\item
    At  this point the  reflection rule can be applied.
    {\GF} forgets everything in {\meta}, (in this case {\tt C1} from the
    language and {\tt c} [{\tt f(x,f(c,c))}] from the domain of the
   	interpretation).
    {\GF} switches back to the context {\tt OBJ}, and assert a new fact
    with wff  {\tt  c = c} [{\tt  f(x,f(c,c))}] and with the empty deplist.
    \end{enumerate}

    The following example shows how {\GF} computes the deplist of a fact derived 
    by a {\tt reflect} command whose arguments contain a fact.
}

\gfrecap{
	Reflection
}

\gfexample+
   <host prompt> cat example.tst
   NAMECONTEXT META;\\
   DECLARE SORT FACT WFF;\\
   DECLARE INDVAR fc [FACT];\\
   DECLARE FUNCONST wffof (FACT) = WFF;\\
   DECLARE PREDCONST THEOREM 1;\\
   DECREP FACT;\\
   DECREP WFF; \\
   REPRESENT \{WFF\} AS WFF; \\
   REPRESENT \{FACT\} AS FACT;\\
   ATTACH wffof TO [FACT = WFF] fact\-get\-wff;\\
   AXIOM M2: forall fc. THEOREM(wffof(fc));\\
   MAKECONTEXT OBJ;\\
   SWITCHCONTEXT OBJ;\\
   DECLARE SENTCONST A;

   <host prompt> GETFOL
   ...

   ***** FETCH example.tst;
   ***** ASSUME A;
   1   A     (1)
   ***** REFLECT M2 1;
   2   A     (1)
+



 
	% ........................... QUICK REFERENCE ..........................
	\newpage
	\appendix
	\section{Open problems}
\label{app-op}

If a {\tt reflect} command is used when a context {\tt META} does not
exist the error produced is that the relevant fact can not be found.
The error message should say that {\tt META} does not exist.

\gap
The {\tt subst} command has problems distinguishing
proof line labels from numbers in equalities.

\gap
The {\tt label} command does not check for overlaps between
labels for proof lines, axioms and theories.



	\newpage
	\newcommand{\syndes}[2]{\item[\parbox{\textwidth}{#1}]\hfill #2}
\newcommand{\module}[1]{\subsection{#1}}

\section{Syntax of commands}

\module{admin}

\begin{description}
\syndes{
   comment \ARG{separator} \OPT{\ARG{text}} \ARG{separator}
}
{}
%3rd-begin%3rd-end

\syndes{
   deflam \ARG{funname} \ARG{var-list} \ARG{form};
}
{}
%3rd-begin%3rd-end

\syndes{
   deflam \ARG{funname} \ARG{var-list} \ARG{form};
}
{}
%3rd-begin%3rd-end

\syndes{
   echo \ARG{separator} \OPT{\ARG{text}} \ARG{separator}
}
{}
%3rd-begin%3rd-end

\syndes{
   hgk \ARG{s-expr};
}
{}
%3rd-begin%3rd-end

\syndes{
   hgk \ARG{s-expr};
}
{}
%3rd-begin%3rd-end

\syndes{
   know natnums \OPT{\ARG{natnum1}, \SEQ \ARG{natnumN}};
}
{}
%3rd-begin%3rd-end

\syndes{
   load \ARG{file};
}
{}
%3rd-begin%3rd-end

\syndes{
   resetprompt;
}
{}
%3rd-begin%3rd-end

\syndes{
   setprompt to \ARG{s-expr};
}
{}
%3rd-begin%3rd-end

\syndes{
   setprompt to \ARG{s-expr};
}
{}
%3rd-begin%3rd-end

\syndes{
   show \ARG{option};
}
{}
%3rd-begin%3rd-end

\end{description}

\module{context}

\begin{description}
\syndes{
  copycontext \ARG{ctx-name};
}
{}
%3rd-begin%3rd-end

\syndes{
	copylex	\ARG{ctx-name};
}
{}
%3rd-begin%3rd-end

\syndes{
  makecontext \ARG{ctx-name};
}
{}
%3rd-begin%3rd-end

\syndes{
  namecontext \ARG{ctx-name};
}
{}
%3rd-begin%3rd-end

\syndes{
  reset;
}
{}
%3rd-begin%3rd-end

\syndes{
  switchcontext \ARG{ctx-name};
}
{}
%3rd-begin%3rd-end

\end{description}

\module{decide}

\begin{description}
\syndes{
  decide \ARG{wff}  \OPT{by \ARG{fact1} \ARG{fact2} \SEQ} using 
  \OPT{\{ \ARG{rewriter} \SEQ \}} \ARG{decider};
}
{}
%3rd-begin%3rd-end

\syndes{
  monad \ARG{wff} \OPT{by \ARG{fact1} \ARG{fact2} \SEQ};
}
{}
%3rd-begin%3rd-end

\syndes{
  monadeq \ARG{wff} \OPT{by \ARG{fact1} \ARG{fact2} \SEQ};
}
{}
%3rd-begin%3rd-end

\syndes{
  ptaut \ARG{wff} \OPT{by \ARG{fact1} \ARG{fact2} \SEQ};
}
{}
%3rd-begin%3rd-end

\syndes{
  taut \ARG{wff} \OPT{by \ARG{fact1} \ARG{fact2} \SEQ};
}
{}
%3rd-begin%3rd-end

\syndes{
  tauteq \ARG{wff} \OPT{by \ARG{fact1} \ARG{fact2} \SEQ};
}
{}
%3rd-begin%3rd-end

\end{description}

\module{eval}

\begin{description}
\syndes{
  assertsimp \ARG{simplabel}; 
}
{}
%3rd-begin%3rd-end

\syndes{
  attach \ARG{indconst}  to \ALT dar [ rep ] \ARG{sexpr};\\
  attach \ARG{sentconst} to T \ALT NIL \ALT UNDEF;\\
  attach \ARG{funconst} \ALT \ARG{predconst} to \ARG{atom};\\
  attach \ARG{funconst}  to [ \ARG{rep1}, \SEQ, \ARG{repN} = \ARG{repM} ]
  \ARG{atom};\\
  attach \ARG{predconst} to [ \ARG{rep1}, \SEQ, \ARG{repN} ] \ARG{atom};
}
{}
%3rd-begin%3rd-end

\syndes{
  decrep \ARG{replabel1} \OPT{\SEQ \ARG{replabelN}};
}
{}
%3rd-begin%3rd-end

\syndes{
  eval \ARG{wff} \ALT \ARG{fact} \ALT \ARG{term} \OPT{by \ARG{simpexpr}};
}
{}
%3rd-begin%3rd-end

\syndes{
  hardware \ARG{indconst} to \ALT dar \ARG{sexpr};\\
  hardware \ARG{indconst} to \ALT dar [ \ARG{rep} ] \ARG{sexpr};
}
{}
%3rd-begin%3rd-end

\syndes{
  let \ARG{\indconst} to \ALT dar [ \ARG{rep} ] \ARG{term};
}
{}
%3rd-begin%3rd-end

\syndes{
  represent \{ \ARG{sort1}, \SEQ, \ARG{sortN} \} as \ARG{rep} \ALT \ARG{*};
}
{}
%3rd-begin%3rd-end

\syndes{
  rewrite \ARG{wff} \ALT \ARG{fact} \ALT \ARG{term} \OPT{by \ARG{simpexpr}};
}
{}
%3rd-begin%3rd-end

\syndes{
  setbasicsimp \ARG{simplabel} at wffs \{ \ARG{wff1} \SEQ \ARG{wffN} \};\\
  setbasicsimp \ARG{simplabel} at facts \{ \ARG{fact1} \SEQ \ARG{factN} \};
}
{}
%3rd-begin%3rd-end

\syndes{
  simplify \ARG{wff} \ALT \ARG{fact} \ALT \ARG{term};
}
{}
%3rd-begin%3rd-end

\end{description}

\module{language}

\begin{description}
\syndes{
   awff \ARG{awff};
}
{}
%3rd-begin%3rd-end

\syndes{
   declare \OPT{\ARG{sentsym} \ALT \ARG{indsym}} \ARG{sym1} \SEQ \ARG{symN}; \\
   declare \OPT{\ARG{funsym}  \ALT \ARG{predsym}} \ARG{sym1} \SEQ \ARG{symN}
   \ARG{arity};\\
   declare \OPT{\ARG{funsym} \ALT \ARG{predsym}} \ARG{sym1} \SEQ \ARG{symN}
   1 \OPT{[ pre \OPT{= \ARG{prbp}} ]};\\
   declare \OPT{\ARG{funsym} \ALT \ARG{predsym}} \ARG{sym1} \SEQ \ARG{symN}
   2 \OPT{[ inf \OPT{= \ARG{lbp} \ARG{rbp}} ] };
}
{}
%3rd-begin%3rd-end

\syndes{
  declare sort \ARG{sym1} \SEQ \ARG{symN};
}
{}
%3rd-begin%3rd-end

\syndes{
  declare indconst \ALT indpar \ALT indvar \ARG{sym1} \SEQ \ARG{symN}
  [ \ARG{sortsym} ]; \\
  declare funconst \ALT funpar \ARG{sym1} \SEQ \ARG{symN}
  ( \ARG{sortsym1} \SEQ \ARG{sortsymN} ) = \ARG{sortsym};\\
}
{}
%3rd-begin%3rd-end

\syndes{
  extension \ARG{sort} by \ARG{extexpr};
}
{}
%3rd-begin%3rd-end

\syndes{
  moregeneral \ARG{sort1} $<$ \ARG{sort2}, \SEQ, \ARG{sortN} $>$;
}
{}
%3rd-begin%3rd-end

\syndes{
  mostgeneral \ARG{sym};
}
{}
%3rd-begin%3rd-end

\syndes{
  setfmap \ARG{funsym} ( \ARG{sym1} \SEQ \ARG{symN} ) = \ARG{sym};
}
{}
%3rd-begin%3rd-end

\syndes{
   term \ARG{term};
}
{}
%3rd-begin%3rd-end

\syndes{
   wff \ARG{wff};
}
{}
%3rd-begin%3rd-end

\end{description}

\module{meta}

\begin{description}
\syndes{
   mattach \ARG{indconst} to \ALT dar \OPT{[rep]}
   \ARG{cname}:\ARG{pname}:\ARG{sort}:\ARG{object};
}
{}
%3rd-begin%3rd-end

\syndes{
  reflect \ARG{M-fact} \ARG{arg1} \ARG{arg2} \SEQ \ARG{argN};
}
{}
%3rd-begin%3rd-end

\end{description}

\module{nd}

\begin{description}
\syndes{
   alle \ALT us \ARG{fact} \OPT{,} \ARG{term1} \ARG{term2} \SEQ;
}
{}
%3rd-begin%3rd-end

\syndes{
   alli \ALT ug \ARG{fact} \OPT{\OPT{,} \ARG{indvar1} \ALT \ARG{indpar1} :} 
                           \ARG{indvar11}
                           \OPT{\OPT{,} \ARG{indvar2} \ALT \ARG{indpar2} :}
                           \ARG{indvar22} \SEQ;
}
{}
%3rd-begin%3rd-end

\syndes{
   ande \ALT ae \ARG{fact} \OPT{,} 1 \ALT 2; \\
   ande \ALT ae \ARG{fact} \OPT{,} 1 \ALT 2 1 \ALT 2 \SEQ;
}
{}
%3rd-begin%3rd-end

\syndes{
   andi \ALT ai \ARG{fact1} \OPT{,} \ARG{fact2}; \\
   andi \ALT ai \ARG{fact11}
                \OPT{conj \ALT cj \ARG{fact12} conj \ALT cj \ARG{fact13} \SEQ}
                \OPT{,}
                \ARG{fact21}
                \OPT{conj \ALT cj \ARG{fact22} conj \ALT cj \ARG{fact23} \SEQ};
}
{}
%3rd-begin%3rd-end

\syndes{
   assume \ARG{wff1} \OPT{\OPT{,} \ARG{wff2} \SEQ};
}
{}
%3rd-begin%3rd-end

\syndes{
   existe \ALT es \ARG{fact}  \OPT{,} \ARG{indvar1} \ALT \ARG{indpar1}
                  \OPT{,} \ARG{indvar2} \ALT \ARG{indpar2} \SEQ;
}
{}
%3rd-begin%3rd-end

\syndes{
  existi \ARG{fact}
   \OPT{\OPT{,} \ARG{term1} :} \ARG{indvar1} \OPT{occ \ARG{n11} \ARG{n12} \SEQ}
   \OPT{\OPT{,} \ARG{term2} :} \ARG{indvar2} \OPT{occ \ARG{n21} \ARG{n22} \SEQ}
   \SEQ;
}
{}
%3rd-begin%3rd-end

\syndes{
   falsee \ALT fe \ARG{fact1} \OPT{,} \ARG{wff};\\
   falsee \ALT fe \ARG{fact1} \OPT{,} \ARG{fact2};
}
{}
%3rd-begin%3rd-end

\syndes{
   falsei \ALT fi \ARG{fact1} \OPT{,} \ARG{fact2}; 
}
{}
%3rd-begin%3rd-end

\syndes{
   iffe \ALT ie \ARG{fact} \OPT{,} 1 \ALT 2;
}
{}
%3rd-begin%3rd-end

\syndes{
   iffi \ALT ii \ARG{fact1} \OPT{,} \ARG{fact2};
}
{}
%3rd-begin%3rd-end

\syndes{
   impe \ALT mp \ARG{fact1} \OPT{,} \ARG{fact2};
}
{}
%3rd-begin%3rd-end

\syndes{
   impi \ALT ded  \ARG{fact1} \OPT{, \ALT imp} \ARG{fact};\\
   impi \ALT ded  \ARG{wff} \OPT{, \ALT imp} \ARG{fact};
}
{}
%3rd-begin%3rd-end

\syndes{
   note \ALT ne \ARG{fact1} \OPT{,} \ARG{wff}; \\
   note \ALT ne \ARG{fact1} \OPT{,} \ARG{fact2};
}
{}
%3rd-begin%3rd-end

\syndes{
   noti \ALT ni \ARG{fact1} \OPT{,} \ARG{wff}; \\
   noti \ALT ni \ARG{fact1} \OPT{,} \ARG{fact2};
}
{}
%3rd-begin%3rd-end

\syndes{
   ore \ALT oe \ARG{fact1} \OPT{,} \ARG{fact2} \OPT{,} \ARG{fact2};
}
{}
%3rd-begin%3rd-end

\syndes{
   ori \ALT oi \ARG{fact} \OPT{,} \ARG{wff} \OPT{,} \OPT{lr \ALT rl};\\
   ori \ALT oi \ARG{fact} \OPT{,} \ARG{fact1} \ALT \ARG{wff1} 
       disj \ALT dj \ARG{fact2} \ALT \ARG{wff2} disj \ALT dj \SEQ
       \OPT{,} \OPT{lr \ALT rl};
}
{}
%3rd-begin%3rd-end

\syndes{
  subst \ARG{fact1} \OPT{with} \ARG{fact2};\\
  subst \ARG{fact1} \OPT{with} \ARG{fact2} \OPT{right \ALT left};\\
  subst \ARG{fact1} \OPT{with} \ARG{fact2} 
                    \OPT{occ \ARG{n1} \ARG{n2} \SEQ} \OPT{right \ALT left};
}
{}
%3rd-begin%3rd-end

\end{description}

\module{parser}

\begin{description}
\syndes{
  backup \ARG{file} open;\\
  backup \ARG{file} close;  
}
{}
%3rd-begin%3rd-end

\syndes{done;}
{}
%3rd-begin%3rd-end

\syndes{
   fetch \ARG{file} \OPT{from \ARG{mark1}} \OPT{to \ARG{mark2}};
}
{}
%3rd-begin%3rd-end

\syndes{
   mark \ARG{sym};
}
{}
%3rd-begin%3rd-end

\syndes{
   probe;\\
   probe \ARG{activity};\\
   probe all;\\
}
{}
%3rd-begin%3rd-end

\syndes{
   unprobe \ARG{activity};\\
   unprobe all;
}
{}
%3rd-begin%3rd-end

\end{description}

\module{proof}

\begin{description}
\syndes{
  axiom \ARG{sym} : \ARG{wff};
}
{}
%3rd-begin%3rd-end

\syndes{
   cancel \OPT{\ARG{label}};
}
{}
%3rd-begin%3rd-end

\syndes{
  copyproof \ARG{prf-name};
}
{}
%3rd-begin%3rd-end

\syndes{
   label fact \ARG{sym};\\
   label fact \ARG{sym} = \ARG{label};
}
{}
%3rd-begin%3rd-end

\syndes{
   makeproof \ARG{prf-name};
}
{}
%3rd-begin%3rd-end

\syndes{
   nameproof \ARG{prf-name};
}
{}
%3rd-begin%3rd-end

\syndes{
  switchproof \ARG{prf-name};
}
{}
%3rd-begin%3rd-end

\syndes{
  theorem \ARG{sym} \ARG{hook};
}
{}
%3rd-begin%3rd-end

\syndes{
  theory \ARG{thlabel} : \ARG{wff1} \OPT{\ARG{wff2} \SEQ};\\
  theory \ARG{thlabel} : \ARG{axlabel1} : \ARG{wff}
                         \OPT{\ARG{axlabel2} : \ARG{wff} \SEQ};
}
{}
%3rd-begin%3rd-end

\end{description}

\module{rules}

\begin{description}
\syndes{
	contract \ALT ctc  \ARG{fact} by \ARG{assumption1} \SEQ  \ARG{assumptionN}; 
}
{}
%3rd-begin%3rd-end

\syndes{
	cut \ARG{fact1} \ARG{fact2};\\
	cut \ARG{fact1} \ARG{fact2} \OPT{keep \ARG{assumption1} \SEQ
	\ARG{assumptionN}};
}
{}
%3rd-begin%3rd-end

\syndes{
	termife \ARG{fact1} \ARG{fact2} \ARG{termif}; \\
	termife \ARG{fact1} \ARG{fact2} \ARG{termif} \OPT{occ \ARG{n1} \ARG{n2}
	\SEQ}; 
}
{}
%3rd-begin%3rd-end

\syndes{
	termifen \ARG{fact1} \ARG{fact2} \ARG{termif}; \\
	termifen \ARG{fact1} \ARG{fact2} \ARG{termif} \OPT{occ \ARG{n1} \ARG{n2}
	\SEQ}; 
}
{}
%3rd-begin%3rd-end

\syndes{
	termifi \ARG{fact1} \ARG{fact2} \ARG{wff} \ARG{term1} \ARG{term2};
}
{}
%3rd-begin%3rd-end

\syndes{
	weaken \ALT wk \ARG{fact} by \ARG{fact1} \SEQ \ARG{factN};
}
{}
%3rd-begin%3rd-end

\syndes{
	wffife \ARG{fact1} \ARG{fact2};
}
{}
%3rd-begin%3rd-end

\syndes{
	wffifen \ARG{fact1} \ARG{fact2};
}
{}
%3rd-begin%3rd-end

\syndes{
	wffifi \ARG{wff} \ARG{fact1} \ARG{fact2};
}
{}
%3rd-begin%3rd-end

\end{description}



	% ............................. BIBLIOGRAPHY ...........................
	\newpage
	\bibliographystyle{alpha}
    \gfbibliography
	
	% ................................ INDEX ...............................
	\newpage
	%............................... USER MANUAL .................................
%.............................................................................

\documentstyle[12pt,rawfonts]{../styfiles/GFmanual}

\title{GETFOL Manual}
\author{\bf Fausto Giunchiglia}
\date{7 March 1994}
\version{2.0}
\abstract{
	  {\GF} is an interactive reasoning system.
	  We use it as an environment for studying epistemological issues.
	  We try to look at questions like: which notions are
	  important for the development of mechanized reasoning systems?
	  What kind of conversations do we want to have with them?
	  What parts of logic should we use to represent such notions?
	  How should logic be embedded in a conversational reasoning system?
	}
\addresses{
     \begin{tabular}[c]{l}
       {\bf Fausto Giunchiglia}           \\
       Mechanized Reasoning Group		  \\
		 IRST, Povo, 38050 Trento, Italy  \\
		 e-mail: {\tt fausto@irst.it}     \\
		 phone: +39 461 314436
	\end{tabular}	
}
\published{
  \begin{tabular}{l}
	  DIST Technical Report No. 92-0010 (1994). \\
	  DIST -- University of Genoa,\\
	  Via Opera Pia 11A, 16145 Genova, Italy.\\ \\
  \end{tabular}
}

%% \newcommand{\gfbibliography}{%
%% \bibliography{/home/tarski/staff/mrg/biblio/bib/a-l,%
%% /home/tarski/staff/mrg/biblio/bib/m-z,userman}}
\newcommand{\gfbibliography}{\bibliography{}}

%% \makeindex
\begin{document}
	%  ............................. COVER ..................................
	\thispagestyle{empty}
	\maketitle

	%  ........................ TABLE OF CONTENTS ...........................
	\newpage
	\pagenumbering{roman}
	\tableofcontents

	\newpage
	\pagenumbering{arabic}
	\pagestyle{headings}

	%  ........................... INTRODUCTION ............................
	\section{Introduction}

The distribution package allows the installation of the following three
applications:
%
\begin{itemize}
	\item
		the {\tt HGKM} interpreter \cite{giunchiglia35}, that is a LISP-like
		language acting as the implementation language of both {\tt FOL}
		and {\tt GETFOL}.
	\item
		the {\tt GETFOL} system \cite{giunchiglia12,giunchiglia29}.
	\item
		the {\tt AGETFOL} system ({\tt GETFOL} with {\em abstraction}
		\cite{giunchiglia7}.
\end{itemize}

From now on, we will write ``{\tt the system}" to mean {\tt HGKM},
{\tt GETFOL} and {\tt AGETFOL}.
The distribution package is provided with a set of general purpose
configuration and installation facilities which allow you to install your
favourite application on your machine.

In order to install {\tt the system} read section~\ref{requirements}
(``{\it Minimal System Requirements}'') and verify that your machine
has all the required features.
Then read section~\ref{instproc} (``{\it The installation procedure}'').
This section gives the instructions to follow in order to install
{\tt the system} on your machine.

If you are an expert you might be interested also in the features
of the installation and configuration facilities.
Section~\ref{sysmod} (``{\it Systems and Modules}'') introduces
to the data structures ({\it systems} and {\it modules}) devised
to store and/or modify the configuration of the applications.


	%  .............................. MODULES ..............................
	% loading introduction to the section
\section{Parser}

This section is intended to explain:
%
\begin{itemize}
	\item
		the main functionalities of the {\GF} scanning primitives,
	\item
		what modifications must be performed in the scanner data
		structures in order to be able to change its behavior ({\em e.g.} to
		define a new escape character).
\end{itemize}

The {\GF} scanner is a {\em backupable} scanner.
This feature is necessary as the parser works in a top-down fashion and,
sometimes, needs to backtrack. 

The {\GF} scanner is able to recognize three types of tokens:
%
\begin{enumerate}
	\item {\tt IDTOKEN},
	\item {\tt NUMTOKEN} and
	\item {\tt DELTOKEN}.
\end{enumerate}

The primitives to scan such tokens are \verb+FOLSYM@+, \verb+NATNUM@+ (actually
no dedicated primitive to scan a \verb+DELTOKEN+ exists).

\verb+TK@+ scans a generic token.

The primitives \verb+SCANSTATUS-GET+ and \verb+SCANSTATUS-RESTORE+ allow
respectively to save and restore the scanner status.
They are necessary to restore the status of the scanner in case of
unsuccessful parsing.

The high level functions \verb+FOLSYM@+, \verb+NATNUM@+, \verb+TK@+ (and other
not mentioned here for brevity) use the general \verb+TOKEN-GET-NEXT+ routine
which is also able to {\em bufferize} the tokens already read (reading from
an input stream is a destructive operation so we need at a some level a
buffer mechanism if we want to perform backtracking).

The buffer is implemented by the array \verb+TOKENARRAY+ whose dimension is
given by the macro \verb+TOKENARRAY-DIMENSION+. The max number of tokens
that can be present in a command line is given by the number returned
by this macro.

\verb+TOKEN-GET-NEXT+ routine is based on the lower level primitive
\verb+TOKEN-SCAN+, which reads the next token from the input stream and returns
its type.
\verb+TOKEN-SCAN+ is able to identify the type of a token reading (via 
\verb+CH-GET-NEXT+) its first character and identifying its type (via
\verb+CHTYPE-GET+).

The characters are divided in the following types:

\begin{enumerate}
	\item
		identifiers ({\tt IDCHAR}): see file
		\verb+ascitab.fol+;
	\item
		numbers (positive integers --- {\tt NUMCHAR}):
		\verb+0 1 2 3 4 5 6 7 8 9+;
	\item
		delimiter {\tt DELCHAR}:
		\verb+( ) , . : ; [ ] { }+;
	\item
		ignored char {\tt IGNCHAR}: see file
		\verb+ascitab.fol+;
	\item
		iddelim {\tt IDDELCHAR}:
        \verb$' * + - / < > = ? @ ^ ` |$;
	\item
		escape characters {\tt ESCCHAR}:
		\verb+\+;
	\item
		special handling {\tt SPECCHAR}
\end{enumerate}

The \verb+IDDELCHAR+s are identifier characters having also the functionality
of a delimiter, that is no \verb+IDTOKEN+ can contain a character of such
type but they by themselves may be considered \verb+IDTOKEN+.
For instance the string \verb+A@B+ will be regarded by the scanner a string
formed by the three distinct \verb+IDTOKEN+ \verb+A @ B+.
The escape characters allows to coerce the type of the following
character to be \verb+IDCHAR+.
For istance the string \verb+A\@B+ will be regarded by the scanner a string
formed by the \verb+IDTOKEN A@B+.

The only think important to know at the user level it is how to modify the type
of a character so that you can, for istance, extend the set of \verb+IDDELCHAR+
with the character "\verb+_+". To do this you have only to change type
declaration for the "\verb+_+" character from \verb+IDCHAR+ to \verb+IDDELCHAR+
in the file {\tt asciitab.fol}.

Finally a low level feature: each time a command line is issued to {\GF} the low
level scanning primitives store each read character in the {\tt TUPLE
SCANBUFARRAY}, so that if a syntactic error in the parsing is detected the rest
of the line will be read and the whole line printed out to show, using an
"\verb+^+", where the syntatic error has been detected. This useful feature
imposes (as it is implemented) a limit in the length of a {\GF} command line as
the dimension of \verb+SCANBUFARRAY+ is fixed by the value returned by the macro
\verb+SCANBUFARRAY-DIMENSION+.


% loading command files
\gfcommand{backup}{backup of a {\GF} session}
\index{backup}

\gfsyntax{
  backup \ARG{file} open;\\
  backup \ARG{file} close;  
}

\gfdescription{
   {\GF} stores in \ARG{file} all the successful commands between the command
   ``{\tt backup \ARG{file} open}" and the command ``{\tt backup \ARG{file}
   close}". 
}

\gfrecap{
Stores in a file all successfull commands type in GETFOL.
}

\gfexample+
   ***** backup file open;
   I am starting to backup onto file
   
   ***** declare sentconst A;
   ***** assume A;
   1   A     (1)
   
   ***** backup file close;
   
   ***** andi 1 1;
   2   A and A     (1)
   
   ***** ^D
   
   >Bye.
   
   <host-prompt> more file
   
   declare sentconst A;
   
   assume A;
   
   <host-prompt>
+

\gfnotes{
   Only Unix file names are supported.
   The file path name can be absolute or relative to the current directory.
   Multiple open backup files can exist simultaneously.
   Files must be explicitly closed to guarantee that all the commands
   are properly stored in the backup file.
}

\gfcommand{done}{exiting a {\GF} session}
\index{done}

\gfsyntax{done;}

\gfdescription{
   Returns the control back to the {\HG} environment~\cite{giunchiglia35}.
   You can get back to {\GF} by typing {\tt (SYSBACK)} at the {\HG} prompt.
}

\gfrecap{
Returns the control back to the HGKM environment.
You can get back to GETFOL by typing (SYSBACK) at the HGKM prompt.
}

\gfexample+
   **** done;
   Returning to host
   NIL
+

\gfnotes{}

\gfcommand{fetch}{fetches {\GF} file}
\index{fetch}

\gfsyntax{
   fetch \ARG{file} \OPT{from \ARG{mark1}} \OPT{to \ARG{mark2}};
}

\gfdescription{
   Redirects the standard input to the file \ARG{file}; all {\GF} commands
   between \ARG{mark1} and \ARG{mark2} are executed.
}

\gfrecap{
Fetches commands from the file `file'.
All commands between `mark1' and `mark2' are executed.
}

\gfexample+
  ***** fetch exmarks.tst;
  ...
  ***** fetch exmarks.tst to m1;
  ...
  ***** fetch exmarks.tst from m1 to m2;
  ...
  ***** fetch exmarks.tst from m2;
+

\gfnotes{
   Only Unix file names are supported.
   The file path name can be absolute or relative to the current directory.
   Nested fetches (and marking) are allowed.
   If no marks are specified, then the whole file is fetched.
   See the command {\tt mark} in this section to set marks in a file.
}


\gfcommand{mark}{sets a mark}
\index{mark}

\gfsyntax{
   mark \ARG{sym};
}

\gfdescription{
   Sets a mark in between a sequence of {\GF} commands.
   It can be used to fetch a file from/to a certain mark (see command 
   {\tt fetch} in this section).
}

\gfrecap{
Sets a mark in a file (see fetch).
}

\gfexample+
   <host-prompt> more example
   NAMECONTEXT META;
   DECLARE SORT FACT WFF;
   DECLARE INDVAR fc [FACT];
   DECLARE FUNCONST wffof (FACT) = WFF;
   DECLARE PREDCONST THEOREM 1;
   comment | here we put the first mark m1 |
   mark m1;
   DECREP FACT;
   DECREP WFF;
   REPRESENT \{WFF\} AS WFF;
   REPRESENT \{FACT\} AS FACT;
   comment | here we put the second mark m2 |
   mark m2;
   ATTACH wffof TO [FACT = WFF] fact\-get\-wff;
   AXIOM M2: forall fc. THEOREM(wffof(fc));
   MAKECONTEXT OBJ;
   SWITCHCONTEXT OBJ;
   DECLARE SENTCONST A;
+



\gfcommand{probe}{verbose mode}
\index{probe}

\gfsyntax{
   probe;\\
   probe \ARG{activity};\\
   probe all;\\
}

\gfdescription{
	Some {\GF} {\em activities} can be executed either in verbose
	or in silent mode.
	In the former, {\GF} displays messages describing the execution of the
	command which are not displayed in the silent mode.
	Examples of activities are given below. 

	{\tt probe} lists the probed commands

	{\tt probe} \ARG{activity} sets commands in the activity to be executed
	in a verbose mode. 

	{\tt probe all} sets all activities to verbose mode.
}

\gfrecap{
Set verbose mode for certain activities.
}

\gfexample+
   ***** probe;
   Probing function set : COMMAND no
   Probing function set : IO yes
   Probing function set : DECLARE yes
   Probing function set : PROOF yes
   Probing function set : ATTACH yes
   Probing function set : SIMPLIFY yes
   Probing function set : SIMPSET yes
   Probing function set : REWRITE yes
   Probing function set : EVAL yes
   Probing function set : CONTEXT yes
   Probing function set : REFLECT yes
   
   ***** declare sentconst A B C;
   A has been declared to be a Sentconst
   B has been declared to be a Sentconst
   C has been declared to be a Sentconst
   
   ***** probe command;
   
   ***** probe;
   probe;
   Probing function set : COMMAND yes
   Probing function set : IO yes
   Probing function set : DECLARE yes
   Probing function set : PROOF yes
   Probing function set : ATTACH yes
   Probing function set : SIMPLIFY yes
   Probing function set : SIMPSET yes
   Probing function set : REWRITE yes
   Probing function set : EVAL yes
   Probing function set : CONTEXT yes
   Probing function set : REFLECT yes
   
   ***** declare sentconst D E F;
   declare sentconst D E F;
   D has been declared to be a Sentconst
   E has been declared to be a Sentconst
   F has been declared to be a Sentconst
   
   ***** unprobe all;
   unprobe all;
   
   ***** declare sentconst H G K;
+

\gfnotes{}

\gfcommand{unprobe}{silent mode}
\index{unprobe}

\gfsyntax{
   unprobe \ARG{activity};\\
   unprobe all;
}

\gfdescription{
   Some {\GF} {\em activities} can be executed either in verbose
   or in silent mode.

   {\tt unprobe} \ARG{activity} sets the silent mode for commands in
   \ARG{activity}.

   {\tt unprobe all} sets all activities to silent mode .
}

\gfrecap{
Set/unset verbose mode for certain activities.  
}

\gfexample+
   ***** declare sentconst D E F;
   declare sentconst B E F;
   D has been declared to be a Sentconst
   E has been declared to be a Sentconst
   F has been declared to be a Sentconst
   
   ***** unprobe all;
   unprobe all;

   ***** probe;
   Probing function set : COMMAND no
   Probing function set : IO no
   Probing function set : DECLARE no
   Probing function set : PROOF no
   Probing function set : ATTACH no
   Probing function set : SIMPLIFY no
   Probing function set : SIMPSET no
   Probing function set : REWRITE no
   Probing function set : EVAL no
   Probing function set : CONTEXT no
   Probing function set : REFLECT no
   
   ***** declare sentconst H G K;
+

\gfnotes{}


	% loading introduction to the section
\newpage
\section{Administration}
\label{sec-adm}

The commands described in this section manipulate the proof checker but do
not modify  the ``logical'' state of the deduction or of the computation.
Among the other things, they can be used to give alternative names to proof lines,
to load {\HG} or {\GF} files, to insert comments in {\GF} files, to show the
logical/computational status of the system.


% loading explanation of commands
\gfcommand{comment}{comments in {\GF}}
\index{comment}

\gfsyntax{
   comment \ARG{separator} \OPT{\ARG{text}} \ARG{separator}
}

\gfdescription{
   Defines a comment \ARG{text} between the two \ARG{separator}s.
   Any token can be a separator.
}

\gfrecap{
Defines a comment between two separators.
Any token can be a separator.
}

\gfexample+
   ***** comment ! this is a comment 
   enclosed between two exclamation marks !
   
   ***** comment ? this is a comment 
   enclosed between two question marks ?
   
   ***** comment com this is a comment 
   enclosed between the two words "com" com
+


\gfcommand{deflam}{defining {\HG} functions}
\index{deflam}

\gfsyntax{
   deflam \ARG{funname} \ARG{var-list} \ARG{form};
}

\gfdescription{
   Defines a {\HG} function at the {\GF} prompt.
   The function \ARG{funname} is defined as the {\HG} \ARG{form}
   with parameters the parameters in the list \ARG{var-list}.
}

\gfrecap{
Defines a HGKM function at the GETFOL prompt.
}

\gfexample+
   ***** deflam wffof (fact) (fact\-get\-wff fact);
+


\gfcommand{echo}{echoes a message to the standard output stream}
\index{echo}

\gfsyntax{
   echo \ARG{separator} \OPT{\ARG{text}} \ARG{separator}
}

\gfdescription{
   Echoes \ARG{text} between the two \ARG{separator}s to the current
   output stream.
   Any token can be a \ARG{separator}.
}

\gfrecap{
Echoes text between the two separators.
}

\gfexample+
   ***** echo ! this is an echo
   enclosed between two exclamation marks !
   this is an echo enclosed between two exclamation marks

   ***** comment ? this is an echo
   enclosed between two question marks ?
   this is an echo enclosed between two question marks

   ***** comment com this is a echo
   enclosed between the two words "com" com
   this is an echo enclosed between the two words "com"
+

\gfcommand{hgk}{{\HG} evaluation}
\index{hgk}

\gfsyntax{
   hgk \ARG{s-expr};
}

\gfdescription{
   Runs the {\HG} evaluator on \ARG{s-expr}.
}

\gfrecap{
Runs the HGKM evaluator on `s-expr'.   
}

\gfexample+
   ***** hgk (LOAD (QUOTE "file"))
+

\gfnotes{
	The command hgk tries to evaluate every element of the {\em
	s-expression} passed as argument; therefore \verb+(LOAD "file")+
	causes an error, if \verb+"file"+ has no value.
}
\gfcommand{know natnums}{allows the use of natural numbers}
\index{know natnums}

\gfsyntax{
	know natnums \OPT{\ARG{natnum1}, \SEQ, \ARG{natnumN}};
}

\gfdescription{
	It allows the use of \ARG{natnum1}, \SEQ, \ARG{natnumN}
	(all natural numbers if no natural numbers are explicitly listed).
	It declares the sort {\tt NATNUMSORT}, the representation 
	{\tt NATNUMREP} and optionally defines the extension of {\tt NATNUMSORT} 
	to be \ARG{natnum1}, \SEQ, \ARG{natnumN} (if explicitly listed).
}

\gfexample+
   ***** know natnums;
   ***** simplify (5=7);
   1   not (5 = 7)   
   ***** know natnums 1 2 3 4;
   Warning! You already know natnums.
   Now the extension of NATNUMSORT is fixed to be : (1 2 3 4)
   ***** simplify (5=7);
   simplify (5=7);
            ^
   SIMPLIFY requires a wff,fact or term here
   ***** simplify (1=3);
   2   not (1 = 3)
+

\gfcommand{load}{loading a {\HG} file}
\index{load}

\gfsyntax{
   load \ARG{file};
}

\gfdescription{
   Each form in \ARG{file} is read by the {\HG} reader and evaluated by the 
   {\HG} evaluator \cite{giunchiglia35}.
}

\gfrecap{
Each form in `file' is read and evaluated by HGKM.
}

\gfexample+
   ***** load example.hgk;
+


\gfcommand{resetprompt}{Resets the user defined prompt}
\index{resetprompt}

\gfsyntax{
   resetprompt;
}

\gfdescription{
   Comes back to the default prompt.
}
\gfrecap{
Comes back to the default prompt.
}

\gfexample+
   CARTOONIA:: resetprompt;

   ***** switchcontext Disneyland;
   You are now using context: Disneyland
   You are switching to a proof with no name.

   *****
+

\gfcommand{setprompt}{Redefines the prompt}
\index{setprompt}

\gfsyntax{
   setprompt to \ARG{s-expr};
}

\gfdescription{
   Sets {\GF}'s prompt  to the value of the s-expression
   \ARG{s-expr}  followed by  ":: ". It is particularly  useful when  you are
   working on multiple contexts as you can set the prompt to the value of
   the current context.
}

\gfrecap{
Sets GETFOL's prompt to `s-expr'.
}

\gfexample+
   ***** setprompt to (QUOTE myprompt);

   myprompt:: setprompt to (QUOTE Tweedledee\&Tweedledum);

   Tweedledee&Tweedledum:: setprompt to (CAPITALIZE (curcname\-get));

   NOTNAMED&:: namecontext Disneyland;
   You have named the current context: Disneyland

   DISNEYLAND:: makecontext Cartoonia;
   You have created the empty context: Cartoonia

   DISNEYLAND:: switchcontext Cartoonia;
   You are now using context: Cartoonia
   You are switching to a proof with no name.

   CARTOONIA:: 
+

\gfnotes{
	The command tries to evaluate the {\em s-expression} passed as argument.
	Failure of the evaluation causes a crash to the {\HG} evalautor.
}

\gfcommand{show}{shows {\GF} information}

\gfsyntax{
   show \ARG{option};
}

\gfdescription{
   Shows {\GF} information. In the example we show some of the
   options implemented in a {\GF} version.
   Notice that ``\verb+show show;+" lists the options supported by the show
   command.
}

\gfrecap{
Shows GETFOL information.
}

\gfexample+
   ***** comment | ******* SHOW SHOW ******* |
   ***** show show;
   The list of show options is the following:

   CONTEXT : WHEREAMI 

   REWRITER : SIMPSET 

   SIMPLIFIER : INT REP 

   DEFINITION : DEFINITION 

   PROOF : PREMISES FACT AXIOM 

   LANGUAGE : LGS MGS MEM SORT TYP SYM 

   SYSTEM : COM SHOW 
   
   ***** comment | ******* SHOW WHEREAMI ******* |
   
   ***** show whereami;
   You are now using an unnamed context.
   You are now using an unnamed proof.
   
   ***** comment | ******* SHOW REP and INT ******* |
   
   ***** decrep PIPPOREP;
   ***** attach A dar [PIPPOREP] a;
   ***** attach C dar [PIPPOREP] c;
   ***** attach B to  [PIPPOREP] b;
   
   ***** show rep PIPPOREP;
   The designators for the representation: PIPPOREP are:
   (c . C) (a . A)
   
   ***** show int A;
   The Indconst A is attached to 'a
   with representation PIPPOREP
   
   ***** comment | ******* SHOW SIMPSET ******* |
   
   ***** setbasicsimp simp1 at wffs
   {forall x1.F1(x1)=x1,
   forall x1 x2.F2(x1 x2)=x1,
   forall x1 x2 x3.F3(x1 x2 x3)=x1};
   
   ***** show simpset simp1;
   Wffs :
   forall x1. (F1(x1) = x1)
   forall x1 x2. (F2(x1,x2) = x1)
   forall x1 x2 x3. (F3(x1,x2,x3) = x1)
   
   ***** setbasicsimp simp2 at facts {1 2};
   
   ***** show simpset simp2;
   Proof lines :  1 2
   
   ***** comment | AX1 and AX2 are two axioms |
   ***** setbasicsimp simp3 at facts {AX1 AX2};
   ***** show simpset simp3;
   Axioms :  AX1 AX2
   
   ***** setcompsimp simpA at simp1 uni simp2 uni simp3;
   ***** show simpset simpA;
   simpA is compound by this list of basic simpsets :
   (simp1 simp2 simp3)
   
   ***** SETCOMPSIMP simpB at simpA dif simp1;
   ***** show simpset simpB;
   simpB is compound by this list of basic simpsets :(simp2 simp3)
   
   
   ***** comment | ******* SHOW PROOF, FACT and AXIOM ******* |
   
   ***** show proof;
   1   forall x1. (F1(x1) = x1)     (1)
   2   forall x1 x2. (F2(x1,x2) = x1)     (2)
   
   ***** label fact identity = 1;
   ***** show fact;
   identity   1
   
   ***** show axiom;
   AX1 : forall x1. (F1(x1) = x1)
   AX2 : forall x1 x2. (F2(x1,x2) = x1)
   
   ***** comment | ******* SHOW LSG, MGS , MEM and SORT ******* |
   
   ***** declare sort a b c d e f g h;
   moregeneral a < b c d e f g h >;
   moregeneral b < c d e f g h >;
   
   ***** show lgs a;
   No sort is strictly lessgeneral than a.
   ***** show lgs b;
   No sort is strictly lessgeneral than b.
   
   ***** show mgs a;
   The only sort strictly moregeneral than a is UNIVERSAL
   ***** show mgs f;
   The only sort strictly moregeneral than f is UNIVERSAL
   
   ***** show mem a;
   No <indsym> is declared to be of sort a.
   
   ***** declare indvar x [a];
   a is a sort
   x has been declared to be an Indvar
   ***** declare indvar y [a];
   a is a sort
   y has been declared to be an Indvar
   ***** show mem a;
   The <indsym>'s declared to be of sort a are
       y  x  
   
   ***** show sort;
   The symbols declared to be sorts are
       b  a  UNIVERSAL  
   
   ***** comment | ******* SHOW TYP, SYM ******* |
   
   ***** show typ INDVAR;
   The symbols declared to be Indvars are
       x  y  
   
   ***** show sym a;
   a is declared to be a sort.
   ***** show sym x;
   x is declared to be an Indvar of sort a.
   
   
   ***** comment | **************** SHOW PREMISES *************** |
   
   ***** declare sentconst A B C;
   ***** assume A B;
   1   A     (1)
   2   B     (2)
   ***** andi 1 2;
   3   A and B     (1 2)
   ***** ori 3 C;
   4   (A and B) or C     (1 2)
   ***** andi 3 4;
   5   (A and B) and ((A and B) or C)     (1 2)
   ***** show premises 5;
   5  (A and B) and ((A and B) or C)  (1 2)
      3  A and B  (1 2)
      4  (A and B) or C  (1 2)
   ***** show premises 5 2;
   5  (A and B) and ((A and B) or C)  (1 2)
      3  A and B  (1 2)
         1  A  (1)
         2  B  (2)
      4  (A and B) or C  (1 2)
         3  A and B  (1 2)
   *****  show premises 5 all;
   5  (A and B) and ((A and B) or C)  (1 2)
      3  A and B  (1 2)
         1  A  (1)
         2  B  (2)
      4  (A and B) or C  (1 2)
         3  A and B  (1 2)
            1  A  (1)
            2  B  (2)
   
   
   ***** comment | ******* SHOW COM ******* |
   
   ***** show com;
   The list of commands is the following:

   META : REFLECT MATTACH 

   CONTEXT : COPYLEX SWITCHCONTEXT COPYCONTEXT NAMECONTEXT MAKECONTEXT 

   DECIDER : DECIDE MONADEQ MONAD TAUTEQ TAUT PTAUT 

   EVAL : EVAL 

   SIMPLIFIER : SIMPLIFY LET HARDWARE REPRESENT ATTACH DECREP 

   REWRITER : REWRITE ASSERTSIMP SETCOMPSIMP SETBASICSIMP UNFOLD FOLD CUT
   CTC CONTRACT WK WEAKEN WFFIFI WFFIFEN WFFIFE TERMIFI TERMIFEN TERMIFE  

   Natural-Deduction : ES EXISTE EXISTI US ALLE UG ALLI IE IFFE II IFFI
   NE NOTE NI NOTI FE FALSEE FI FALSEI OE ORE OI ORI AE ANDE AI ANDI MP
   IMPE DED IMPI SUBST THEOREM ASSUME   

   DEFINITION : DEFINE 

   PROOF : LABEL CANCEL AXIOM THEORY 

   LANGUAGE : SWITCHPROOF COPYPROOF NAMEPROOF MAKEPROOF EXTENSION WFF
   AWFF TERM MOREGENERAL MOSTGENERAL SETFMAP DECLARE  

   SYSTEM : COPYLEX RESET PAGER RESETPROMPT SETPROMPT KNOW HGK SHOW ECHO
   COMMENT UNPROBE PROBE DEFLAM LOAD MARK FETCH DONE BACKUP  
+

 	;;;;;;;;;;;;;;;;;;;;;;;;;;;;;;;;;;;;;;;;;;;;;;;;;;;;;;;;;;;;;;;;;;;;;;;;;;;;;;
;;
;; FOL version 2.001
;; This file is an FOL source file: language.cfg
;; Date: Wed Oct 20 10:46:03 MET 1993
;;
;;;;;;;;;;;;;;;;;;;;;;;;;;;;;;;;;;;;;;;;;;;;;;;;;;;;;;;;;;;;;;;;;;;;;;;;;;;;;;
;;                                                                          ;;
;;   Copyright (c) 1986-1987 by Richard Weyhrauch.  All rights reserved.    ;;
;;   Copyright (c) 1987-1988 by Fausto Giunchiglia.  All rights reserved.   ;;
;;                                                                          ;;
;;   This software is being provided to you, the LICENSEE, by Richard       ;;
;;   Weyhrauch and Fausto Giunchiglia, the AUTHORS, under certain rights    ;;
;;   and obligations.  By obtaining, using and/or copying this software,    ;;
;;   you indicate that you have read, understood, and will comply with      ;;
;;   the following terms and conditions:                                    ;;
;;                                                                          ;;
;;   THE AUTHORS MAKE NO REPRESENTATIONS OF WARRANTIES, EXPRESS OR          ;;
;;   IMPLIED.  By way of example, but not limitation, THE AUTHORS MAKE      ;;
;;   NO REPRESENTATIONS OR WARRANTIES OF MERCHANTABILITY OF FITNESS FOR     ;;
;;   ANY PARTICULAR PURPOSE OR THAT THE USE OF THE LICENSED SOFTWARE        ;;
;;   COMPONENTS OR DOCUMENTATION WILL NOT INFRINGE ANY PATENTS,             ;;
;;   COPYRIGHTS, TRADEMARKS OR OTHER RIGHTS.                                ;;
;;                                                                          ;;
;;   The AUTHORS shall not be held liable for any direct, indirect or       ;;
;;   consequential damages with respect to any claim by LICENSEE or any     ;;
;;   third party on account of or arising from this Agreement or use of     ;;
;;   this software.  Permission to use, copy, modify and distribute this    ;;
;;   software and its documentation for any purpose and without fee or      ;;
;;   royalty is hereby granted, provided that the above copyright notice    ;;
;;   and disclaimer appears in and on ALL copies of the software and        ;;
;;   documentation, whether original to the AUTHORS or a modified           ;;
;;   version by LICENSEE.                                                   ;;
;;                                                                          ;;
;;   The name of the AUTHORS may not be used in advertising or publicity    ;;
;;   pertaining to distribution of the software without specific, written   ;;
;;   prior permission.  Notice must be given in supporting documentation    ;;
;;   that such distribution is by permission of the AUTHORS.  The AUTHORS   ;;
;;   make no representations about the suitability of this software for     ;;
;;   any purpose.  It is provided "AS IS" without express or implied        ;;
;;   warranty.  Title to copyright to this software and to any associated   ;;
;;   documentation shall at all times remain with the AUTHORS and LICENSEE  ;;
;;   agrees to preserve same.  LICENSEE agrees to place the appropriate     ;;
;;   copyright notice on any such copies.                                   ;;
;;                                                                          ;;
;;;;;;;;;;;;;;;;;;;;;;;;;;;;;;;;;;;;;;;;;;;;;;;;;;;;;;;;;;;;;;;;;;;;;;;;;;;;;;

;*************************************************************************;
;                                                                         ;
;                    "LANGUAGE" MODULE CONFIGURATION FILE                 ;
;                                                                         ;
;*************************************************************************;

(MODULE-INIT        'LANGUAGE)
(MODULE-SET-NAME    'LANGUAGE "LANGUAGE")
(MODULE-SET-MODE    'LANGUAGE 'COMPILED)

(MODULE-SET-SRCDIR 'LANGUAGE (PATH-CONCAT (SYS-GET-SRCDIR 'GETFOL) "language"))
(MODULE-SET-OBJDIR 'LANGUAGE (SYS-GET-OBJDIR 'GETFOL))
(MODULE-SET-DOCDIR 'LANGUAGE (PATH-CONCAT (SYS-GET-DOCDIR 'GETFOL) "language"))

(MODULE-SET-DOCFILE 'LANGUAGE
   (PATH-CONCAT  (MODULE-GET-DOCDIR 'LANGUAGE) "language.tex"))


;;;     GETHGKM special variables declaration
(MODULE-ADD-FILE 'LANGUAGE   "vlang.cl"     ""         'INTERPRETED)

;;;     Label Spaces
(MODULE-ADD-FILE 'LANGUAGE   "labelspa.hgk" "labspah"  'COMPILED)
(MODULE-ADD-FILE 'LANGUAGE   "labelspa.fol" "labspaf"  'COMPILED)
(MODULE-ADD-FILE 'LANGUAGE   "labelspa.rp"  "labspar"  'COMPILED)

;;;     Signature, symbols and sorts
(MODULE-ADD-FILE 'LANGUAGE   "symls.hgk"    "symlsh"   'COMPILED)
(MODULE-ADD-FILE 'LANGUAGE   "sym.hgk"      "symh"     'COMPILED)
(MODULE-ADD-FILE 'LANGUAGE   "symls.fol"    "symlsf"   'COMPILED)
(MODULE-ADD-FILE 'LANGUAGE   "sym.fol"      "symf"     'COMPILED)
(MODULE-ADD-FILE 'LANGUAGE   "sym.rp"       "symr"     'COMPILED)

;;;     Expressions
(MODULE-ADD-FILE 'LANGUAGE   "exp.hgk"      "exph"     'INTERPRETED)
(MODULE-ADD-FILE 'LANGUAGE   "exp.rp"       "expr"     'COMPILED)
(MODULE-ADD-FILE 'LANGUAGE   "exp.fol"      "expf"     'COMPILED)

;;;     Sorts
(MODULE-ADD-FILE 'LANGUAGE   "sort.hgk"     "sorth"    'COMPILED)
(MODULE-ADD-FILE 'LANGUAGE   "sort.fol"     "sortf"    'COMPILED)

;;;     Probe file
(MODULE-ADD-FILE 'LANGUAGE   "problang.fol" "problaf"  'COMPILED)

;;;     Command files
(MODULE-ADD-FILE 'LANGUAGE   "decsymls.hgk" "decsymlh" 'COMPILED)
(MODULE-ADD-FILE 'LANGUAGE   "decsymls.fol" "decsymlf" 'COMPILED)
(MODULE-ADD-FILE 'LANGUAGE   "language.fol" "langf"    'COMPILED)
(MODULE-ADD-FILE 'LANGUAGE   "cmdlang.fol"  "cmdlangf" 'COMPILED)

;;;     Defaults for the language
(MODULE-ADD-FILE 'LANGUAGE   "langdflt.fol" "langdflf" 'COMPILED)

;;;     Show files
(MODULE-ADD-FILE 'LANGUAGE   "showlang.fol" "showlanf" 'COMPILED)
(MODULE-ADD-FILE 'LANGUAGE   "showlang.rp"  "showlanr" 'COMPILED)


(MODULE-ADD-FILE 'LANGUAGE   "skolem.hgk"   "skolh"    'COMPILED)


;;;     Initialization files
(MODULE-ADD-FILE 'LANGUAGE   "ilang.fol"    ""         'INTERPRETED)

 	;;;;;;;;;;;;;;;;;;;;;;;;;;;;;;;;;;;;;;;;;;;;;;;;;;;;;;;;;;;;;;;;;;;;;;;;;;;;;;
;;
;; FOL version 2.001
;; This file is an FOL source file: proof.cfg
;; Date: Wed Oct 20 10:47:27 MET 1993
;;
;;;;;;;;;;;;;;;;;;;;;;;;;;;;;;;;;;;;;;;;;;;;;;;;;;;;;;;;;;;;;;;;;;;;;;;;;;;;;;
;;                                                                          ;;
;;   Copyright (c) 1986-1987 by Richard Weyhrauch.  All rights reserved.    ;;
;;   Copyright (c) 1987-1988 by Fausto Giunchiglia.  All rights reserved.   ;;
;;                                                                          ;;
;;   This software is being provided to you, the LICENSEE, by Richard       ;;
;;   Weyhrauch and Fausto Giunchiglia, the AUTHORS, under certain rights    ;;
;;   and obligations.  By obtaining, using and/or copying this software,    ;;
;;   you indicate that you have read, understood, and will comply with      ;;
;;   the following terms and conditions:                                    ;;
;;                                                                          ;;
;;   THE AUTHORS MAKE NO REPRESENTATIONS OF WARRANTIES, EXPRESS OR          ;;
;;   IMPLIED.  By way of example, but not limitation, THE AUTHORS MAKE      ;;
;;   NO REPRESENTATIONS OR WARRANTIES OF MERCHANTABILITY OF FITNESS FOR     ;;
;;   ANY PARTICULAR PURPOSE OR THAT THE USE OF THE LICENSED SOFTWARE        ;;
;;   COMPONENTS OR DOCUMENTATION WILL NOT INFRINGE ANY PATENTS,             ;;
;;   COPYRIGHTS, TRADEMARKS OR OTHER RIGHTS.                                ;;
;;                                                                          ;;
;;   The AUTHORS shall not be held liable for any direct, indirect or       ;;
;;   consequential damages with respect to any claim by LICENSEE or any     ;;
;;   third party on account of or arising from this Agreement or use of     ;;
;;   this software.  Permission to use, copy, modify and distribute this    ;;
;;   software and its documentation for any purpose and without fee or      ;;
;;   royalty is hereby granted, provided that the above copyright notice    ;;
;;   and disclaimer appears in and on ALL copies of the software and        ;;
;;   documentation, whether original to the AUTHORS or a modified           ;;
;;   version by LICENSEE.                                                   ;;
;;                                                                          ;;
;;   The name of the AUTHORS may not be used in advertising or publicity    ;;
;;   pertaining to distribution of the software without specific, written   ;;
;;   prior permission.  Notice must be given in supporting documentation    ;;
;;   that such distribution is by permission of the AUTHORS.  The AUTHORS   ;;
;;   make no representations about the suitability of this software for     ;;
;;   any purpose.  It is provided "AS IS" without express or implied        ;;
;;   warranty.  Title to copyright to this software and to any associated   ;;
;;   documentation shall at all times remain with the AUTHORS and LICENSEE  ;;
;;   agrees to preserve same.  LICENSEE agrees to place the appropriate     ;;
;;   copyright notice on any such copies.                                   ;;
;;                                                                          ;;
;;;;;;;;;;;;;;;;;;;;;;;;;;;;;;;;;;;;;;;;;;;;;;;;;;;;;;;;;;;;;;;;;;;;;;;;;;;;;;

;*************************************************************************;
;                                                                         ;
;                    "PROOF" MODULE CONFIGURATION FILE                    ;
;                                                                         ;
;*************************************************************************;

(MODULE-INIT        'PROOF)
(MODULE-SET-NAME    'PROOF   "PROOF")
(MODULE-SET-MODE    'PROOF   'COMPILED)

(MODULE-SET-SRCDIR  'PROOF  (PATH-CONCAT (SYS-GET-SRCDIR 'GETFOL) "proof"))
(MODULE-SET-OBJDIR  'PROOF  (SYS-GET-OBJDIR 'GETFOL))
(MODULE-SET-DOCDIR  'PROOF  (PATH-CONCAT (SYS-GET-DOCDIR 'GETFOL) "proof"))

(MODULE-SET-DOCFILE 'PROOF
    (PATH-CONCAT (MODULE-GET-DOCDIR 'PROOF) "inclfile"    "proof.tex"))



;;;   GETHGKM special variables declaration
(MODULE-ADD-FILE   'PROOF   "vproof.cl"       ""             'INTERPRETED)

;;;   Reason
(MODULE-ADD-FILE    'PROOF  "reason.hgk"      "reasonh"      'COMPILED)

;;;   Proof-steps
(MODULE-ADD-FILE    'PROOF  "pline.hgk"       "plineh"       'COMPILED)
(MODULE-ADD-FILE    'PROOF  "pline.fol"       "plinef"       'COMPILED)
(MODULE-ADD-FILE    'PROOF  "pline.rp"        "pliner"       'COMPILED)

;;;   Facts
(MODULE-ADD-FILE    'PROOF  "fact.hgk"        "facth"        'COMPILED)
(MODULE-ADD-FILE    'PROOF  "fact.fol"        "factf"        'COMPILED)
(MODULE-ADD-FILE    'PROOF  "fact.rp"         "factr"        'COMPILED)

;;;   Axioms
(MODULE-ADD-FILE    'PROOF  "axiom.hgk"       "axiomh"       'COMPILED)
(MODULE-ADD-FILE    'PROOF  "axiom.fol"       "axiomf"       'COMPILED)
(MODULE-ADD-FILE    'PROOF  "axiom.rp"        "axiomr"       'COMPILED)

;;;   Show & probe proofs.
(MODULE-ADD-FILE    'PROOF  "showprf.rp"      "showprfr"     'COMPILED)
(MODULE-ADD-FILE    'PROOF  "probprf.fol"     "probprf"      'COMPILED)

;;;   Proofs
(MODULE-ADD-FILE    'PROOF  "proof.hgk"       "proofh"       'COMPILED)
(MODULE-ADD-FILE    'PROOF  "proof.fol"       "prooff"       'COMPILED)
(MODULE-ADD-FILE    'PROOF  "proof.rp"        "proofr"       'COMPILED)

;;;   Command files
(MODULE-ADD-FILE    'PROOF  "cmdproof.fol"    "cmdprff"      'COMPILED)
(MODULE-ADD-FILE    'PROOF  "cmdlabel.fol"    "cmdlabef"     'COMPILED)

;;;   Initialization files
(MODULE-ADD-FILE    'PROOF  "iproof.fol"      ""             'INTERPRETED)

 	;;;;;;;;;;;;;;;;;;;;;;;;;;;;;;;;;;;;;;;;;;;;;;;;;;;;;;;;;;;;;;;;;;;;;;;;;;;;;;
;;
;; FOL version 2.001
;; This file is an FOL source file: nd.cfg
;; Date: Wed Oct 20 10:46:59 MET 1993
;;
;;;;;;;;;;;;;;;;;;;;;;;;;;;;;;;;;;;;;;;;;;;;;;;;;;;;;;;;;;;;;;;;;;;;;;;;;;;;;;
;;                                                                          ;;
;;   Copyright (c) 1986-1987 by Richard Weyhrauch.  All rights reserved.    ;;
;;   Copyright (c) 1987-1988 by Fausto Giunchiglia.  All rights reserved.   ;;
;;                                                                          ;;
;;   This software is being provided to you, the LICENSEE, by Richard       ;;
;;   Weyhrauch and Fausto Giunchiglia, the AUTHORS, under certain rights    ;;
;;   and obligations.  By obtaining, using and/or copying this software,    ;;
;;   you indicate that you have read, understood, and will comply with      ;;
;;   the following terms and conditions:                                    ;;
;;                                                                          ;;
;;   THE AUTHORS MAKE NO REPRESENTATIONS OF WARRANTIES, EXPRESS OR          ;;
;;   IMPLIED.  By way of example, but not limitation, THE AUTHORS MAKE      ;;
;;   NO REPRESENTATIONS OR WARRANTIES OF MERCHANTABILITY OF FITNESS FOR     ;;
;;   ANY PARTICULAR PURPOSE OR THAT THE USE OF THE LICENSED SOFTWARE        ;;
;;   COMPONENTS OR DOCUMENTATION WILL NOT INFRINGE ANY PATENTS,             ;;
;;   COPYRIGHTS, TRADEMARKS OR OTHER RIGHTS.                                ;;
;;                                                                          ;;
;;   The AUTHORS shall not be held liable for any direct, indirect or       ;;
;;   consequential damages with respect to any claim by LICENSEE or any     ;;
;;   third party on account of or arising from this Agreement or use of     ;;
;;   this software.  Permission to use, copy, modify and distribute this    ;;
;;   software and its documentation for any purpose and without fee or      ;;
;;   royalty is hereby granted, provided that the above copyright notice    ;;
;;   and disclaimer appears in and on ALL copies of the software and        ;;
;;   documentation, whether original to the AUTHORS or a modified           ;;
;;   version by LICENSEE.                                                   ;;
;;                                                                          ;;
;;   The name of the AUTHORS may not be used in advertising or publicity    ;;
;;   pertaining to distribution of the software without specific, written   ;;
;;   prior permission.  Notice must be given in supporting documentation    ;;
;;   that such distribution is by permission of the AUTHORS.  The AUTHORS   ;;
;;   make no representations about the suitability of this software for     ;;
;;   any purpose.  It is provided "AS IS" without express or implied        ;;
;;   warranty.  Title to copyright to this software and to any associated   ;;
;;   documentation shall at all times remain with the AUTHORS and LICENSEE  ;;
;;   agrees to preserve same.  LICENSEE agrees to place the appropriate     ;;
;;   copyright notice on any such copies.                                   ;;
;;                                                                          ;;
;;;;;;;;;;;;;;;;;;;;;;;;;;;;;;;;;;;;;;;;;;;;;;;;;;;;;;;;;;;;;;;;;;;;;;;;;;;;;;

;*************************************************************************;
;                                                                         ;
;                    "ND" MODULE CONFIGURATION FILE                       ;
;                                                                         ;
;*************************************************************************;

(MODULE-INIT        'ND)
(MODULE-SET-NAME    'ND  "ND")
(MODULE-SET-MODE    'ND  'COMPILED)

(MODULE-SET-SRCDIR  'ND  (PATH-CONCAT (SYS-GET-SRCDIR 'GETFOL) "nd"))
(MODULE-SET-OBJDIR  'ND  (SYS-GET-OBJDIR 'GETFOL))
(MODULE-SET-DOCDIR  'ND
     (PATH-CONCAT (SYS-GET-DOCDIR 'GETFOL) "inclfile"  "nd"))
(MODULE-SET-DOCFILE 'ND
     (PATH-CONCAT (MODULE-GET-DOCDIR 'ND)  "inclfile"  "nd.tex"))


;;;   GETHGKM special variables declaration
(MODULE-ADD-FILE 'ND "vnd.cl"       ""         'INTERPRETED)              

;;;   Inference rules for nd
(MODULE-ADD-FILE 'ND "fapfnd.hgk"   "fapfndh"  'COMPILED)
(MODULE-ADD-FILE 'ND "fapfnd.fol"   "fapfndf"  'COMPILED)

;;;   Command files
(MODULE-ADD-FILE 'ND "cmdnd.fol"    "cmdndf"   'COMPILED)

;;;   Initialization files
(MODULE-ADD-FILE 'ND "ind.fol"      ""         'INTERPRETED)

	;;;;;;;;;;;;;;;;;;;;;;;;;;;;;;;;;;;;;;;;;;;;;;;;;;;;;;;;;;;;;;;;;;;;;;;;;;;;;;
;;
;; GETFOL version 1.001
;; This file is a GETFOL source file: rules.cfg
;; Date: Thu Nov 11 14:42:52 MET 1993
;;
;;;;;;;;;;;;;;;;;;;;;;;;;;;;;;;;;;;;;;;;;;;;;;;;;;;;;;;;;;;;;;;;;;;;;;;;;;;;;;
;;                                                                          ;;
;;   Copyright (c) 1987-1988 by Fausto Giunchiglia.  All rights reserved.   ;;
;;                                                                          ;;
;;   This software is being provided to you, the LICENSEE, by Fausto        ;;
;;   Giunchiglia, the AUTHOR, under certain rights and obligations.         ;;
;;   By obtaining, using and/or copying this software, you indicate that    ;;
;;   you have read, understood, and will comply with the following terms    ;;
;;   and conditions:                                                        ;;
;;                                                                          ;;
;;   THE AUTHOR MAKES NO REPRESENTATIONS OF WARRANTIES, EXPRESS OR          ;;
;;   IMPLIED.  By way of example, but not limitation, THE AUTHOR MAKES      ;;
;;   NO REPRESENTATIONS OR WARRANTIES OF MERCHANTABILITY OF FITNESS FOR     ;;
;;   ANY PARTICULAR PURPOSE OR THAT THE USE OF THE LICENSED SOFTWARE        ;;
;;   COMPONENTS OR DOCUMENTATION WILL NOT INFRINGE ANY PATENTS,             ;;
;;   COPYRIGHTS, TRADEMARKS OR OTHER RIGHTS.                                ;;
;;                                                                          ;;
;;   The AUTHOR shall not be held liable for any direct, indirect or        ;;
;;   consequential damages with respect to any claim by LICENSEE or any     ;;
;;   third party on account of or arising from this Agreement or use of     ;;
;;   this software.  Permission to use, copy, modify and distribute this    ;;
;;   software and its documentation for any purpose and without fee or      ;;
;;   royalty is hereby granted, provided that the above copyright notice    ;;
;;   and disclaimer appears in and on ALL copies of the software and        ;;
;;   documentation, whether original to the AUTHOR or a modified            ;;
;;   version by LICENSEE.                                                   ;;
;;                                                                          ;;
;;   The name of the AUTHOR may not be used in advertising or publicity     ;;
;;   pertaining to distribution of the software without specific, written   ;;
;;   prior permission.  Notice must be given in supporting documentation    ;;
;;   that such distribution is by permission of the AUTHOR.  The AUTHOR     ;;
;;   makes no representations about the suitability of this software for    ;;
;;   any purpose.  It is provided "AS IS" without express or implied        ;;
;;   warranty.  Title to copyright to this software and to any associated   ;;
;;   documentation shall at all times remain with the AUTHOR and LICENSEE   ;;
;;   agrees to preserve same.  LICENSEE agrees to place the appropriate     ;;
;;   copyright notice on any such copies.                                   ;;
;;                                                                          ;;
;;;;;;;;;;;;;;;;;;;;;;;;;;;;;;;;;;;;;;;;;;;;;;;;;;;;;;;;;;;;;;;;;;;;;;;;;;;;;;

;*************************************************************************;
;                                                                         ;
;                    "RULES" MODULE CONFIGURATION FILE                    ;
;                                                                         ;
;*************************************************************************;

(MODULE-INIT       'RULES)
(MODULE-SET-NAME   'RULES "RULES")
(MODULE-SET-MODE   'RULES 'COMPILED)

(MODULE-SET-SRCDIR 'RULES (PATH-CONCAT (SYS-GET-SRCDIR 'GETFOL) "rules"))
(MODULE-SET-OBJDIR 'RULES (SYS-GET-OBJDIR 'GETFOL))
(MODULE-SET-DOCDIR 'RULES (PATH-CONCAT (SYS-GET-DOCDIR 'GETFOL) "rules"))

(MODULE-SET-DOCFILE 'RULES
  (PATH-CONCAT (MODULE-GET-DOCDIR 'RULES) "rules.tex"))

;;;
;;;                !!!FILES WITHOUT DEFSUB!!!
;;;

;;;  If rules
(MODULE-ADD-FILE 'RULES "fapfif.fol"   "fapfiff"  'COMPILED)
(MODULE-ADD-FILE 'RULES "cmdif.fol"    "cmdiff"   'COMPILED)

;;;  Structural rules
(MODULE-ADD-FILE 'RULES "fapfstr.fol"  "fapfstrf" 'COMPILED)
(MODULE-ADD-FILE 'RULES "cmdstr.fol"   "cmdstrf"  'COMPILED)

;;;  Initialization files
(MODULE-ADD-FILE 'RULES "iif.fol"      ""         'INTERPRETED)
(MODULE-ADD-FILE 'RULES "istr.fol"     ""         'INTERPRETED)

 	% introduction to the deciders
\newpage
\section{Introduction}

The distribution package allows the installation of the following three
applications:
%
\begin{itemize}
	\item
		the {\tt HGKM} interpreter \cite{giunchiglia35}, that is a LISP-like
		language acting as the implementation language of both {\tt FOL}
		and {\tt GETFOL}.
	\item
		the {\tt GETFOL} system \cite{giunchiglia12,giunchiglia29}.
	\item
		the {\tt AGETFOL} system ({\tt GETFOL} with {\em abstraction}
		\cite{giunchiglia7}.
\end{itemize}

From now on, we will write ``{\tt the system}" to mean {\tt HGKM},
{\tt GETFOL} and {\tt AGETFOL}.
The distribution package is provided with a set of general purpose
configuration and installation facilities which allow you to install your
favourite application on your machine.

In order to install {\tt the system} read section~\ref{requirements}
(``{\it Minimal System Requirements}'') and verify that your machine
has all the required features.
Then read section~\ref{instproc} (``{\it The installation procedure}'').
This section gives the instructions to follow in order to install
{\tt the system} on your machine.

If you are an expert you might be interested also in the features
of the installation and configuration facilities.
Section~\ref{sysmod} (``{\it Systems and Modules}'') introduces
to the data structures ({\it systems} and {\it modules}) devised
to store and/or modify the configuration of the applications.


% user commands for deciders
;;;;;;;;;;;;;;;;;;;;;;;;;;;;;;;;;;;;;;;;;;;;;;;;;;;;;;;;;;;;;;;;;;;;;;;;;;;;;;
;;
;; GETFOL version 2.001
;; Date: Wed Nov 10 15:08:45 MET 1993
;;
;; This    FOL file was modified in GETFOL version 2.001
;;
;;;;;;;;;;;;;;;;;;;;;;;;;;;;;;;;;;;;;;;;;;;;;;;;;;;;;;;;;;;;;;;;;;;;;;;;;;;;;;
;;
;; FOL version 2.001
;; This file is an FOL source file: decide.cfg
;; Date: Wed Oct 20 10:43:51 MET 1993
;;
;;;;;;;;;;;;;;;;;;;;;;;;;;;;;;;;;;;;;;;;;;;;;;;;;;;;;;;;;;;;;;;;;;;;;;;;;;;;;;
;;                                                                          ;;
;;   Copyright (c) 1986-1987 by Richard Weyhrauch.  All rights reserved.    ;;
;;   Copyright (c) 1987-1988 by Fausto Giunchiglia.  All rights reserved.   ;;
;;                                                                          ;;
;;   This software is being provided to you, the LICENSEE, by Richard       ;;
;;   Weyhrauch and Fausto Giunchiglia, the AUTHORS, under certain rights    ;;
;;   and obligations.  By obtaining, using and/or copying this software,    ;;
;;   you indicate that you have read, understood, and will comply with      ;;
;;   the following terms and conditions:                                    ;;
;;                                                                          ;;
;;   THE AUTHORS MAKE NO REPRESENTATIONS OF WARRANTIES, EXPRESS OR          ;;
;;   IMPLIED.  By way of example, but not limitation, THE AUTHORS MAKE      ;;
;;   NO REPRESENTATIONS OR WARRANTIES OF MERCHANTABILITY OF FITNESS FOR     ;;
;;   ANY PARTICULAR PURPOSE OR THAT THE USE OF THE LICENSED SOFTWARE        ;;
;;   COMPONENTS OR DOCUMENTATION WILL NOT INFRINGE ANY PATENTS,             ;;
;;   COPYRIGHTS, TRADEMARKS OR OTHER RIGHTS.                                ;;
;;                                                                          ;;
;;   The AUTHORS shall not be held liable for any direct, indirect or       ;;
;;   consequential damages with respect to any claim by LICENSEE or any     ;;
;;   third party on account of or arising from this Agreement or use of     ;;
;;   this software.  Permission to use, copy, modify and distribute this    ;;
;;   software and its documentation for any purpose and without fee or      ;;
;;   royalty is hereby granted, provided that the above copyright notice    ;;
;;   and disclaimer appears in and on ALL copies of the software and        ;;
;;   documentation, whether original to the AUTHORS or a modified           ;;
;;   version by LICENSEE.                                                   ;;
;;                                                                          ;;
;;   The name of the AUTHORS may not be used in advertising or publicity    ;;
;;   pertaining to distribution of the software without specific, written   ;;
;;   prior permission.  Notice must be given in supporting documentation    ;;
;;   that such distribution is by permission of the AUTHORS.  The AUTHORS   ;;
;;   make no representations about the suitability of this software for     ;;
;;   any purpose.  It is provided "AS IS" without express or implied        ;;
;;   warranty.  Title to copyright to this software and to any associated   ;;
;;   documentation shall at all times remain with the AUTHORS and LICENSEE  ;;
;;   agrees to preserve same.  LICENSEE agrees to place the appropriate     ;;
;;   copyright notice on any such copies.                                   ;;
;;                                                                          ;;
;;;;;;;;;;;;;;;;;;;;;;;;;;;;;;;;;;;;;;;;;;;;;;;;;;;;;;;;;;;;;;;;;;;;;;;;;;;;;;

;*************************************************************************;
;                                                                         ;
;                    "DECIDE" MODULE CONFIGURATION FILE                   ;
;                                                                         ;
;*************************************************************************;

(MODULE-INIT        'DECIDE)
(MODULE-SET-NAME    'DECIDE  "DECIDE")
(MODULE-SET-MODE    'DECIDE  'COMPILED)

(MODULE-SET-SRCDIR  'DECIDE (PATH-CONCAT (SYS-GET-SRCDIR 'GETFOL) "decide"))
(MODULE-SET-OBJDIR  'DECIDE (SYS-GET-OBJDIR 'GETFOL))
(MODULE-SET-DOCDIR  'DECIDE (PATH-CONCAT (SYS-GET-DOCDIR 'GETFOL) "decide"))
(MODULE-SET-DOCFILE 'DECIDE
        (PATH-CONCAT (MODULE-GET-DOCDIR 'GETFOL) "inclfile" "deciders.tex"))


;;;   GETHGKM special variables declaration
(MODULE-ADD-FILE 'DECIDE "vdecide.cl"   ""         'INTERPRETED)

;;;   Low level functions for deciders
(MODULE-ADD-FILE 'DECIDE "orelse.hgk"   "orelseh"  'COMPILED)
(MODULE-ADD-FILE 'DECIDE "decide.hgk"   "decideh"  'COMPILED)
(MODULE-ADD-FILE 'DECIDE "ptaut.hgk"    "ptauth"   'COMPILED)
(MODULE-ADD-FILE 'DECIDE "ptauteq.hgk"  "ptauteqh" 'COMPILED)
(MODULE-ADD-FILE 'DECIDE "tautren.hgk"  "tautrenh" 'COMPILED)
(MODULE-ADD-FILE 'DECIDE "phexp.hgk"    "phexph"   'COMPILED)
(MODULE-ADD-FILE 'DECIDE "reduce.hgk"   "reduceh"  'COMPILED)

;;;   High level functions for deciders
(MODULE-ADD-FILE 'DECIDE "ptaut.fol"    "ptautf"   'COMPILED)
(MODULE-ADD-FILE 'DECIDE "ptauteq.fol"  "ptauteqf" 'COMPILED)
(MODULE-ADD-FILE 'DECIDE "phexp.fol"    "phexpf"   'COMPILED)
(MODULE-ADD-FILE 'DECIDE "reduce.fol"   "reducef"  'COMPILED)
(MODULE-ADD-FILE 'DECIDE "assert.fol"   "assertf"  'COMPILED)
(MODULE-ADD-FILE 'DECIDE "peval.fol"    "pevalf"   'COMPILED)
(MODULE-ADD-FILE 'DECIDE "termif.fol"   "termiff"  'COMPILED)
(MODULE-ADD-FILE 'DECIDE "tautren.fol"  "tautrenf" 'COMPILED)
(MODULE-ADD-FILE 'DECIDE "skolem.fol"   "skolemf"  'COMPILED)
(MODULE-ADD-FILE 'DECIDE "nnf.fol"      "nnff"     'COMPILED)

;;;   Command files
(MODULE-ADD-FILE 'DECIDE "cmdecide.fol" "cmdecidf" 'COMPILED)

;;;   Initialization files
(MODULE-ADD-FILE 'DECIDE "idecide.fol"  ""         'INTERPRETED)

\gfcommand{monad}{first order decider for monadic formulas}
\index{monad}

\gfsyntax{
  monad \ARG{wff} \OPT{by \ARG{fact1} \ARG{fact2} \SEQ};
}

\gfdescription{
It tries to establish whether the input formula is deducible from the
specified facts by using {\tt nnf}, {\tt reduce}, {\tt phexp} and finally
{\tt ptaut}.
}

\gfrecap{
It tries to establish whether the input formula is deducible from the
specified facts by using nnf, reduce, phexp and finally ptaut.
}

\gfexample+
   ***** declare predconst P 1;
   ***** declare funconst f 3;
   ***** declare indvar x y z;
   ***** declare indpar a b;
   
   *****comment | *** MONAD EXAMPLES *** |
   ***** monad forall x. exists y. (P(x) imp P(y));
   1   forall x. exists y. (P(x) imp P(y))
   ***** monad exists y. forall x. (P(x) imp P(y));
   2   exists y. forall x. (P(x) imp P(y))
   ***** monad exists y. forall x. ((P(x) imp P(y)) or P(x));
   3   exists y. forall x. ((P(x) imp P(y)) or P(x))
   
   ***** monad forall x. exists y.
    wffif P(trmif P(y) then x else y)
     then P(trmif P(y) 
             then trmif P(y) 
                   then x 
                   else trmif P(y) 
                         then x 
                         else y 
             else y) 
          or TRUE
     else P(y) or TRUE;
   4   forall x. exists y. 
       (wffif P(trmif P(y) then x else y) 
         then (P(trmif P(y) 
                 then 
                 (trmif P(y) 
                   then x 
                   else (trmif P(y) then x else y)) 
               else y) or TRUE) else (P(y) or TRUE))
   ***** monad forall x. exists y. wffif P(x) then P(y) else not P(y);
   5   forall x. exists y. (wffif P(x) then P(y) else (not P(y)))  
   ***** monad forall y. exists x. (P(f(a,b,x)) or not P(f(a,b,y)));
   6   forall y. exists x. (P(f(a,b,x)) or (not P(f(a,b,y))))
   ***** monad exists z. forall y. exists x. (P(f(z,b,x)) or not P(f(x,b,z)));
   7   exists z. forall y. exists x. (P(f(z,b,x)) or (not P(f(x,b,z)))) 
   ***** monad exists z. forall y. exists x. (P(f(z,b,x)) or not P(f(x,b,y)));
   7   exists z. forall y. exists x. (P(f(z,b,x)) or (not P(f(x,b,y)))) 
+
   
\gfnotes{
   The name ``{\tt monad}" is due to the fact that the monadic predicate
calculus is subset of the class of formulas decided by such a decider.
}   

\gfcommand{monadeq}{first order decider for monadic formulas with equality}
\index{monadeq}

\gfsyntax{
  monadeq \ARG{wff} \OPT{by \ARG{fact1} \ARG{fact2} \SEQ};
}


\gfdescription{
It tries to establish whether the input formula is deducible from the
specified facts by using {\tt nnf}, {\tt reduce}, {\tt phexp} and finally
{\tt ptauteq}.
}

\gfrecap{
It tries to establish whether the input formula is deducible from the
specified facts by using nnf, reduce, phexp and finally ptauteq.
}

\gfexample+
   ***** declare predconst P 1;
   ***** declare funconst f 3;
   ***** declare indvar x y z;
   ***** declare indpar a b;

   *****comment | *** MONADEQ EXAMPLES *** |
   ***** monadeq forall x. exists y. (x=y);
   1   forall x. exists y. (x=y)
   ***** monadeq forall x.  forall y. (x=y imp (P(x) imp P(y)));
   2   forall x.  forall y. (x=y imp (P(x) imp P(y)))
   ***** monad (x=y imp (P(x) imp P(y))) by 2;
   3   (x=y imp (P(x) imp P(y)))  (2)
+

\gfcommand{ptaut}{tautological decider}
\index{ptaut}
\label{sec-decproc}

\gfsyntax{
  ptaut \ARG{wff} \OPT{by \ARG{fact1} \ARG{fact2} \SEQ};
}

\gfdescription{
It decides the quantifier-free formulas provable only by means of the
propositional deductive machinery.
}

\gfrecap{
It decides the quantifier-free formulas provable only by means of the
propositional deductive machinery.
}

\gfexample+
   ***** declare sentconst A B;
   ***** ptaut (A imp (B imp A));
   1   A imp (B imp A)
   ***** assume A B;
   2   A     (2)
   3   B     (3)
   ***** ptaut (A imp (B imp A)) by 2 3;
   4   A imp (B imp A)  (2 3)
   ***** ptaut (A and B) by 2 3;
   5   A and B  (2 3)
   ***** ptaut A by 3;
   PTAUT couldn't prove that A
   is a logical consequence of facts.
   
   ***** declare predconst P 1;
   ***** declare indconst c;
   ***** declare indvar x;

   ***** ptaut (A imp (P(c) imp A));
   6   A imp (P(c) imp A)     
   ***** ptaut (A imp (forall x.P(x) imp A));
   The formula passed to PTAUT is not propositional !
+   

\gfcommand{taut}{tautological first order decider}
\index{taut}

\gfsyntax{
  taut \ARG{wff} \OPT{by \ARG{fact1} \ARG{fact2} \SEQ};
}

\gfdescription{
  The class of formulae decided by {\tt taut} is the set of first order
  formulas provable using only the introduction and elimination rules for 
  the sentential connectives plus the following rule ({\em congruence-rule}):\\
  %
  \[
      \fraz{\forall x A(x)}{\forall y A(y)}
  \]
}

\gfrecap{
The class of formulae decided by `taut' is the set of first order formulas
provable using only the introduction and elimination rules for the 
sentencial connectives plus the following rule (congruence rule)::
              +------------------------------------+
              | forall x. A(x) imp forall y. A(y)  |
              +------------------------------------+
}

\gfexample+
   ***** declare sentconst A;
   ***** declare predconst P 1;
   ***** declare indconst c;

   ***** taut (A imp (P(c) imp A));
   1   A imp (P(c) imp A)

   ***** declare indvar x y [S1];
   ***** declare indvar z [S2];

   ***** taut forall x.P(x) iff forall y.P(y);
   2   forall x.P(x) iff forall y.P(y);
   ***** taut forall x.P(x) iff forall z.P(z);
   TAUT couldn't prove that forall x. P(x) iff forall z. P(z)
   is a tautology.
+

\gfcommand{tauteq}{tautological decider with equality}
\index{tauteq}

\gfsyntax{
  tauteq \ARG{wff} \OPT{by \ARG{fact1} \ARG{fact2} \SEQ};
}

\gfdescription{
  The class of formulae decided by {\tt tauteq} is the set of formulas provable
  using:
  \begin{itemize}
  \item the introduction and elimination rules for the sentential connectives;
  \item the {\em congruence-rule};
  \item the following axioms schemata for equality:
  $$
  \begin{array}{l}
    x=x\\
    (x=y\ \imp\ y=x)\\
    ((x=y\ \con\ y=z)\ \imp\ x=z)\\
    ((x_1=y_1\ \con \ldots \con\ x_n=y_n)\ \imp\ 
    (P(x_1, \ldots \ ,x_n)\ \liff\ P(y_1, \ldots \ ,y_n)))
  \end{array}
  $$
  %
  corresponding to {\em reflexivity, symmetry, transitivity} and {\em substitution 
  into predicates}.
  \end{itemize}
}

\gfrecap{
The class of formulae decided by {\tt tauteq} is the set of formulas provable
using:
* the introduction and elimination rules for the sentential connectives;
* the ``congruence-rule'';
* the following axioms schemata for equality:
     +-------------------------------------------------------------------------+
     |   x = x                                                                 |
     |   x = y imp y = x                                                       |
     |   (x = y and y = z) imp x = z                                           |
     |   (x1 =y1 and ... and xN = yN) imp (P(x1, ..., xN) iff P(y1, ..., yN))  |
     +-------------------------------------------------------------------------+
  corresponding to ``reflexivity'', ``symmetry'', ``transitivity'' and
  ``substitution into predicates''.
}

\gfexample+
   ***** declare predconst P 1;
   ***** declare funconst f 1;
   ***** declare indvar x y;
   ***** declare indvar z;
   ***** tauteq x=x;
   1   x=x
   ***** tauteq x=y imp y=x; 
   2   x=y imp y=x
   ***** tauteq ((x=y and y=z) imp x=z);
   3   (x=y and y=z) imp x=z
   ***** tauteq (x=y imp (P(x) or not P(y)));
   4   x=y imp (P(x) or not P(y))
   ***** tauteq (f(x)=f(y) imp (P(f(x)) iff P(f(y))));
   5   f(x)=f(y) imp (P(f(x)) iff P(f(y)))
   ***** tauteq x=y imp f(x)=f(y);
   TAUTEQ couldn't prove that (x = y) imp (f(x) = f(y))
   is a tautology.
+

 	% introduction to the semantic simplification's section
\newpage
\section{Semantic simplification}
\label{sec-comp}

The subsections \ref{sec-ss-intro}, \ref{sec-ss-model} and \ref{sec-ss-repr} 
of this section have been taken from \cite{rww3}.


\subsection{Introduction}
\label{sec-ss-intro}

{\GF} is intended to express a variety of methods of human reasoning.
Though the word "reasoning" usually connotes a logical deductive process of 
using facts and assertions to obtain conclusions, much of human intelligence 
relies more upon observation than upon deduction.
We look at a book. The book is seen to be "green", as an immediate observation,
not as a deduction involving, say, analysis of wavelengths of light and 
sensory receptors  in the eye. Similarly, humans cross streets without 
conscious analysis of the traffic flow, add numbers without resorting to basic
set theory, and play chess without considering each move in terms of the 
geometry of the board. 

Any system which hopes to express a variety of reasoning processes, therefore 
needs a method of doing purely computational tasks.
In {\GF}, the {\bf semantic interpretation mechanism}, which provides this 
ability, consists of two parts:
\begin{itemize}
\item {\GF}'s {\bf semantic attachment mechanism} permits the user to define a
      ``correspondence'' between the various constants (function symbols,
      predicate constants, individual constants) of the language and
      corresponding objects of the programming language {\HG}.
\item facts about the {\HG} structure can be used directly in the proof
      via the {\bf semantic simplification mechanism}, eliminating the 
      necessity of a possibly complicated deduction.
\end{itemize}
For example, obvious attachments to the function symbol $+$ and to the 
individual constants $17$, $34$, $51$ would allow to conclude $17+34=51$ in 
one step, instead of computing $34$ successors of $17$.
In order to explain this more clearly we first give an informal account of the 
technical details.

\subsection{``Intended'' and ``computational'' models}
\label{sec-ss-model}

The declarations made by a {\GF} user specify a first order language 
$L=\langle P,F,C\rangle$, where $P$ is the list of {\predconst}s, $F$ the list 
of {\funconst}s, and $C$ the list of {\indconst}s (see section \ref{sec-decl}).

A model for such a language is a structure $M=\langle D,P',F',C'\rangle$ where
$D$ is a set and $P'$,$F'$ and $C'$ are lists of predicates over $D$, functions 
over $D$, and individuals of $D$ such that the arities of the symbols in $P$ and 
$F$ match the arities of the predicates and functions at the correspondent 
positions in $P'$ and $F'$.
The idea here is that the language $L$ is used for making statements about 
structures such as $M$ (what we call {\bf ``intended'' or ``standard'' model}). 
In particular, when the user writes down a theory in {\GF}, he generally has 
in mind some particular model for his language, and the axioms of his theory 
are intended to express the properties of this particular model.

The fact that {\GF} is really a {\HG} program running in a LISP 
environment, inspires the following idea: some parts of a model for a {\GF} 
language can often be expressed computationally in the sense that the elements 
of $D$ can be represented by s-expressions, and the predicates and functions 
on $D$ can be represented by {\HG} functions and predicates.
It should then be possible to use the computational representation to aid 
{\GF} deductions concerning the model.
For example, suppose the theory we are interested in, is first order number 
theory, and the model that we have in mind is the set of natural numbers 
together with the operations of successor, addition and multiplication.
The numerals have natural representations as {\HG} numbers, and the 
functions in question have {\tt PLUS1}, {\tt PLUS}, {\tt TIMES} as their {\HG}
counterparts.
As mentioned above it should then be possible to use the computational 
representation to provide swift deductions of such statements as $25+37=52$.

The semantic attachment mechanism in {\GF} allows the user to set up these 
computational representations of his subject matter, and the semantic
interpretation mechanism allows to use these representations to aid deduction 
in {\GF}.

With the above overview in mind, let us proceed to the details.

Given a language $L=\langle P,F,C\rangle$ and a model 
$M=\langle D,P',F',C'\rangle$, we define an interpretation function $I$.
For each {\term} $t$ of $L$ in which no free variable occurs, $I(t)$ is the 
individual in $D$ which $t$ denotes.
In particular we define the interpretation of an {\indconst} $c$ to be the 
individual $c'$ in $D$, and where $f$ is a {\funconst}, and the interpretation 
of {\term} $t_1,\ldots,t_n$ are defined, we inductively define the 
interpretation of the {\term} 
$f(t_1,\ldots,t_n)$ to be $f'(I(t_1),\ldots,I(t_n))$.
We may extend the interpretation function to formulas (again without free 
variables) over $L$ by defining $I(w)$ to be the object {\tt TRUE} exactly 
when the formula $w$ is true of the model (for a technical definition 
see \cite{kleene2}).

When $f'$ is the function in a model corresponding to the {\funconst} $f$ in 
$L$, we will also say that $f'$ is the interpretation of $f$, and similarly 
for predconsts.

Now we define a {\bf computational model} to be an object
$K=\langle D',P'',F'',C''\rangle$, where it is understood that $D'$ is a set
of s-expressions, and $P''$, $F''$ and $C''$ are lists of {\HG} predicates,
functions and s-expressions respectively, with the appropriate restrictions on 
arities.

From the extensional point of view, a computational model is for a language
just like a set-theoretic model for a language, except that we do not require
that the functions and predicates concerned be total; that is functions and
predicates may be undefined (non-terminating) for some elements
of $D'$.

We define an {\bf attachment map} $att$ from terms and formulas of $L$ into
$K$ in a manner exactly analogous to the definition of $I$ given above.

We have one last map to worry about, the map {\bf $rep$} which gives, for each
object in the domain $D'$ of the computational model $K$, the object it
represents in the domain $D$ of the model $M$.

Now we may define precisely the meaning of attachments made in the {\GF}
system: the attachment of an {\indconst} $c$ to an s-expression $c''$
signifies that $c$ and $c''$ represent the same object in the model, that is
to say, $I(c)=rep(c'')$.
Similarly, the attachment of a {\funconst} $f$ to a {\HG} function $f''$
signifies that the result of applying $f''$ to an s-expression $c''$ which
represents an individual $c'$ in the model, is a s-expression which represents
the individual $f'(c')$ in the model.
The analogous statements hold for attachments to {\predconst}s.

The above conditions are equivalent to the statement that the 
diagram in figure \ref{fig-ss} commutes.

\begin{figure}[htb]
\begin{center}
\setlength{\unitlength}{0.0125in}%
\begin{picture}(288,260)(92,540)
\thicklines
\put(340,580){\oval(80,80)}
\put(160,580){\oval(80,80)}
\put(160,760){\oval(80,80)}
\put(200,580){\vector( 1, 0){100}}
\put(194,726){\vector( 1,-1){112}}
\put(160,720){\vector( 0,-1){100}}
\put(340,559){\makebox(0,0)[b]{\raisebox{0pt}[0pt][0pt]{\twlrm model}}}
\put(340,577){\makebox(0,0)[b]{\raisebox{0pt}[0pt][0pt]{\twlrm of intended}}}
\put(340,595){\makebox(0,0)[b]{\raisebox{0pt}[0pt][0pt]{\twlrm Domain }}}
\put(160,562){\makebox(0,0)[b]{\raisebox{0pt}[0pt][0pt]{\twlrm sexpr}}}
\put(160,580){\makebox(0,0)[b]{\raisebox{0pt}[0pt][0pt]{\twlrm {\HG}}}}
\put(160,752){\makebox(0,0)[b]{\raisebox{0pt}[0pt][0pt]{\twlrm Terms}}}
\put(160,770){\makebox(0,0)[b]{\raisebox{0pt}[0pt][0pt]{\twlrm {\GF}}}}
\put(250,560){\makebox(0,0)[b]{\raisebox{0pt}[0pt][0pt]{\twlrm Representation}}}
\put(280,660){\makebox(0,0)[b]{\raisebox{0pt}[0pt][0pt]{\twlrm I}}}
\put(120,660){\makebox(0,0)[b]{\raisebox{0pt}[0pt][0pt]{\twlrm attachment}}}
\end{picture}
\end{center}
\label{fig-ss}
\caption{intended model - computational model mappings}
\end{figure}


\subsection{Multiple representation functions}
\label{sec-ss-repr} 

The semantic attachment mechanism allows several representation of the model 
by {\HG} s-expressions to be in force at the same time.
We will seek to motivate this aspect of the semantic attachment mechanism 
by means of an example: consider a theory of chess with includes a general 
theory of lists as a subtheory (this subtheory would be applied in arguments 
about lists of pieces, lists of game positions and so on).
The intended model of such a theory includes at least two kinds of objects: 
chess positions and lists.
Lists and positions form disjoint domains in the model, though it may be 
possible to build lists of chess position.
If we are going to build a computational representation of this model, we will
need to represent positions and lists by s-expressions in such a way that no 
s-expression represents both a list and a position.
The natural representation of a chess position as an s-expression is as a 
list of eight lists, each of which is a list of eight piece names (one of 
which is "empty" or some such), and the natural representation of lists as 
s-expressions is the direct representation as {\HG} lists.
This representation scheme cannot be used, since it will not be possible to 
decide whether a given list of eight lists of eight piece names represents a 
chess board or a list of list of pieces. 
That is to say, the map $rep$ will not be well defined. 
It is of course not hard to solve this problem by the use of some slightly 
fancier coding, but a general solution to the problem of disambiguating 
computational representations is available.
Suppose that the intended model of a {\GF} theory $T$ includes the disjoint 
domains $D_1,\ldots,D_n$, and suppose further that we have a different coding 
function for each of these domains.
That is we have $n$ different {\bf representation functions} $rep_i$ which map 
the domain of s-expressions into domain of the model, with the property that 
the range of $rep_i$ is a subset of $D_i$.
Then it is possible that a single s-expression codes two different objects 
$d_i$, $d_j$ in the model, but as long as we know what coding function $rep_i$
to apply, there is no ambiguity. 


Then the definition of the $att$ map may be extended to take account of the 
possibility of multiple representations in the following way: the domain of 
the $att$ map will still consist of the set of {\GF} terms and formulas, but
its range will now lie in the set of pairs of the form $\langle$ representation
function, s-expression $\rangle$.

The soundness condition for the $att$ map is now that, when 
$att(t)=\langle rep, c'' \rangle$, we have $rep(c'')=I(t)$.
In order to specify this new more complicated $att$ map, the user of the {\GF}
system must give representation information concerning his attachments.

Specifically, each representation function must be given a name and when the 
attachment to an {\indconst} is given, the name of the associate 
representation function must be given as well.
Similarly, when the attachment $f''$ to a {\funconst} $f$ is specified, the 
(names of the) representations of its arguments and of the value it returns 
must be given, and when the attachment to a {\predconst} is specified, the 
representations of its arguments must also be specified.

The significance of specifying that the representations of the arguments and 
value of the attachment $f''$ to a {\funconst} $f$ are 
$R_1,\ldots,R_n$ and $R_{n+1}$ respectively, is that 
$R_{n+1}(f''(c''_1,\ldots,c''_n))=f'(R_1(c''_1),\ldots,R_n(c''_n))$ 
where $f'$ is the interpretation of $f$, whenever $c''_1$,..,$c''_n$ are 
s-expressions in the domains of $R_1$,..,$R_n$.
The same holds for attachments to {\predconst}, mutatis mutandis.
Given the attachments with representation information for individual symbols,
the map $att$ on the domain of terms and formulas is defined inductively in 
the obvious way: if $f$ is attached to $f''$ and the declared representation 
of the arguments of $f''$ are $R_1,\ldots,R_n$ and terms $t_1$,..,$t_n$ have 
attachments with representations $R_1$,..,$R_n$ then 
$att(f(t_1,\ldots,t_n))=f''(att(t_1),\ldots,att(t_n))$.
Under this definition the diagram above commutes for each individual 
representation function.

Note that if the representation of the attachment of any term $t_i$ does not 
match that of its place in the argument list, then 
$f''(att(t_1),\ldots,att(t_n))$ cannot be expected to represent the 
interpretation of $f(t_1,\ldots,t_n)$.
The reason for this is that the correctness of a computation which purports to
represent a mathematical function depends on the representation of the 
arguments of the function as data objects.
For example, no one would expect a floating point multiplication algorithm to 
behave correctly if its arguments were encoded as integers rather than 
floating point numbers.

Finally, note that the attachment map, as well as the s-expressions which 
represent functions, may be partial.
The user is never required to provide an attachment for any {\GF} symbol, nor is
any attachment to a {\funconst} or {\predconst} required to be complete.

The semantic simplification mechanism will use whatever information is 
available and if there will be insufficient information, it will return this 
fact to the user. 




% user commands for semantic simplification
\gfcommand{attach}{semantic attachment}
\index{attach}

\gfsyntax{
  attach \ARG{indconst}  to \ALT dar [ rep ] \ARG{sexpr};\\
  attach \ARG{sentconst} to T \ALT NIL \ALT UNDEF;\\
  attach \ARG{funconst} \ALT \ARG{predconst} to \ARG{atom};\\
  attach \ARG{funconst}  to [ \ARG{rep1}, \SEQ, \ARG{repN} = \ARG{repM} ]
  \ARG{atom};\\
  attach \ARG{predconst} to [ \ARG{rep1}, \SEQ, \ARG{repN} ] \ARG{atom};
}

\gfdescription{
  Defines the attachment for the {\GF} constants.
  \ARG{repI} can be a representation function or an asterisk; if it is an
  asterisk or no representation is specified, then the default representation
  function {\tt UNIVERSALREP} is taken.
  \ARG{indconst}s can be attached either ``one way'' (using {\tt to}) or ``two ways''
  (using {\tt dar}).
  The ``two ways'' attachment tells the semantic interpretation mechanism 
  that whenever \ARG{sexpr} is computed as the {\HG} representation of 
  a term $t$, then the attached {\GF} \ARG{indconst} should be returned as the 
  simplified version of $t$.
  That is, not only \ARG{sexpr} is the {\HG} representation of \ARG{indconst}, but 
  \ARG{indconst} is the preferred {\GF} name of (the intended model value denoted 
  by) the {\HG} object \ARG{sexpr}.
  \ARG{sentconst}s can be attached to the three possible truth values
  corresponding to true, false and undefined\cite{kleene1}.
  This is done by attaching a sentconst to {\tt T}, {\tt NIL}, {\tt UNDEF}
  respectively.
  \ARG{funconst}s and \ARG{predconst}s can be attached to a {\HG} \ARG{atom}
  \cite{giunchiglia35}.
  \ARG{atom} will be used as the identifier of a {\HG} function, whose number of 
  arguments is supposed to match the arity of the {\GF} symbol.
}

\gfrecap{
Defines the attachment for the GETFOL constants.
`repI' can be a representation function or an asterisk; if it is an
asterisk or no representation is specified, then the default representation
function `UNIVERSALREP' is taken.
`indconst's can be attached either ``one way'' (using `to') or ``two ways''
(using `dar').
The ``two ways'' attachment tells the semantic interpretation mechanism 
that whenever `sexpr' is computed as the HGKM representation of 
a term `t', then the attached GETFOL `indconst' should be returned as the 
simplified version of `t'.
That is, not only `sexpr' is the HGKM representation of `indconst', but 
`indconst' is the preferred GETFOL name of (the intended model value denoted 
by) the HGKM object `sexpr'.
`sentconst's can be attached to the three possible truth values
corresponding to true, false and undefined.
This is done by attaching a sentconst to `T', `NIL', `UNDEF'
respectively.
`funconst's and `predconst's can be attached to a HGKM `atom'
`atom' will be used as the identifier of a HGKM function, whose number of 
arguments is supposed to match the arity of the GETFOL symbol.
}

\gfexample+
   ***** declare indconst a b;
   ***** declare sentconst s;
   ***** declare funconst f 1;
   ***** declare predconst p 1;
   ***** decrep rep;

   ***** attach a to a;
   a attached to 'a
   ***** attach a dar a;
   a attached to 'a
   a is the preferred name of a
   ***** attach a dar [rep]a;
   a attached to 'a
   ***** attach a dar [rep]b;
   a is already an preferred name in this representation
   ***** attach b dar [rep]a;
   a has already a preferred name in this representation

   ***** COMMENT | deflam defines an HGKM function |;
   ***** deflam f(x) x;
   ***** attach f to f;
   f attached to f
   ***** deflam p(x) (IF (EQUAL x (QUOTE a))TRUE FALSE);
   ***** attach p to p;
   p attached to p
   attach f to [repin=repout]f;
   f attached to f
   ***** deflam p1(x) (IF (EQUAL x (QUOTE b))TRUE FALSE);
   ***** attach p to [repin]p1;
   p attached to p1
   ***** attach p to [repin]p1;
   p has already an attachment with these representation informations
   ***** attach s to T;
   s attached to 'T
   ***** attach s to NIL;
   s has already an attachment
+

\gfcommand{decrep}{representation declaration}
\index{decrep}

\gfsyntax{
  decrep \ARG{replabel1} \OPT{\SEQ \ARG{replabelN}};
}

\gfdescription{
  It declares \ARG{replabelI} to be representation functions.\\
  The only builtin representation functions are {\tt NATNUMREP}, {\tt TRUTHREP}
  and {\tt UNIVERSALREP}, the representation functions for natural numbers, for 
  truth values and for default representation respectively.
  Numerals have a builtin attachment to {\HG} numbers in the representation 
  function {\tt NUMERALREP}.
}

\gfrecap{
It declares `replabelI' to be representation functions.
The only builtin representation functions are `NATNUMREP', `TRUTHREP'
and `UNIVERSALREP', the representation functions for natural numbers, for 
truth values and for default representation respectively.
Numerals have a builtin attachment to HGKM numbers in the representation 
function `NUMERALREP'.
}

\gfexample+
   ***** decrep rep1 rep2;
+

\gfnotes{
  Since the intended model itself appears nowhere in the {{\GF}} system, there
  is no need for the user to give any detailed information about the nature of 
  the representation maps which he has in mind.
  {\tt NATNUMREP} is known by {\GF} only after typing {\tt know natnums}.
}

\gfcommand{hardware}{semantic attachment to values of a s-expression}
\index{hardware}

\gfsyntax{
  hardware \ARG{indconst} to \ALT dar \ARG{sexpr};\\
  hardware \ARG{indconst} to \ALT dar [ \ARG{rep} ] \ARG{sexpr};
}

\gfdescription{
  This command is similar to the attach command for \ARG{indconst}.
  The difference is that, if \ARG{sexpr} changes value over time, then so does
  the value of the attachment.
  It is a "dynamic attachment" in the sense that it is attached to the values the
  \ARG{sexpr} assumes over time.
}

\gfrecap{
This command is similar to the attach command for `indconst'.
The difference is that, if `sexpr' changes value over time, then so does
the value of the attachment.
It is a "dynamic attachment" in the sense that it is attached to the values the
`sexpr' assumes over time.
}

\gfexample+
   ***** declare indconst clock t0 t1;
   ***** attach t0 to 0;
   t0 attached to '0
   ***** attach t1 to 1;
   t1 attached to '1
   ***** hardware clock to time;
   clock attached to time
   ***** done;
   >(SETQ time 0)
   0
   >(GETFOL)
   Hi!  Glad your back.  What would you like to talk about now?
   ***** simplify clock = t0;
   1   clock = t0     
   ***** done;
   >(SETQ time 1)
   1
   >(GETFOL)
   Hi!  Glad your back.  What would you like to talk about now?
   ***** simplify clock = t1;
   2   clock = t1     
+

\gfnotes{
  This command gives the possibility of changing the intended model. 
}

\gfcommand{represent}{default representation for sorts}
\index{represent}
\label{sec-rep-sort}

\gfsyntax{
  represent \{ \ARG{sort1}, \SEQ, \ARG{sortN} \} as \ARG{rep} \ALT \ARG{*};
}

\gfdescription{
  Sets the default representation for \ARG{sort1}, \SEQ, \ARG{sortN} to be
  \ARG{rep} ({\tt UNIVERSALREP} if \ARG{*} is specified).
  The default representation is used by the reflect command (see section
  \ref{sec-refl}).
}

\gfrecap{
Sets the default representation for `sort1', ..., `sortN' to be `rep'
(`UNIVERSALREP' if `*' is specified).
The default representation is used by the `reflect' command.
}

\gfexample+
   ***** declare sort s t;
   ***** decrep rep;
   ***** represent {s t} as rep;
   ***** represent {s t} as rep;
   s has already a default representation
+

\gfcommand{simplify}{semantic simplification}
\index{simplify}
\label{sec-simplify}

\gfsyntax{
  simplify \ARG{wff} \ALT \ARG{fact} \ALT \ARG{term};
}

\gfdescription{
  If \ARG{term} is provided as argument, three steps are performed:\\
  %
  \begin{itemize}
  \item 
    the interpretation of \ARG{term} in the computational model is computed;
  \item a preferred name for the interpretation is found;
  \item the equality of \ARG{term} with the preferred name is asserted as the next
    line of the proof.
    No action is taken if \ARG{term} has no interpretation in the computational
    model (has an undefined interpretation) or a preferred name for its
    interpretation does not exist.
  \end{itemize}
  %
  If {\em wff} is provided as an argument, two steps are performed:
  %
  \begin{itemize}
  \item the interpretation of {\em wff} in the computational model is computed 
    (in this case the interpretation will be {\tt TRUE}, {\tt FALSE} or an 
    undefined truth value)
  \item if the interpretation of {\em wff} is {\tt TRUE}, {\em wff} 
    is asserted as the next line of the proof, if it is {\tt FALSE} 
    the negation of {\em wff} is asserted.
  \end{itemize}
  %
  When \ARG{fact} is provided as argument, the simplify command works on the wff
  of \ARG{fact}.
}

\gfrecap{
  If `term' is provided as argument, three steps are performed:
      * the interpretation of `term' in the computational model is computed;
      * a preferred name for the interpretation is found;
      * the equality of $term$ with the preferred name is asserted as the next
        line of the proof. No action is taken if `term' has no interpretation
        in the computational model (has an undefined interpretation) or a
        preferred name for its interpretation does not exist.
  If `wff' is provided as an argument, two steps are performed:
      * the interpretation of `wff' in the computational model is computed 
        (in this case the interpretation will be `TRUE', `FALSE' or an 
        undefined truth value)
      * if the interpretation of `wff' is `TRUE', `wff' is asserted as the next
        line of the proof, if it is `FALSE' the negation of `wff' is asserted.
  When `fact' is provided as argument, the simplify command works on the wff
  of `fact'.
}

\gfexample+
   ***** declare indconst a b c;
   ***** decrep REP;
   ***** attach a dar [REP]a;
   a attached to 'a
   a is the preferred name of a
   ***** attach b dar [REP]b;
   b attached to 'b
   b is the preferred name of b
   ***** attach c to [REP]c;
   c attached to 'c
   ***** declare funconst F 1;
   F has been declared to be a Funconst
   ***** DEFLAM F(x) (IF (EQ x (QUOTE a)) (QUOTE b) 
                         (IF (EQ x (QUOTE b)) (QUOTE c)
                          (QUOTE UNDEF&)));
   ***** attach F to [REP=REP]F;
   F attached to F
   ***** simplify F(a);
   1   F(a) = b    
   ***** simplify F(b);
   F(b) : No simplification is possible.
   ***** simplify F(c);
   F(c) : No simplification is possible.
   ***** declare predconst P 1;
   P has been declared to be a Predconst
   ***** DEFLAM P(x) (IF (EQ x (QUOTE a)) TRUE 
                         (IF (EQ x (QUOTE b)) FALSE 
                          (QUOTE UNDEF&)));
   ***** attach P to [REP]P;
   P attached to P
   ***** simplify P(a);
   2   P(a)  
   ***** simplify P(b);
   3   not P(b)    
   ***** simplify P(c);
   P(c) : No simplification is possible.
   ***** extension UNIVERSAL by {a b c};
   Now the extension of UNIVERSAL is fixed to be : (a b c)
   ***** declare indvar x;
   UNIVERSAL is a sort
   x has been declared to be an Indvar
   ***** simplify exists x.F(x)=b;
   4   exists x. (F(x) = b)
   ***** simplify forall x.P(x);
   5   not forall x. P(x) 
+

\gfnotes{
  In the case of sorts with extensions (see the command {\tt extension} in section 
  \ref{sec-sort}) quantification is considered as {\bf bounded quantification}.
  In other words, let $P$ be a predicate and $x$ an indvar of sort $S$, where
  $S$ has extension $\{s_1,\ldots,s_n\}$.
  Then the following equivalences hold:
  $$ 
  \forall x P(x)\liff (P(s_1) \con \ldots \con P(s_n))
  $$
  $$
  \exists x P(x)\liff (P(s_1) \dis \ldots \dis P(s_n))
  $$
  %
  The command explicitly unfolds universal/existential statements into 
  their propositional equivalents. 
}



% introduction to the syntactic simplification's section
\newpage
\section{Syntactic simplification}
\label{sec-rew}

The subsections \ref{sec-rew-intro} and \ref{sec-rew-simpset} 
of this section have been taken from \cite{rww3}.


\subsection{Introduction}
\label{sec-rew-intro}

The basic idea  of syntactic simplification is repeated substitution of 
selected equalities and equivalences into a given expression.
More precisely, let $E$ be a set of universally quantified equations and 
equivalences ("rewrite rules"), so members of $E$ look like:
\begin{itemize}
\item $\forall\ {\vec x}.(t_1=t_2)$
\item $\forall\ {\vec y}.(F_1\ \liff \ F_2)$
\end{itemize}
where ${\vec x}$ and ${\vec y}$ are the {\indvar} sequences $x_1$...$x_n$
and $y_1$..,$y_m$, $t_1$ and $t_2$ are {\term}s, and $F_1$, $F_2$ are {\wff}s.

A match, or an immediate simplification, of a {{\GF}} expression $exp$ consists
 of replacing an occurrence of $t_1[x \leftarrow u]$($F_1[y \leftarrow v]$) 
in $exp$ by $t_2[x \leftarrow u]$($F_2[y \leftarrow v]$), where $u$($v$) is a 
sequence of terms and where $\leftarrow$ indicates substitution.

There are two problems to solve:
\begin{enumerate}
\item There may be more than one equation (or equivalence) whose left half 
      matches a given expression, so one has to establish a precedence 
      hierarchy for matching.

\item The order used by the algorithm to consider the subexpressions of a 
      given expression.
\end{enumerate}

{{\GF}}'s solution to the first problem is the following ordering expression:
each simplification expression (i.e., left half of a rewrite rule) is 
regarded as a linear string of atoms.
Each atom is either:

\begin{itemize}
\item a {\bf constant} (which is not bound by the universal quantifier in the
                       prefix);

\item an {\bf old variable} (which is bound by the universal quantifier in 
                            the prefix and which has occurred before in the 
                            linear string);

\item a {\bf new variable} (which is bound by the universal quantifiers in the
                           prefix and which has not occurred before in the 
                           linear string);
\end{itemize}

If we think of concatenating different atoms to a given initial string, then 
the atoms have this precedence ordering:
\begin{center}
constants $<$ old variables $<$ new variables
\end{center}
and expressions are ordered lexicographically in accordance with this 
ordering on atoms.

Let's consider, for example, the precedence relations among the simplification
 expressions :
$f(a,b,b)$, $f(a,b,c)$, $f(a,a,x)$, $f(a,x,x)$, $f(a,x,y)$, $f(x,x,x)$,
$f(x,x,y)$, 
where $f$, $a$, $b$, $c$ are constants and $x$, $y$ are variables.

The last four expressions are linearly ordered:
$$
f(a,x,x)<f(a,x,y)<f(x,x,x)f<(x,x,y)
$$
and each of the first three expressions is less than $f(a,x,x)$ and 
incomparable to the other two of the first three expressions:
$$
f(a,b,b)<f(a,x,x)
$$
$$
f(a,b,c)<f(a,x,x)
$$
$$
f(a,a,x)<f(a,x,x)
$$
Together with transitivity, these inequalities completely define the 
precedence relation.

As far as regard the second problem, {{\GF}}'s syntactic simplification code 
basically considers subexpressions of $exp$ in the usual left-to-right order.
The exceptions occur after a subexpression $exp'$ has been matched (and 
substituted for).
The algorithm then begins again at the subexpression one level above $exp'$.

The syntactic simplification algorithm has the usual problems of rewrite rules.
A typical difficulty is the infinitely recurring substitutions: 
for example if one uses $\forall\ x.x+y=y+x$ as simplification equation, 
the algorithm will attempt to make this substitution without end.

\subsection{Simplification sets}
\label{sec-rew-simpset}

Syntactic simplification in {\GF} is performed by using 
{\bf syntactic simplification sets} (called {\bf simpsets} from now on).
Simpsets contain a label ({\em simplabel}) to identify the
rewrite rules used to rewrite expressions.
{\GF} has built-in simpsets, but the user can define his own ones: he can specify a 
set of formulae or facts as rewrite rules in a {\bf basic simspset}
or he can compose already defined simpsets in {\bf compound simpsets}.

The {{\GF}} builtin simpsets (see figure \ref{fig-simpset}) are
{\bf \tt LPROPTREE}, {\bf \tt LQUANTREE}, {\bf \tt LARGIFTREE} and 
{\tt LOGICTREE}.
{\tt LPROPTREE} contains a set of rewrite rules 
corresponding to basic logical equivalences (e.g. $P\con\neg P\liff \bot$). 
{\tt LQUANTREE} contains a set of rewrite rules 
corresponding to logic equivalences for quantified formulas.
{\tt LARGIFTREE} contains a set of rewrite rules corresponding to logic
equalities and equivalences for conditional terms and formulas.
{\tt LOGICTREE} is the union of all the previous builtin simpsets.

\newpage

\begin{figure}[htbp]
\begin{center}
\fbox{
\parbox{16cm}{
$$
\begin{array}{ll}
\neg \neg P \liff P    & \ \ \                         \\
\neg {\tt TRUE} \liff \bot   & \ \ \  \neg \bot \liff {\tt TRUE}    \\
P \con \bot \liff \bot & \ \ \  \bot \con P\liff \bot   \\
P \con {\tt TRUE}  \liff  P  & \ \ \  {\tt TRUE}  \con P\liff P     \\
\neg P\con P\liff\bot  & \ \ \  P \con \neg P\liff \bot \\
P\con P\liff P         & \ \ \                         \\
P     \dis \bot \liff P  & \ \ \  \bot \dis P \liff P       \\
P \dis {\tt TRUE} \liff {\tt TRUE}   & \ \ \  {\tt TRUE}  \dis P \liff {\tt TRUE}   \\
\neg P \dis P\liff {\tt TRUE}  & \ \ \  P \dis \neg P \liff {\tt TRUE}  \\
P     \dis P     \liff P & \ \ \                         \\
P \imp \bot\liff\neg P & \ \ \  \bot \imp P\liff {\tt TRUE}   \\
P\imp {\tt TRUE}\liff {\tt TRUE}   & \ \ \  {\tt TRUE}  \imp P     \liff P\\
\neg P \imp P \liff P  & \ \ \  P \imp \neg P\liff\neg p\\
P \imp P\liff {\tt TRUE}     & \ \ \                         \\
P\liff \bot\liff\neg P  & \ \ \  \bot \liff P\liff\neg P  \\
P\liff {\tt TRUE}\liff P      & \ \ \  {\tt TRUE}  \liff P\liff P     \\
\neg P \liff P\liff\bot & \ \ \  P\liff\neg P\liff\neg\bot\\
P\liff P\liff {\tt TRUE}      & \ \ \                         \\
\end{array}
$$
}}
\fbox{
\parbox{16cm}{
$$
\begin{array}{ll}
\forall x.{\tt TRUE}  \liff {\tt TRUE} & \ \ \  \forall x.\bot \liff \bot    \\
\exists x.{\tt TRUE}  \liff {\tt TRUE} & \ \ \  \exists x.\bot \liff \bot    \\
\end{array}
$$
}}
\vspace{0.3cm} 
\fbox{
\parbox{16cm}{
$$
\begin{array}{ll}
\forall x y.{\em trmif}\ \bot{\em then}\ x\ {\em else}\ y = y    & \ \ \ 
\forall x y.{\em trmif}\ {\tt TRUE}\ {\em then}\ x\ {\em else}\ y\ = x \\
\forall x.  {\em trmif}\ P\ {\em then}\ x\ {\em else}\ x = x & \ \ \  \\
{\em wffif}\ \bot\ {\em then}\ P1\ {\em else}\ P2\ \liff P2 & \ \ \ 
{\em wffif}\ {\tt TRUE}\ {\em then}\ P1\ {\em else}\ P2\ \liff\ P1 \\
{\em wffif}\ P\ {\em then}\ P1\ {\em else}\ P1\ \liff\ P1 & \ \ \  \\
\end{array}
\\
$$
}}
\caption{The rewrite rules of {\tt LPROPTREE}, {\tt LQUANTREE}
\label{fig-simpset}
and {\tt LARGIFTREE}.}
\end{center}
\end{figure}


% user commands for syntactic simplification
\gfcommand{assertsimp}{simpset command}
\index{assertsimp}

\gfsyntax{
  assertsimp \ARG{simplabel}; 
}

\gfdescription{
  It generates a proof step for each formula contained in the rewrite rules
  of \ARG{simplabel}, which cannot be a builtin simpset.
}

\gfrecap{
It generates a proof step for each formula contained in the rewrite rules
of `simplabel', which cannot be a builtin simpset.
}

\gfexample+
   ***** declare indconst a b;
   ***** declare predconst q r 1;
   ***** declare indvar x;
   ***** setbasicsimp s1 at wffs {q(a), forall x.(q(x) iff r(x))};
   ***** assume a=b;
   1   a = b     (1)
   ***** assume q(b);
   2   q(b)     (2)
   ***** setbasicsimp s2 at facts {1,2};
   ***** setbasicsimp s2 at facts {1};
   s2 is already the label of a simpset
   ***** setcompsimp s4 at s1 uni s2;
   ***** assertsimp s1;
   3   q(a)     
   4   forall x. (q(x) iff r(x))     
   ***** assertsimp s2;
   s2 does not contain any wff to assert.
   ***** assertsimp s4;
   5   q(a)
   6   forall x. (q(x) iff r(x))
   ***** assertsimp LOGICTREE;
   LOGICTREE is the label of a builtin simpset, you can't assert it.
+
\gfcommand{rewrite}{syntactic simplifier command}
\index{rewrite}

\gfsyntax{
  rewrite \ARG{wff} \ALT \ARG{fact} \ALT \ARG{term} \OPT{by \ARG{simpexpr}};
}

\gfdescription{
  It rewrites the given expression by using the union of the rewrite rules 
  indicated by the \ARG{simpexpr}.
  If \ARG{simpexpr} is not specified, {\tt LOGICTREE} is used.
  If \ARG{term} is provided as an argument, two steps are performed:
  %
  \begin{itemize}
  \item \ARG{term} is rewritten by using the set of rewrite rules 
    indicated by \ARG{simpexpr}.
  \item the equality of \ARG{term} with its rewritten form is asserted as the next 
    line of the proof. The dependencies depend on the simpsets actually used
    during the syntactic simplification.
  \end{itemize}
  %
  If \ARG{wff} is provided as argument, also two steps are performed:
  %
  \begin{itemize}
  \item \ARG{wff} is rewritten by the set of rewrite rules of the
    \ARG{simpexpr} result.
  \item if \ARG{wff} is rewritten to {\tt TRUE}, \ARG{wff} is asserted as the
    next line in the proof, if \ARG{wff} is rewritten to {\tt FALSE},
    $\neg$ \ARG{wff} is asserted, otherwise the equivalence of \ARG{wff} with 
    its rewritten form is asserted.
    The dependencies depend on the simpsets used during the syntactic
    simplification.
  \end{itemize}
  %
  When \ARG{fact} is provided as an argument, the rewrite command works on the wff
  of the \ARG{fact}.
}

\gfrecap{
It rewrites the given expression by using the union of the rewrite rules 
indicated by the `simpexpr'.
If `simpexpr' is not specified, LOGICTREE is used.
}

\gfexample+
   ***** declare indconst A,B;
   ***** declare indvar X,Y;
   ***** declare funconst F 2;
   ***** declare funconst G 1;
   ***** declare sentconst P;
   ***** assume forall X . F(X,A) = A;
   1   forall X. (F(X,A) = A)     (1)
   ***** assume forall X . F(X,X) = G(X);
   2   forall X. (F(X,X) = G(X))     (2)
   ***** assume forall X Y . F(X,Y) =Y;
   3   forall X Y. (F(X,Y) = Y)     (3)
   ***** axiom F1:forall X . F(X,A) = A;
   F1 : forall X. (F(X,A) = A)
   ***** axiom F2:forall X . F(X,X) = G(X);
   F2 : forall X. (F(X,X) = G(X))
   ***** axiom F3:forall X Y. F(X,Y) = Y;
   F3 : forall X Y. (F(X,Y) = Y)
   ***** setbasicsimp S1 at facts {1};
   ***** setbasicsimp S2 at facts {2};
   ***** setbasicsimp S3 at facts {3};
   ***** setbasicsimp S4 at facts {F1};
   ***** setbasicsimp S5 at facts {F2};
   ***** setbasicsimp S6 at facts {F3};
   ***** setbasicsimp SIMPEQ at wffs {forall X.(X=X iff TRUE)};
   ***** setcompsimp S7 at S1 uni S2 uni S3;
   ***** rewrite F(A,A) by S6;
   4   F(A,A) = A     
   ***** rewrite F(A,A) by S5;
   5   F(A,A) = G(A)     
   ***** rewrite F(A,A) by S4;
   6   F(A,A) = A     
   ***** rewrite F(A,A) by S1;
   7   F(A,A) = A     (1)
   ***** rewrite F(A,A) by S2;
   8   F(A,A) = G(A)     (2)
   ***** rewrite F(A,A) by S3;
   9   F(A,A) = A     (3)
   ***** rewrite F(A,A) by S7;
   10   F(A,A) = A     (1)
   ***** rewrite F(B,B) by S1 uni S3;
   11   F(B,B) = B     (3)
   ***** rewrite F(B,B) by S1;
   F(B,B): No simplification is possible
   ***** rewrite not TRUE by S1;
   not TRUE: No simplification is possible
   ***** rewrite not TRUE;
   12   not (not TRUE)     
   ***** rewrite TRUE imp (P imp X=X);
   13   (TRUE imp (P imp (X = X))) iff (P imp (X = X)) 
   ***** rewrite TRUE imp (P imp X=X) by SIMPEQ uni LOGICTREE;
   14   TRUE imp (P imp (X = X))
   ***** rewrite F(A,A) by S7;
   15   F(A,A) = A     (1)
   ***** rewrite F(A,A)=A by S7;
   16   F(A,A) = A  iff (A = A)   (1)
   ***** rewrite F(A,A)=A by S7 uni SIMPEQ;
   17   F(A,A) = A     (1)
   ***** rewrite F(A,A)=G(A) by S7;
   18   (F(A,A) = G(A)) iff (A = G(A))     (1)
   ***** rewrite F(B,B) by S7;
   19   F(B,B) = G(B)     (2)
   ***** rewrite F(B,B)=G(B) by S7 uni SIMPEQ;
   20   F(B,B) = G(B)     (2)
   ***** rewrite F(B,B)=G(B) and F(A,A)=A by S7 uni SIMPEQ uni LOGICTREE;
   21   (F(B,B) = G(B)) and (F(A,A) = A)     (1 2)
   ***** rewrite F(A,A) by S7 dif S1 ;
   22   F(A,A) = G(A)     (2)
   ***** rewrite F(A,A) by S7 dif (S1 uni S2);
   23   F(A,A) = A     (3)
   ***** rewrite F(A,A)=A by S3 dif (S1 uni S2) uni SIMPEQ;
   24   F(A,A) = A     (3)
+
   


% introduction to the syntactic/semantic simplification's section
\newpage
\section{Syntactic and semantic simplification}

Some of the commands perform both syntactic and semantic simplifications.


% user commands for the syntactic/semantic simplification's section
\gfcommand{eval}{mixed simplifier command}
\index{eval}
\label{sec-eval}

\gfsyntax{
  eval \ARG{wff} \ALT \ARG{fact} \ALT \ARG{term} \OPT{by \ARG{simpexpr}};
}

\gfdescription{
  This command evaluates the expression (\ARG{wff}, the wff of \ARG{fact}
  or \ARG{term} respectively) by combining the semantic evaluation of the
  expression in the simulation structure and the syntactical rewriting
  performed by using the union of the rewrite rules indicated by 
  \ARG{simpexpr}.
}

\gfrecap{
  This command evaluates the expression (`wff', the wff of `fact'
  or `term' respectively) by combining the semantic evaluation of the
  expression in the simulation structure and the syntactical rewriting
  performed by using the union of the rewrite rules indicated by 
  `simpexpr'.
}

\gfexample+
   ***** declare indconst a b c;
   ***** decrep REP;
   ***** attach a dar [REP]a;
   a attached to 'a
   a is the preferred name of a
   ***** attach b dar [REP]b;
   b attached to 'b
   b is the preferred name of b
   ***** attach c dar [REP]c;
   c attached to 'c
   c is the preferred name of c
   ***** declare funconst G 2;
   ***** declare indvar x y;
   ***** setbasicsimp S at wffs {forall x y.G(x y)=x};
   ***** declare predconst P 1;
   ***** DEFLAM P(x) (IF (EQ x (QUOTE a)) TRUE 
                         (IF (EQ x (QUOTE b)) FALSE 
                          (QUOTE UNDEF&)));
   ***** attach P to [REP]P;
   P attached to P
   ***** eval P(G(a,G(b,c))) by S;
   1   P(G(a,G(b,c)))     
   ***** eval P(G(b,c)) and P(c) by S;
   2   not (P(G(b,c)) and P(c))
   ***** eval P(G(c,a)) by S;
   3   P(G(c,a)) iff P(c)   
   ***** eval forall x.P(G(x x)) by S;
   4   forall x. P(G(x,x)) iff forall x. P(x)     
   ***** extension UNIVERSAL by {a b c};
   Now the extension of UNIVERSAL is fixed to be : (a b c)
   ***** eval forall x.P(G(x x)) by S;
   5   not forall x. P(G(x,x)) 
+

\gfnotes{
  In the case of sorts with extensions (see the command {\tt extension} in section 
  \ref{sec-sort}) quantification is considered as  {\bf bounded quantification}.
  In other words, let $P$ be a predicate and $x$ an indvar of sort $S$, where
  $S$ has extension $\{s_1,\ldots,s_n\}$.
  Then the following equivalences hold:
  $$ 
  \forall x(P(x)\liff P(s_1) \con \ldots P(s_n))
  $$
  $$
  \exists x(P(x)\liff P(s_1) \dis \ldots P(s_n))
  $$
  %
  The command explicitly unfolds universal/existential statements into 
  their propositional equivalents. The expansion is performed 
  syntactically, that is the formula $\forall x P(x)$ [$\exists x P(x)$]
  is rewritten as $P(s_1) \con \ldots \con P(s_n)$
  [$P(s_1) \dis \ldots \dis P(s_n)$].
  The expansion mechanism embedded in {\tt simplify} is not used by {\tt eval}.
}


\gfcommand{let}{evaluation plus attachment}
\index{let}

\gfsyntax{
  let \ARG{\indconst} to \ALT dar [ \ARG{rep} ] \ARG{term};
}

\gfdescription{
  This command evaluates the \ARG{term} using the mixed evaluation mechanism
  of the {\tt eval} command.
  If the evaluation returns a {\HG} representation for \ARG{term}, then it is
  attached to  \ARG{indconst} with representation function \ARG{rep}. Then the 
  equality of \ARG{indconst} with \ARG{term} is asserted as the next line in the
  proof, otherwise an error message is given. If \ARG{rep} is not specified the
  representation is the default representation. If this does not happen {\GF}
  outputs an error message.
}

\gfrecap{
This command evaluates the `term' using the mixed evaluation mechanism
of the {\tt eval} command.
If the evaluation returns a HGKM representation for `term', then it is
attached to  `indconst' with representation function `rep'. Then the 
equality of `indconst' with `term' is asserted as the next line in the
proof, otherwise an error message is given. If `rep' is not specified the
representation is the default representation. If this does not happen GETFOL
outputs an error message.
}

\gfexample+
   ***** declare indconst a b c;
   ***** attach b to b;
   b attached to 'b
   ***** attach c to c;
   c attached to 'c
   ***** declare funconst h 2;
   ***** DEFLAM h(x y) (QUOTE d);
   ***** attach h to h;
   h attached to h
   ***** let a dar h(b c);
   a attached to 'd
   a is the preferred name of d
   1   a = h(b,c)     
+



 	% loading introduction to the section
\newpage
\section{Multiple contexts}
\label{sec-cxt}

\subsection{Introduction}

\begin{quote}\em
  ... When reasoning, people seem to be able to switch focus of their
  attention and make always some sort of local reasoning ...
  \cite{giunchiglia2}
\end{quote}

Structuring the knowledge into {\em distinguished partial descriptions of the
world}, has been hinted as a cognitively plausible hypotheses. 
Distinct partial descriptions can be represented by {\GF} {\bf contexts}.
A {\GF} context contains its own language defined by a set of declarations, 
its own axioms and definitions, and its own computational model.
Reasoning can be performed within a context. 
You can type any command defined so far within any context in {\GF}.
Multiple proofs can be performed within a context.
When you work in {\GF} you are always in one context.
The context in which you are working in is called the {\em current context}.
When you enter the system the current context is empty and without name.
If you want to leave the context to work in another one, you have to give the
context a name to refer to it later (by the command {\tt namecontext}).
You can create a new context by using {\tt makecontext}, and  switch to it
by using {\tt switchcontext}.
The context you switch to then becomes the current context.


% loading explanation of commands
\gfcommand{copycontext}{Multiple contexts manipulation}
\index{copycontext}

\gfsyntax{
  copycontext \ARG{ctx-name};
}

\gfdescription{
  A new context with name \ARG{ctx-name} is created, and the current context
  is copied in it.
}

\gfrecap{
  A new context with name `ctx-name' is created, and the current context
  is copied in it.
}

\gfexample+
   ***** show whereami;
   You are now using an unnamed context.
   You are now using an unnamed proof.
   ***** namecontext C1;
   You have named the current context: C1
   ***** show whereami;
   You are now using context: C1
   You are now using an unnamed proof.
   ***** declare sentconst A;
   ***** makecontext C2;
   You have created the empty context: C2
   ***** switchcontext C2;
   You are now using context: C2
   ***** declare indvar A;
+
\gfcommand{copylex}{Language declaration through contexts}
\index{copylex}

\gfsyntax{
	copylex	\ARG{ctx-name};
}

\gfdescription{
	This command copies in the current context all the symbols and sorts
	declared in the context {\em ctx-name}.
	The command has no effects in the case there is at least a symbol in the
	current context that has the same name as a symbol in the context
	{\em ctx-name}.
}

\gfrecap{
This command copies in the current context all the symbols and sorts
declared in the context `ctx-name'.
The command has no effects in the case there is at least a symbol in the
current context that has the same name as a symbol in the context `ctx-name'.
}


\gfexample+
   ***** declare indconst a b;
   ***** declare sentconst A B;
   ***** declare sort S1 S2;
   ***** namecontext C1;
   You have named the current context: C1
   ***** makecontext C2;
   You have created the empty context: C2
   ***** switchcontext C2;
   You are now using context: C2
   You are switching to a proof with no name.
   ***** probe declare;
   ***** copylex C1;
   S1 has been declared to be a sort
   S2 has been declared to be a sort
   A has been declared to be a Sentconst
   B has been declared to be a Sentconst
   a has been declared to be an Indconst
   b has been declared to be an Indconst
   ***** copylex C1;
   COPYLEX cannot be done: A has already been declared
   ***** copylex C2;
   You cannot copy the lex of the current context
+

\gfcommand{makecontext}{Multiple contexts' manipulation}
\index{makecontext}

\gfsyntax{
  makecontext \ARG{ctx-name};
}

\gfdescription{
  A new empty context  with name  \ARG{ctx-name} is created.
}

\gfrecap{
  A new empty context  with name  `ctx-name' is created.
}

\gfexample+
   ***** show whereami;
   You are now using an unnamed context.
   You are now using an unnamed proof.
   ***** namecontext C1;
   You have named the current context: C1
   ***** show whereami;
   You are now using context: C1
   You are now using an unnamed proof.
   ***** declare sentconst A;
   ***** makecontext C2;
   You have created the empty context: C2
   ***** switchcontext C2;
   You are now using context: C2
   ***** declare indvar A;
+

\gfcommand{namecontext}{Multiple contexts' manipulation}
\index{namecontext}

\gfsyntax{
  namecontext \ARG{ctx-name};
}

\gfdescription{
  If the current context has no  name,  it is named with \ARG{ctx-name}.
}

\gfrecap{
  If the current context has no  name,  it is named with `ctx-name'.
}

\gfexample+
   ***** show whereami;
   You are now using an unnamed context.
   You are now using an unnamed proof.
   ***** namecontext C1;
   You have named the current context: C1
   ***** show whereami;
   You are now using context: C1
   You are now using an unnamed proof.
   ***** declare sentconst A;
   ***** makecontext C2;
   You have created the empty context: C2
   ***** switchcontext C2;
   You are now using context: C2
   ***** declare indvar A;
+
\gfcommand{reset}{{\GF} reset}
\index{reset}

\gfsyntax{
  reset;
}

\gfdescription{
  Resets the whole {\GF} system.
}

\gfrecap{
  Resets the whole GETFOL system.
}

\gfexample+
   ***** namecontext c; nameproof p;
   You have named the current context: c
   You have named the current proof: p
   
   ***** show whereami;
   You are now using context: c
   You are now using the proof: p
   
   ***** declare sentconst A;
   A has been declared to be a Sentconst
   ***** assume A;
   1   A     (1)
   ***** show proof;
   1   A     (1)
   ***** makecontext c1;
   You have created the empty context: c1
   ***** switchcontext c1;
   You are now using context: c1
   You are switching to a proof with no name.
   
   ***** reset;
   Resetting the whole GETFOL-system
   
   ***** show whereami;
   You are now using an unnamed context.
   You are now using an unnamed proof.
   
   ***** show proof;
+

\gfcommand{switchcontext}{Multiple contexts' manipulation}
\index{switchcontext}

\gfsyntax{
  switchcontext \ARG{ctx-name};
}

\gfdescription{ 
  Switches from the current context to the context \ARG{ctx-name} which
  becomes the current context.
}

\gfrecap{ 
  Switches from the current context to the context `ctx-name' which
  becomes the current context.
}

\gfexample+
   ***** show whereami;
   You are now using an unnamed context.
   You are now using an unnamed proof.
   ***** namecontext C1;
   You have named the current context: C1
   ***** show whereami;
   You are now using context: C1
   You are now using an unnamed proof.
   ***** declare sentconst A;
   ***** makecontext C2;
   You have created the empty context: C2
   ***** switchcontext C2;
   You are now using context: C2
   ***** declare indvar A;
+

 	% loading introduction to the section
\newpage
\section{Metareasoning}
\label{sec-meta}

\subsection{Introduction}

A special context is {\meta}.
{\GF} recognizes {\meta} as a metatheory of all the other contexts.
The context {\meta} can be used to perform {\em metareasoning}, that is to
describe other contexts and to reason about them. 
Metareasoning in {\meta} is performed by employing the following novel
features:
%
\begin{itemize}
  \item
    the metatheory is, in general, distinct from the object theories it
    describes; 
  \item
    the link between the metalanguage and the object language is not performed
    by encoding, but rather by naming \cite{giunchiglia3}.
    Naming is implemented by using the commands which implement reasoning in
    the computational model of a context (see section \ref{sec-comp}).
    These features are available to the user by the commands \C{attach},
    \C{simplify}, \C{eval} etc.  
  \item
    Metareasoning and object reasoning can be mixed via the reflection rule 
    \cite{giunchiglia3}:

    \begin{equation}
      R_{down}
      \fraz{\der{M} Theorem(``w'')}{\der{O} w}
      \label{refl}
    \end{equation}

    where $M$ and $O$ stand for {\meta} and object theory respectively. 
    This rule is implemented in the {\GF} command \C{reflect}.
    The command knows that some form of metareasoning must be performed in
    {\meta} to deduce the metastatement $Theorem(``w'')$.
    The command can use the reflection rule (\ref{refl}) to assert a new proof
    line in the object level context (the context in which object level 
    reasoning is performed and where the command \C{reflect} is typed in).
  \item
    Any context can be the object level context, {\meta} itself.
    The amalgamation of the object and meta level is a particular case of {\GF}
    metareasoning. 
\end{itemize}

In {\meta}, the user is free to declare any language, any set of axioms and to 
define any computational model. This amounts to say that {\meta} is the 
``metatheory'' of a theory represented in a context as far as the user defines
the appropriate attachments and axioms.
A special unary predicate symbol which can be declared in {\meta} is
{\tt THEOREM}: this is the predicate recognized as meaning theoremhood by the
the reflect rule (\ref{refl}) in the command {\tt reflect}.



% loading explanation of commands
\gfcommand{mattach}{semantic meta attachment}
\index{mattach}

\gfsyntax{
   mattach \ARG{indconst} to \ALT dar \OPT{[rep]}
   \ARG{cname}:\ARG{pname}:\ARG{sort}:\ARG{object};
}

\gfdescription{
   Defines an attachment for a constant of the context {\em meta}.

   \ARG{rep}, if present, can be a representation function or \verb+*+.
   If it is \verb+*+ or no representation is specified, then the default representation function 
   {\tt UNIVERSALREP} is taken.
   \ARG{indconst} is a symbol declared to be an INDCONST in {\em meta}; \ARG{cname} is the name
   of the context to which \ARG{object} belongs; \ARG{pname} is the name of the proof in which
   \ARG{object} is present; \ARG{sort} is a sort of the meta-context associated to one of the
   syntactic categories reported with the {\tt reflect} command; \ARG{object} is an object of
   type \ARG{sort}.

   This command implements the mechanism of ``naming'' symbols or objects belonging to the 
   context \ARG{cname}, {\em ie.} the creation of names denoting objects of \ARG{cname}.
   \ARG{indconst} can be attached ``one way'' (using {\tt to}) or ``two ways'' (using {\tt dar}).
   The ``one way'' attachment tells the semantic interpretation mechanism that \ARG{indconst} is
   the name in {\meta} of the {\GF} object in the context \ARG{cname} corresponding to
   \ARG{object}.
   The two ways attachment tells the semantic interpretation mechanism that whenever the 
   (data structure representing) the \ARG{object} is computed as the representation of a 
   term {\em t}, then \ARG{indconst} should be returned as the simplified version of {\em t}.
}

\gfrecap{
This command implements the mechanism of ``naming'' symbols or objects belonging to the 
context `cname', ie. the creation of names denoting objects of `cname'.
`indconst' can be attached ``one way'' (using `to') or ``two ways'' (using `dar').
The ``one way'' attachment tells the semantic interpretation mechanism that `indconst' is
the name in meta of the GETFOL object in the context `cname' corresponding to
`object'.
The two ways attachment tells the semantic interpretation mechanism that whenever the 
(data structure representing) the `object' is computed as the representation of a 
term `t', then `indconst' should be returned as the simplified version of `t'.
}

\gfexample+
   ***** namecontext META;
   ***** nameproof P1;

   ***** declare indconst sc [SENTCONST];
   ***** declare indconst ic [INDCONST];
   ***** declare indconst vl [FACT];
   ***** declare indconst f1 [FACT];

   ***** DECREP  SENTCONST INDCONST FACT;

   ***** represent { SENTCONST } as SENTCONST;
   ***** represent { INDCONST } as INDCONST;
   ***** represent { FACT } as FACT;

   ***** makecontext C;
   ***** switchcontext C;
   ***** declare indconst c;
   ***** declare sentconst A;
   ***** nameproof P1;
   You have named the current proof: P1

   ***** assume c=c;
   1   c = c     (1)

   ***** makeproof P2;
   You have created the empty proof: P2

   ***** switchproof P2;
   You are now using the proof: P2

   ***** assume A imp A;
   1   A imp A     (1)

   ***** label fact ax = 1;

   ***** switchcontext META;

   ***** MATTACH sc TO  C::SENTCONST:A;
   ctext-get: I changed context to: C
   ctext-get: I changed context to: META
   sc attached to 'A

   ***** MATTACH ic DAR C:P2:INDCONST:c;
   ctext-get: I changed context to: C
   ctext-get: I changed context to: META
   ic attached to 'c
   ic is the preferred name of c

   ***** MATTACH vl DAR [SENTCONST] C:P1:FACT:1;
   ctext-get: I changed context to: C
   proof-get: I changed proof to: P1
   proof-get: I changed proof to: P2
   ctext-get: I changed context to: META
   vl attached to '(1 (= c c) (1) ASSUME (%WFF% = c c))
   vl is the preferred name of (1 (= c c) (1) ASSUME (%WFF% = c c))

   ***** MATTACH f1 TO  C:P2:FACT:1;
   ctext-get: I changed context to: C
   ctext-get: I changed context to: META
   f1 attached to '(1 (imp A A) (1) ASSUME (%WFF% imp A A))

   ***** MATTACH f1 DAR C:P2:FACT:ax;
   ctext-get: I changed context to: C
   ctext-get: I changed context to: META
   f1 attached to '(1 (imp A A) (1) ASSUME (%WFF% imp A A))
   f1 is the preferred name of (1 (imp A A) (1) ASSUME (%WFF% imp A A))

+

\gfnotes{}


\gfcommand{reflect}{reflection}
\index{reflect}
\label{sec-refl}

\gfsyntax{
  reflect \ARG{M-fact} \ARG{arg1} \ARG{arg2} \SEQ \ARG{argN};
}

\gfdescription{
  In the   following description we  call ``object context''  the 
  context where the {\tt reflect} command is executed.

  \ARG{M-fact} is any fact of  the context {\meta} whose wff is of the
  form $\forall x_1 x_2\ldots x_n A(x_1, x_2,\ldots,x_n)$, $(n \geq 0)$,
  where the sorts of the variables $x_1, x_2,\ldots,x_n$ 
  correspond to some {\GF} syntactic category ({\em term, wff, fact} ... ).
  For any syntactic category corresponding to a sort in {\meta},
  {\GF} provides the necessary parsing routine.
  This parsing routine is necessary to run the reflect command.
  The relation between sorts in {\meta} and the associated parsed syntactic
  category is the following:

  \begin{figure}
    \begin{tabular}{|l|l|}
   \hline
   {\bf sort}&  {\bf syntactic category} \hspace{7cm} \\ \hline \hline
   SENTCONST &  a {\em sentconst}; \\
   QUANT     &  a {\em quant} (quantifier: {\tt forall} or {\tt exists}); \\
   SORT      &  a symbol declared as a sort; \\
   DECSYM    &  any declared symbol: {\em sym};\\
   FACT      &  a {\em fact};\\
   WFF       &  a {\em wff}; \\
   WFFIF     &  a {\em wffif};\\
   QUANTWFF  &  a {\em quantwff} (of the form  {\tt forall ... }  or 
                {\tt exists  ... });\\
   AWFF      &  an atomic wff (a wff of the form {\tt P( ... ))};\\
   TERM      &  a {\em term}; \\
   TERMIF    &  a {\em termif};\\
   INDSYM    &  a symbol declared as an {\em indconst} or {\em indvar} or
                {\em indpar}; \\
   INDVAR    &  a symbol declared as an {\em indvar};\\
   INDPAR    &  a symbol declared as an {\em indpar};\\
   INDCONST  &  a symbol declared as an {\em indconst};\\
   SENTSYM   &  a symbol declared as a {\em sentconst} or a {\em sentpar}; \\
   SENTPAR   &  a symbol declared as a {\em sentpar}; \\
   SENTCONST &  a symbol declared as a {\em sentconst}; \\
   APPLSYM   &  a symbol declared as a {\em funconst, funpar, predconst, 
                predpar} \\
             &  or a boolean connective; \\ 
   PREDSYM   &  a symbol declared as a {\em predconst} or {\em predpar}; \\
   PREDPAR   &  a symbol declared as a {\em predpar}; \\
   PREDCONST &  a symbol declared as a {\em predconst}; \\
   FUNSYM    &  a symbol declared as a {\em funconst} or {\em funpar}; \\
   FUNPAR    &  a symbol declared as a {\em funpar}; \\
   FUNCONST  &  a symbol declared as a {\em funconst}; \\ \hline 
   \end{tabular}
   \caption{
	Sorts for {\GF} syntactic categories.
   }
   \end{figure}

   If, for example,  the sort of the variable $x_1$  is SENTCONST,  we
   expect $x_1$ to denote any object that is declared  as  a {\em sentconst}
   in the object  context .  Therefore \ARG{arg1}, \ARG{arg2}, \SEQ, \ARG{argN}
   are objects in the object context, such that all \ARG{arg$_i$} are of
   the syntactic category corresponding to the sort of the variables $x_i$
   in \ARG{M-fact}.

   In the following we give an explanation of the major steps performed during
   the execution of the command {\tt reflect} \cite{giunchiglia3}~\footnote{
   This description is the generalization of the example reported below
   and copied from \cite{giunchiglia3}.}:

   \begin{enumerate}
   \item 
     In the object context, when \C{REFLECT} is executed, {\GF}, after parsing 
     \C{REFLECT}, knows that the next argument is the label of a fact in 
     {\meta}. 
     Thus, {\GF} switches context automatically and goes to {\meta}.
   \item
     In {\meta}, the first argument of the command (\ARG{M-fact}) is parsed 
     and the formula of the fact whose label is \ARG{M-fact} is returned:
     $\forall x_1 x_2\ldots x_n A(x_1, x_2,\ldots,x_n)$.
     The variables $x_1, x_2,\ldots,x_n$ must be instantiated to constants
     in {\meta} which will be the names of the objects in the object context
     \ARG{arg1}, \ARG{arg2}, \SEQ, \ARG{argN}.
     The syntactic type of these objects must correspond to the sort of the 
     variables in {\meta}.
     For instance, if the sort of $x_1$ is WFF, then it must be instantiated
     to a constant denoting a {\it wff} in the object context. At this point
     {\GF} knows that \ARG{arg1} must be a {\it wff} in the object context,
     and so {\em arg}$_1$ can be parsed.
     This step provides {\GF} with the information needed to parse \ARG{arg1},
     \SEQ, \ARG{argN} in the object context.
   \item 
     {\GF} switches to the object context and parses  \ARG{arg1}, \ARG{arg2}, 
     \SEQ, \ARG{argN}, using the parsing functions of the syntactic categories
     corresponding to the variables $x_1, x_2,\ldots,x_n$ respectively.
   \item 
     {\GF} switches to {\meta} and automatically declares $n$ constants in 
     {\meta}. Let them be $c_1, c_2, ... c_n$, with the sorts of $x_1, x_2, 
     ... x_n$ respectively.
     Any constant in $c_1, ... c_n$ is automatically ``attached'' to the objects
     returned by parsing \ARG{arg1}, \ARG{arg2}, \SEQ, \ARG{argN} respectively.
     The representation associated to these constants is the representation
     declared for their sorts (see the command {\bf represent}, section
     \ref{sec-rep-sort}).
   \item 
     Still in {\meta}, an universal elimination is performed on 
     $\forall x_1 x_2\ldots x_n A(x_1, x_2,\ldots,x_n)$ obtaining
     $A(c_1, \ldots ,c_n)$.
   \item 
     Still in {\meta}, the formula $A(c_1,...,c_n)$ is automatically evaluated
     by the command {\tt eval} (see section \ref{sec-eval}).
     In this step {\GF}, by using the command {\bf eval}, performs metalevel
     reasoning by computation in the model.
     This metalevel reasoning is used to compute a formula of the form 
     $Theorem(``w'')$.
     If the metalevel reasoning does not lead to $Theorem(``w'')$, then an error
     message is returned. Otherwise the evaluation of the ground term ``$w$''
     gives $w$, the formula of a fact to be asserted in the object context by
     $R_{down}$
   \item
     At this point the reflection rule $R_{down}$ can be applied.
     Thus {\GF} forgets the constants $c_1 ... c_n$ declared in {\meta} and the
     attachments, switches back to the object context and asserts a new fact
     whose formula is $w$ and whose dependencies are the union of the 
     dependencies of the facts in {\em arg$_1$ ... arg$_n$} if there are any.
   \end{enumerate}

	Let us define two context {\tt OBJ} and {\tt META} as follows:

	\gfsourcefile{example.tst}{
	  NAMECONTEXT META; \\
	  DECLARE SORT TERM WFF;\\
	  DECLARE PREDCONST THEOREM 1;\\
	  DECLARE FUNCONST mkequal (INDVAR, INDVAR) = WFF;\\
	  DECLARE INDVAR x [TERM];\\
	  AXIOM M1: forall x. THEOREM(mkequal(x,x));\\
	  DECREP TERM;\\
	  DECREP WFF;\\
	  REPRESENT \{TERM\} AS TERM;\\
	  REPRESENT \{WFF\} AS WFF;\\
	  ATTACH mkequal TO [TERM,TERM=WFF] mkequ;\\
	  MAKECONTEXT OBJ;\\
	  SWITCHCONTEXT OBJ;\\
	  DECLARE INDCONST c;\\
	  DECLARE INDVAR   x;\\
	  DECLARE FUNCONST f 2;
	}

	Let's now type the following lines in {\GF}:

	\begin{quote}\tt
	   ***** FETCH example.tst;
	   ...

	   ***** REFLECT M1 c;
	   1  c = c;
	   ***** REFLECT M1 f(x,f(c,c));
	   2  f(x,f(c,c)) = f(x,f(c,c))
	\end{quote}

	Let us describe  step by step what happened during  the execution of
	the two  previous command \C{REFLECT}.

	\begin{enumerate}
	\item
    In {\tt OBJ} the word \C{REFLECT}  is parsed.
    The next argument must be  the name of a fact  in {\meta}.  Thus {\GF}
    automatically switches context and goes to {\meta}.
	\item
    In  {\meta}, {\tt M1} is parsed  and the axiom
    {\tt forall  x.THEOREM(mkequ(x,x))} is returned.  The variable {\tt x}
    in {\tt M1} must be instantiated  to a constant  in {\meta} which will
    be the name of a symbol of the language  of {\tt OBJ}.   Since the
    sort of  $x$ is {\em TERM}, then  the symbol of  the language  of {\tt
    OBJ}  must have  syntactic  type {\em term}   (it must be  a term).
	\item
    {\GF} switches to the context {\tt OBJ}.
    In  {\tt OBJ} the   term  {\tt  c} [{\tt f(x,f(c,c))} in  the second case]
    is  parsed.  
	\item
    Since no more arguments are needed, {\GF} switches back to {\meta}.
    In {\meta} a new constant, say {\tt C1} of sort {\tt TERM} is created and
    added to the language of {\meta}.
    {\tt   C1} is {\em attached} to the term {\tt c}  [{\tt f(x,f(c,c))}] 
    of the language of {\tt OBJ}.  
    The representation associated to {\tt C1} is {\tt TERM} which is associated
    to the sort {\tt TERM} by the command {\tt REPRESENT} in example.tst.
    In this way,  {\tt C1} is defined  as the name in  {\meta} 
    of {\tt c} [{\tt f(x,f(c,c))}].   
	\item
    Still in {\meta}, an   universal elimination is performed on {\tt  M1}
    obtaining\\
    {\tt THEOREM(mkequ(C1,C1))}.
	\item
    Still in {\meta}, {\tt  THEOREM(mkequ(C1,C1))} is evaluated in {\meta}'s 
    model.
    {\tt THEOREM} has no interpretation, {\tt mkequ(C1,C1)} evaluates to 
    {\tt c = c} [{\tt f(x,f(c,c)) = f(x,f(c,c))}], namely {\tt mkequ(C1,C1)}
    turns out to be the name of {\tt c = c} [{\tt f(x,f(c,c)) = f(x,f(c,c))}].
    So the result of this step is something that could be written as 
    {\tt THEOREM(``c =  c'')}.
    [{\tt THEOREM(``f(x,f(c,c)) = f(x,f(c,c))'')}],  where {\tt ``c = c''}
    [{\tt ``f(x,f(c,c)) = f(x,f(c,c))''}]  should be read as  the  name  of
    {\tt c = c} [{\tt (f(f(c,c),c),f(x,c)) = f(x,f(c,c))}]
	\item
    At  this point the  reflection rule can be applied.
    {\GF} forgets everything in {\meta}, (in this case {\tt C1} from the
    language and {\tt c} [{\tt f(x,f(c,c))}] from the domain of the
   	interpretation).
    {\GF} switches back to the context {\tt OBJ}, and assert a new fact
    with wff  {\tt  c = c} [{\tt  f(x,f(c,c))}] and with the empty deplist.
    \end{enumerate}

    The following example shows how {\GF} computes the deplist of a fact derived 
    by a {\tt reflect} command whose arguments contain a fact.
}

\gfrecap{
	Reflection
}

\gfexample+
   <host prompt> cat example.tst
   NAMECONTEXT META;\\
   DECLARE SORT FACT WFF;\\
   DECLARE INDVAR fc [FACT];\\
   DECLARE FUNCONST wffof (FACT) = WFF;\\
   DECLARE PREDCONST THEOREM 1;\\
   DECREP FACT;\\
   DECREP WFF; \\
   REPRESENT \{WFF\} AS WFF; \\
   REPRESENT \{FACT\} AS FACT;\\
   ATTACH wffof TO [FACT = WFF] fact\-get\-wff;\\
   AXIOM M2: forall fc. THEOREM(wffof(fc));\\
   MAKECONTEXT OBJ;\\
   SWITCHCONTEXT OBJ;\\
   DECLARE SENTCONST A;

   <host prompt> GETFOL
   ...

   ***** FETCH example.tst;
   ***** ASSUME A;
   1   A     (1)
   ***** REFLECT M2 1;
   2   A     (1)
+



 
	% ........................... QUICK REFERENCE ..........................
	\newpage
	\appendix
	\section{Open problems}
\label{app-op}

If a {\tt reflect} command is used when a context {\tt META} does not
exist the error produced is that the relevant fact can not be found.
The error message should say that {\tt META} does not exist.

\gap
The {\tt subst} command has problems distinguishing
proof line labels from numbers in equalities.

\gap
The {\tt label} command does not check for overlaps between
labels for proof lines, axioms and theories.



	\newpage
	\newcommand{\syndes}[2]{\item[\parbox{\textwidth}{#1}]\hfill #2}
\newcommand{\module}[1]{\subsection{#1}}

\section{Syntax of commands}

\module{admin}

\begin{description}
\syndes{
   comment \ARG{separator} \OPT{\ARG{text}} \ARG{separator}
}
{}
%3rd-begin%3rd-end

\syndes{
   deflam \ARG{funname} \ARG{var-list} \ARG{form};
}
{}
%3rd-begin%3rd-end

\syndes{
   deflam \ARG{funname} \ARG{var-list} \ARG{form};
}
{}
%3rd-begin%3rd-end

\syndes{
   echo \ARG{separator} \OPT{\ARG{text}} \ARG{separator}
}
{}
%3rd-begin%3rd-end

\syndes{
   hgk \ARG{s-expr};
}
{}
%3rd-begin%3rd-end

\syndes{
   hgk \ARG{s-expr};
}
{}
%3rd-begin%3rd-end

\syndes{
   know natnums \OPT{\ARG{natnum1}, \SEQ \ARG{natnumN}};
}
{}
%3rd-begin%3rd-end

\syndes{
   load \ARG{file};
}
{}
%3rd-begin%3rd-end

\syndes{
   resetprompt;
}
{}
%3rd-begin%3rd-end

\syndes{
   setprompt to \ARG{s-expr};
}
{}
%3rd-begin%3rd-end

\syndes{
   setprompt to \ARG{s-expr};
}
{}
%3rd-begin%3rd-end

\syndes{
   show \ARG{option};
}
{}
%3rd-begin%3rd-end

\end{description}

\module{context}

\begin{description}
\syndes{
  copycontext \ARG{ctx-name};
}
{}
%3rd-begin%3rd-end

\syndes{
	copylex	\ARG{ctx-name};
}
{}
%3rd-begin%3rd-end

\syndes{
  makecontext \ARG{ctx-name};
}
{}
%3rd-begin%3rd-end

\syndes{
  namecontext \ARG{ctx-name};
}
{}
%3rd-begin%3rd-end

\syndes{
  reset;
}
{}
%3rd-begin%3rd-end

\syndes{
  switchcontext \ARG{ctx-name};
}
{}
%3rd-begin%3rd-end

\end{description}

\module{decide}

\begin{description}
\syndes{
  decide \ARG{wff}  \OPT{by \ARG{fact1} \ARG{fact2} \SEQ} using 
  \OPT{\{ \ARG{rewriter} \SEQ \}} \ARG{decider};
}
{}
%3rd-begin%3rd-end

\syndes{
  monad \ARG{wff} \OPT{by \ARG{fact1} \ARG{fact2} \SEQ};
}
{}
%3rd-begin%3rd-end

\syndes{
  monadeq \ARG{wff} \OPT{by \ARG{fact1} \ARG{fact2} \SEQ};
}
{}
%3rd-begin%3rd-end

\syndes{
  ptaut \ARG{wff} \OPT{by \ARG{fact1} \ARG{fact2} \SEQ};
}
{}
%3rd-begin%3rd-end

\syndes{
  taut \ARG{wff} \OPT{by \ARG{fact1} \ARG{fact2} \SEQ};
}
{}
%3rd-begin%3rd-end

\syndes{
  tauteq \ARG{wff} \OPT{by \ARG{fact1} \ARG{fact2} \SEQ};
}
{}
%3rd-begin%3rd-end

\end{description}

\module{eval}

\begin{description}
\syndes{
  assertsimp \ARG{simplabel}; 
}
{}
%3rd-begin%3rd-end

\syndes{
  attach \ARG{indconst}  to \ALT dar [ rep ] \ARG{sexpr};\\
  attach \ARG{sentconst} to T \ALT NIL \ALT UNDEF;\\
  attach \ARG{funconst} \ALT \ARG{predconst} to \ARG{atom};\\
  attach \ARG{funconst}  to [ \ARG{rep1}, \SEQ, \ARG{repN} = \ARG{repM} ]
  \ARG{atom};\\
  attach \ARG{predconst} to [ \ARG{rep1}, \SEQ, \ARG{repN} ] \ARG{atom};
}
{}
%3rd-begin%3rd-end

\syndes{
  decrep \ARG{replabel1} \OPT{\SEQ \ARG{replabelN}};
}
{}
%3rd-begin%3rd-end

\syndes{
  eval \ARG{wff} \ALT \ARG{fact} \ALT \ARG{term} \OPT{by \ARG{simpexpr}};
}
{}
%3rd-begin%3rd-end

\syndes{
  hardware \ARG{indconst} to \ALT dar \ARG{sexpr};\\
  hardware \ARG{indconst} to \ALT dar [ \ARG{rep} ] \ARG{sexpr};
}
{}
%3rd-begin%3rd-end

\syndes{
  let \ARG{\indconst} to \ALT dar [ \ARG{rep} ] \ARG{term};
}
{}
%3rd-begin%3rd-end

\syndes{
  represent \{ \ARG{sort1}, \SEQ, \ARG{sortN} \} as \ARG{rep} \ALT \ARG{*};
}
{}
%3rd-begin%3rd-end

\syndes{
  rewrite \ARG{wff} \ALT \ARG{fact} \ALT \ARG{term} \OPT{by \ARG{simpexpr}};
}
{}
%3rd-begin%3rd-end

\syndes{
  setbasicsimp \ARG{simplabel} at wffs \{ \ARG{wff1} \SEQ \ARG{wffN} \};\\
  setbasicsimp \ARG{simplabel} at facts \{ \ARG{fact1} \SEQ \ARG{factN} \};
}
{}
%3rd-begin%3rd-end

\syndes{
  simplify \ARG{wff} \ALT \ARG{fact} \ALT \ARG{term};
}
{}
%3rd-begin%3rd-end

\end{description}

\module{language}

\begin{description}
\syndes{
   awff \ARG{awff};
}
{}
%3rd-begin%3rd-end

\syndes{
   declare \OPT{\ARG{sentsym} \ALT \ARG{indsym}} \ARG{sym1} \SEQ \ARG{symN}; \\
   declare \OPT{\ARG{funsym}  \ALT \ARG{predsym}} \ARG{sym1} \SEQ \ARG{symN}
   \ARG{arity};\\
   declare \OPT{\ARG{funsym} \ALT \ARG{predsym}} \ARG{sym1} \SEQ \ARG{symN}
   1 \OPT{[ pre \OPT{= \ARG{prbp}} ]};\\
   declare \OPT{\ARG{funsym} \ALT \ARG{predsym}} \ARG{sym1} \SEQ \ARG{symN}
   2 \OPT{[ inf \OPT{= \ARG{lbp} \ARG{rbp}} ] };
}
{}
%3rd-begin%3rd-end

\syndes{
  declare sort \ARG{sym1} \SEQ \ARG{symN};
}
{}
%3rd-begin%3rd-end

\syndes{
  declare indconst \ALT indpar \ALT indvar \ARG{sym1} \SEQ \ARG{symN}
  [ \ARG{sortsym} ]; \\
  declare funconst \ALT funpar \ARG{sym1} \SEQ \ARG{symN}
  ( \ARG{sortsym1} \SEQ \ARG{sortsymN} ) = \ARG{sortsym};\\
}
{}
%3rd-begin%3rd-end

\syndes{
  extension \ARG{sort} by \ARG{extexpr};
}
{}
%3rd-begin%3rd-end

\syndes{
  moregeneral \ARG{sort1} $<$ \ARG{sort2}, \SEQ, \ARG{sortN} $>$;
}
{}
%3rd-begin%3rd-end

\syndes{
  mostgeneral \ARG{sym};
}
{}
%3rd-begin%3rd-end

\syndes{
  setfmap \ARG{funsym} ( \ARG{sym1} \SEQ \ARG{symN} ) = \ARG{sym};
}
{}
%3rd-begin%3rd-end

\syndes{
   term \ARG{term};
}
{}
%3rd-begin%3rd-end

\syndes{
   wff \ARG{wff};
}
{}
%3rd-begin%3rd-end

\end{description}

\module{meta}

\begin{description}
\syndes{
   mattach \ARG{indconst} to \ALT dar \OPT{[rep]}
   \ARG{cname}:\ARG{pname}:\ARG{sort}:\ARG{object};
}
{}
%3rd-begin%3rd-end

\syndes{
  reflect \ARG{M-fact} \ARG{arg1} \ARG{arg2} \SEQ \ARG{argN};
}
{}
%3rd-begin%3rd-end

\end{description}

\module{nd}

\begin{description}
\syndes{
   alle \ALT us \ARG{fact} \OPT{,} \ARG{term1} \ARG{term2} \SEQ;
}
{}
%3rd-begin%3rd-end

\syndes{
   alli \ALT ug \ARG{fact} \OPT{\OPT{,} \ARG{indvar1} \ALT \ARG{indpar1} :} 
                           \ARG{indvar11}
                           \OPT{\OPT{,} \ARG{indvar2} \ALT \ARG{indpar2} :}
                           \ARG{indvar22} \SEQ;
}
{}
%3rd-begin%3rd-end

\syndes{
   ande \ALT ae \ARG{fact} \OPT{,} 1 \ALT 2; \\
   ande \ALT ae \ARG{fact} \OPT{,} 1 \ALT 2 1 \ALT 2 \SEQ;
}
{}
%3rd-begin%3rd-end

\syndes{
   andi \ALT ai \ARG{fact1} \OPT{,} \ARG{fact2}; \\
   andi \ALT ai \ARG{fact11}
                \OPT{conj \ALT cj \ARG{fact12} conj \ALT cj \ARG{fact13} \SEQ}
                \OPT{,}
                \ARG{fact21}
                \OPT{conj \ALT cj \ARG{fact22} conj \ALT cj \ARG{fact23} \SEQ};
}
{}
%3rd-begin%3rd-end

\syndes{
   assume \ARG{wff1} \OPT{\OPT{,} \ARG{wff2} \SEQ};
}
{}
%3rd-begin%3rd-end

\syndes{
   existe \ALT es \ARG{fact}  \OPT{,} \ARG{indvar1} \ALT \ARG{indpar1}
                  \OPT{,} \ARG{indvar2} \ALT \ARG{indpar2} \SEQ;
}
{}
%3rd-begin%3rd-end

\syndes{
  existi \ARG{fact}
   \OPT{\OPT{,} \ARG{term1} :} \ARG{indvar1} \OPT{occ \ARG{n11} \ARG{n12} \SEQ}
   \OPT{\OPT{,} \ARG{term2} :} \ARG{indvar2} \OPT{occ \ARG{n21} \ARG{n22} \SEQ}
   \SEQ;
}
{}
%3rd-begin%3rd-end

\syndes{
   falsee \ALT fe \ARG{fact1} \OPT{,} \ARG{wff};\\
   falsee \ALT fe \ARG{fact1} \OPT{,} \ARG{fact2};
}
{}
%3rd-begin%3rd-end

\syndes{
   falsei \ALT fi \ARG{fact1} \OPT{,} \ARG{fact2}; 
}
{}
%3rd-begin%3rd-end

\syndes{
   iffe \ALT ie \ARG{fact} \OPT{,} 1 \ALT 2;
}
{}
%3rd-begin%3rd-end

\syndes{
   iffi \ALT ii \ARG{fact1} \OPT{,} \ARG{fact2};
}
{}
%3rd-begin%3rd-end

\syndes{
   impe \ALT mp \ARG{fact1} \OPT{,} \ARG{fact2};
}
{}
%3rd-begin%3rd-end

\syndes{
   impi \ALT ded  \ARG{fact1} \OPT{, \ALT imp} \ARG{fact};\\
   impi \ALT ded  \ARG{wff} \OPT{, \ALT imp} \ARG{fact};
}
{}
%3rd-begin%3rd-end

\syndes{
   note \ALT ne \ARG{fact1} \OPT{,} \ARG{wff}; \\
   note \ALT ne \ARG{fact1} \OPT{,} \ARG{fact2};
}
{}
%3rd-begin%3rd-end

\syndes{
   noti \ALT ni \ARG{fact1} \OPT{,} \ARG{wff}; \\
   noti \ALT ni \ARG{fact1} \OPT{,} \ARG{fact2};
}
{}
%3rd-begin%3rd-end

\syndes{
   ore \ALT oe \ARG{fact1} \OPT{,} \ARG{fact2} \OPT{,} \ARG{fact2};
}
{}
%3rd-begin%3rd-end

\syndes{
   ori \ALT oi \ARG{fact} \OPT{,} \ARG{wff} \OPT{,} \OPT{lr \ALT rl};\\
   ori \ALT oi \ARG{fact} \OPT{,} \ARG{fact1} \ALT \ARG{wff1} 
       disj \ALT dj \ARG{fact2} \ALT \ARG{wff2} disj \ALT dj \SEQ
       \OPT{,} \OPT{lr \ALT rl};
}
{}
%3rd-begin%3rd-end

\syndes{
  subst \ARG{fact1} \OPT{with} \ARG{fact2};\\
  subst \ARG{fact1} \OPT{with} \ARG{fact2} \OPT{right \ALT left};\\
  subst \ARG{fact1} \OPT{with} \ARG{fact2} 
                    \OPT{occ \ARG{n1} \ARG{n2} \SEQ} \OPT{right \ALT left};
}
{}
%3rd-begin%3rd-end

\end{description}

\module{parser}

\begin{description}
\syndes{
  backup \ARG{file} open;\\
  backup \ARG{file} close;  
}
{}
%3rd-begin%3rd-end

\syndes{done;}
{}
%3rd-begin%3rd-end

\syndes{
   fetch \ARG{file} \OPT{from \ARG{mark1}} \OPT{to \ARG{mark2}};
}
{}
%3rd-begin%3rd-end

\syndes{
   mark \ARG{sym};
}
{}
%3rd-begin%3rd-end

\syndes{
   probe;\\
   probe \ARG{activity};\\
   probe all;\\
}
{}
%3rd-begin%3rd-end

\syndes{
   unprobe \ARG{activity};\\
   unprobe all;
}
{}
%3rd-begin%3rd-end

\end{description}

\module{proof}

\begin{description}
\syndes{
  axiom \ARG{sym} : \ARG{wff};
}
{}
%3rd-begin%3rd-end

\syndes{
   cancel \OPT{\ARG{label}};
}
{}
%3rd-begin%3rd-end

\syndes{
  copyproof \ARG{prf-name};
}
{}
%3rd-begin%3rd-end

\syndes{
   label fact \ARG{sym};\\
   label fact \ARG{sym} = \ARG{label};
}
{}
%3rd-begin%3rd-end

\syndes{
   makeproof \ARG{prf-name};
}
{}
%3rd-begin%3rd-end

\syndes{
   nameproof \ARG{prf-name};
}
{}
%3rd-begin%3rd-end

\syndes{
  switchproof \ARG{prf-name};
}
{}
%3rd-begin%3rd-end

\syndes{
  theorem \ARG{sym} \ARG{hook};
}
{}
%3rd-begin%3rd-end

\syndes{
  theory \ARG{thlabel} : \ARG{wff1} \OPT{\ARG{wff2} \SEQ};\\
  theory \ARG{thlabel} : \ARG{axlabel1} : \ARG{wff}
                         \OPT{\ARG{axlabel2} : \ARG{wff} \SEQ};
}
{}
%3rd-begin%3rd-end

\end{description}

\module{rules}

\begin{description}
\syndes{
	contract \ALT ctc  \ARG{fact} by \ARG{assumption1} \SEQ  \ARG{assumptionN}; 
}
{}
%3rd-begin%3rd-end

\syndes{
	cut \ARG{fact1} \ARG{fact2};\\
	cut \ARG{fact1} \ARG{fact2} \OPT{keep \ARG{assumption1} \SEQ
	\ARG{assumptionN}};
}
{}
%3rd-begin%3rd-end

\syndes{
	termife \ARG{fact1} \ARG{fact2} \ARG{termif}; \\
	termife \ARG{fact1} \ARG{fact2} \ARG{termif} \OPT{occ \ARG{n1} \ARG{n2}
	\SEQ}; 
}
{}
%3rd-begin%3rd-end

\syndes{
	termifen \ARG{fact1} \ARG{fact2} \ARG{termif}; \\
	termifen \ARG{fact1} \ARG{fact2} \ARG{termif} \OPT{occ \ARG{n1} \ARG{n2}
	\SEQ}; 
}
{}
%3rd-begin%3rd-end

\syndes{
	termifi \ARG{fact1} \ARG{fact2} \ARG{wff} \ARG{term1} \ARG{term2};
}
{}
%3rd-begin%3rd-end

\syndes{
	weaken \ALT wk \ARG{fact} by \ARG{fact1} \SEQ \ARG{factN};
}
{}
%3rd-begin%3rd-end

\syndes{
	wffife \ARG{fact1} \ARG{fact2};
}
{}
%3rd-begin%3rd-end

\syndes{
	wffifen \ARG{fact1} \ARG{fact2};
}
{}
%3rd-begin%3rd-end

\syndes{
	wffifi \ARG{wff} \ARG{fact1} \ARG{fact2};
}
{}
%3rd-begin%3rd-end

\end{description}



	% ............................. BIBLIOGRAPHY ...........................
	\newpage
	\bibliographystyle{alpha}
    \gfbibliography
	
	% ................................ INDEX ...............................
	\newpage
	%............................... USER MANUAL .................................
%.............................................................................

\documentstyle[12pt,rawfonts]{../styfiles/GFmanual}

\title{GETFOL Manual}
\author{\bf Fausto Giunchiglia}
\date{7 March 1994}
\version{2.0}
\abstract{
	  {\GF} is an interactive reasoning system.
	  We use it as an environment for studying epistemological issues.
	  We try to look at questions like: which notions are
	  important for the development of mechanized reasoning systems?
	  What kind of conversations do we want to have with them?
	  What parts of logic should we use to represent such notions?
	  How should logic be embedded in a conversational reasoning system?
	}
\addresses{
     \begin{tabular}[c]{l}
       {\bf Fausto Giunchiglia}           \\
       Mechanized Reasoning Group		  \\
		 IRST, Povo, 38050 Trento, Italy  \\
		 e-mail: {\tt fausto@irst.it}     \\
		 phone: +39 461 314436
	\end{tabular}	
}
\published{
  \begin{tabular}{l}
	  DIST Technical Report No. 92-0010 (1994). \\
	  DIST -- University of Genoa,\\
	  Via Opera Pia 11A, 16145 Genova, Italy.\\ \\
  \end{tabular}
}

%% \newcommand{\gfbibliography}{%
%% \bibliography{/home/tarski/staff/mrg/biblio/bib/a-l,%
%% /home/tarski/staff/mrg/biblio/bib/m-z,userman}}
\newcommand{\gfbibliography}{\bibliography{}}

%% \makeindex
\begin{document}
	%  ............................. COVER ..................................
	\thispagestyle{empty}
	\maketitle

	%  ........................ TABLE OF CONTENTS ...........................
	\newpage
	\pagenumbering{roman}
	\tableofcontents

	\newpage
	\pagenumbering{arabic}
	\pagestyle{headings}

	%  ........................... INTRODUCTION ............................
	\section{Introduction}

The distribution package allows the installation of the following three
applications:
%
\begin{itemize}
	\item
		the {\tt HGKM} interpreter \cite{giunchiglia35}, that is a LISP-like
		language acting as the implementation language of both {\tt FOL}
		and {\tt GETFOL}.
	\item
		the {\tt GETFOL} system \cite{giunchiglia12,giunchiglia29}.
	\item
		the {\tt AGETFOL} system ({\tt GETFOL} with {\em abstraction}
		\cite{giunchiglia7}.
\end{itemize}

From now on, we will write ``{\tt the system}" to mean {\tt HGKM},
{\tt GETFOL} and {\tt AGETFOL}.
The distribution package is provided with a set of general purpose
configuration and installation facilities which allow you to install your
favourite application on your machine.

In order to install {\tt the system} read section~\ref{requirements}
(``{\it Minimal System Requirements}'') and verify that your machine
has all the required features.
Then read section~\ref{instproc} (``{\it The installation procedure}'').
This section gives the instructions to follow in order to install
{\tt the system} on your machine.

If you are an expert you might be interested also in the features
of the installation and configuration facilities.
Section~\ref{sysmod} (``{\it Systems and Modules}'') introduces
to the data structures ({\it systems} and {\it modules}) devised
to store and/or modify the configuration of the applications.


	%  .............................. MODULES ..............................
	% loading introduction to the section
\input{parser/intropar}

% loading command files
\input{parser/backup}
\input{parser/done}
\input{parser/fetch}
\input{parser/mark}
\input{parser/probe}
\input{parser/unprobe}

	% loading introduction to the section
\newpage
\input{admin/introadm.tex}

% loading explanation of commands
\input{admin/comment}
\input{admin/deflam}
\input{admin/echo}
\input{admin/hgk}
\input{admin/know}
\input{admin/load}
\input{admin/resprmpt}
\input{admin/setprmpt}
\input{admin/show}

 	;;;;;;;;;;;;;;;;;;;;;;;;;;;;;;;;;;;;;;;;;;;;;;;;;;;;;;;;;;;;;;;;;;;;;;;;;;;;;;
;;
;; FOL version 2.001
;; This file is an FOL source file: language.cfg
;; Date: Wed Oct 20 10:46:03 MET 1993
;;
;;;;;;;;;;;;;;;;;;;;;;;;;;;;;;;;;;;;;;;;;;;;;;;;;;;;;;;;;;;;;;;;;;;;;;;;;;;;;;
;;                                                                          ;;
;;   Copyright (c) 1986-1987 by Richard Weyhrauch.  All rights reserved.    ;;
;;   Copyright (c) 1987-1988 by Fausto Giunchiglia.  All rights reserved.   ;;
;;                                                                          ;;
;;   This software is being provided to you, the LICENSEE, by Richard       ;;
;;   Weyhrauch and Fausto Giunchiglia, the AUTHORS, under certain rights    ;;
;;   and obligations.  By obtaining, using and/or copying this software,    ;;
;;   you indicate that you have read, understood, and will comply with      ;;
;;   the following terms and conditions:                                    ;;
;;                                                                          ;;
;;   THE AUTHORS MAKE NO REPRESENTATIONS OF WARRANTIES, EXPRESS OR          ;;
;;   IMPLIED.  By way of example, but not limitation, THE AUTHORS MAKE      ;;
;;   NO REPRESENTATIONS OR WARRANTIES OF MERCHANTABILITY OF FITNESS FOR     ;;
;;   ANY PARTICULAR PURPOSE OR THAT THE USE OF THE LICENSED SOFTWARE        ;;
;;   COMPONENTS OR DOCUMENTATION WILL NOT INFRINGE ANY PATENTS,             ;;
;;   COPYRIGHTS, TRADEMARKS OR OTHER RIGHTS.                                ;;
;;                                                                          ;;
;;   The AUTHORS shall not be held liable for any direct, indirect or       ;;
;;   consequential damages with respect to any claim by LICENSEE or any     ;;
;;   third party on account of or arising from this Agreement or use of     ;;
;;   this software.  Permission to use, copy, modify and distribute this    ;;
;;   software and its documentation for any purpose and without fee or      ;;
;;   royalty is hereby granted, provided that the above copyright notice    ;;
;;   and disclaimer appears in and on ALL copies of the software and        ;;
;;   documentation, whether original to the AUTHORS or a modified           ;;
;;   version by LICENSEE.                                                   ;;
;;                                                                          ;;
;;   The name of the AUTHORS may not be used in advertising or publicity    ;;
;;   pertaining to distribution of the software without specific, written   ;;
;;   prior permission.  Notice must be given in supporting documentation    ;;
;;   that such distribution is by permission of the AUTHORS.  The AUTHORS   ;;
;;   make no representations about the suitability of this software for     ;;
;;   any purpose.  It is provided "AS IS" without express or implied        ;;
;;   warranty.  Title to copyright to this software and to any associated   ;;
;;   documentation shall at all times remain with the AUTHORS and LICENSEE  ;;
;;   agrees to preserve same.  LICENSEE agrees to place the appropriate     ;;
;;   copyright notice on any such copies.                                   ;;
;;                                                                          ;;
;;;;;;;;;;;;;;;;;;;;;;;;;;;;;;;;;;;;;;;;;;;;;;;;;;;;;;;;;;;;;;;;;;;;;;;;;;;;;;

;*************************************************************************;
;                                                                         ;
;                    "LANGUAGE" MODULE CONFIGURATION FILE                 ;
;                                                                         ;
;*************************************************************************;

(MODULE-INIT        'LANGUAGE)
(MODULE-SET-NAME    'LANGUAGE "LANGUAGE")
(MODULE-SET-MODE    'LANGUAGE 'COMPILED)

(MODULE-SET-SRCDIR 'LANGUAGE (PATH-CONCAT (SYS-GET-SRCDIR 'GETFOL) "language"))
(MODULE-SET-OBJDIR 'LANGUAGE (SYS-GET-OBJDIR 'GETFOL))
(MODULE-SET-DOCDIR 'LANGUAGE (PATH-CONCAT (SYS-GET-DOCDIR 'GETFOL) "language"))

(MODULE-SET-DOCFILE 'LANGUAGE
   (PATH-CONCAT  (MODULE-GET-DOCDIR 'LANGUAGE) "language.tex"))


;;;     GETHGKM special variables declaration
(MODULE-ADD-FILE 'LANGUAGE   "vlang.cl"     ""         'INTERPRETED)

;;;     Label Spaces
(MODULE-ADD-FILE 'LANGUAGE   "labelspa.hgk" "labspah"  'COMPILED)
(MODULE-ADD-FILE 'LANGUAGE   "labelspa.fol" "labspaf"  'COMPILED)
(MODULE-ADD-FILE 'LANGUAGE   "labelspa.rp"  "labspar"  'COMPILED)

;;;     Signature, symbols and sorts
(MODULE-ADD-FILE 'LANGUAGE   "symls.hgk"    "symlsh"   'COMPILED)
(MODULE-ADD-FILE 'LANGUAGE   "sym.hgk"      "symh"     'COMPILED)
(MODULE-ADD-FILE 'LANGUAGE   "symls.fol"    "symlsf"   'COMPILED)
(MODULE-ADD-FILE 'LANGUAGE   "sym.fol"      "symf"     'COMPILED)
(MODULE-ADD-FILE 'LANGUAGE   "sym.rp"       "symr"     'COMPILED)

;;;     Expressions
(MODULE-ADD-FILE 'LANGUAGE   "exp.hgk"      "exph"     'INTERPRETED)
(MODULE-ADD-FILE 'LANGUAGE   "exp.rp"       "expr"     'COMPILED)
(MODULE-ADD-FILE 'LANGUAGE   "exp.fol"      "expf"     'COMPILED)

;;;     Sorts
(MODULE-ADD-FILE 'LANGUAGE   "sort.hgk"     "sorth"    'COMPILED)
(MODULE-ADD-FILE 'LANGUAGE   "sort.fol"     "sortf"    'COMPILED)

;;;     Probe file
(MODULE-ADD-FILE 'LANGUAGE   "problang.fol" "problaf"  'COMPILED)

;;;     Command files
(MODULE-ADD-FILE 'LANGUAGE   "decsymls.hgk" "decsymlh" 'COMPILED)
(MODULE-ADD-FILE 'LANGUAGE   "decsymls.fol" "decsymlf" 'COMPILED)
(MODULE-ADD-FILE 'LANGUAGE   "language.fol" "langf"    'COMPILED)
(MODULE-ADD-FILE 'LANGUAGE   "cmdlang.fol"  "cmdlangf" 'COMPILED)

;;;     Defaults for the language
(MODULE-ADD-FILE 'LANGUAGE   "langdflt.fol" "langdflf" 'COMPILED)

;;;     Show files
(MODULE-ADD-FILE 'LANGUAGE   "showlang.fol" "showlanf" 'COMPILED)
(MODULE-ADD-FILE 'LANGUAGE   "showlang.rp"  "showlanr" 'COMPILED)


(MODULE-ADD-FILE 'LANGUAGE   "skolem.hgk"   "skolh"    'COMPILED)


;;;     Initialization files
(MODULE-ADD-FILE 'LANGUAGE   "ilang.fol"    ""         'INTERPRETED)

 	;;;;;;;;;;;;;;;;;;;;;;;;;;;;;;;;;;;;;;;;;;;;;;;;;;;;;;;;;;;;;;;;;;;;;;;;;;;;;;
;;
;; FOL version 2.001
;; This file is an FOL source file: proof.cfg
;; Date: Wed Oct 20 10:47:27 MET 1993
;;
;;;;;;;;;;;;;;;;;;;;;;;;;;;;;;;;;;;;;;;;;;;;;;;;;;;;;;;;;;;;;;;;;;;;;;;;;;;;;;
;;                                                                          ;;
;;   Copyright (c) 1986-1987 by Richard Weyhrauch.  All rights reserved.    ;;
;;   Copyright (c) 1987-1988 by Fausto Giunchiglia.  All rights reserved.   ;;
;;                                                                          ;;
;;   This software is being provided to you, the LICENSEE, by Richard       ;;
;;   Weyhrauch and Fausto Giunchiglia, the AUTHORS, under certain rights    ;;
;;   and obligations.  By obtaining, using and/or copying this software,    ;;
;;   you indicate that you have read, understood, and will comply with      ;;
;;   the following terms and conditions:                                    ;;
;;                                                                          ;;
;;   THE AUTHORS MAKE NO REPRESENTATIONS OF WARRANTIES, EXPRESS OR          ;;
;;   IMPLIED.  By way of example, but not limitation, THE AUTHORS MAKE      ;;
;;   NO REPRESENTATIONS OR WARRANTIES OF MERCHANTABILITY OF FITNESS FOR     ;;
;;   ANY PARTICULAR PURPOSE OR THAT THE USE OF THE LICENSED SOFTWARE        ;;
;;   COMPONENTS OR DOCUMENTATION WILL NOT INFRINGE ANY PATENTS,             ;;
;;   COPYRIGHTS, TRADEMARKS OR OTHER RIGHTS.                                ;;
;;                                                                          ;;
;;   The AUTHORS shall not be held liable for any direct, indirect or       ;;
;;   consequential damages with respect to any claim by LICENSEE or any     ;;
;;   third party on account of or arising from this Agreement or use of     ;;
;;   this software.  Permission to use, copy, modify and distribute this    ;;
;;   software and its documentation for any purpose and without fee or      ;;
;;   royalty is hereby granted, provided that the above copyright notice    ;;
;;   and disclaimer appears in and on ALL copies of the software and        ;;
;;   documentation, whether original to the AUTHORS or a modified           ;;
;;   version by LICENSEE.                                                   ;;
;;                                                                          ;;
;;   The name of the AUTHORS may not be used in advertising or publicity    ;;
;;   pertaining to distribution of the software without specific, written   ;;
;;   prior permission.  Notice must be given in supporting documentation    ;;
;;   that such distribution is by permission of the AUTHORS.  The AUTHORS   ;;
;;   make no representations about the suitability of this software for     ;;
;;   any purpose.  It is provided "AS IS" without express or implied        ;;
;;   warranty.  Title to copyright to this software and to any associated   ;;
;;   documentation shall at all times remain with the AUTHORS and LICENSEE  ;;
;;   agrees to preserve same.  LICENSEE agrees to place the appropriate     ;;
;;   copyright notice on any such copies.                                   ;;
;;                                                                          ;;
;;;;;;;;;;;;;;;;;;;;;;;;;;;;;;;;;;;;;;;;;;;;;;;;;;;;;;;;;;;;;;;;;;;;;;;;;;;;;;

;*************************************************************************;
;                                                                         ;
;                    "PROOF" MODULE CONFIGURATION FILE                    ;
;                                                                         ;
;*************************************************************************;

(MODULE-INIT        'PROOF)
(MODULE-SET-NAME    'PROOF   "PROOF")
(MODULE-SET-MODE    'PROOF   'COMPILED)

(MODULE-SET-SRCDIR  'PROOF  (PATH-CONCAT (SYS-GET-SRCDIR 'GETFOL) "proof"))
(MODULE-SET-OBJDIR  'PROOF  (SYS-GET-OBJDIR 'GETFOL))
(MODULE-SET-DOCDIR  'PROOF  (PATH-CONCAT (SYS-GET-DOCDIR 'GETFOL) "proof"))

(MODULE-SET-DOCFILE 'PROOF
    (PATH-CONCAT (MODULE-GET-DOCDIR 'PROOF) "inclfile"    "proof.tex"))



;;;   GETHGKM special variables declaration
(MODULE-ADD-FILE   'PROOF   "vproof.cl"       ""             'INTERPRETED)

;;;   Reason
(MODULE-ADD-FILE    'PROOF  "reason.hgk"      "reasonh"      'COMPILED)

;;;   Proof-steps
(MODULE-ADD-FILE    'PROOF  "pline.hgk"       "plineh"       'COMPILED)
(MODULE-ADD-FILE    'PROOF  "pline.fol"       "plinef"       'COMPILED)
(MODULE-ADD-FILE    'PROOF  "pline.rp"        "pliner"       'COMPILED)

;;;   Facts
(MODULE-ADD-FILE    'PROOF  "fact.hgk"        "facth"        'COMPILED)
(MODULE-ADD-FILE    'PROOF  "fact.fol"        "factf"        'COMPILED)
(MODULE-ADD-FILE    'PROOF  "fact.rp"         "factr"        'COMPILED)

;;;   Axioms
(MODULE-ADD-FILE    'PROOF  "axiom.hgk"       "axiomh"       'COMPILED)
(MODULE-ADD-FILE    'PROOF  "axiom.fol"       "axiomf"       'COMPILED)
(MODULE-ADD-FILE    'PROOF  "axiom.rp"        "axiomr"       'COMPILED)

;;;   Show & probe proofs.
(MODULE-ADD-FILE    'PROOF  "showprf.rp"      "showprfr"     'COMPILED)
(MODULE-ADD-FILE    'PROOF  "probprf.fol"     "probprf"      'COMPILED)

;;;   Proofs
(MODULE-ADD-FILE    'PROOF  "proof.hgk"       "proofh"       'COMPILED)
(MODULE-ADD-FILE    'PROOF  "proof.fol"       "prooff"       'COMPILED)
(MODULE-ADD-FILE    'PROOF  "proof.rp"        "proofr"       'COMPILED)

;;;   Command files
(MODULE-ADD-FILE    'PROOF  "cmdproof.fol"    "cmdprff"      'COMPILED)
(MODULE-ADD-FILE    'PROOF  "cmdlabel.fol"    "cmdlabef"     'COMPILED)

;;;   Initialization files
(MODULE-ADD-FILE    'PROOF  "iproof.fol"      ""             'INTERPRETED)

 	;;;;;;;;;;;;;;;;;;;;;;;;;;;;;;;;;;;;;;;;;;;;;;;;;;;;;;;;;;;;;;;;;;;;;;;;;;;;;;
;;
;; FOL version 2.001
;; This file is an FOL source file: nd.cfg
;; Date: Wed Oct 20 10:46:59 MET 1993
;;
;;;;;;;;;;;;;;;;;;;;;;;;;;;;;;;;;;;;;;;;;;;;;;;;;;;;;;;;;;;;;;;;;;;;;;;;;;;;;;
;;                                                                          ;;
;;   Copyright (c) 1986-1987 by Richard Weyhrauch.  All rights reserved.    ;;
;;   Copyright (c) 1987-1988 by Fausto Giunchiglia.  All rights reserved.   ;;
;;                                                                          ;;
;;   This software is being provided to you, the LICENSEE, by Richard       ;;
;;   Weyhrauch and Fausto Giunchiglia, the AUTHORS, under certain rights    ;;
;;   and obligations.  By obtaining, using and/or copying this software,    ;;
;;   you indicate that you have read, understood, and will comply with      ;;
;;   the following terms and conditions:                                    ;;
;;                                                                          ;;
;;   THE AUTHORS MAKE NO REPRESENTATIONS OF WARRANTIES, EXPRESS OR          ;;
;;   IMPLIED.  By way of example, but not limitation, THE AUTHORS MAKE      ;;
;;   NO REPRESENTATIONS OR WARRANTIES OF MERCHANTABILITY OF FITNESS FOR     ;;
;;   ANY PARTICULAR PURPOSE OR THAT THE USE OF THE LICENSED SOFTWARE        ;;
;;   COMPONENTS OR DOCUMENTATION WILL NOT INFRINGE ANY PATENTS,             ;;
;;   COPYRIGHTS, TRADEMARKS OR OTHER RIGHTS.                                ;;
;;                                                                          ;;
;;   The AUTHORS shall not be held liable for any direct, indirect or       ;;
;;   consequential damages with respect to any claim by LICENSEE or any     ;;
;;   third party on account of or arising from this Agreement or use of     ;;
;;   this software.  Permission to use, copy, modify and distribute this    ;;
;;   software and its documentation for any purpose and without fee or      ;;
;;   royalty is hereby granted, provided that the above copyright notice    ;;
;;   and disclaimer appears in and on ALL copies of the software and        ;;
;;   documentation, whether original to the AUTHORS or a modified           ;;
;;   version by LICENSEE.                                                   ;;
;;                                                                          ;;
;;   The name of the AUTHORS may not be used in advertising or publicity    ;;
;;   pertaining to distribution of the software without specific, written   ;;
;;   prior permission.  Notice must be given in supporting documentation    ;;
;;   that such distribution is by permission of the AUTHORS.  The AUTHORS   ;;
;;   make no representations about the suitability of this software for     ;;
;;   any purpose.  It is provided "AS IS" without express or implied        ;;
;;   warranty.  Title to copyright to this software and to any associated   ;;
;;   documentation shall at all times remain with the AUTHORS and LICENSEE  ;;
;;   agrees to preserve same.  LICENSEE agrees to place the appropriate     ;;
;;   copyright notice on any such copies.                                   ;;
;;                                                                          ;;
;;;;;;;;;;;;;;;;;;;;;;;;;;;;;;;;;;;;;;;;;;;;;;;;;;;;;;;;;;;;;;;;;;;;;;;;;;;;;;

;*************************************************************************;
;                                                                         ;
;                    "ND" MODULE CONFIGURATION FILE                       ;
;                                                                         ;
;*************************************************************************;

(MODULE-INIT        'ND)
(MODULE-SET-NAME    'ND  "ND")
(MODULE-SET-MODE    'ND  'COMPILED)

(MODULE-SET-SRCDIR  'ND  (PATH-CONCAT (SYS-GET-SRCDIR 'GETFOL) "nd"))
(MODULE-SET-OBJDIR  'ND  (SYS-GET-OBJDIR 'GETFOL))
(MODULE-SET-DOCDIR  'ND
     (PATH-CONCAT (SYS-GET-DOCDIR 'GETFOL) "inclfile"  "nd"))
(MODULE-SET-DOCFILE 'ND
     (PATH-CONCAT (MODULE-GET-DOCDIR 'ND)  "inclfile"  "nd.tex"))


;;;   GETHGKM special variables declaration
(MODULE-ADD-FILE 'ND "vnd.cl"       ""         'INTERPRETED)              

;;;   Inference rules for nd
(MODULE-ADD-FILE 'ND "fapfnd.hgk"   "fapfndh"  'COMPILED)
(MODULE-ADD-FILE 'ND "fapfnd.fol"   "fapfndf"  'COMPILED)

;;;   Command files
(MODULE-ADD-FILE 'ND "cmdnd.fol"    "cmdndf"   'COMPILED)

;;;   Initialization files
(MODULE-ADD-FILE 'ND "ind.fol"      ""         'INTERPRETED)

	;;;;;;;;;;;;;;;;;;;;;;;;;;;;;;;;;;;;;;;;;;;;;;;;;;;;;;;;;;;;;;;;;;;;;;;;;;;;;;
;;
;; GETFOL version 1.001
;; This file is a GETFOL source file: rules.cfg
;; Date: Thu Nov 11 14:42:52 MET 1993
;;
;;;;;;;;;;;;;;;;;;;;;;;;;;;;;;;;;;;;;;;;;;;;;;;;;;;;;;;;;;;;;;;;;;;;;;;;;;;;;;
;;                                                                          ;;
;;   Copyright (c) 1987-1988 by Fausto Giunchiglia.  All rights reserved.   ;;
;;                                                                          ;;
;;   This software is being provided to you, the LICENSEE, by Fausto        ;;
;;   Giunchiglia, the AUTHOR, under certain rights and obligations.         ;;
;;   By obtaining, using and/or copying this software, you indicate that    ;;
;;   you have read, understood, and will comply with the following terms    ;;
;;   and conditions:                                                        ;;
;;                                                                          ;;
;;   THE AUTHOR MAKES NO REPRESENTATIONS OF WARRANTIES, EXPRESS OR          ;;
;;   IMPLIED.  By way of example, but not limitation, THE AUTHOR MAKES      ;;
;;   NO REPRESENTATIONS OR WARRANTIES OF MERCHANTABILITY OF FITNESS FOR     ;;
;;   ANY PARTICULAR PURPOSE OR THAT THE USE OF THE LICENSED SOFTWARE        ;;
;;   COMPONENTS OR DOCUMENTATION WILL NOT INFRINGE ANY PATENTS,             ;;
;;   COPYRIGHTS, TRADEMARKS OR OTHER RIGHTS.                                ;;
;;                                                                          ;;
;;   The AUTHOR shall not be held liable for any direct, indirect or        ;;
;;   consequential damages with respect to any claim by LICENSEE or any     ;;
;;   third party on account of or arising from this Agreement or use of     ;;
;;   this software.  Permission to use, copy, modify and distribute this    ;;
;;   software and its documentation for any purpose and without fee or      ;;
;;   royalty is hereby granted, provided that the above copyright notice    ;;
;;   and disclaimer appears in and on ALL copies of the software and        ;;
;;   documentation, whether original to the AUTHOR or a modified            ;;
;;   version by LICENSEE.                                                   ;;
;;                                                                          ;;
;;   The name of the AUTHOR may not be used in advertising or publicity     ;;
;;   pertaining to distribution of the software without specific, written   ;;
;;   prior permission.  Notice must be given in supporting documentation    ;;
;;   that such distribution is by permission of the AUTHOR.  The AUTHOR     ;;
;;   makes no representations about the suitability of this software for    ;;
;;   any purpose.  It is provided "AS IS" without express or implied        ;;
;;   warranty.  Title to copyright to this software and to any associated   ;;
;;   documentation shall at all times remain with the AUTHOR and LICENSEE   ;;
;;   agrees to preserve same.  LICENSEE agrees to place the appropriate     ;;
;;   copyright notice on any such copies.                                   ;;
;;                                                                          ;;
;;;;;;;;;;;;;;;;;;;;;;;;;;;;;;;;;;;;;;;;;;;;;;;;;;;;;;;;;;;;;;;;;;;;;;;;;;;;;;

;*************************************************************************;
;                                                                         ;
;                    "RULES" MODULE CONFIGURATION FILE                    ;
;                                                                         ;
;*************************************************************************;

(MODULE-INIT       'RULES)
(MODULE-SET-NAME   'RULES "RULES")
(MODULE-SET-MODE   'RULES 'COMPILED)

(MODULE-SET-SRCDIR 'RULES (PATH-CONCAT (SYS-GET-SRCDIR 'GETFOL) "rules"))
(MODULE-SET-OBJDIR 'RULES (SYS-GET-OBJDIR 'GETFOL))
(MODULE-SET-DOCDIR 'RULES (PATH-CONCAT (SYS-GET-DOCDIR 'GETFOL) "rules"))

(MODULE-SET-DOCFILE 'RULES
  (PATH-CONCAT (MODULE-GET-DOCDIR 'RULES) "rules.tex"))

;;;
;;;                !!!FILES WITHOUT DEFSUB!!!
;;;

;;;  If rules
(MODULE-ADD-FILE 'RULES "fapfif.fol"   "fapfiff"  'COMPILED)
(MODULE-ADD-FILE 'RULES "cmdif.fol"    "cmdiff"   'COMPILED)

;;;  Structural rules
(MODULE-ADD-FILE 'RULES "fapfstr.fol"  "fapfstrf" 'COMPILED)
(MODULE-ADD-FILE 'RULES "cmdstr.fol"   "cmdstrf"  'COMPILED)

;;;  Initialization files
(MODULE-ADD-FILE 'RULES "iif.fol"      ""         'INTERPRETED)
(MODULE-ADD-FILE 'RULES "istr.fol"     ""         'INTERPRETED)

 	% introduction to the deciders
\newpage
\input{decide/intro}

% user commands for deciders
\input{decide/decide}
\input{decide/monad}
\input{decide/monadeq}
\input{decide/ptaut}
\input{decide/taut}
\input{decide/tauteq}

 	% introduction to the semantic simplification's section
\newpage
\input{eval/introsema}

% user commands for semantic simplification
\input{eval/attach}
\input{eval/decrep}
\input{eval/hardware}
\input{eval/represent}
\input{eval/simplify}

% introduction to the syntactic simplification's section
\newpage
\input{eval/introsynt}

% user commands for syntactic simplification
\input{eval/assertsimp}
\input{eval/rewrite}


% introduction to the syntactic/semantic simplification's section
\newpage
\input{eval/introsynsema.tex}

% user commands for the syntactic/semantic simplification's section
\input{eval/eval}
\input{eval/let}


 	% loading introduction to the section
\newpage
\input{context/introcon}

% loading explanation of commands
\input{context/copycontext}
\input{context/copylex}
\input{context/makecontext}
\input{context/namecontext}
\input{context/reset}
\input{context/switchcontext}

 	% loading introduction to the section
\newpage
\input{meta/intromet.tex}

% loading explanation of commands
\input{meta/mattach}
\input{meta/reflect}

 
	% ........................... QUICK REFERENCE ..........................
	\newpage
	\appendix
	\section{Open problems}
\label{app-op}

If a {\tt reflect} command is used when a context {\tt META} does not
exist the error produced is that the relevant fact can not be found.
The error message should say that {\tt META} does not exist.

\gap
The {\tt subst} command has problems distinguishing
proof line labels from numbers in equalities.

\gap
The {\tt label} command does not check for overlaps between
labels for proof lines, axioms and theories.



	\newpage
	\newcommand{\syndes}[2]{\item[\parbox{\textwidth}{#1}]\hfill #2}
\newcommand{\module}[1]{\subsection{#1}}

\section{Syntax of commands}

\module{admin}

\begin{description}
\syndes{
   comment \ARG{separator} \OPT{\ARG{text}} \ARG{separator}
}
{}
%3rd-begin%3rd-end

\syndes{
   deflam \ARG{funname} \ARG{var-list} \ARG{form};
}
{}
%3rd-begin%3rd-end

\syndes{
   deflam \ARG{funname} \ARG{var-list} \ARG{form};
}
{}
%3rd-begin%3rd-end

\syndes{
   echo \ARG{separator} \OPT{\ARG{text}} \ARG{separator}
}
{}
%3rd-begin%3rd-end

\syndes{
   hgk \ARG{s-expr};
}
{}
%3rd-begin%3rd-end

\syndes{
   hgk \ARG{s-expr};
}
{}
%3rd-begin%3rd-end

\syndes{
   know natnums \OPT{\ARG{natnum1}, \SEQ \ARG{natnumN}};
}
{}
%3rd-begin%3rd-end

\syndes{
   load \ARG{file};
}
{}
%3rd-begin%3rd-end

\syndes{
   resetprompt;
}
{}
%3rd-begin%3rd-end

\syndes{
   setprompt to \ARG{s-expr};
}
{}
%3rd-begin%3rd-end

\syndes{
   setprompt to \ARG{s-expr};
}
{}
%3rd-begin%3rd-end

\syndes{
   show \ARG{option};
}
{}
%3rd-begin%3rd-end

\end{description}

\module{context}

\begin{description}
\syndes{
  copycontext \ARG{ctx-name};
}
{}
%3rd-begin%3rd-end

\syndes{
	copylex	\ARG{ctx-name};
}
{}
%3rd-begin%3rd-end

\syndes{
  makecontext \ARG{ctx-name};
}
{}
%3rd-begin%3rd-end

\syndes{
  namecontext \ARG{ctx-name};
}
{}
%3rd-begin%3rd-end

\syndes{
  reset;
}
{}
%3rd-begin%3rd-end

\syndes{
  switchcontext \ARG{ctx-name};
}
{}
%3rd-begin%3rd-end

\end{description}

\module{decide}

\begin{description}
\syndes{
  decide \ARG{wff}  \OPT{by \ARG{fact1} \ARG{fact2} \SEQ} using 
  \OPT{\{ \ARG{rewriter} \SEQ \}} \ARG{decider};
}
{}
%3rd-begin%3rd-end

\syndes{
  monad \ARG{wff} \OPT{by \ARG{fact1} \ARG{fact2} \SEQ};
}
{}
%3rd-begin%3rd-end

\syndes{
  monadeq \ARG{wff} \OPT{by \ARG{fact1} \ARG{fact2} \SEQ};
}
{}
%3rd-begin%3rd-end

\syndes{
  ptaut \ARG{wff} \OPT{by \ARG{fact1} \ARG{fact2} \SEQ};
}
{}
%3rd-begin%3rd-end

\syndes{
  taut \ARG{wff} \OPT{by \ARG{fact1} \ARG{fact2} \SEQ};
}
{}
%3rd-begin%3rd-end

\syndes{
  tauteq \ARG{wff} \OPT{by \ARG{fact1} \ARG{fact2} \SEQ};
}
{}
%3rd-begin%3rd-end

\end{description}

\module{eval}

\begin{description}
\syndes{
  assertsimp \ARG{simplabel}; 
}
{}
%3rd-begin%3rd-end

\syndes{
  attach \ARG{indconst}  to \ALT dar [ rep ] \ARG{sexpr};\\
  attach \ARG{sentconst} to T \ALT NIL \ALT UNDEF;\\
  attach \ARG{funconst} \ALT \ARG{predconst} to \ARG{atom};\\
  attach \ARG{funconst}  to [ \ARG{rep1}, \SEQ, \ARG{repN} = \ARG{repM} ]
  \ARG{atom};\\
  attach \ARG{predconst} to [ \ARG{rep1}, \SEQ, \ARG{repN} ] \ARG{atom};
}
{}
%3rd-begin%3rd-end

\syndes{
  decrep \ARG{replabel1} \OPT{\SEQ \ARG{replabelN}};
}
{}
%3rd-begin%3rd-end

\syndes{
  eval \ARG{wff} \ALT \ARG{fact} \ALT \ARG{term} \OPT{by \ARG{simpexpr}};
}
{}
%3rd-begin%3rd-end

\syndes{
  hardware \ARG{indconst} to \ALT dar \ARG{sexpr};\\
  hardware \ARG{indconst} to \ALT dar [ \ARG{rep} ] \ARG{sexpr};
}
{}
%3rd-begin%3rd-end

\syndes{
  let \ARG{\indconst} to \ALT dar [ \ARG{rep} ] \ARG{term};
}
{}
%3rd-begin%3rd-end

\syndes{
  represent \{ \ARG{sort1}, \SEQ, \ARG{sortN} \} as \ARG{rep} \ALT \ARG{*};
}
{}
%3rd-begin%3rd-end

\syndes{
  rewrite \ARG{wff} \ALT \ARG{fact} \ALT \ARG{term} \OPT{by \ARG{simpexpr}};
}
{}
%3rd-begin%3rd-end

\syndes{
  setbasicsimp \ARG{simplabel} at wffs \{ \ARG{wff1} \SEQ \ARG{wffN} \};\\
  setbasicsimp \ARG{simplabel} at facts \{ \ARG{fact1} \SEQ \ARG{factN} \};
}
{}
%3rd-begin%3rd-end

\syndes{
  simplify \ARG{wff} \ALT \ARG{fact} \ALT \ARG{term};
}
{}
%3rd-begin%3rd-end

\end{description}

\module{language}

\begin{description}
\syndes{
   awff \ARG{awff};
}
{}
%3rd-begin%3rd-end

\syndes{
   declare \OPT{\ARG{sentsym} \ALT \ARG{indsym}} \ARG{sym1} \SEQ \ARG{symN}; \\
   declare \OPT{\ARG{funsym}  \ALT \ARG{predsym}} \ARG{sym1} \SEQ \ARG{symN}
   \ARG{arity};\\
   declare \OPT{\ARG{funsym} \ALT \ARG{predsym}} \ARG{sym1} \SEQ \ARG{symN}
   1 \OPT{[ pre \OPT{= \ARG{prbp}} ]};\\
   declare \OPT{\ARG{funsym} \ALT \ARG{predsym}} \ARG{sym1} \SEQ \ARG{symN}
   2 \OPT{[ inf \OPT{= \ARG{lbp} \ARG{rbp}} ] };
}
{}
%3rd-begin%3rd-end

\syndes{
  declare sort \ARG{sym1} \SEQ \ARG{symN};
}
{}
%3rd-begin%3rd-end

\syndes{
  declare indconst \ALT indpar \ALT indvar \ARG{sym1} \SEQ \ARG{symN}
  [ \ARG{sortsym} ]; \\
  declare funconst \ALT funpar \ARG{sym1} \SEQ \ARG{symN}
  ( \ARG{sortsym1} \SEQ \ARG{sortsymN} ) = \ARG{sortsym};\\
}
{}
%3rd-begin%3rd-end

\syndes{
  extension \ARG{sort} by \ARG{extexpr};
}
{}
%3rd-begin%3rd-end

\syndes{
  moregeneral \ARG{sort1} $<$ \ARG{sort2}, \SEQ, \ARG{sortN} $>$;
}
{}
%3rd-begin%3rd-end

\syndes{
  mostgeneral \ARG{sym};
}
{}
%3rd-begin%3rd-end

\syndes{
  setfmap \ARG{funsym} ( \ARG{sym1} \SEQ \ARG{symN} ) = \ARG{sym};
}
{}
%3rd-begin%3rd-end

\syndes{
   term \ARG{term};
}
{}
%3rd-begin%3rd-end

\syndes{
   wff \ARG{wff};
}
{}
%3rd-begin%3rd-end

\end{description}

\module{meta}

\begin{description}
\syndes{
   mattach \ARG{indconst} to \ALT dar \OPT{[rep]}
   \ARG{cname}:\ARG{pname}:\ARG{sort}:\ARG{object};
}
{}
%3rd-begin%3rd-end

\syndes{
  reflect \ARG{M-fact} \ARG{arg1} \ARG{arg2} \SEQ \ARG{argN};
}
{}
%3rd-begin%3rd-end

\end{description}

\module{nd}

\begin{description}
\syndes{
   alle \ALT us \ARG{fact} \OPT{,} \ARG{term1} \ARG{term2} \SEQ;
}
{}
%3rd-begin%3rd-end

\syndes{
   alli \ALT ug \ARG{fact} \OPT{\OPT{,} \ARG{indvar1} \ALT \ARG{indpar1} :} 
                           \ARG{indvar11}
                           \OPT{\OPT{,} \ARG{indvar2} \ALT \ARG{indpar2} :}
                           \ARG{indvar22} \SEQ;
}
{}
%3rd-begin%3rd-end

\syndes{
   ande \ALT ae \ARG{fact} \OPT{,} 1 \ALT 2; \\
   ande \ALT ae \ARG{fact} \OPT{,} 1 \ALT 2 1 \ALT 2 \SEQ;
}
{}
%3rd-begin%3rd-end

\syndes{
   andi \ALT ai \ARG{fact1} \OPT{,} \ARG{fact2}; \\
   andi \ALT ai \ARG{fact11}
                \OPT{conj \ALT cj \ARG{fact12} conj \ALT cj \ARG{fact13} \SEQ}
                \OPT{,}
                \ARG{fact21}
                \OPT{conj \ALT cj \ARG{fact22} conj \ALT cj \ARG{fact23} \SEQ};
}
{}
%3rd-begin%3rd-end

\syndes{
   assume \ARG{wff1} \OPT{\OPT{,} \ARG{wff2} \SEQ};
}
{}
%3rd-begin%3rd-end

\syndes{
   existe \ALT es \ARG{fact}  \OPT{,} \ARG{indvar1} \ALT \ARG{indpar1}
                  \OPT{,} \ARG{indvar2} \ALT \ARG{indpar2} \SEQ;
}
{}
%3rd-begin%3rd-end

\syndes{
  existi \ARG{fact}
   \OPT{\OPT{,} \ARG{term1} :} \ARG{indvar1} \OPT{occ \ARG{n11} \ARG{n12} \SEQ}
   \OPT{\OPT{,} \ARG{term2} :} \ARG{indvar2} \OPT{occ \ARG{n21} \ARG{n22} \SEQ}
   \SEQ;
}
{}
%3rd-begin%3rd-end

\syndes{
   falsee \ALT fe \ARG{fact1} \OPT{,} \ARG{wff};\\
   falsee \ALT fe \ARG{fact1} \OPT{,} \ARG{fact2};
}
{}
%3rd-begin%3rd-end

\syndes{
   falsei \ALT fi \ARG{fact1} \OPT{,} \ARG{fact2}; 
}
{}
%3rd-begin%3rd-end

\syndes{
   iffe \ALT ie \ARG{fact} \OPT{,} 1 \ALT 2;
}
{}
%3rd-begin%3rd-end

\syndes{
   iffi \ALT ii \ARG{fact1} \OPT{,} \ARG{fact2};
}
{}
%3rd-begin%3rd-end

\syndes{
   impe \ALT mp \ARG{fact1} \OPT{,} \ARG{fact2};
}
{}
%3rd-begin%3rd-end

\syndes{
   impi \ALT ded  \ARG{fact1} \OPT{, \ALT imp} \ARG{fact};\\
   impi \ALT ded  \ARG{wff} \OPT{, \ALT imp} \ARG{fact};
}
{}
%3rd-begin%3rd-end

\syndes{
   note \ALT ne \ARG{fact1} \OPT{,} \ARG{wff}; \\
   note \ALT ne \ARG{fact1} \OPT{,} \ARG{fact2};
}
{}
%3rd-begin%3rd-end

\syndes{
   noti \ALT ni \ARG{fact1} \OPT{,} \ARG{wff}; \\
   noti \ALT ni \ARG{fact1} \OPT{,} \ARG{fact2};
}
{}
%3rd-begin%3rd-end

\syndes{
   ore \ALT oe \ARG{fact1} \OPT{,} \ARG{fact2} \OPT{,} \ARG{fact2};
}
{}
%3rd-begin%3rd-end

\syndes{
   ori \ALT oi \ARG{fact} \OPT{,} \ARG{wff} \OPT{,} \OPT{lr \ALT rl};\\
   ori \ALT oi \ARG{fact} \OPT{,} \ARG{fact1} \ALT \ARG{wff1} 
       disj \ALT dj \ARG{fact2} \ALT \ARG{wff2} disj \ALT dj \SEQ
       \OPT{,} \OPT{lr \ALT rl};
}
{}
%3rd-begin%3rd-end

\syndes{
  subst \ARG{fact1} \OPT{with} \ARG{fact2};\\
  subst \ARG{fact1} \OPT{with} \ARG{fact2} \OPT{right \ALT left};\\
  subst \ARG{fact1} \OPT{with} \ARG{fact2} 
                    \OPT{occ \ARG{n1} \ARG{n2} \SEQ} \OPT{right \ALT left};
}
{}
%3rd-begin%3rd-end

\end{description}

\module{parser}

\begin{description}
\syndes{
  backup \ARG{file} open;\\
  backup \ARG{file} close;  
}
{}
%3rd-begin%3rd-end

\syndes{done;}
{}
%3rd-begin%3rd-end

\syndes{
   fetch \ARG{file} \OPT{from \ARG{mark1}} \OPT{to \ARG{mark2}};
}
{}
%3rd-begin%3rd-end

\syndes{
   mark \ARG{sym};
}
{}
%3rd-begin%3rd-end

\syndes{
   probe;\\
   probe \ARG{activity};\\
   probe all;\\
}
{}
%3rd-begin%3rd-end

\syndes{
   unprobe \ARG{activity};\\
   unprobe all;
}
{}
%3rd-begin%3rd-end

\end{description}

\module{proof}

\begin{description}
\syndes{
  axiom \ARG{sym} : \ARG{wff};
}
{}
%3rd-begin%3rd-end

\syndes{
   cancel \OPT{\ARG{label}};
}
{}
%3rd-begin%3rd-end

\syndes{
  copyproof \ARG{prf-name};
}
{}
%3rd-begin%3rd-end

\syndes{
   label fact \ARG{sym};\\
   label fact \ARG{sym} = \ARG{label};
}
{}
%3rd-begin%3rd-end

\syndes{
   makeproof \ARG{prf-name};
}
{}
%3rd-begin%3rd-end

\syndes{
   nameproof \ARG{prf-name};
}
{}
%3rd-begin%3rd-end

\syndes{
  switchproof \ARG{prf-name};
}
{}
%3rd-begin%3rd-end

\syndes{
  theorem \ARG{sym} \ARG{hook};
}
{}
%3rd-begin%3rd-end

\syndes{
  theory \ARG{thlabel} : \ARG{wff1} \OPT{\ARG{wff2} \SEQ};\\
  theory \ARG{thlabel} : \ARG{axlabel1} : \ARG{wff}
                         \OPT{\ARG{axlabel2} : \ARG{wff} \SEQ};
}
{}
%3rd-begin%3rd-end

\end{description}

\module{rules}

\begin{description}
\syndes{
	contract \ALT ctc  \ARG{fact} by \ARG{assumption1} \SEQ  \ARG{assumptionN}; 
}
{}
%3rd-begin%3rd-end

\syndes{
	cut \ARG{fact1} \ARG{fact2};\\
	cut \ARG{fact1} \ARG{fact2} \OPT{keep \ARG{assumption1} \SEQ
	\ARG{assumptionN}};
}
{}
%3rd-begin%3rd-end

\syndes{
	termife \ARG{fact1} \ARG{fact2} \ARG{termif}; \\
	termife \ARG{fact1} \ARG{fact2} \ARG{termif} \OPT{occ \ARG{n1} \ARG{n2}
	\SEQ}; 
}
{}
%3rd-begin%3rd-end

\syndes{
	termifen \ARG{fact1} \ARG{fact2} \ARG{termif}; \\
	termifen \ARG{fact1} \ARG{fact2} \ARG{termif} \OPT{occ \ARG{n1} \ARG{n2}
	\SEQ}; 
}
{}
%3rd-begin%3rd-end

\syndes{
	termifi \ARG{fact1} \ARG{fact2} \ARG{wff} \ARG{term1} \ARG{term2};
}
{}
%3rd-begin%3rd-end

\syndes{
	weaken \ALT wk \ARG{fact} by \ARG{fact1} \SEQ \ARG{factN};
}
{}
%3rd-begin%3rd-end

\syndes{
	wffife \ARG{fact1} \ARG{fact2};
}
{}
%3rd-begin%3rd-end

\syndes{
	wffifen \ARG{fact1} \ARG{fact2};
}
{}
%3rd-begin%3rd-end

\syndes{
	wffifi \ARG{wff} \ARG{fact1} \ARG{fact2};
}
{}
%3rd-begin%3rd-end

\end{description}



	% ............................. BIBLIOGRAPHY ...........................
	\newpage
	\bibliographystyle{alpha}
    \gfbibliography
	
	% ................................ INDEX ...............................
	\newpage
	%............................... USER MANUAL .................................
%.............................................................................

\documentstyle[12pt,rawfonts]{../styfiles/GFmanual}

\title{GETFOL Manual}
\author{\bf Fausto Giunchiglia}
\date{7 March 1994}
\version{2.0}
\abstract{
	  {\GF} is an interactive reasoning system.
	  We use it as an environment for studying epistemological issues.
	  We try to look at questions like: which notions are
	  important for the development of mechanized reasoning systems?
	  What kind of conversations do we want to have with them?
	  What parts of logic should we use to represent such notions?
	  How should logic be embedded in a conversational reasoning system?
	}
\addresses{
     \begin{tabular}[c]{l}
       {\bf Fausto Giunchiglia}           \\
       Mechanized Reasoning Group		  \\
		 IRST, Povo, 38050 Trento, Italy  \\
		 e-mail: {\tt fausto@irst.it}     \\
		 phone: +39 461 314436
	\end{tabular}	
}
\published{
  \begin{tabular}{l}
	  DIST Technical Report No. 92-0010 (1994). \\
	  DIST -- University of Genoa,\\
	  Via Opera Pia 11A, 16145 Genova, Italy.\\ \\
  \end{tabular}
}

%% \newcommand{\gfbibliography}{%
%% \bibliography{/home/tarski/staff/mrg/biblio/bib/a-l,%
%% /home/tarski/staff/mrg/biblio/bib/m-z,userman}}
\newcommand{\gfbibliography}{\bibliography{}}

%% \makeindex
\begin{document}
	%  ............................. COVER ..................................
	\thispagestyle{empty}
	\maketitle

	%  ........................ TABLE OF CONTENTS ...........................
	\newpage
	\pagenumbering{roman}
	\tableofcontents

	\newpage
	\pagenumbering{arabic}
	\pagestyle{headings}

	%  ........................... INTRODUCTION ............................
	\input{intro/intro.tex}

	%  .............................. MODULES ..............................
	\input{parser/parser}
	\input{admin/admin}
 	\input{language/language}
 	\input{proof/proof}
 	\input{nd/nd}
	\input{rules/rules}
 	\input{decide/deciders}
 	\input{eval/evaluator}
 	\input{context/context}
 	\input{meta/meta}
 
	% ........................... QUICK REFERENCE ..........................
	\newpage
	\appendix
	\input{appendix/openprob.tex}

	\newpage
	\input{appendix/commands.tex}

	% ............................. BIBLIOGRAPHY ...........................
	\newpage
	\bibliographystyle{alpha}
    \gfbibliography
	
	% ................................ INDEX ...............................
	\newpage
	\input{userman.ind}
\end{document}

\end{document}

\end{document}

\end{document}
