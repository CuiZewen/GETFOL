\section{Parser}

This section is intended to explain:
%
\begin{itemize}
	\item
		the main functionalities of the {\GF} scanning primitives,
	\item
		what modifications must be performed in the scanner data
		structures in order to be able to change its behavior ({\em e.g.} to
		define a new escape character).
\end{itemize}

The {\GF} scanner is a {\em backupable} scanner.
This feature is necessary as the parser works in a top-down fashion and,
sometimes, needs to backtrack. 

The {\GF} scanner is able to recognize three types of tokens:
%
\begin{enumerate}
	\item {\tt IDTOKEN},
	\item {\tt NUMTOKEN} and
	\item {\tt DELTOKEN}.
\end{enumerate}

The primitives to scan such tokens are \verb+FOLSYM@+, \verb+NATNUM@+ (actually
no dedicated primitive to scan a \verb+DELTOKEN+ exists).

\verb+TK@+ scans a generic token.

The primitives \verb+SCANSTATUS-GET+ and \verb+SCANSTATUS-RESTORE+ allow
respectively to save and restore the scanner status.
They are necessary to restore the status of the scanner in case of
unsuccessful parsing.

The high level functions \verb+FOLSYM@+, \verb+NATNUM@+, \verb+TK@+ (and other
not mentioned here for brevity) use the general \verb+TOKEN-GET-NEXT+ routine
which is also able to {\em bufferize} the tokens already read (reading from
an input stream is a destructive operation so we need at a some level a
buffer mechanism if we want to perform backtracking).

The buffer is implemented by the array \verb+TOKENARRAY+ whose dimension is
given by the macro \verb+TOKENARRAY-DIMENSION+. The max number of tokens
that can be present in a command line is given by the number returned
by this macro.

\verb+TOKEN-GET-NEXT+ routine is based on the lower level primitive
\verb+TOKEN-SCAN+, which reads the next token from the input stream and returns
its type.
\verb+TOKEN-SCAN+ is able to identify the type of a token reading (via 
\verb+CH-GET-NEXT+) its first character and identifying its type (via
\verb+CHTYPE-GET+).

The characters are divided in the following types:

\begin{enumerate}
	\item
		identifiers ({\tt IDCHAR}): see file
		\verb+ascitab.fol+;
	\item
		numbers (positive integers --- {\tt NUMCHAR}):
		\verb+0 1 2 3 4 5 6 7 8 9+;
	\item
		delimiter {\tt DELCHAR}:
		\verb+( ) , . : ; [ ] { }+;
	\item
		ignored char {\tt IGNCHAR}: see file
		\verb+ascitab.fol+;
	\item
		iddelim {\tt IDDELCHAR}:
        \verb$' * + - / < > = ? @ ^ ` |$;
	\item
		escape characters {\tt ESCCHAR}:
		\verb+\+;
	\item
		special handling {\tt SPECCHAR}
\end{enumerate}

The \verb+IDDELCHAR+s are identifier characters having also the functionality
of a delimiter, that is no \verb+IDTOKEN+ can contain a character of such
type but they by themselves may be considered \verb+IDTOKEN+.
For instance the string \verb+A@B+ will be regarded by the scanner a string
formed by the three distinct \verb+IDTOKEN+ \verb+A @ B+.
The escape characters allows to coerce the type of the following
character to be \verb+IDCHAR+.
For istance the string \verb+A\@B+ will be regarded by the scanner a string
formed by the \verb+IDTOKEN A@B+.

The only think important to know at the user level it is how to modify the type
of a character so that you can, for istance, extend the set of \verb+IDDELCHAR+
with the character "\verb+_+". To do this you have only to change type
declaration for the "\verb+_+" character from \verb+IDCHAR+ to \verb+IDDELCHAR+
in the file {\tt asciitab.fol}.

Finally a low level feature: each time a command line is issued to {\GF} the low
level scanning primitives store each read character in the {\tt TUPLE
SCANBUFARRAY}, so that if a syntactic error in the parsing is detected the rest
of the line will be read and the whole line printed out to show, using an
"\verb+^+", where the syntatic error has been detected. This useful feature
imposes (as it is implemented) a limit in the length of a {\GF} command line as
the dimension of \verb+SCANBUFARRAY+ is fixed by the value returned by the macro
\verb+SCANBUFARRAY-DIMENSION+.
