\gfcommand{termife}{term conditional elimination}
\index{termife}

\gfsyntax{
	termife \ARG{fact1} \ARG{fact2} \ARG{termif}; \\
	termife \ARG{fact1} \ARG{fact2} \ARG{termif} \OPT{occ \ARG{n1} \ARG{n2}
	\SEQ}; 
}

\gfdescription{
	\renewcommand{\arraystretch}{0.5}
	\[
	\mbox{{\em tif}} \  E \ \ 
	\fraz{A \ \ \ \ \ B(\mbox{{\em termif}} \ A \ \mbox{{\em then}} \ t_1 \
	\mbox{{\em else}} \ t_2)} 
	{B(t_1)}
	\]
	\renewcommand{\arraystretch}{1}

	If \ARG{termif} is {\tt iftrm A then t1 else t2}, \ARG{fact1}'s wff is
	{\tt W(iftrm A then t1 else t2)} and {\em fact}$_2$'s wff is {\tt A},  then
	the rule deduces {\tt W(t1)}.
	If {\em termif} is not a subexpression of \ARG{fact1}'s wff, then no
	substitution is performed.
	Individual occurrences can be substituted by specifying the optional 
	\ARG{n1} \ARG{n2}, \SEQ, where \ARG{n1}, \ARG{n2}, \SEQ are the occurrences
	to be substituted.
	Without this option, all occurrences are substituted. 
	The dependencies of the derived wff are the union of those of \ARG{fact1}
	and \ARG{fact2}. 
}

\gfrecap{
Term conditional elimination.
}

\gfexample+
   ***** declare sentconst A;
   ***** declare predconst P 1;
   ***** declare indpar a b;
   ***** assume P(trmif A then a else b);
   1   P(trmif A then a else b)     (1)
   ***** assume A;
   2   A     (2)
   ***** termife 1 2 trmif A then a else b;
   3   P(a)     (1 2)
+

\gfnotes{}
