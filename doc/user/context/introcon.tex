\section{Multiple contexts}
\label{sec-cxt}

\subsection{Introduction}

\begin{quote}\em
  ... When reasoning, people seem to be able to switch focus of their
  attention and make always some sort of local reasoning ...
  \cite{giunchiglia2}
\end{quote}

Structuring the knowledge into {\em distinguished partial descriptions of the
world}, has been hinted as a cognitively plausible hypotheses. 
Distinct partial descriptions can be represented by {\GF} {\bf contexts}.
A {\GF} context contains its own language defined by a set of declarations, 
its own axioms and definitions, and its own computational model.
Reasoning can be performed within a context. 
You can type any command defined so far within any context in {\GF}.
Multiple proofs can be performed within a context.
When you work in {\GF} you are always in one context.
The context in which you are working in is called the {\em current context}.
When you enter the system the current context is empty and without name.
If you want to leave the context to work in another one, you have to give the
context a name to refer to it later (by the command {\tt namecontext}).
You can create a new context by using {\tt makecontext}, and  switch to it
by using {\tt switchcontext}.
The context you switch to then becomes the current context.
